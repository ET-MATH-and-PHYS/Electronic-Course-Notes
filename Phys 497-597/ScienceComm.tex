%%%%%%%%%%%%%%%%%%%%% Chapter 2.1
\chapter{General Advice}



\section{Writing the Laboratory Report}

\subsection{Where to Begin}

You must consider four things:
\begin{itemize}
    \item[1)] Audience 
    \item[2)] Format
    \item[3)] Mechanics
    \item[4)] Politics
\end{itemize}

Keep in mind that the \textbf{goal of scientific writing is either to inform or persuade}.


\textbf{Audience:} Your document must make a connection with your audience. It is important to decide who will read your report.


\textbf{Goal=Inform:} Efficiency is the style that communicates the most amount of information in the least amount of reading time. Avoid writing a murder mystery where you save the ``best for last." Place important details where they stand out.


\textbf{Goal = Persuade:} Efficiency presents logical arguments in the most convincing manner. You may withhold your conclusions until later in the text. This allows you time to establish credibility with your audience.

Consider the following:
\begin{itemize}
    \item Who will read the document?
    \item What do they know about the subject?
    \item Why will they read the document? What information is sought?
    \item How will they read the document? Will they read from start to finish?
\end{itemize}


\textbf{Format:} While you cannot control the format, you can control the style; including word choices, complexity of illustrations, etc.


\textbf{Mechanics:} This entails the rules of grammar and punctuation.


\textbf{Politics:} In an ideal world you ``stay honest." However, political constraints will force you to face your ethical responsibilities as a scientist and act accordingly.


\subsection{Stylistic Tools}

The first tool you should consider is the \textbf{structure} of the document. This includes the organization of details, depth of the presentation, transitions between ideas, and emphasis. If the structure fails, your readers are lost.


\textbf{Language} is another important stylistic tool. The thoughtful selection of words and the arrangement of those words in sentences can make the difference between a useful paper and a document that misses the mark completely. Consider the following:
\begin{itemize}
    \item \textbf{Precision:} Write what you mean
    \item \textbf{Clarity:} avoid saying what you don't mean
    \item \textbf{Forthright:} be sincere and straightforward
    \item \textbf{Use familiar language}
    \item \textbf{Make every word count}
    \item \textbf{Fluid writing:} smooth transition between different sections and ideas.
\end{itemize}

The integration of \textbf{illustrations} can also help to convey your topic effectively.


\textbf{Structure:} Basically, a scientific paper has a beginning, middle, and end.

The \textbf{beginning} of the documet has one task: to prepare the reader for understanding the document's middle. The single most important component of the beginning is the \textbf{title}. The title identifies the field of study and goes on to separate the document from all others in the field. Your title should say what the document is about and should entice your audience to actually read through the paper. Avoid phrases that your audience might not recognize.

You must also provide a \textbf{summary} or \textbf{abstract} for your report. \textbf{This component should give away the show from the beginning}. The summary is where you emphasize the most important details. Your summary should describe the goals of the work, give a descriptive summary of the results, and how these results were achieved. The first sentence of the summary orients the audience toward the identity of the work.

In writing your \textbf{introduction} you could consider:
\begin{itemize}
    \item \textbf{What} exactly is the work? (Goal or Objective)
    \item \textbf{Why} is the work important? (Motivation)
    \item \textbf{What} is needed to understand the work?
    \item \textbf{How} will the work be presented?
\end{itemize}

Note that the introduction may not have to answer all of these questions. However, it should specify the scope (aspects of the project) and limitations (assumptions that restrict the boundaries of the work). You should demonstrate why your work is important and provide some incentive for the reader to read your work or convince funding agencies to give you money. You should try to instill this importance/curiosity in your readers. 

The amount of background material you provide depends on your audience. Providing historical background should be done only \textbf{if it serves your readers}. In many documents, other kinds of information, such as definitions of key terms, are more important. Some background information may be more appropriate elsewhere in the document. The introduction is the background material that applies to the entire document.

It is good practice to map the document in the introduction, providing the reader with an idea of how the ideas and topics in the document will flow.


The \textbf{middle} is where the bulk of information is found and the outcomes are presented and interpretations are developed. You may take a \textbf{chronological approach}. You may choose to use a \textbf{spatial} approach, such that you develop the ideas according to the physical shape or form of the object. A common approach in Science is to follow the \textbf{flow of a variable}. 

You should envision a path before you start writing and read your document to see if it works for the subject matter and audience. 

Section titles should be clear and you should avoid one-word headings. Consider developing titles that parallel each other.


At the \textbf{end} of the document you provide closure, an analysis of the most important results, and future perspectives on your work. In the end of the document, your should treat the results as a whole. Do not introduce new material at this stage. Leave your readers with a sense of the impact of your work on the proverbial ``Big Picture."


\subsection{Format}

Although the exact style and names might be different, you should find that the contents of the sections are equivalent. 

\begin{itemize}
    \item[1.] \textbf{Title:} The title should be short and straightforward to appeal to a general reader but detailed enough to properly reflect the contents of the article. Think about keywords and use recognizable, searchable terms. Avoid the use of non-standard abbreviations and symbols.
    \item[2.] \textbf{Authorship:} Full names and affilitations for all the authors should be included. Everyone who made a significant contribution to the conception, design or implementation of the work should be listed as co-authors. THe corresponding author has the responsibilsity to include all (and only) co-authors. The corresponding author also signs a copyright licence on behalf of all the authors. If there are more than 10 co-authors on the manuscript, the corresponding author should provide a statement to specify the contribution of each co-author. Identify co-corresponding authors on your manuscript's first page and also mention this in your comments to the editor and/or cover letter.
    \item[3.] \textbf{Abstract:} The abstract should be a single paragraph (50-250 words) that summarises the content of the article. It should set out briefly and clearly the main objectives and results of the work; it should give the reader a clear idea of what has been achieved. Like your title, make sure you use recognisable searchable terms and keywords.
    \item[4.] \textbf{Keywords:} Think carefully of those terms that will help identify what you did and where the work is relevant. Avoid abbreviations and plural forms if possible. Be specific and avoid overly general terms and concepts.
    \item[5.] \textbf{Introduction:} An introduction should `set the scene' of the work. Context is \textbf{EVERYTHING} and your introductino should clearly explain both the nature of the problem under investigation and its background. What are you doing and why is it significant? How is this work related to our current knowledge and what new knowledge will your work generate? What might be the broader implications being careful not to overstate this? It should start off general and then focus in to the specific research question you are investigating. Describe if there are any challenges or controversies involved in the topic. Ensure you include all relevant references. Define any important terms or concepts that might not be familiar to your readers. THe concluding sentence or two should be a concise statement of what the paper will accomplish. What \textbf{EXACTLY} is being tested or investigated?
    \item[6.] \textbf{Experimental:} You should provide descriptions of the experiments in enough detail so that a skilled researcher is able to repeat them. Standard techniques and methods used throughout the work should just be stated at the beginning of the section; descriptions of these are not needed. Any unusual hazards about the procedures or equipment should be clearly identifies.
    \item[7.] \textbf{Results \& Discussion:} Your results should be organized into an orderly and logical sequence. Only the most relevant results should be described in the text; to highlight the most important points. Figures, tables, and equations should be used for purposes of clarity and brevity. Data should not be reproduced in more than one form, for example in both figures and tables, without good reason. 

        The purpose of the discussion is to explain the meaning of your results and why they are important. You should state the impact of your results compared with recent work and relate it back to the problem or question you posed in your introduction. Ensure claims are backed up by evidence and explain any complex arguments.
    \item[8.] \textbf{Conclusions:} State the most important outcome of your work clearly and concisely. Place the outcome in some context; the ``big picture." Relate your outcome to the goal and motivation described in the introduction. Discuss where there are opportunities for future work. Do not repeat ideas from the introduction. A conclusion can be short.
    \item[9.] \textbf{Conflicts of interest:} Ensure a conflicts of interest statement is included in your manuscript here. If no conflicts exist, please state that `There are no conflicts to declare'. 
    \item[10.] \textbf{Acknowledgements:} Contributors (that are not included as co-authors) may be acknowledged; they should be as brief as possible. All sources of funding should be declared.
    \item[11.] \textbf{References:} THe list of papers, book chapters, theses, reports, webpages, etc. is provided at the end of your paper. In this course, references are cited in the order in which they appear in the paper. For a webpage citation use the following format:
        \begin{quotation}
            Last, FM. (Year, Month, Date Published). Article title [Type of blog post]. Retrieved (on Year, Month, Date) from URL.
        \end{quotation}
\end{itemize}

\subsection{Paper elements}

\begin{itemize}
    \item[1.] \textbf{Footnotes:} Footnotes relating to the title and/or authors, including affilitations, should appear at the very bottom of the first page of the article.
    \item[2.] \textbf{Data tables:} You will usually not include ALL your data in the final report. Select the data which are most relevant. You need to include sufficient evidence to demonstrate the methodology that was used is robust and appropriate. You want to show that you have followed sound metrological principles, for example calibration and assessment of uncertainties. You will also want to arrange the results in a manner that conveys the outcomes clearly and compactly. Some results may be better suited to presentation as a figure rather than a table, and you need to decide what is more effective while remaining transparent and honest.
        The table will contain columns of information and a useful identifier should be centered above each column. Make sure to include units if necessary. Uncertainty should be incorporated into th etable entries. THis can be done for each individual entry, as a separate column, or, if the uncertainties are the same for all entries, at the top or bottom of the column. A table may have an associated caption, although this is not always done. There must be a reference to the table in the main body of the text. The font size for the table may be smaller than the main body in order to accommodate all the entries. It is not good practice to split data tables, and it may be better to change the orientation of the table from portrait to landscape to avoid splitting the table.
    \item[3.] \textbf{Figures and images:} It is worth spending some effort to create meaningful images. Figures are always accompanied by a figure caption that is attached to the figure itself. Figures never have titles. The caption should be a summary of what is being shown/plotted and anything of significance. The full explanation of the figure is in the body of the paper. If you are providing a plot or graph, make sure to include axis titles with units. Use a font size that is easy to read, which might be a few points large than you use in the document. 

        Uncertainties should be provided on the plots when appropriate. Place major and minor tick marks on the plot. Use a range for the axes that uses the full width and height of the plot so that you avoid squishing the data into a very small area of the figure.
    \item[4.] \textbf{Equations:} The equations should be labeled and in a separate line of the text. The equation itself is right justified and the equation number is left justified. 
\end{itemize}

\subsection{Other considerations}


\textbf{Inclusive language:} Inclusive language acknowledges diversity, conveys respect to all people, is sensitive to differences, and promotes equal opportunities. Authors should ensure that writing is free from bias.




\section{How To Speak}


\section{Writing A Scientific Paper}

\subsection{10 Steps}

\textbf{Step 1: Write a Vision Statement} - What is the key message of your paper? Be able to articulate it in one sentence, because it's a sentence you'll come back to a few times throughout the paper. Think of your paper as a press release: what would the subhead be? 

The vision statement should guide your next important decision: where are you submitting? Once you choose a journal, check the website for requirements with regards to formatting, length limits, and figures.


\textbf{Step 2: Don't Start at the Beginning} - If you start with the introduction, by the time everything else is written, you will likely have to rewrite both sections.


\textbf{Step 3: Storyboard the Figures} - Figures are the best place to start, because they form the backbone of your paper. The first figure should inspire the reader to want to learn about your discovery.

Consider ``storyboarding" where all figures are laid out on boards. One approach is to put the vision statement on the first slide of a powerpoint, and all your results on subsequent slides. To start, simply include all data, without concern for order or importance. Subsequent passes can evaluate consolidation of data sets and relative importance. The figures should be arranged in a logical order to support your hypothesis statement. Notably, this order may or may not be the order in which you took the data.


\textbf{Step 4: Write the Methods Section} - The methods section is the most important to write accurately. ANy results in your paper should be replicable based on the methods section. If you've developed an entirely new experimental method, write it out in excruciating detail, including setup, controls, and protocols, also manufacturers and part numbers, if appropriate. If you're building on a previous study, there's no need to repeat all of those details; that's what references are for.

The methods section is simply a record of what you did (no results!). 


\textbf{Step 5: Write the Results and Discussion Section} - This section(s) should form the bulk of your paper-by storyboarding your figures, you already have an outline! 

A good place to start is to write a few paragraphs about each figure, explaining: 1. the result (this should be void of interpretation), 2. the relevance of the result to your hypothesis statement (interpretation is beginning to appear), and 3. the relevance to the field (this is completely your opinion). Whenever possible, you should be quantitative and specific, especially when comparing to prior work. Additionally, any experimental errors should be calculated and error bars should be included on experimental results along with replicate analysis.

You can use this section to help readers understand how your research fits in the context of other ongoing work and explain how your study adds to the body of knowledge. This section should smoothly transition into the conclusion.


\textbf{Step 6: Write the conclusion} - Summarize everything you have already written. Emphasize the most important findings from your study and restate why they matter. State what you learned and end with the moost important thing you want the reader to take away from the paper-again, your vision statement. 


\textbf{Step 7: Now Write the Introduction} - The introduction sets the stage of your article. The introduction gives a view of your researh from 30,000 feet: it defines the problem in the context of a larger field; it reviews what other research groups have done to move forward on the problem (the literature review); and it lays out your hypothesis, which may include your expectations about what the study will contribute to the body of knowledge. THe majority of your references will be located in the introduction.


\textbf{Step 8: Assemble References} - References serve multiple roles in a manuscript:
%%
\begin{itemize}
    \item[1)] To enable a reader to get more detailed information on a topic that has been previously published.
    \item[2)] To support statements that are not common knowledge or may be contentious.
    \item[3)] To recognize others working in the field, such as those who came before you and laid the groundwork for your work as well as more recent discoveries. The selection of these papers is where you need to be particularly conscientious. You need to make sure that your references include both foundational papers as well as recent works.
\end{itemize}


\textbf{Step 9: Write the Abstract} - The abstract is the elevator pitch for your article. Most abstracts are 150-300 words, or approximately 10-20 sentences. It should describe the importance of the field, the challenge that your research addresses, how your research solves the challenge, and its potential future impact. It should include any key quantitative metrics.


\textbf{Step 10: The Title Comes Last} - The title should capture the essence of the paper.  If someone was interested in your topic, what phrase or keywords would they type into a search engine?


\subsection{First-Class Paper Writing}

\textbf{Keep your message clear}. This is even more important when there is a multidisciplinary group of authors. Authors should sit together (preferably in person) and seek consensus --- not only in the main message, but also in the selection of data, the visual presentation and information necessary to transmit a strong message.


Writers should put their results into a global context to demonstrate what makes those results significant or original. 

There is a narrow line between speculation and evidence-based conclusions. Speculation is okay in the discussion, but only to a degree! In the conclusion, include a one-or-two sentence statement on the research you plan to do in the future and on what else needs to be explored.


\textbf{Create a logical framework}. In each paragraph the first sentence defines the context, the body contains the new idea, and the final sentence offers a conclusion. For the whole paper, the introduction sets the context, the results present the content, and the discussion brings home the conclusion.

It is crucial to focus your paper on a single key message, which you communicate in the title. Everything in the paper should logically and structurally support that idea.


\textbf{State your case with confidence}. Answering one central question --- What did you do? --- is the key to finding the structure of a piece. Every section of the manuscript needs to support that one fundamental idea.


\textbf{Beware the curse of `zombie nouns'}. Scientific writing should be factual, concise, and evidence-based, but it should also be creative, original, and engaging.

We should engage readers' emotions and avoid formal, impersonal language. Still, there's a balance. Once the paper has a clear message, try some vivid language to help to tell the story.


\textbf{Prune that purple prose}. Make the writing only as complex as it needs to be. 

They need to explain why the findings are interesting and how they affect a wider understanding of the topic. Authors should also reassess the existing literature and consider whether their findings open the door for future work. And, in making clear how robust their findings are, they must convince readers that they’ve considered alternative explanations. 


\textbf{Aim for a wide audience}. Articles with clear, succinct, declarative titles are more likely to get picked up by social media or the popular press. Make your point clearly and concisely---if possible in non-specialist language, so that readers from other fields can quickly make sense of it.
