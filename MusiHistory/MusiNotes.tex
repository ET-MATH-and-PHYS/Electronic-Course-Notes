\documentclass[12pt, a4paper, twoside, openright, titlepage]{book}
\usepackage[utf8]{inputenc}
\raggedbottom
%%%%%%%%%%%%%%%%% Book Formatting Comments:

%%%%%%%%%%%%%%%%%%%%%%%%%%%%%%%%%%%%% for Part

%%%%%%%%%%%%%%%%%%%%%% for chapter

%%%%%%%%%%%%%%%%%%%% for section








%%%%%% PACKAGES %%%%%%%
\usepackage{hyperref}
\hypersetup{
    colorlinks,
    citecolor=black,
    filecolor=black,
    linkcolor=black,
    urlcolor=black
}
\usepackage{amsmath} % Math display options
\usepackage{amssymb} % Math symbols
\usepackage{amsfonts} % Math fonts
\usepackage{amsthm}
\usepackage{mathtools} % General math tools
\usepackage{array} % Allows you to write arrays
\usepackage{empheq} % For boxing equations
\usepackage{mathabx}
\usepackage{mathrsfs}
\usepackage{nameref}

\usepackage{soul}
\usepackage[normalem]{ulem}

\usepackage{txfonts}
\usepackage{cancel}
\usepackage[toc, page]{appendix}
\usepackage{titletoc,tocloft}
\setlength{\cftchapindent}{1em}
\setlength{\cftsecindent}{2em}
\setlength{\cftsubsecindent}{3em}
\setlength{\cftsubsubsecindent}{4em}
\usepackage{titlesec}

\titleformat{\section}
  {\normalfont\fontsize{25}{15}\bfseries}{\thesection}{1em}{}
\titleformat{\section}
  {\normalfont\fontsize{20}{15}\bfseries}{\thesubsection}{1em}{}
\setcounter{secnumdepth}{1}  
  
  

\newcommand\numberthis{\refstepcounter{equation}\tag{\theequation}} % For equation labelling
\usepackage[framemethod=tikz]{mdframed}

\usepackage{tikz} % For drawing commutative diagrams
\usetikzlibrary{cd}
\usetikzlibrary{calc}
\tikzset{every picture/.style={line width=0.75pt}} %set default line width to 0.75p

\usepackage{datetime}
\usepackage[margin=1in]{geometry}
\setlength{\parskip}{1em}
\usepackage{graphicx}
\usepackage{float}
\usepackage{fancyhdr}
\setlength{\headheight}{15pt} 
\pagestyle{fancy}
\lhead[\leftmark]{}
\rhead[]{\leftmark}

\usepackage{enumitem}

\usepackage{url}
\allowdisplaybreaks

%%%%%% ENVIRONMENTS %%%
\definecolor{purp}{rgb}{0.29, 0, 0.51}
\definecolor{bloo}{rgb}{0, 0.13, 0.80}



%%\newtheoremstyle{note}% hnamei
%{3pt}% hSpace above
%{3pt}% hSpace belowi
%{}% hBody fonti
%{}% hIndent amounti
%{\itshape}% hTheorem head fonti
%{:}% hPunctuation after theorem headi
%{.5em}% hSpace after theorem headi
%{}% hTheorem head spec (can be left empty, meaning ‘normal’)i


%%%%%%%%%%%%% THEOREM STYLES

\newtheoremstyle{BigTheorem}
{20pt}
{20pt}
{\slshape}
{}
{\Large\color{purp}\bfseries}
{.}
{\newline}
{\thmname{#1}\thmnumber{ #2}\thmnote{ (#3)}}



\newtheoremstyle{TheoremClassic}
{15pt}
{15pt}
{\slshape}
{}
{\bfseries}
{.}
{.5em}
{}

\newtheoremstyle{Definitions}
{15pt}
{15pt}
{\slshape}
{}
{\bfseries}
{.}
{.5em}
{\thmname{#1}\thmnumber{ #2}\thmnote{ (#3)}}


\newtheoremstyle{Remarks}
{10pt}
{10pt}
{\upshape}
{}
{\bfseries}
{.}
{.5em}
{}

\newtheoremstyle{Examples}
{10pt}
{10pt}
{\upshape}
{}
{\bfseries}
{.}
{.5em}
{}


%%%%%%%%%%%%% THEOREM DEFINITIONS

\theoremstyle{BigTheorem}
\newtheorem{namthm}{Theorem}
\newtheorem{conj}[namthm]{Conjecture}

\theoremstyle{TheoremClassic}
\newtheorem{thm}{Theorem}[section]
\newtheorem*{thm*}{Theorem}
\newtheorem{lem}[thm]{Lemma}
\newtheorem{cor}[thm]{Corollary}
\newtheorem{prop}[thm]{Proposition}
\newtheorem{claim}[thm]{Claim}


\theoremstyle{Definitions}
\newtheorem{defn}{Definition}[section]
\newtheorem{axi}[defn]{Axiom}
\newtheorem{cust}[defn]{}
\newtheorem{cons}[defn]{Construction}
\newtheorem{props}[defn]{Properties}
\newtheorem{proc}[defn]{Process}
\newtheorem*{law}{Law}


\theoremstyle{Examples}
\newtheorem{eg}{Example}[section]
\newtheorem{noneg}[eg]{Non-Example}
\newtheorem{xca}[eg]{Exercise}


\theoremstyle{Remarks}
\newtheorem{rmk}{Remark}[section]
\newtheorem{qst}[rmk]{Question}
\newtheorem*{ans}{Answer}
\newtheorem{obs}[rmk]{Observation}
\newtheorem{rec}[rmk]{Recall}
\newtheorem{summ}[rmk]{Summary}
\newtheorem{nota}[rmk]{Notation}
\newtheorem{note}[rmk]{Note}



\renewcommand{\qedsymbol}{$\blacksquare$}


\numberwithin{equation}{section}

\newenvironment{qest}{
    \begin{center}
        \em
    }
    {
    \end{center}
    }

%%%%%% MACROS %%%%%%%%%
%% New Commands
\newcommand{\ip}[1]{\langle#1\rangle} %%% Inner product
\newcommand{\abs}[1]{\lvert#1\rvert} %%% Modulus
\newcommand\diag{\operatorname{diag}} %%% diag matrix
\newcommand\tr{\mbox{tr}\.} %%% trace
\newcommand\C{\mathbb C} %%% Complex numbers
\newcommand\R{\mathbb R} %%% Real numbers
\newcommand\Z{\mathbb Z} %%% Integers
\newcommand\Q{\mathbb Q} %%% Rationals
\newcommand\N{\mathbb N} %%% Naturals
\newcommand\F{\mathbb F} %%% An arbitrary field
\newcommand\ste{\operatorname{St}} %%% Steinberg Representation
\newcommand\GL{\mathbf{GL}} %%% General Linear group
\newcommand\SL{\mathbf{SL}} %%% Special linear group
\newcommand\gl{\mathfrak{gl}} %%% General linear algebra
\newcommand\G{\mathbf{G}} %%% connected reductive group
\newcommand\g{\mathfrak{g}} %%% Lie algebra of G
\newcommand\Hbf{\mathbf{H}} %%% Theta fixed points of G
\newcommand\X{\mathbf{X}} %%% Symmetric space X
\newcommand{\catname}[1]{\normalfont\textbf{#1}}
\newcommand{\Set}{\catname{Set}} %%% Category set
\newcommand{\Grp}{\catname{Grp}} %%% Category group
\newcommand{\Rmod}{\catname{R-Mod}} %%% Category r-modules
\newcommand{\Mon}{\catname{Mon}} %%% Category monoid
\newcommand{\Ring}{\catname{Ring}} %%% Category ring
\newcommand{\Topp}{\catname{Top}} %%% Category Topological spaces
\newcommand{\Vect}{\catname{Vect}_{k}} %%% category vector spaces'
\newcommand\Hom{\mathbf{Hom}} %%% Arrows

\newcommand{\map}[2]{\begin{array}{c} #1 \\ #2 \end{array}}

\newcommand{\Emph}[1]{\textbf{\ul{\emph{#1}}}}

\newcommand{\mapsfrom}{\mathrel{\reflectbox{\ensuremath{\mapsto}}}}


%% Math operators
\DeclareMathOperator{\ran}{Im} %%% image
\DeclareMathOperator{\aut}{Aut} %%% Automorphisms
\DeclareMathOperator{\spn}{span} %%% span
\DeclareMathOperator{\ann}{Ann} %%% annihilator
\DeclareMathOperator{\rank}{rank} %%% Rank
\DeclareMathOperator{\ch}{char} %%% characteristic
\DeclareMathOperator{\ev}{\bf{ev}} %%% evaluation
\DeclareMathOperator{\sgn}{sign} %%% sign
\DeclareMathOperator{\id}{Id} %%% identity
\DeclareMathOperator{\supp}{Supp} %%% support
\DeclareMathOperator{\inn}{Inn} %%% Inner aut
\DeclareMathOperator{\en}{End} %%% Endomorphisms
\DeclareMathOperator{\sym}{Sym} %%% Group of symmetries


%% Diagram Environments
\iffalse
\begin{center}
    \begin{tikzpicture}[baseline= (a).base]
        \node[scale=1] (a) at (0,0){
          \begin{tikzcd}
           
          \end{tikzcd}
        };
    \end{tikzpicture}
\end{center}
\fi




\newdateformat{monthdayyeardate}{%
    \monthname[\THEMONTH]~\THEDAY, \THEYEAR}
%%%%%%%%%%%%%%%%%%%%%%%


%%%%%% BEGIN %%%%%%%%%%


\begin{document}

%%%%%% TITLE PAGE %%%%%

\begin{titlepage}
    \centering
    \scshape
    \vspace*{\baselineskip}
    \rule{\textwidth}{1.6pt}\vspace*{-\baselineskip}\vspace*{2pt}
    \rule{\textwidth}{0.4pt}
    
    \vspace{0.75\baselineskip}
    
    {\LARGE History of Rock `n' Roll}
    
    \vspace{0.75\baselineskip}
    
    \rule{\textwidth}{0.4pt}\vspace*{-\baselineskip}\vspace{3.2pt}
    \rule{\textwidth}{1.6pt}
    
    \vspace{2\baselineskip}
    Stats \\
    \vspace*{3\baselineskip}
    \monthdayyeardate\today \\
    \vspace*{5.0\baselineskip}
    
    {\scshape\Large Elijah Thompson, \\ Physics and Math Honors\\}
    
    \vspace{1.0\baselineskip}
    \textit{Solo Pursuit of Learning}
\end{titlepage}

%%%%%%%%%%%%%%%%%%%%%%%
\tableofcontents


%%%%%%%%%%%%%%%%%%%% C1
\chapter{Popular and Country \& Western Musics}

\begin{qst}{}{}
    What do we talk about when we talk about \Emph{music}?
\end{qst}
\begin{itemize}
    \item Do you talk about the lyrics?
    \item Do you talk about the melody?
    \item Do you talk about the rhythm?
    \item Do you talk about the harmonies?
    \item etcetera.
\end{itemize}

\begin{defn}{Conventional Popular Music History}{}
    Conventionally, before there was rock there was: \begin{itemize}
        \item Mainstream and popular music (middle class whites)
        \item Rhythm and blues (black people)
        \item Country and western (lower class whites, usually in the south)
    \end{itemize}
\end{defn}


\section{Regionalism in Music}

Music could be identified as belonging to a specific place. Most music was accessed by hearing it live, or playing it yourself. You could also buy sheet music and play it yourself, or know it from memory.

\subsection{Radio \& Film}

\begin{rmk}{}{}
    With radio, you could live in a small town and hear the same music as someone in New York or Los Angeles, forging a National audience. 

    \Emph{``Breaking boundaries"}
\end{rmk}

\begin{note}{}{}
    ``Popular music," aimed at the white middle class, aired on national radio. \begin{enumerate}
        \item[$\drsh$] Leaves out minorities and their music/cultures
    \end{enumerate}
\end{note}

\begin{rmk}{}{}
    Country \& Western, and Rhythm \& Blues were not aired on national radio as they were associated with lower income whits and black people, respectively.
\end{rmk}

Recall that this is occuring during segregation.

\subsection{Radio Before 1945}

\begin{rmk}{}{}
    It was considered ``unethical" to play records on the air.
    \begin{qst}{}{}
        Why? It was thought as ``fooling" people into believing a performance was live when it wasn't.
    \end{qst}
\end{rmk}

\begin{rmk}{}{}
    Network radio programs popular in the 30s and 40s aften made the jump to television in the 50s.
\end{rmk}
\begin{eg}{}{}
    ``Amos `n' Andy" was a radio show turned into a Sitcom. Originally Amos `n' Andy involved white voice actors performing minstralcy over radio.

    Whites are depicted as educated and elequent while a majority of black characters are played off of minstralcy stereotypes such as being uneducated or follish, and they were made the butt of the joke.
\end{eg}

\begin{qst}{}{}
    What does this say about popular culture in early 1950s America?
\end{qst}

\section{Development of Popular Music at the Turn of the 20th Century (Early 1900s)}


\subsection{Tin Pan Alley} 


\begin{rmk}{}{}
    \Emph{Tin Pan Alley} was turning popular music into a commodity, selling sheat music and creating new songs to sell.
\end{rmk}


Tin Pan Alley was a sheet music business in NYC. At the time the song was percieved as separate from the performance.

The publisher wants the song to be permormed by as many different artists as possible:
\begin{itemize}
    \item Reason: economical - to increase the demand for the sheet music.
\end{itemize}

\begin{rmk}{}{}
    At this time popular forms for music began to arise:
\end{rmk}
\begin{eg}{}{}
    The \Emph{AABA} format was very popular with A $=$ verse, and B $=$ bridge. 
    \begin{eg}{}{}
        ``Over the Rainbow": The verse always starts with ``Somewhere over the rainbow" while the bridge starts with ``Someday I'll wish upon a star." Although the verses had different lyrics, they have very similar flows: long low then short high. The bridge is much more ``bouncy" than the verse: short low - short high - short low etcetera, then a long drifting end.

        The song was often used as a song of hope during the great depression and WWII, and even to this dat.
    \end{eg}
\end{eg}

This was an early example of repititon in music used to make a song more memorable - i.e. to make it an \Emph{ear-worm} to increase sales.

\section{Big Band Era 1935-1945}

\begin{rmk}{}{}
    The purpose of these groups were as dance band - to create music to dance to.
\end{rmk}

\begin{defn}{Instruments}{}
    Big bands often involved bass, drums, piano, guitar, and a horn section. They were typically led by an instrumentalist, and the solo would be past around the band: \begin{eg}{}{}
        \leavevmode
        \begin{itemize}
            \item Benny Goodman
            \item Tommy Dorsey
            \item Jimmy Dorsey
            \item Glen Miller
        \end{itemize}
    \end{eg}
\end{defn}

\begin{note}{}{}
    Vocals are not the star in these bands. Often there were no vocals or only a vocal solo.
\end{note}

\begin{eg}{}{}
    Tommy Dorsey with ``Boogie Woogie" - 1938. Tommy Dorsey was the main instrumentalist, playing a similar role to a conductor for part. Tommy Dorsey is made to stand out (at the front, different colored clothes, etc.) The solo is ``passed around" throughout the band.
\end{eg}

\subsection{Bing Crosby}

Considered by many as the most important pop singer of the 1930s and 1940s.

\begin{rmk}{}{}
    He was a ``Crooner" - a singer who sings very close to the mic to bring out a ``bass" or ``baritone" sound. Somewhat quiet and projective.
\end{rmk}

He was both a singer and an actor.

\begin{note}{}{}
    Notable song: ``White Christmas," which was a chart topper in 1942 and 1945.
\end{note}

Bing Crosby was marted with this ``Nice guy" image - the ``friendly uncle."

\begin{eg}{}{}
    ``Swnging on a Star" - 1945.

    Still big band, but vocal centric. Supporting background vocalists and nonsensical innoffensive lyrics.
\end{eg}


\subsection{Andrews Sisters \& Mills Brothers}

They sang in harmony and were also accompanied by a band.

THe were a pre-cursor to ``Doo-Wop" and ``Girl Groups" of the 60s.

\begin{eg}{}{}
    ``Rum and Coca-Cola" - 1945 by the Andrews Sisters

    They sung together, adding a sense of fullness.
\end{eg}


\subsection{Frank Sinatra}


\begin{note}{}{}
    Bing Crosby was an exception not the rule as a solo singer in the Big Band Era.
\end{note}

Sinatra built on Crosby's accomplishments and establishes a new model: \begin{itemize}
    \item ``The Pop Singer"
\end{itemize}

The singer is now the star. Frank Sinatra was seen as a ``rebellious teen idol."

As with Crosby he started out as a big band singer, going solo in 1943.

\begin{eg}{}{}
    ``I've got a crush on you" - 1948


    Big band instrumentation still present, but the singer is the star.
\end{eg}

\subsection{Singers In, Bands Out}


Sinatra spurs a movement of imatators. The Big Band Era comes to an end with financial pressures (hard to sustain and pay such a large group) and new trends in taste take their toll.

This works in the favour of Tin Pan Alley as ``singers need songs"

\begin{rmk}{}{}
    The new brand of singer-centric pop music as often family-oriented.
\end{rmk}

\begin{eg}{}{}
    ``If I knew you were comin' I'd've Baked a Cake" by Eileen Barton

    Portarys gener rols of the time. Somewhat ``silly" lyricism.
\end{eg}

\begin{eg}{}{}
    Patti Page - ``(How Much Is) That Doggie in the Window" - 1953

    Very family oriented. Dog barking, ``gimmicky"
\end{eg}

\begin{note}{}{}
    This music is very much aimed at middle class white people.
\end{note}



\section{Country \& Western}

Music is still ``regional." Until after 1945 country and western music remains regional, unlike popular music which benefitted from national radio exposure.

\begin{rmk}{}{}
    Country $=$ folk styles of the southeast and appalachia.
\end{rmk}

\begin{rmk}{}{}
    Western $=$ from the southwest and west associated with the Prairies.
\end{rmk}

\subsection{Country}


Record companies labeled it ``\Emph{Hillbilly music}." Recordings were often ``on-site" in the fact that they would go to communities with their recording devices to get songs.

For instgance, we have early ``on-site" recording by \Emph{Ralph Peer}.

\begin{eg}{}{}
    ``Can the circle be unbroken?" - 1935 by \Emph{The Carter Family}.

    Less and simpler instrumentation than the Big Bands with Sinatra and Crosby.

    Just a guitar and maybe some other stringed instruments such as a fiddle or banjo.
\end{eg}


\subsection{Western}

Labeled as ``\Emph{cowboy music}" by recording companies. Was popularized in Hollywood.

\begin{eg}{}{}
    \Emph{Jimmie Rodgers'} Yodel Style singing in ``Blue Todel" - 1927


    Again simple instrumentation, with just Jimmie playing a guitar. Some more ``offensive lyrics" than those found with big band singers: e.g. ``I'm gonna shoot poor Thelma just to see her jump and fall."
\end{eg}


\subsection{Authenticity?}

In 1927 Jimmie Rodgers Entertainers (his original band) auditioned for Victor recording company in Bristol, Tennessee. The audition didn't go well, but then Rodgers auditioned solo and \Emph{Ralph Peer} a talent scout for Victor ``produced" Rodgers. 

\begin{rmk}{}{}
    Their goal with Rodgers was to make ``music that sounded old yet was original enough to be copyrighted and sold"
\end{rmk}

\begin{note}{}{}
    Yodeling speaks to a long history of black and white interaction in the south.
\end{note}

\begin{note}{}{}
    Rodgers understood if he went along with Peers' idea of the new old-time craze, he would make considerable money.
\end{note}


\subsection{The Spread of Country \& Western}

There was some national airplay by the mid to late 1930s. The war also boosted the spread as soldiers from the north and south shared music.

During the war southerners often moved north to find work, bringing their music with them.

\begin{rmk}{}{}
    After WWII, Nashville TN became the C\&W center.
\end{rmk}

\subsection{Hank Williams}

Hank WIlliams was thought of as a ``pure country" cinger.

He had tremendous success until his early death at the age of 33.

\begin{eg}{}{}
    ``Hey Good Looking'" - 1951

    Follows the AABA formula. Instrumentation very string focused, with singer and guitarist one and the same.
\end{eg}


\subsection{Bluegrass}


\begin{eg}{}{}
    Bill Monroe and His Blue Grass Boys: 

    First performed at the Grand Old Opry in 1939, but popularity didn't follow until the late 1940s.

    They were virtuosos (experts) on their respective instruments: \begin{itemize}
        \item Earl Scruggs on Banjo
        \item Bill Monroe on Mandolin
        \item Robert ``Chubby" Wise on Fiddle
    \end{itemize}

\end{eg}

\begin{eg}{}{}
    ``Bluegrass Breakdown" - approximately 1945

    Very fast and precise picking.
\end{eg}

\begin{qst}{}{}
    What have we learned so far?
\end{qst}

\begin{itemize}
    \item Music made by white people in the first half of the twntieth century?
\end{itemize}






\section{Amos `N` Andy Article}

The primary argument in favor of Amos `n' Andy is that is depicts a richly textured Harlem community as its center, filled with all types of black people.

Much of the controversy surrounding the show stems from its original radio show, where Correll and Gosden were white actors using their facility with doing voices based on stereotypes from minstrel shows to fill Amos `n' Andy. Andy was seen as a dumb, shiftless layabout, Kingfish a greedy two-bit hustler, and Lightnin' a goofball ``naif."

\begin{qst}{}{}
    Is representation worth anything if it's primarily being used to prop up the majority's belief in its own superiority?
\end{qst}

The NAACP eventually got the show removed from the airwaves. Even with its drawbacks and reliance on stereotypes, the show was the only one on TV at the time where black people were shown in any position other than servitude.





\section{Somewhere Over the Rainbow Significance}

From the moment it was written, for the 1939 film The Wizard of Oz, Over the Rainbow took on a sginificance greater than the moment it was intended to sountrack. This can be attributed the the composer Harold Arlen and lyricist Yip Harburg who had experienced quite hard times, Yarburg being born poor as the son of Jewish immigrants. His business was wiped out in the Great Depression, but this gave an opportunity for his writing to shine through.

A special recording of the song by Judy Garland and the Tommy Dorsey Orchestra was pressed and sent to American soldiers as a promise of better times to come during WWII. Garland's version also became an anthem for LGBTQ+ individuals.



%%%%%%%%%%%%%%%%%%%% C2
\chapter{Rhythm \& Blues Musics}


\begin{qst}{Motivating Question}{}
    Is there such a thing as white music? Black music?
\end{qst}

\section{What is R\&B?}


Recall R\&B stands for Rhythm and Blues.

\subsection{Early R\&B}

Arising as a by product of racial segregation, most white listeners had no familiarity with rhythm and blues in the 1940s and 1950s.

The idea is that if two groups remain separated, their music and culture cannot intermingle, so the music stays regional.

\begin{rmk}{}{}
    However, in 1923 \Emph{Bessie Smith's} ``Down Hearted Blues" sold a million copies, putting blues in the mainstream, but hthis did not last long.
\end{rmk}

This implies there would be some intermingling of music.

\begin{rmk}{}{}
    Record companies wanted to repeat or piggy back on Smith's success, and hence searched for other artists with her style.

    They called this genre ``Rural blues."
\end{rmk}

\begin{note}{}{}
    In Smith's 1923 recording, it is quite granny and the treble/bass cannot shine through due to the limitations of the equipment (before electric recording).
\end{note}

\subsection{Robert Johnson (1930s blues)}

\begin{eg}{}{}
    ``Cross Road Blues" - 1936
    \begin{note}{}{}
        Johnson sang to his own accompaniment (guitar). This was advantageous as it gave a flexible style which allowed for structural changes on the fly (because it's just you, there's no one following yhou)
    \end{note}
\end{eg}

\begin{rmk}{}{}
    His take on the blues would become the bedrock model for many future popular musicians (see for example Eric Clampton)/

    That is, Johnson's way of doing R\&B would become the way of doing rock. Became the model for people to follow in terms of his way of playing blues.
\end{rmk}

\begin{rmk}{}{}
    Johnson recorded ``Cross Road Blues" and 26 other songs over 5 days in 1936 and 1937 in Texas.
\end{rmk}


In 1941 \Emph{Alan Lomax} (a song hunter/collector) came looking for Johnson only to find out Johnson died in 1938 (at 27). Legend has it that Johnson sold his soul for musical prowess.

Local people told Lomax to go find \Emph{Muddy Waters} instead.

\section{Muddy Waters}

Muddy built his own guitar at age 17 using a box and a stick: ``Couldn't do much with it, but that's how you learn"


A couple years later he bought a Stella guitar for \$2.50, and learned to play listening to the radio.


His first `real' guitar was a Silvertone (sears catalogue) bought for \$11.

\begin{note}{}{}
    He only made 22 cents/hour driving a tractor on a plantation. ALthough slavery had been abolished, its effects are still very prominent.
\end{note}

\begin{rmk}{}{}
    He was said to play all night from ``can to can't" in his chack.
\end{rmk}


\subsection{Chess \& Muddy}

In the fields behind his shack, Muddy was recorded by Lomax. This is the first time Muddy hears himself, and he realizes that he could be a musician.


In 1947 Muddy records under a label called \Emph{Aristocrat records}, and the resulting record sells over 60,000 copies - the most ever by an artist on this label.

Leonard Chess buys Aristocrat, and soon thereafter Chess becomes the premiere independent blues label.

\begin{note}{}{}
    This is happening in Chicago.

    Many plantation workers migrated north in hopes of better prospects, and this is what Muddy did.
\end{note}

\begin{rmk}{}{}
    Muddy was said to have had the first electric band on the South side of Chicago with Jimmy Rogers and Little Walter.
\end{rmk}

\subsection{Chess Records}

\begin{rmk}{}{}
    Artists at Chess: Muddy Waters, Howlin' Wolf, Jimmy Rogers, Little Walter, Bo Diddley, Chuck Berry, Elmore James, Sonny Boy Williamson, Clarence Gatemouth, Koko Taydor, Buddy Guy, Otis Rush, John Lee Hooker, Robert Nighthawk, Billy Boy Arnold, Etta James.
\end{rmk}


Leonard Chess did everythin at the label, including fixing the toilets - very much a D.I.U type of operation

\begin{rmk}{}{}
    To singers and songwriters the company was ``Plantation Chess" with Leonard as master (reminded them of plantations in the South)

    Note most if not all artists were black while Leonard was white.
\end{rmk}

Like ``sharecropping," artists were often not paid in cash but in goods and services. Leonard may pay their medical bills, or buy them a car, etcetera. Not terrible, but was definitely not good either.

\begin{rmk}{Documentary}{}
    It all started with money - ``money was the magnet." Buddy Guy started out picking cotton in New Orleans. Chicago was where many immigrants and southerners went to start a new life and make money. In particular, black southerners wanting to leave plantations.
\end{rmk}

Muddy's style changed quite clearly after he arrived in Chicago.

\begin{defn}{Focused Listening}{}
    \leavevmode
    \begin{enumerate}
        \item Lyrics: Is the message to be taken litterally, or does it have some other meaning?
        \item The music: What is in the instrumentation? Is it fast? Is it slow? How does it make you feel?
        \item How do these things go together?
    \end{enumerate}
\end{defn}


\section{Black Radio}

\begin{note}{}{}
    In Urban areas, blues musicians stuck to a more prearranged structure because they played in bands (electric guitar, bass, drums, piano, harmonica, and vocals) This gives the Chicago blues a structure and establishes the Chicago blues style.
\end{note}

This was unlike Muddy Waters and Robert Johnson when in the south - they're playing by themselves.

\begin{rmk}{}{}
    Major blues scenes were in both Memphis and Chicago.
\end{rmk}

As television took away radio listeners, radio adapted by reverting to a regional approach and creating black stations (which white kids could hear!)

\section{Atlantic R\&B}


Was a record company in NYC with artists such as Ruth Brown, aka ``The Queen of R\&B," and Big Joe Turner.

\begin{note}{}{}
    The instrumentation of Chess in comparison to Atlantic is quite stripped down.
\end{note}


\begin{eg}{}{}
    Ruth Brown: More instruments, akin to Big Bands. Faster pace than Muddy Waters, with lots of energy. Bass, drums, piano, trumpet, and other brass instruments. 

    Her eyes go wide open at certain parts in the song, looking right at the audience.

    Lots of repetition.
\end{eg}

\subsection{Hokum Blues}

Lots of sexual double-entendres: 
\begin{eg}{}{}
    \leavevmode
    \begin{itemize}
        \item ``Let me play with your poodle"
        \item ``Hound Dog" featuring lyrics ``snooping' round my door," ``wag your tail," and ``feed you."
        \item ``Shake, rattle, and roll" by Big Joe Turner follows the Hokum Blues tradition.
    \end{itemize}
\end{eg}

\begin{eg}{}{}
    Big Joe Turner: \begin{itemize}
        \item ``Get outta that bed, wash your face and hands"
        \item ``Shake, Rattle, and roll"
        \item ``One eyed cat peepin' in a seafood store"
        \item over the hill and way down underneath"
    \end{itemize}
\end{eg}

\section{Karl Hagstrom Miller's Analysis of Regionalism}

He says southern musicians performed a very large variety of music in the early 1900s. This included not only blues, ballads, ragtime, and string band music, but also music popular throughout the nation: sentimental ballds, Tin Pan Alley tunes, and Broadway hits.

They embraced pop music. That is, he is suggesting regionalism of sound doesn't really exist.

He says we see the segregating of sound in the 1880s to 1920s:

\begin{rmk}{Contributing Factors}{}
    \leavevmode
    \begin{itemize}
        \item Segregation legislation
        \item American Folklore Society (est. 1888) as authority on ``racial and cultural authenticity"
        \item Mass production of sheet music and then recordings.
    \end{itemize}
\end{rmk}

\textbf{Quote:} ``at the beginning of my story black and white performers regularly employed racialized sounds. By the end most listeners expected artists to embody them"

That is, musicians would play a part, but this would be done for so long that people expected them to actually be these parts. Akin to expecting someone actin a role to be that person. 


He says it literally payed to be able to play anything and everything. A musician wouldn't really be just a say blues musician but would rather be able to play many styles as it was more economical.


Conforming to a stereotype was a successful strategy for those coming from the south to the north. It was also profitable for record companies to sell ``race" (code for black) and ``old-time" (code for C\&W) records.


This strategy assumed musical tastes to be quite narrow as opposed to broad.

\begin{note}{}{}
    Miller says race and old-time records corresponded to the musical lives of no particular sets of artists or audiences. THis was an invention of \Emph{white corporate men}.
\end{note}

Record companies made up styles and genres, and presumed people would listen to them based on their skin colour or economic class - this still happens.


This took advantage of the fact that people could ``buy local" and made it a selling strategy:

\begin{note}{}{}
    Access: Think of the fact that if you only have access to a certain kind of music that's all you're going to listen to. 

    It forces a stereotype upon people by companies.
\end{note}

\begin{rmk}{}{}
    Both the Carters and Jimmie Rodgers listened to records of a wide variety of music.
\end{rmk}

Distinctions in styles often had more to do with marketing than the music itself.

\section{Toward Rock `n' Roll}

The strongest ingredient to rock `n' roll is R\&B.


\subsection{Ahmet Ertegun and Jerry Wexler}

Executives of Atlantic records. In \Emph{Cashbox} (music trades magazine) they predicted the rise of the blues. They calle it ``Cat music"

\textbf{Quote:} ``Up to date blues with a beat, and infectious catch phrases, and danceable rhythms .. It has to kick and it has to have a message for the sharp youngsters who dig it."

\subsection{Rich Cohen}

\textbf{Quote:} ``For years, in the black community rock and roll had been slang for fucking. It appeared on race records as early as the 1920s."

\begin{eg}{}{}
    1922: Trixie Smith with ``My Daddy Rocks Me"
\end{eg}

\begin{eg}{}{}
    1951: The Dominoes with ``Sixty Minute Man":


    ``They call me loving Dom, I rock and roll `em all night long, I'm the sixty minute man."
\end{eg}

Another example is the ``Ann Powers, Good Booty Love and Sex, Black and White, Body and Soul in American Music" magazine.



\section{Robert Johnson Article}

\begin{rmk}{}{}
    Robert Johnson was a musician fabled to have traveled to a local crossroads and made a deal with the devil: sell his soul, and achieve untold musical success.
\end{rmk}

For the most part during his life Johnson played street corners, juke joints, and Saturday night dances, receiving little recognition. 

Johnson had two known recording sessions: one in San Antonio in 1936, and one in Dallas in 1937 - producing 29 songs. At the age of 27 he died due to poisoning.

\begin{note}{}{}
    A 1961 release of Johnson's music by Columbia Records brought it to the fore front. Numerous important and influtential musicians covered his songs including Rolling Stones, Led Zeppelin, and Eric Clapton.
\end{note}

Johnson's family has long been frustrated by the mythology surrounding the musician.

Annye Anderson, Johnson's step-sister, describes Johnson as a ``cool older sibling." She remembers Johnson helping her learn to read and play music.

Lauterbach says that the mythology around Johnson arose as a combination of people seeing how quickly he as=cquired his exceptional guitar skills, and the lyrics of his songs, among them \emph{Hellhound on my Trail, Me} and the \emph{Devil Blues} and \emph{Crossroad Blues}.

\begin{eg}{}{}
    Me and the Devil Blues: ``I said hello satan, I believe it's time to go," ``Me and the devil were walking side by side," 
\end{eg}

Lauterbach says ``just simply that view of playing the blues was looked at, metaphorically, as selling your soul to the devil."

\begin{note}{}{}
    Lauterbach remarks, ``... when you get into the literal aspect of it, I just find it a bit weird and racist that so many people can run with this idea that this talented, ingenious guy must have been endowed with these powers supernaturally instead of being great and working really hard at it."
\end{note}

Anderson says her family lost Johnson twice - once when he died, and again when the mythology surrounding him was created.


\section{Muddy Waters, 1915-1983, Article}

Muddy Waters was known for transforming the Delta's back-country blues into electric blues. Muddy Waters nickname came partially from his grandmother for his love to play in the mud, and later the Waters portion came from playmates.

\begin{rmk}{}{}
    Muddy Waters original band, considered the first electric band, had deep ``Delta" roots and two brilliant musicians in a harp player Little Walter Jacobs and guitarist Jimmy Rogers. 

    However, Muddy Waters is possibly most famous for the blues he played alone.
\end{rmk}

Many musicians tried to reproduce Muddy's sound, none succeeding, but main found their own sound in that process. \Emph{Otis Rush} is an example of such a musician, as well as other Chicagoans including \Emph{Son Seals} and \Emph{Jummy Johnson}, who like Muddy came from the south, but this time inspired by Muddy's work with blues label Chess records.


Muddy's music even reached England, inspiring Brian Jones to learn the slide guitar, and from Muddy's single ``Rollin' Stone," Jones named his band formed with Mick Jagger and Keith Richards. This song also inspired Bob Dylan's ``Like a Rolling Stone."


\begin{rmk}{}{}
    \Emph{Eric Clapton} called Muddy Waters his father, arranging for him to come along on his 1979 cross-country stadium tour. Muddy also felt fondly of Clapton.
\end{rmk}

Johnny Winter also made payments on the debt he owed Muddy Waters by producing Muddy's commercially successful albums for Blue Sky in the late 1970s. 

\begin{note}{}{}
    Muddy's timing, his phrasing, his razorsharp intonation and command of the subtlest shadings of pitch, and his vocabulary of vocal effects, from the purest falsetto to the grittiest roar, put him in a class by himself.
\end{note}

\begin{rmk}{}{}
    It is important to note that Muddy Waters learned blues in the back country, brought his down-home blues to the city and added electricity and a solid backbeat, thereby \Emph{laying the groundwork for rock \& roll}. 
\end{rmk}

Muddy's story began in Rolling Fork, in the Southern Mississippi Delta near Highway 61, where he was born on April 4th, 1915. After his parents separated at six months old Muddy went with his grandmother to the north to live with her on Stovall Plantation near Clarksdale Mississippi. This is precisely where \Emph{John Lee Hooker} and many other future blues stars grew to maturity.


Muddy started on the harmonica, but was drawn to the guitar and the music men like \Emph{Son House} and \Emph{Robert Johnson} were making with it. At the age of 17 Muddy sold a horse to buy his first guitar, a Stella.


By 1941, when \Emph{Alan Lomax} showed up in Clarksdale and recorded Muddy for the Library of Congress, Muddy was the most esteemed guitarist in his part of the Delta. Lomax returned in 1942 and recorded Muddy again. In 1943, Muddy left Mississippi for Chicago.


Muddy bought his first electric gitar in 1944, and by 1946 he was gigging regular with \Emph{Jimmy Rogers} and \Emph{Little Walter}. By 1949 Muddy and his band, including now \Emph{Baby Face Leroy Foster}, were packing crowds in Chicago's South Side and recording for Aristocrat Records. Leonard and Phil Chess who were owners of Aristocrat records soon bought it out, making Chess recors, which had Muddy's ``Roling' Stone" as its first release.

\begin{rmk}{}{}
    In 1958 Muddy played the first electric blues heard in England and launched a rhythm and blues movement that gave birth to groups like the Rolling Stones and the Yardbirds.

    Moreover, in 1960 at a Newport Folk Festival, Muddy introduced young white America to his music.
\end{rmk}

\subsection{Eric Clapton Quote}

``I'd never heard electric Delta blues before, I couldn't believe it. It changed everything." "I never heard anyone who even attempted to play like Muddy Waters, It was impossible because it was the subtlest of them all: it wasn't very fancy or fast. It was just the deepest"

\subsection{Marshal Chess Quote}

White people started packing blues as ``folk music," and appropriating it.


``Muddy's music always makes you feel something ... it's definitely music that cuts through and makes you feel."

\subsection{Keith Richards Quote}

``He was more than a guitar player, more than a signer, more than a writer. It was all him. It's the hoochie-coochie man."

\subsection{Peter Wolf Quote}

After driving Muddy and his band to their apartment, and seeing that it was stationed at the end of town in the red-light district Wolf said ``I'll never forget that. That's when I started to get a sense of the great injustice that was going on in music ... to see the plaster falling off the walls and the creaky old beds in this flea-bitten hotel, and here were these heroic men, these great, great artists."




\section{Rosetta Tharpe: ``A queer black woman invented rock-and-roll" Article}


\begin{rmk}{}{}
    \Emph{Sister Rosetta Tharpe} (1915-1973) was an innovative but underappreciated 20th-century gospel singer and guitarist.


    In 2013, \Emph{Brittany Howard}, backed by the singer-guitarist \Emph{Felicia Collins} and the drummer \Emph{Questlove}, inducted Tharpe into the Rock and Roll Hall of Fame.
\end{rmk}

\begin{defn}{}{}
    The \Emph{Tharpe meme} commonly appears as an image of Tharpe juxtaposed with a phrase identifying her as a ``queer black woman" who ``invented rock and roll." 
\end{defn}

The Tharpe meme responds to the erasures of both black queers and black women from rock-and-roll history. It succinctly challenges the ``organized forgeeting" that enabled Americans in the 1950s to embrace Elvis Presley as the ``King of Rock and Roll" and in the 1970s to identify ``rock" with white British bands such as the Beatles, the Who, and the Rolling Stones, even as these musicians were themselves citing African American influences.

The Tharpe meme is part of the larger Black Rock counternarrative, but it also articulates U.S. pop music history to political and social agendas that coalesce around the intersection of issues of cultural appropriation, anti-black racism, and non-normative sexualities. 

\begin{defn}{}{}
    Lipsitz defines \Emph{counter-memory} as ``a way of remembering and forgetting that starts with the local, the immediate, and the personal," looking``to the past for the hidden histories excluded from dominant narratives"
\end{defn}

In naming Tharpe as rock and roll's (singular) inventor, the meme jettisons historiographical nuance in favor of a performative hyperbole that both secures its shareability and serves users' political and aesthetic ends.



Tharpe's gendered exclusion from the popular notion of guitar virtuosity was something she confronted in her lifetime, as both men and women understood her as extraordinary for her ability to play the instrument ``like a man." In the counter-history it was revealed that it was not Tharpe who played like a man, but men who played like Tharpe.


\begin{rmk}{}{}
    Rosetta Tharpe rose from humble origins in Cotton Plant, Arkansas to become the U.S.' first gospel superstar and one of its most influential and innovative guitarists.
\end{rmk}

In 1938 Tharpe pivoted to a secular career that involved a deal with Decca Records that would last into the 1950s. 

Tharpe performed regularly in the UK and Europe beginning in 1957, influencing the youthful musicians, and was a musical hero to the likes of Little Richard, Johnny Cash, Bob Dylan, Etta James, and Elvis Presley. 

Tharpe died in Philadelphia in 1973, and rested in an unmarked grave until 2009, following a benefit concert that raised funds for a headstone.

\begin{rmk}{}{}
    Quote: ``The rock-and-roll past haunst and shapes the present, just as the present is the dilating lens through which the meaning of the past comes into focus"
\end{rmk}




%%%%%%%%%%%%%%%%%%%% C3
\chapter{The First Wave of Rock and Roll}

\section{Making Waves: Early Rock and Roll}

\begin{rmk}{}{}
    Rhythm and blues is undoubtably the largest influence on what would become rock and roll.
\end{rmk}

\begin{rec}{}{}
    America emerged from WWII as the new world leader, both economically and politically.


    It had faired much better than the previous world leader, England, in the war, only suffering one on soil attack - pearl Harbour.
\end{rec}

Many white middle class Americans experienced an improved standard of living following the war.

\begin{rmk}{Excerpt: The Fabulous 50s}{}
    Large postwar economic boom due to seeling to European countries who needed aid post war.

    An emphasis on women as ``homemakers" during the period. With middle class whites moving to new suburbs, funding for civic improvements fell.

    Those outside the financial boom remained in declining urban areas with increasing poverty and crime.
\end{rmk}

\begin{eg}{}{}
    1950s - ``Leave it to Beaver" Suburban life sitcom
    

    Focus on family suburban life, with white suburban teens going through their daily lives.
\end{eg}

\subsection{A New Normalcy}

Post WWII we see a movement to return to normal - focus on family life, health, education, happiness, leisure, and luxuries.

\begin{note}{}{}
    Most notably adult responsibilities were put off, making a new stage of life: the teenager.
\end{note}

\begin{qst}{}{}
    What are the characteristics of a ``teenager"?
\end{qst}
\begin{itemize}
    \item Puberty and hormones, which came fast bodily and mental changes.
\end{itemize}

\subsection{Teenagers}

From this new phase of life came a pop culture dedicated solely to teenagers:\begin{itemize}
    \item Fashion
    \item Magazing
    \item Movies
    \item Slang (Daddy-O!)
    \item Dancing
    \item Music (that isn't their parents)
\end{itemize}


We see films with characters who portray teenage frustation:

\begin{eg}{}{}
    Johnny in the Wild One (1953)

    ``What are you rebelling against" - ``What do yah got"
\end{eg}

\begin{eg}{}{}
    James Dean in Rebel Without a Cause (1955)

    Represents ``all-american teenager"

    A teen rebelling by drinking
\end{eg}

\begin{eg}{}{}
    Blackboard Jungle (1955)

    Portrays antisocial behaviours of youth
\end{eg}


\section{1950s America}

Conservative (puritanical) values. An absence of sexuality on TV. Suburbs and suburban lifem along with cars (and their radios for listening to music). The Cold War and sniffing out communism. The space race. Science fiction. Modern Civil RIghts. Rock `n' Roll. The red scare and Macarthyism - promoted ``snitching" and suspicion of one's neighbor.

Science fiction played often on public interest in the possibility of extraterrestrial life - eg. ``Invasion of the Body Snatchers"

\section{White Teens and `Black Radio'}

Two key radio shows played a pivotal role in getting black music into the ears of white teens: \begin{enumerate}
    \item Alan Freed's ``The Moondog Show" on WJW in Cleveland launched in 1951
    \item Dewey Phillip's `Red, Hot, and Blue' on WHBQ in Memphis established in 1949
\end{enumerate}

This was easily accessable through the wide availability of the small transistor radio. It was seen as ``exotic, dangerous, and sexual" to listen to among white teens. An act of social rebellion especially from their parents. Also a way to resist assimilation.

\subsection{Major Labels}

\begin{itemize}
    \item Decca 
    \item Mercury 
    \item RCA-Victor 
    \item Columbia 
    \item Capital 
    \item MGM
\end{itemize}

All had their own manufacturing plants and distribution networks.

\begin{note}{}{}
    At this time what soon would be rock `n' roll was regulated to independent labels.


    A mechanism to break into the boardrooms of the big labels was needed in order for rock to spread.
\end{note}

\subsection{Independent Labels}

They had to ``farm out" production. That is they needed reciprocal distribution agreements with other indie regional labels to have their recordings produced and distributed.

They also needed to partake in aggressive marketing:

\begin{eg}{}{}
    \Emph{Payola} - paying DJs to get their music played on the radio
\end{eg}

Often these labels were promoting R\&B


\subsection{Charting Success}

\begin{rec}{}{}
    Roots of rock: Pop, C\&W, R\&B
\end{rec}

Means to track success: 
\begin{itemize}
    \item Cashbox (for jukeboxes): track what is getting played
    \item Billboard: track radio plays
\end{itemize}

\begin{rec}{}{}
    Intended audience of certain genres: \begin{itemize}
        \item R\&B = `race' records directed at black urban audiences
        \item C\&W = `hillbilly' records directed at rural or urban low income whites
        \item Pop = mainstream for the white middle class
    \end{itemize}
\end{rec}


\section{Crossover}

Non-mainstream recordings needed to crossover and land on the popular chart to garner success. That is, charting in R\&B or C\&W AND Pop


Can apply to a song or the recording itself

\begin{rec}{}{}
    A song is the music on paper that can be performed by anyone
\end{rec}

\begin{rec}{}{}
    A recording is the whole package - the performance on the record that one can listen to
\end{rec}

From 1950-1953: 10\% of R\&B crosses over


In 1954 25\% of R\&B crosses over

Finally in 1958 94\% of R\&B crosses over. That is R\&B becomes popular as measured by the charts of success.


\section{The First Waves}

\subsection{characteristics}

\begin{rmk}{Rhythm}{}
    There is an emphasis on beats 2 and 4 (one - TWO - three - FOUR)
\end{rmk}

\begin{rmk}{Lyrically}{}
    Focused on romance, dance, school, music, sex (which was often sanitized by the white artists who covered black artists)
\end{rmk}

\begin{rmk}{}{}
    Rck `n' roll can be essentially said to be R\&B with some added elements from other traditions, i.e. C\&W and pop
\end{rmk}

\begin{rmk}{Form}{}{}
    Many early rock songs resembled 12-bar blues or verse-chorus-verse-chorus, etcetera etcetera, structure.
\end{rmk}

\subsection{Four Common Formal Types}

\begin{enumerate}
    \item \Emph{Simple verse} - all verses based on the same music; no chorus (essentially same thing over and over)
    \item \Emph{Simple verse-chorus} - verse and choruses based on the same music, so you have different lyrics, but the underlying music is the same
    \item \Emph{Contrasting verse-chorus} - verses and choruses based on different music. A clear difference, musically, between the verse and chorus
    \item \Emph{AABA} - verses and bridge based on different music
\end{enumerate}
\begin{note}{}{}
    The 12-bar blues may be a basis for any of the sections in any of the above forms.
\end{note}

\subsection{Rock `n' Roll?}

\begin{itemize}
    \item Fats Domino and his ``Blueberry Hill" - he played piano and was the main singer
    \item Chuck Berry and his ``Maybelline" - guitar soloist and singer with extensive virtuosity with the guitar
    \item Little Richard with ``Lucille" - fast piano playing and energetic vocals
    \item Bill Haley and His Comets with ``Rock around the Clock TOnight" - quite a few instruments, almost big band esque
\end{itemize}

\textbf{Quote:} ``Rock and roll is just rhythm and blues. It's the same music I've been playing for 15 years in New Orleans" - Fats Domino

\subsection{Chuck Berry}

One of the most imitated guitarists in rock - characterized by jumping up and down, bending back and forth, etc.

Used double stops in guitar solos (pioneered it): playing two notes simultaneously


He used both blues' bends and country's speed combined in his guitar style 

Blue grass and R\&B influences.


\subsection{Little Richard}

Said, even from himself, to be the originator of rock and roll: ``I am the architect of rock and roll," ``I feel from the bottom of my heart that I'm the original inventor and architect of rock and roll"

``Rock and roll is really boogie-woogie, it's rhythm \& blues up tempo"

``Older people didn't like it - they simply said it was bad for the children"

Little Richard was the only musically inclined of 12 children. He made a contract with RCA-Victor, and was with them before Elvis, but if you were black it was called ``Camden"(?) They didn't push black records, unlike white records.

He worked at a restuarant that he wasn't allowed to eat at or go to the restroom in


Little Richard's song ``Tootie Frootie" is said to be the first rock `n' roll song - lots of double-entendres, called n-word music by white people.

They gave it to a white artist to ``clean up" and because they didn't like the popularity Little Richard was having with white youth

He took a break to be ``closer to god" in the late 1950s.


Little Richard in fact inspired the Beatles.

\begin{note}{}{}
    Little Richard initiated gospel styles into R\&B
\end{note}

\begin{rmk}{}{}
    Little Richard made the mold for rock and roll which was co-opted by many white musicians.
\end{rmk}

Many things recorded at J\&M record.

Little Richard's charisma was clear in his music. 


Rock and roll was more than music, it was a way of being.


\section{Little Richard Is Everywhere: Article}



Commonly, Little Richard performed with a dense horn section as well as \emph{two} drummers.

Little Richard was born Richard Penniman in Macon, Georgia, and later died in Tullahoma Tennessee at the age of 87 from bone cancer. 


\begin{rmk}{}{}
    Little Richard fused together New Orleans R\&B, boogie woogie's driving left handed bass shuffle rhythm, and pyrotechnical gospel, to deliver rock `n' roll.
\end{rmk}

Little Richard's breakout 1955 single was ``Tuttie Frutti." It is said that Little Richard's exuberant piano went on to inspire the likes of Jerry Lee Lewis and Elton John.

Little Richard's singing beared the influence of gospel dynamos like \Emph{Sister Rosetta Tharpe}, Mahalia Jackson, and Brother Joe May. Moreover, his singing would inspire vocalists such as Otis Redding and Same Cooke.


Richard often performed with a saxophonist Lee Allen and a drummer Earl Palmer. Palmer's swinging on the kick drum against the 4/4 snare in 1957's ``Lucille" would become the inspiration to the advent of much 1960s rock music.


Even with the supression Little Richard faced due to his race, by 1959 when he parted from his label Specialty Records, he had nine Top 40 pop singles and at least 17 singles on the Top 40 R\&B chart.

\begin{rmk}{}{}
    It can not be overstated that Little Richard contributed to the desegregation of popular music in the 1950s, inspiring black and white teenagers to buy his records and to attend his concerts in the same venue, sometimes even in the segregated south.
\end{rmk}

Often Little Richard liked to sing about graphic sex. But because of the repressive conservatism of the era, Richard's subject matter was often restricted and hemmed in.

It is important to note that although the names in Little Richard's songs were feminine, and he used female pronouns for them, Little Richard was gay and overtly femme. He spent some of his teenage years on the underground drag revues in the persona of \Emph{Princess LaVonne}.

\begin{note}{}{}
    Little Richard's gender non-conformism was truly remarkable due to the profound risk and stigmatization in the 1950s - Cold War conservatism, racism, and homophobia. It should be noted that Richard drew on a long line of queer and lesbian black female performers who preceded him, from blues women like \Emph{Bessie Smith}, to gospel figures like \Emph{Rosetta Tharpe}.
\end{note}

Little Richard sometimes played up being effeminate to make himself \emph{non-threatening} to white audiences, and thereby ensure his commercial survival.


The danger of Richard's larger-than-life queeny excess was that he could come off as a racial caricature that reaffirmed the terms of white supremacy, rather than rebel against it. Though Richard's records made millions in the 1950s, the music industry defiled his work by endorsing tepid, white-bread covers of his hits, sung by mediocre artists like Pat Boone, that often earned more than the originals.



Riddled by personal demons that partly stemmed from the internalized homophobic residue of growing up in the South and being stigmatized as a freak, Richard would take himself in and out of the closet throughout his professional career, careening between poles of sin and salvation.




%%%%%%%%%%%%%%%%%%%% C4
\chapter{The Second Wave of Rock and Roll}



\textbf{Quote:} ``If I could find a white man who had the [sic] sound and the [sic] feel, I could make a billion dollars" - Sam Phillips, Sun Records

\section{Elvis' Early Years}

\textbf{Quote:} ``We were a religious family going around together to sing at camp meetigns and revivals" - Elvis' musical upbringing in gospel singing.


Elvis was an avid listener of Black radio stations in Memphis - R\&B influence. Elvis was also a country fan, and loved Jimmi Rodgers


As a teenager he bought his clothes from Memphis leading clothing store, Lonsky Brothers. All together Elvis' musical and cultural ingluences are diverse.

\begin{rmk}{}{}
    According to Elvis' drummer, their breakout hit ``That's All Right (Mama)" (1954), was what cemented their signature sound - he describes it as a happy accident.
\end{rmk}


\subsection{Sam Phillips and Elvis}

Sam Phillips was looking for individualism to the extreme. ``I knew I was going to get some black folks in that studio one way or the other"

Memphis in the mid 50s was a black house town. ``[Sam Phillips] was to pull out whatever was inside [insecure musicians]"

\begin{rmk}{}{}
    Elvis' rise coincided with the proliferation of TV. He took advantage of this with his appearance on the Milton Birle show.

    How he gained the moniker ``Elvis the Pelvis"
\end{rmk}

After airing the Hound Dog milton recieved 700,000 ``pan letters" - painmail of parents against Elvis' ``provocative" dancing, the way he gyrates his hips.

\section{Anatomically Correct Rock Doll Method}

\begin{defn}{Method}{}
    The idea of this analysis method is to consider how you respond to music when you listen to it. What responses does the song ellicit? \begin{itemize}
        \item Mind (intellectual)
        \item Heart (emotional)
        \item Genitalia (sexual)
        \item Feet (dancing/movement)
    \end{itemize}
\end{defn}

\section{Elvis}

\begin{qst}{}{}
    Why is Elvis considered ``the king of rock and roll"?
\end{qst}
\begin{itemize}
    \item Was the first to repeatedly have hits on all three charts simultaneously - Pop, C\&W, and R\&B
    \item The first rock and roll artist to draw attention from major labels - previously rock and roll was regulated to individual labels causing limited circulation
    \item As a measure of his importance, Sun Records sold his rights for \$35,000 to RCA
\end{itemize}

\subsection{Elvis' Team}

\begin{itemize}
    \item Guitar: Scotty Moore
    \item Bass: Bill Black
    \item RCA production including guitarist Chet Atkins
    \item Manager: Colonel Tom Parker (took 20\%-50\% cut)
    \item He had an entourage called the \Emph{Memphis mafia}
\end{itemize}

Elvis was also frequently in movies (four before 1958), so he crossed between mediums

\subsection{The Comeback}

From 1961-1967 Elvis had no public performances, but was in 2 to 3 movies per year. In 1968 the ``King" returns with a live TV special from Las Vegas. However, this comeback was short lived as in the 1970s Elvis developed some self-destructive behaviour, such as a drug addiction.

Elvis then died on AUgust 16th, 1977 - he had died not of a heart attack but of a drug overdose; these were legally obtained from his doctors.

\begin{rmk}{}{}
    Elvis had 107 top 40 singles* (approximately)
\end{rmk}

\textbf{Quote:} ``I didn't think Presly was as good as the Everly boys" - Chuck Berry

\textbf{Quote:} ``I felt the music wasn't that goot to care about. [Elvis] caused a lot of the populace (white people) to start listening to a lot of music they probably wouldn't have" ... ``All Elvis was doing was copying our kind of music" - Ray Charles


\section{Rockabilly}

Influences of R\&B, Country, and gospel. Rockabilly often excluded drums and vocals were less `gritty.'

\begin{eg}{}{}
    \leavevmode
    \begin{itemize}
        \item The Everly Brothers with ``All I have to do is dream" (1958)
        \item Jerry Lee Lewis with ``Great Balls of Fire" (1957)
        \item Ricky Nelson with ``Lonesome Town" (1958)
        \item Buddy Holly with ``Everyday" (1957)
        \item Roy Orbison with ``Crying" (1961)
    \end{itemize}
\end{eg}
Many of these were seen as teen heart throbes.


\section{Other Parallel Movements}

\begin{itemize}
    \item Doo-Wop
    \item Teen Idols
    \item Sweet Soul
    \item Folk Revival
    \item Girl Groups
    \item Surf Music
\end{itemize}

\begin{note}
    Following stylistic differences is more like adhering to rough guidelines rather than hard rules.
\end{note}



\begin{rmk}{}{}
    Lyrically and musically many of these styles may overlap.
\end{rmk}

Moreover, one can't help but to have influences between styles. What's happening in one style may influence another.

\begin{rec}{}{}
    Recall that style labels are largely used to aid in marketing.

    Sometimes may depend on the groups name as to which style they are labelled as.
\end{rec}

\begin{rmk}{}{}
    Contexualized listening involves recognizing whate sounds are, what they're doing, and making connections to previous lived experiences and music history knowledge.
\end{rmk}


\subsection{Doo-Wop}

Generally speaking, it fuses gospel and R\&B. Groups were informally formed on teh street using no instruments, and instead their voices to mimic instruments.

However, it was common for them to be recorded in a studio with a professional band as a backing track.

There were two types of bands: \begin{enumerate}
    \item up-tempo - usually happy
    \item romantic ballad - usually slower
\end{enumerate}

\begin{rmk}{}{}
    Curiously, many bands were named after birds:


    e.g. The Orioles, the Raves, the Flamingos, the Swallows, the Crows, the Cardinals, the Robins, the Penguins, the Larks, the Pelicons, the Swans, the Wrens, and the Sparrows.
\end{rmk}

\begin{note}{}{}
    Music is a two way street - influence can go both ways.
\end{note}

Bands often used onomatopoeias to simulate instruments. Often used by black musicians who could not afford proper instruments. Many werer very badly exploited by record labels: e.g. Sunny Till


The music was often slow and ``jazzy." A very popular Doo-Wop group was ``The Teenagers"


\subsection{Crossing over the the Mainstream}

\begin{eg}{}{}
    The Platters: the first doo-wop act to have lasting pop-chart success:

    18 top 40 hits from 1955-1960, with ``The Great Pretender" hitting \#1 in December 1955
\end{eg}

\begin{eg}{}{}
    The Coasters - ``Yakety Yak" hits \#1 in June 1958
\end{eg}

\begin{eg}{}{}
    The Chords' 1954 ``Sh-Boom" was whitewashed by the Crew Cuts and became \#1 for both
\end{eg}

The whitewashing of a song by an African American group to make it more marketable to white audiences was very common.

\begin{rmk}{Excerpt}{}
    A few white DJs began to play R\&B music for white teen audiences using rock and roll to hide its black origins. E.g. \Emph{Alan Freed}

    He mainly played black vocal groups (e.g. Doo-wop), causing it to gain vast popularity with white audiences. Record companies decided to hire white bands to do covers as they would be more marketable rather than paying these black groups.


    Known as cover versions - almost alwasy surpassed the original recordings due to the amount of promotions they had.
\end{rmk}

\begin{eg}{}{}
    The Del-Vikings (a mixed race group during segregation) ``Come go with me" (1956)

    A mix between an emotional and dancing/movement song.
\end{eg}


\section{Analysis Tips}

\begin{itemize}
    \item Identify instrumentation: i.e. drums, guitar, bass, piano, sax
    \item What are these instruments doing and how do they effect you the listener: \begin{itemize}
            \item Singing: Are they whispering? Yelling? Somewhere in between?
            \item Does it sound like they're labouring or in pain? Or, like it's easy for them?
            \item Melody: Is it catchy? Memorable? Something you can sing along to?
            \item Each Instrument: What are they doing? Why?
    \end{itemize}
    \item Tempo: Is it fast? Slow? Somewhere in between? How does it make you feel?
    \item Dynamics: Really loud? Quiet? Somewhere in between? Does it change throughout?
    \item Structures: Is there a repeated catchy part of the song? (maybe the chorus?)
\end{itemize}

\begin{qst}{}{}
    What might we find if we used this method to analyze, say, ten doo-wop songs from this period?
\end{qst}

\begin{eg}{}{}
    Ben E. King's ``Stand by Me" 
    \begin{note}{}{}
        A trademark of the ``sweet soul" style is the addition of a string section to what otherwise would be a slower R\&B song
    \end{note}
    Singer straining slightly/sounds sad/pained. Lyrics express longing and love. Music not too suitable for dancing, nor very erotic. String playing and vibrato emphasizing the emotion.
\end{eg}

\begin{eg}{}{}
    Ray Charles ``Hit the Road Jack"

    Up beat catchy tune. Dancy with a very bouncy beat. Piano playing in a bouncing style. Singer up spirited voice, king of jokey.
\end{eg}

\begin{eg}{}{}
    Sam Cook ``Bring it hom to me"

    Slower tempo. Lyrics drawn out and slightly pained. Longing - string instruments draw it out.
\end{eg}


\subsection{Girl Groups}

\begin{itemize}
    \item The Crystals with ``Do Doo Ron Ron"
    \item The Shirelles with ``Will you love me tomorrow"
    \item The Chiffons with ``One fine day"
    \item The Ronettes with ``Be my baby"
    \item The shangri-las with ``Walkin' in the sand"
\end{itemize}

\begin{eg}{}{}
    The Ronettes and ``Be my baby"

    Lyrics are about love, loudish music, and slightly dancy
\end{eg}

\subsection{The Beach Boys}

Front man \Emph{Brian Wilson} was a big fan of producer Phil Specter. Influenced by ``Be my baby," with ``Don't worry baby" being his take on Specter's style.

Interpreted as surf rock
\begin{eg}{}{}
    ``Don't worry baby"

    Long drawn out vocals. Slightly strained high singing with supporting harmonies.
\end{eg}


\subsection{Peter Paul and Mary}

\begin{rmk}{}{}
    Often in folk the music is a vehicle for a social justice message
\end{rmk}

\begin{eg}{}{}
    ``If I had a Hammer"

    Lyrics carry a warning of ``danger" and talk about promoting love between all people of the land.
\end{eg}

\section{Sanitizing Rock in the Early 1960s}

Generally a whitening of rock and roll. The Teen Idol arose out of this sanitization: ``well-dressed, well-behaved rock"

An example of which was Chubby Checker's 1960 ``The Twist": Fun, wholesome, lacking any edge.

Mainstays of the Dick Clark's American Bandstand were artists like Chubby: a dance centric TV show.


Considered the era of ``Payola": pay for play on the radio. Demostrates the power of bigger players with bigger budgets to dictate what becomes popular.

Teen idols rose to popularity in the late 50s early 60s as they fit the mold of well dressed well behaved rock.

\begin{eg}{}{}
    Paul Anka, Bobby Vee, Bobby Vinton, Tommy Sands, Brenda Lee, Leslie Gore, Conrie Francis.
\end{eg}

\begin{eg}{}{}
    Paul Anka with ``Diana"

    Lyrics about loving a girl - emotional. No movement on stage, no presence.
\end{eg}

\section{The End of An Era}

Little RIchard left rock in the late 50s to be with God. Jerry Lee Lewis had a relationship scandal with his 13 year old cousin causing him to be shunned - forced to cancel tour early, and was boycotted by records.

Elvis leaves for the military in the late 50s. Buddy Holly left his band the Crickets and got married before dying in a plane crash during his last tour. Paul Anka was brought in to finish the tour - hard edges of rock and roll were flattened.

\subsection{1960: The End of Classic Rock}

\begin{itemize}
    \item Chuck Berry in Jail (1961)
    \item Jerry Lee Lewis in scandal after marrying his young cousin (1958)
    \item Little Richard decides to become a minister (1957)
    \item Elvis enlists in the army (1958)
    \item Buddy Holly dies in airplane accident (1959)
    \item Bill Haley, Fats Domino, Every Brothers, fade from the spotlight.
\end{itemize}





\section{Elvis' Top 40 Hits}

\begin{enumerate}
    \item ``Heartbreak Hotel" 03/10/56, Position 1
    \item ``I Was the One" 03/17/56, Position 19
    \item ``Blue Suede Shoes" 04/28/56, Position 20
    \item ``I Want You, I Need You, I Love You" 06/02/56, Position 1
    \item ``My Baby Left Me" 06/09/56, Position 31
    \item ``Don't Be Cruel" 08/04/56, Position 1
    \item ``Hound Dog" 08/04/56, Position 1
    \item ``Love Me Tender" 10/20/56, Position 1
    \item ``Anyway You Want Me (That's How I Will Be)" 11/10/56, Position 20
    \item ``Love Me" 11/34/56, Position 2
    \item ``When My Blue Moon Turns to Gold Again" 12/29/56, Position 19
    \item ``Poor Boy" 01/05/57, Position 24
    \item ``Too Much" 01/26/57, Position 1
    \item ``Playing for Keeps" 02/09/57, Position 21
    \item ``All Shook Up" 04/06/57, Position 1
    \item ``(There'll be) Peace in the Valley (for Me)" 04/29/57, Position 25
    \item ``(Let Me Be Your) Teddy Bear" 06/24/57, Position 1
    \item ``Loving You" 07/08/57, Position 20
    \item ``Jailhouse Rock" 10/14/57, Position 1
    \item ``Treat Me Nice" 10/21/57, Position 18
    \item ``Don't" 01/27/58, Position 1
    \item ``I Beg of You" 02/03/58, Position 8
    \item ``Wear My Ring Around Your Neck" 04/21/58, Position 2
    \item ``Doncha' Think It's Time" 05/05/58, Position 15
    \item ``Hard Headed Woman" 06/30/58, Position 1
    \item ``Don't Ask Me Why" 07/14/58, Position 25
    \item ``One Night" 11/10/58, Position 4
    \item ``I Got Stung" 11/10/58, Position 8
    \item ``(Now and Then There's) a Fool Such as I" 03/30/59, Position 2
    \item ``I Need Your Love Tonight" 03/30/59, Position 4
    \item ``A Big Hunk o' Love" 07/13/59, Position 1
    \item And more
\end{enumerate}


\section{Chubby Checker's ``The Twist" Article}

Checker's ``The Twist" was the number 1 single in the Billboard Hot 100 in 1960 and stayed there for 3 weeks.

``The Twist" is an example of a foundational record of the ``dance-craze." The Twist's only goal/purpose was to pound hard and communicate excitement. 

\begin{rmk}{}{}
    The Twist was originally written by \Emph{Hank Ballard} and recorded by his band \Emph{Hank Ballard And The Midnighters}. Dick Clark liked the song but didn't think Ballard was ``approachable" enough to feature on American Bandstand. In auditions for the part Clark found Ernest Evens (Chubby Checker) who he recorded the song with which shortly hit \#1 a week after it was performed live for the first time.
\end{rmk}

The Twist was later remixed and rereleased multiple times, even having a rab release in the 1980s.


\section{Ray Charles' ``Hit the Road Jack" Article"}

Ray Charles'  ``Hit the Road Jack" Hit \#1 on the Billboard Hot 100 in 1961 for 2 weeks.

The song involves a woman who knows she's stuck with someone who is worthless, and she's kicking him out. The guy begs and pleads for another chance, but he knows it's hopeless. They're following a script and they know it. (Maybe they've had this argument before. Maybe this really is the end. But the contours of it will always be familiar).

The star of the song isn't Charles by \Emph{Margie Hendrix} leader of the \Emph{Raelettes}, Charles' trio of backup singers. A few years later Charles fired Hendrix, and although Charles survived his addictions Hendrix did not, dying in 1973.


\section{The Marcels' ``Blue Moon" Article}

The Marcels' ``Blue Moon" Hit \#1 on the Billboard Hot 100 in 1961 for 3 weeks.

Richard Rodgers and Lorenz Hart had written it for a never-released Hollywood musical in 1934. It had been a hit for Billy Eckstine in 1947, for Mel Torme in 1949, and for Elvis Presley in 1956. All of these versions of the song were sad and tender and romantic.

The Marcels performed it as a dizzy cartoon burp. The Marcels' version is an act of ``beautiful vandalism."

\begin{rmk}{}{}
    The Marcels was a mixed race doo-wop quintet from Pittsburgh. THey had never recorded anything before ``Blue Moon," and they'd never go on to have another real hit.
\end{rmk}

None of the Marcels gives any consideration to the son'g emotional focus. They just mine the song for hooks.


Elvis Presley's cover of Blue Moon is much slower and more somber. The lyrics are drawn out and the instrumentation quiet and controlled. An emotional song (under this interpretation). The instrumentation is very minimalist, with muted picking in the background to provide a beat.


Billie Holiday's cover of Blue Moon is slow with piano accompiament and other backing instruments. Evidently an emotional rendition. Saxophone solo. Lots of vibrato in the singing.





\chapter{The British Invasion}



\begin{rmk}{}{}
    This is about a brief period in American history, 1964-1966, in which American listening audiences could not get enough of British rock `n' roll bands.


    Consequently, British rock `n' roll steamrolled the US. This was remarkable because these bands were playing distinctly American music, in particular R\&B.
\end{rmk}

\begin{rec}{}{}
    Recipe for rock `n' roll is R\&B, plus less majorly C\&W and pop. 

    Due to this, for many R\&B musicians the only thing that changed for them was the name of the genre.
\end{rec}

\begin{note}{}{}
    British musicians did have homegrown influences, but the imprint of American R\&B is very apparent.
\end{note}

\begin{eg}{}{}
    The Rolling Stones were blues aficionados, the Beatles were inspired by Little RIchard and Chuck Berry, and they styled their hair like Elvis for a time.
\end{eg}


\section{Charlie Christian}

Considered the first rock guitarist by many accounts. He influenced many musicians such as Muddy Waters, as well as many UK artists.

\begin{note}{}{}
    The initial migration of the blues is in the US: South to North.

    Next there is a transatlantic transmission of the music, influencing a generation of UK youth. During the time many UK rock artists were hailed as originators of the form, although they can very much be considered as derivative.
\end{note}

\subsection{Charlie Christian Excerpt}

\Emph{John Hammon} discovered Charlie Christian in Oklahoma and helped to define the American popular and jazz music landscape. Hammon put Charlie together with \Emph{Benny Goodman}, the most famous white band leader in the Country at the time: hated guitar players. 


In the early 1930s, jazz guitar players were the acoustic guitar players in Big Bands, who would play chords - they were not loud enough for the melodies.

Gibson amps came out in the late 1930s, so now the guitar was no longer a background instrument - need more power? Turn the amp up!

There was a direct connection between the tone of Charlie's instrumentation and the blues. \Emph{Jack White} (of the white stripes) says that ``blues is the root of a all of it."

Due to the migration of south to north in 1920s, Chicago absorbed many black migrants, with Muddy Waters being one of them, going to Chicago to make records.


\section{Civil Rights}

\subsection{Kennedy Excerpt}

\textbf{Quote:} ``It ought to be possible for american students of any color to be able to go to school without military backup. It ought to be possible for american consumers of any color to be able to go to receive equal service. It ought to be possible for american citizens of any color to be able to go to register to vote and participate in a fair election."

\begin{rmk}{Stats}{}
    With black children and adults versus white children and adults, Kennedy stated the following statistics: \begin{itemize}
        \item a 1/2 chance of completing high school
        \item a 1/3 chance of completing college
        \item a 1/3 chance of becoming a professional man
        \item a 2 times chance of becoming unemployed
        \item a 1/7 chance of earning 10,000 a year
        \item a 7 year shorter life expectancy
        \item a 1/2 prospect of earning
    \end{itemize}
\end{rmk}

\textbf{Quote:} ``This nation will not be fully free until all of its citizens are free"


\subsection{MLK Excerpt}

I have a dream speech: ``I have a dream that one day on the red hills of Georgia, the sons of former slaves and the sons of former slave owners will be able to sit down together at the table of brotherhood.

I have a dream that one day even the state of Mississippi, a state sweltering with the heat of injustice, sweltering with the heat of oppression will be transformed into an oasis of freedom and justice.

I have a dream that my four little children will one day live in a nation where they will not be judged by the color of their skin but by the content of their character. I have a dream today.

I have a dream that one day down in Alabama with its vicious racists, with its governor having his lips dripping with the words of interposition and nullification, one day right down in Alabama little black boys and black girls will be able to join hands with little white boys and white girls as sisters and brothers. I have a dream today.

I have a dream that one day every valley shall be exalted, every hill and mountain shall be made low, the rough places will be made plain, and the crooked places will be made straight, and the glory of the Lord shall be revealed, and all flesh shall see it together.

This is our hope. This is the faith that I go back to the South with. With this faith, we will be able to hew out of the mountain of despair a stone of hope. With this faith we will be able to transform the jangling discords of our nation into a beautiful symphony of brotherhood. With this faith we will be able to work together, to pray together, to struggle together, to go to jail together, to stand up for freedom together, knowing that we will be free one day."

\begin{note}{}{}
    MLK Jr. was assassinated in 1968, and JFK was assassinated in 1963 (months before the British Invasion).
\end{note}


\begin{rmk}{}{}
    The British Invasion starts following the assassination of the US president and the start of the Civil Rights Movement.
\end{rmk}


\section{Meanwhile In the UK}

\begin{rec}{}{}
    After WWII the US is mostly unscathed and comes out as the world's foremost power.

    The UK on the other hand sustained substantial loss of life and destruction of infrastructure - needed to rebuild.
\end{rec}

\begin{note}{}{}
    American music served as a means to engage with American culture for British youth: \begin{eg}{}{}
        Jimmy Page, Led Zeppelin's guitar, learned about the US through the music of Chuck Berry
    \end{eg}
\end{note}

At the same time in the US R\&B music was being replaced by Teen Idols - safe and lackluster.

\subsection{Britain Excerpt}

In the 1950s the sun sets on the British Empire, with America becoming the dominant world power. England was poor now so as many couldn't afford the plane fee, rock `n' roll acted as a portal for UK youth to explore US culture. Jimmy Page says he learned of US culture through Chuck Berry recordings: eg. Hamburgers. Chuck Berry's music portrayed the US as a mecha for music and freedom, it didn't matter if it actually was those things.


By 1963 US rock n roll was dominated by clean white Teen Idols. The energy and thrill of rock had dissipated, replaced by a safe, grownup sountrack.

There was no sense that these teen bands would be around for long - always ready for the next thing. 


\begin{note}{}{}
    At the time the idea of musicians from England taking over American rock n roll was inconcievable.
\end{note}

On \Emph{Feb 7th, 1964} the Beatles set off for a tour in America. After the release of ``I Want To Hold Your Hand" the Beatles arrived in the US to a teenage frenzy - a glimpse at beatlemania.


CBS and other news stations joked about the band, thinking they weren't going to last. The teenagers found their voice - `this was our music'. Reporters like other adults looked down on musicians and youth culture.

\begin{note}{}{}
    The Beatles were from the port town Liverpool - lots of things were brought over from AMerica through the town.
\end{note}

\begin{rmk}{}{}
    \Emph{Everly Brothers}' harmonies influenced the Beatles.
\end{rmk}


The Beatles were resistant to the press. The defined the ``perfect 3 minute record," and 73 million people watched them perform on the Sullivan show.

\begin{note}{}{}
    It wasn't just about the music - they were distinctly different from anything Americans had experienced. Their style, hair, how they fought back against the press, etc.
\end{note}

\begin{rmk}{Ed Sullivan Beatles Performance}{}
    The performance was of ``I Want To Hold Your Hand" - it was an emotional and dancing type of song with drums, guitar, bass, and singing. The tempo changes depending on the section of the song, and the Beatles, mainly Lennon and McCartney sing in harmony. The audience was primarly female.
\end{rmk}

\subsection{Rapid Ascension}

\begin{note}{}{}
    In 1956, in 1 week Elvis had 9 songs in the top 100. In 1964, in 1 week the Beatles had 14 songs in the top 100.
\end{note}

In April 1964, the top 5 songs were all by the Beatles.

\subsection{Beatlemania}

Like Elvis' colonel Tom Parker, the Beatles had an excellent manager \Emph{Brian Epstein}. They had a 50,000 dollor promotional budget for the US. and they distributed pre-recorded interviews and records to major US djs prior to the tour. This gave listeners the impression that the Beatles were in their town, on their local radio station. 


Across in the US, in cities such as NY, Chicago, and LA, \Emph{5 million} bumper stickers were distributed that read ``The Beatles Are Coming.

\begin{rmk}{}{}
    ``I Want To Hold Your Hand," their first single, hit \#1 on January 15th 1964
\end{rmk}

When they arrived on February 7, 1964 at Kennedy Airport they were greeted by 10,000 screaming fans. On February 9, 1964, they played on the Ed Sullivan show to 73 million viewers.

\begin{note}{}{}
    They also appeared in films such as ``A Hard Day's Night" in 1964, and ``Help!" in 1965
\end{note}

In 1965 they played the Shea stadium, home of the NY Mets, to 65,000 fans.

\begin{rmk}{}{}
    They also toured internationally in 1964 visiting the UK, Europe, Australia, Hong Kong, and 24 US cities.
\end{rmk}

\subsection{The Beatles Excerpt}

Hard Days Night encapsulated Beatlemania. ``They seemed to always be either touring, making a movie, or making a record." Companies also started creating merchandise of the Beatles to sell to teenagers. Even in black communities the Beatles were everywhere. ``Selling more than anyone had ever sold before, and playing venues bigger than anyone had ever played before."

\section{The Beatles: American Influences}

\begin{rmk}{}{}
    It was common for British Invasion bands to cover their American influences.
\end{rmk}

\begin{eg}{}{}
    The Beatles covered Chuck Berry, Little Richard, Buddy Holly, Ray Charles, Isley Brothers, Carl Perkins, and early Motown.
\end{eg}

\begin{eg}{}{}
    Motown band Smokey and the Miracles' ``You Really Got a Hold on Me" was covered by the Beatles.
\end{eg}

\begin{note}{}{}
    The Beatles adopted the Buddy Holly band format: two guitars, electric bass, and drums. They helped popularize this format as the ``defacto" style of a rock band.
\end{note}

\textbf{Vocal Influences:} Little Richard was a vocal influence for Paul McCartney.

\begin{note}{}{}
    The Beatles were known for their harmonies and this shows the Everly Brothers' influence on McCartney's and Lennon's close tenor harmonies.
\end{note}


\section{Other British Boys}

\begin{enumerate}
    \item Manfred Man - ``Doo Wah Diddy" 1964
    \item Herman's Hermits - ``I'm Into Something Good" 1964
    \item The Zombies - ``She's Not There" 1964
    \item The Kinks - ``You Really Got Me" 1964
    \item The Yardbirds - ``For Your Love" 1965
\end{enumerate}

\subsection{British Invasion Sound/Style}

All wearing suits or other formal like attire. Primary singer/instrumentalist with backup instrumentalists joining in for harmonies and repitition. Usually the band format of two guitars, an electric bass, and a drum. Often singing in songs was related/adjacent to love - in particular teen love. ALl musicians have a mic to join in on harmonies.

\begin{qst}{}{}
    Can you as a present listener find a past live performance engaging?
\end{qst}


\begin{eg}{}{}
    House of the rising sun by the Animals. I like the vocal intonation and story telling like singing. The bellching also gives chills. Though there isn't a ton of action/movement on stage from the signer, there is a lot of energy in the vocals.
\end{eg}

British bands had a pension for traditional songs.


For instance, the Rolling Stones were obsessed with the blues.

\subsection{Rolling Stones Excerpt}

``blues had a natural attraction" in England - different to anything they had heard before. The Rolling Stones revered the blues. The Rolling Stones were marketed as the ``bad boys" of English rock n roll.


The Stons covered Muddy Waters on the Dean Martin Show in 1964. During the show the Rolling Stones called out Howlin' Wolf as one of their ``great inspirers," and they invited Holwin' Wolf to perform on \Emph{Shindig}. Their first few records in the US were recorded in Chicago, the heart of R\&B. McJager sang sometimes with an ``American accent," inspired by his idols.


\section{The Rolling Stones}

The Stones were marketed as the bad boys, or anti-Beatles. In reality they were friendly with the Beatles.

\textbf{Style:} The Stones' lyrics were often sexually suggestive, they had longer hair, and a scruffy/unkempt look. They were also more influenced by the blues, especially the Chicago blues (eg: Chess Records artists like Muddy Waters, Howlin' Wolf)


They melded blues with their own brand of blues based rock n roll

\begin{eg}{}{}
    ``Satisfaction" 1965 was what made Kieth to be consiered a riff meister. Prior to this song Brian Jones provided all the riffs. Keith Richards often used a tuning similar to what Muddy Waters used - it put an ``attitude" on his sound like a ``machine gun"

    It gave the impression of cocky rebellious music - ``get the fuck out of my way"


    The band began to ditch their suits for a more colourful style. They changed up both their style and sound on occasion with Brian Jones playing the sitar at some points.
\end{eg}

\begin{eg}{}{}
    ``Yesterday" by Paul McCartney - a solo performance. Themes of longing and desparation.
\end{eg}



\section{``I Want To Hold Your Hand" Stereogum article}


On February 1st, 1964, The Beatles' ``I Want to Hold Your Hand" hit number \#1 on the charts, and stayed there for seven weeks.


Between 1964 (when they first showed up on American shores) and 1970 (when they broke up), the Beatles had 20 singles that appeared at \#1 on the Billboard Hot 100. They spent a grand total of 59 weeks at \#1. As a commercial entity in America they really only lasted six years, but they spent more than a year of that time at \#1.

After they broke up, every member of the band had multiple solo songs that made it to \#1. As solo artists they continued hitting \#1 on the charts into the late 1980s.


\begin{rmk}{}{}
    The band was very popular in the UK through 1963, but \Emph{Capitol} (the American branch of their label) didn't want to release their music in the US. Instead the band released a couple of singles on smaller labels in the US in 1963. The real start was when a Washington DC radio DJ started playing ``I Want To Hold Your Hand" on the request of a fan. Following suit Capitol rush released ``I Want To Hold Your Hand" the day after Christmas 1963.
\end{rmk}


By February, 1964, when the band flew to NY, thousands of fans were waiting for them at the airport. Moreover, when they played the song on the Ed Sullivan Show, 73 million people (34\% of the entire US at the time) watched them.



\section{``(I Can't Get No) Satisfaction" Stereogum article}

The Rolling Stones' ``(I Can't Get No) Satisfaction" hit \#1 on the Billboard's hot 100 on July 10, 1965, and stayed there for 4 weeks.

\begin{note}{}{}
    The guitar riff in the song was supposed to be horns. It is said that Keith Richards slept next to a guitar and a tape recorder, and when he woke up he saw that a whole side of tape had been recorded containing the legendary riff, and a lot of snoring.
\end{note}

When he wrote the riff Richards, who was obsessed with electric Chicago blues, thought it would work better on horns than guitar. He used a Gibson Fuzz Box for the distorted sound, and used it because he wanted to show horn players how the riff should sound.


The Rolling Stones were on tour in America when Richards and Mick Jagger got together to write ``Satisfaction." Originally they could not compete with their British Invasion peers, only hitting \#1 a few times in the UK.


Their manager was \Emph{Andrew Loog Oldham}. He instructed them to write their own songs, since previously the Stones mainly wanted to cover songs from American bluesmen (Chuck Berry, Bo Diddley, etc.). ``Satisfaction" was seen as shaggy and unkempt, not exactly the work of professional songwriters.


Almost from the start the Stones had been marketed as a rowdier, more dangerous alternative to the Beatles. ``Satisfaction is a song about not getting laid.



\chapter{America's Ambassadors of Rock, Part 1}


Much of what is about to be discussed is happening concurrently with the British Invasion. For some there was fierce competition while for others there was mutual adoration (ex: the Beatles and the Beach Boys)

\begin{note}{}{}
    Whereas British Invasion bands are quite uniform, American bands were quite diverse at this time. This was in part due to regionalism - America is a very large country especially in comparison to Britain.
\end{note}


\begin{qst}{}{}
    What comes to mind when you hear the label ``folk music"?
\end{qst}

White R\&B like music, often with a social justice message.

\begin{rmk}{}{}
    At this time NY had a thriving folk scene.
\end{rmk}

\section{Bob Dylan}

He was part of NY's Greenwich Village Folk scene - around Washinton square park. A scene was a neighborhood associated with the arts at the time (it is now very expensive to live in).

He was initially a well respected folk singer in the US, but not well known outside that community. To most audiences folk was associated with the Kingston Trio or Peter, Paul, and Mary.

\begin{note}{}{}
    Dylan wrote ``Blowin' in the Wind" which was a hit for Peter, Paul, and Mary in 1963.
\end{note}

\begin{eg}{}{}
    Dylan plays it solo with a guitar and harmonica. Relatively simple guitar strumming - intellectual style (the lyrics mean something deeper).

    Peter, Paul, and Mary sing in harmony with two guitars.
\end{eg}


\begin{rmk}{}{}
    Hit songs were not the focus for most folk artists. Instead, folk artists tended to have success in album charts as opposed to singles.


    They valued collections of songs as a whole - a trend that woud catch on for rock and roll.
\end{rmk}

\begin{rmk}{}{}
    Dylan's Idol was \Emph{Woody Guthrie} who wrote about social justice. Initially Dylan followed along with Guthrie in this regard, which garnered him respect in the folk scene, but he began to write about himself. Due to this he was criticized by his folk peers for replacing the ``we" with ``me"
\end{rmk}

When Dylan later goes electric it signals a firm crossover to rock.

\begin{eg}{}{}
    Dylan's song ``Mr. Tambourine Man" as played by the Byrds: The Byrds had a British Invasion type of style - they have mop like haircuts, are wearing suits, and have the formula oftwo guitars, a bass guitar, and drums. The song starts off with a 12-string guitar which has a very distinct sound and serves as the source for the Beatles ``A Hard Day's Night." There's also of course folk (as it is a Dylan song) so there's this sense of hybridization - folk rock. The recording of the song was recorded by the Wrecking Crew a studio band who also recorded songs for Girl groups and surf music (eg: the Beach Boys).

    Additionally, a young \Emph{David Crosby} plays guitar in the Byrds and he would go on to form Crosby Stil Nash and Young (CSNY) - ``play a song for me"
\end{eg}


\subsection{Dylan Goes Electric}


Dylan's 1965 album (his fifth) ``Bringin' It All Back Home" exemplifies his musical transition, with half of it featuring electric instruments. The first hit single of this album is ``Subterranean Homesick Blues."

The video of ``Subterranean Homesick Blues" shows Dylan holding flashcards of particular phrases in the song. He looks to be standing in a work/construction area. About life and capitalism - how strangling it is.

The followup to this album is the album ``Highway 61 Revisited" featuring ``Like a Rolling Stone" which hit \#2 in the US in 1965.

\subsection{``Highway 61 Revisited" Excerpt}

The album confirmed times were changing for popular music. Dylan was chaffing against the old left wanting him to be a voice for the new left. The album had a punk rock energy. Unlike any other rock and roll or music of the time. Like a Rolling Stone was a six to seven minute hit single. His transition made it seem like there were no more folk musicians.


\Emph{Levon and the Hawks} were working as a roaming R\&B band at this time, and Dylan was looking for an electric group to back him in upcoming shows and the Hawks had appeared on his radar. The Hawks would later go on to become the \Emph{Band}. This new sound shows an interesting contrast to the sound of ``blowin' in the wind" from just a few years earlier (1963). 


\section{Simon \& Garfunkel}

IT is said their debut album, Wednesday Morning 3 A.M., 1964, sold poorly. Due to this they separated with Simon moving to London to go solo and Garfunkel going to grad school. 

\begin{rmk}{}{}
    When the Byrds ``Mr. Tambouring Man" and Dylan's ``Like a Rolling Stone" became hits, someone at Columbia Records decided to revisit the Simon and Garfunkel album.

    The Sound of Silence was reworked and rereleased without their knowledge in 1965. It was given a folk rock makeover featuring electric guitar, bass guitar, and drums, added on top of the original recording.
\end{rmk}

Before the change we hear harmonies and simple acoustic guitar instrumentation.

After the change it sounds a bit more full, with backing drums, bass, and electric guitar.


\begin{eg}{}{}
    In the 2003 film Old School, Will Ferrel shoots himself in the neck with a tranq gun. When he falls in the pool Sound of Silence begins to play along with the flash back.
    \begin{qst}{}{}
        Why did they choose this song?
    \end{qst}
    Ferrel is experiencing a trip, something very associated with the 60s, but why not pick a more psychadelic centered song? Simon \& Garfunkel provided music for the Graduate which has some shared imagery - swimming pool scene with Sound of Silence playing.
\end{eg}


\section{Folk Rock}

The Turtles covered Dylan's ``It Ain't Me Babe" in 1965, and ``Let Me Be" in 1965 as well, following the formula of folk rock. They went on to pop success with ``Happy Together" in 1967. There was a similar structure of guitars and drums, but with an added brass instrument.

Sonny \& Cher with ``I got You Babe" 1965 - a back and forth between the two signers. Cher would go on to be a successful solo artist with 1989's ``Turn Back Time" and 1998's ``Believe" - an autotune happy song. 


The Mama's and the Papa's were influenced by rock, doo wop, and Peter, Paul, and Mary: 

\begin{eg}{}{}
    ``California Dreamin'" 1966. It was referenced in a TV show under the same name, and in Adele's ``Hello." It is escapist, lyrically. Musically it is very focused on vocals, but there are also drums and even a flute solo. The video is a staged TV performance and they are lipsyncing (badly): only a guitar on stage but you can still hear drums. It has some rock elements, but isn't exactly British Invasion style.
\end{eg}


\section{Phil Spector}

Arguably the first celebrity music producer. Recall ``Be My Baby" 1964, and the Ronettes. That was produced after the Beatles arrived.

\begin{note}{}{}
    American music didn't stop during the British Invasion, though it did suffer some setbacks.
\end{note}

Spector went on to record more hits, competing with the British Invasion.

\begin{eg}{}{}
    A Chistmas Gift For You, from Phil Spector. Was released on the day of JFK's assassination, and did not do well initially.
\end{eg}


\begin{eg}{}{}
    Wrote ``You've Lost that Lovin' Feeling" (1965) and ``Unchained Melody" (1965) for the \Emph{Righteous Brothers}.
\end{eg}

He produced the Beatles' album Let It Be (1970), George Harrison's All Things Must Pass (1970), and John Lennon's Imagine (1971). The Beatles were fans of his.

\begin{note}{}{}
    Phil Spector is in jail for murder. We must be careful not to cast him as a role model.
\end{note}

\subsection{Phil Spector Excerpt}


The orchestra he got on ``Be My Baby" was massive. Considered the first ``rock star producer." The Teddy Bears was his first band (his highschool mates) with it having the first record he ever made.


His method would be come to be known as the ``Wall of Sound." He started his own record label, which started the idea of the Spector Wall of Sound. Lots of his music was aimed towards love.


He hired: 3 or 5 keyboard players, 2 drummers, 3 bass players, 3 organ players, 3 guitar players, 3 woodwinds, 3 saxophones, and the LA choir.


\begin{eg}{}{}
    ``You've Lost That Lovin' Feeling" by the Righteous Brothers. Two main singers with a large isntrumental and vocal accompianment/backing.

    This was sung on the Jimmy Fallon show with Tom Cruise, who sang it in Top Gun along with others.
\end{eg}


\section{Brian Wilson}

The Beach Boys were made up of the brothers Brian, Carl, and Dennis Wilson and their cousin Mike Love, and friend Al Gordeen. They started off as a rock and roll band, sounding along the veins of Chuck Berry, but less edgy, and singing a lot about surfing and cars. In 1964 Brian Wilson stops touring with the Beach Boys as it is taking a toll on his mental health. He dedicates his time to writing and recording - his song writing becomes increasingly more sophisticated. 

The song structures become more complex and less predictable, with instrumentation becoming varied, including orchestra instruments.

\begin{eg}{}{}
    ``Help Me Rhonda" (1965) and ``California Girls" (1965) versus 1966 album ``Pet Sounds"
\end{eg}

Brian Wilson had a lot of admiration for Phil Spector using the same musicians, the Wrecking Crew, and the same studio, Gold Star. One of the musicians is bassist \Emph{Carol Kaye} who used a pick with bass playing, giving a percussive sound

\subsection{Brian Wilson Video}

Denny Wilson on the drums, Al Gordeen on Rhythm guitar, Carl Wilson on lead guitar, and Brian Wilson on bass. The backing track for the Beach Boys in the mid 1960s was often outsourced. ``Summer Days [and Summer Nights]" is an example. ``Pet Sounds" was so influential that George Martin said that the Beatles ``Sgt. Pepper" was an attempt to equal it.

Hall Blaine (drummer), was the leader of the Wrecking Crew.

\begin{eg}{}{}
    ``God Only Knows" (1966) - Contains Saxophone, trumpet, jinngle bells/tambourine, piano, violin section, wooden spoons, drums, bass, and guitar.
\end{eg}


The Beach Boys gave us a california sound.


\section{Back In New York}

The Lovin' Spoonful with ``Summer in the City" (1966), The (Young) Rascals with ``Good Lovin'" (1965) (drummer quite prominent with his speed and precision), The Jersey Boys' The Four Seasons, continued to have success without changing their sound and with a doo wop rock style.


\section{The Northwest: Garage Bands}

The Kingsmen with ``Louie, Louie" (1963) - peaks during Beatlemania. Sound attributed the amateurish performances - rough around the edges. Some though the lyrics were profane causing an FBI investigation, but they couldn't determine what the singer was saying exactly since the singer wasn't close enough to the mic for the recording. The lyrics are quite mumbled, but the song is seen as a dancing/movement song.

Paul Revere and the Raiders hosted a Shindig type TV show Where The Action Is - gives the northwest their own show.



\section{The Monkees}

A made for TV band - music was performed by professional musicians and the actors sung. Some members regretted not being able to be more active musically. Unexpectedly, their records were very popular with \#1 hits in the US and UK. 

\begin{eg}{}{}
    ``I'm a Believer" (1966). Actors not very good at acting as musicians. The drummer adjusts the mic while the drums are still going, the pianist plays the same few notes over and over with hardly any hand movement. It was a ``boy band" in the current sense - meant to be swooned for, and percieved as goofy and playful.
\end{eg}








\section{``Good Vibrations" Stereogum article}

On December 10, 1966, the Beach Boys' ``Good Vibrations" hit \#1 on the Billboard's Hot 100, and remained there for one week. ``Good Vibrations" was a triumph of one man's single-minded possibly self-destructive obsession. The Beach Boys' Brian Wilson has this song idea, but he hadn't finished it in his head when he started writing it. So he spent months in studios, finding sounds and discovering new ways to patch them together. Consequently he spent tons of money, tracked down obscure electronic isntruments, and alienated som of the people in his life.


``Good Vibrations" was one of the most expensive songs ever recorded for a long period of time, with an estimated cost between \$25,000 and \$50,000.

\begin{note}{}{}
    Brian Wilson shaped the song as he was recording it, using tape-splicing to edit pieces of tracks together. The song begins verse-chorus-verse, but then explodes into a number of different mini-movements.
\end{note}

\begin{rmk}{}{}
    On the song you can hear seesawing staccato cello hums, reverb-drowned harpsichors, the eerie whine of the primitive synth known as the elecro-theremin, as well as many others.
\end{rmk}

The group is singing about being newly and deliriously in love, so the son'g atomized structure mirrors that feeling. The lyrics were done by Mike Love, Wilson's cousin and bandmate.








\chapter{America's Ambassadors of Rock, Part 2: Motown and Southern Soul}


\begin{rec}{}{}
    R\&B was often used as an all-encompasing label for music made by African Americans, and Soul music is often used in the same way.
\end{rec}

Regionalism is also used to identify a particular sound: eg. Southern Soul.


Motown was its own style label:

\section{Motown Excerpt}

``Dancing in the Street," a Motown song. Motown was started by Berry Gordie, and Smokey Robinson was also a founding member. Motown was founded on a solid and cotrolled gospel sound: big beat - lots of bass, tambourine, drums, etc.

The \Emph{Funk Brothers} were a very important group in Motown. Their rhythm section was quite consistent, with \Emph{James Jamerson} on basslines, along with the consistent drums.


\section{Motown}

Motown was established in 1959 in Detroit by Berry Gordy Jr. He was a Jazz fan, but recognized that there was no money in it. Instead he targeted cross-over markets, i.e. wanted music that would be in demand by black and white audiences.

\begin{note}{}{}
    Gordy started out as a song writer, writing for ex-boxer Jackie Wilson in the late 1950s.
\end{note}

\begin{rmk}{}{}
    The first Motown hit was ``Money (That's What I Want)" performed by Barret Strong (1960)
\end{rmk}


\section{The Motown Model}

Gordy observed that Chuck Berry's success was due to the fact that the originals were embraced by white audiences, to avoid white artists covering songs by black artists. With this in mind he wanted music that would appeal to a broad audience and to do this he had to target it towards white teens.

Gordy applied an assembly line mentality to Motown, similar to the \Emph{Brill Building Model}. The idea was to divide the labour of music into separate sections to dispearse about certain groups.

\begin{rmk}{}{}
    Motown can be considered to have eras organized by the principle songwriters. For example, from the start up to 1964 was the Gordy, Milliam ``Mickey" Stevenson, and William ``Smokey" Robinson era. From 1964-1967 was the H-D-H era, Brian Holland, Lamont Dozier, and Eddie Holland. Post 1967 was the Norman Whitfield era.
\end{rmk}


\subsection{Holland-Dozier-Holland (HDH)}

From 1965 to 1967 Motown produced 28 songs in the top 20, 12 going to \#1, five of which being by the \Emph{Four Tops}. In 1966 75\% of all Motown releases hit the charts in an industry in which the average was 10\%.


\subsection{``Create, Make, Sell"}

Motown can be considered a \Emph{vertically integrated business}. The concept was that all operations happened in teh domain of one business, with the benefit of quality control.

\textbf{Create:} Writing, producing, recording


\textbf{Make:} Manufacturing, inventory, delivery, billing


\textbf{Sell:} Getting airplay, marketing, advertising


\begin{note}{}{}
    Gordy learned electronics in the Military and did the wiring for Hitsville USA (Motown). His father, George, fixed up the walls and plaster, and his wife, Ray, soundproofed the studio with discarded curtains.
\end{note}

\subsection{Hitsville Map}


The garage was the recording studio (studio A, or the snake pit). Partitions were placed around musicians to minimize sound bleed. Guitars and electric bass fed directly to the control room (no amplifiers) - this reduced sound bleed and gave Motown a distinct sound. It also made the sound of the instruments clearer.


The first floor had the control room and business offices, and it looked down on the musicians in the garage.


The Hallway closet was a vocal booth (to isolate and control sound).

The downstairs bathroom was the first echo chamber (no flush!), for reverb. A singer has their vocals sent through a mic to a speaker in the echo chamber, and then reverbed vocals are recorded by additional mics in the chamber. They then decide on the amount of the dry signal (no reverd) to mix with the wet signal (reverb)


The attic was a second echo chamber, but you could pick up traffic sounds.


\textbf{Quote:} ``Hitsville had an atmosphere that allowed people to explore creatively and gave them the courage not to be afraid to make mistakes. In fact, I sometimes encouraged mistakes. Everything starts as an idea and as far as I was concerned there were not stupid ones ... I never wanted anyone to feel how I felt in school - dumb" - Gordy


\subsection{``Writing" Process of Motown}


Gordy began by giving out chord sheets but was after feel, so adherence to sheet music was loose. There was room for interpretation. Generally sessions started by locking in a drumbeat, then Gordy would hum a line for the musicians to play. Gordy encouraged musicians to ad-lib until he heard something he liked.


\subsection{The Funk Brothers}


Initially musicians were paid as little as 5 dollars per side (1 recording). Eventually they earned 25k-60k a year. Musicians did not get album credits until the 1970s. They were also always payed a flat fee regardless of time put in or how successful a song became - no royalties system.

Notebale Funk Brothers are: \begin{itemize}
    \item James Jamerson - bassist
    \item Benny Benjamin - Drummer 
    \item Earl Van Dyke - piano
\end{itemize}

The tracks instrumentation were the essential element of the Motown sound - the Funk Brothers.


\subsection{Choreography}

Gordy wanted to project an image of class and sophistication. Dance movements were refined and graceful.

\begin{eg}{}{}
    The Temptation Walk - gliding, slow-ish leg movements.
\end{eg}

Motown artists were also fashion trend setters.


\subsection{A Few Hits From Hittsville}

\begin{itemize}
    \item ``Please Mr. Postman" by the Marvelettes (1961)
    \item ``Do You Love Me" by the Contours (1962)
    \item ``Heatwave" by Martha and the Vandellas (1963)
    \item ``Baby Love" by the Supremes (1964)
    \item ``I Can't Help Myself (Sugar Pie, Honey Bunch)" by the Four Tops (1965)
\end{itemize}

\begin{rmk}{}{}
    The Supremes lead singer, Diana Ross, would go on to have a successful solo career.
\end{rmk}


Motown was not a part of the RIAA (Recording Industry Association of America) the group that certifies record sales. Instead, Motown spray painted their own gold records. After chipping off some of the gold paint, Marvin Gaye discovered one of his records was actually a Supremes record. Marvin Gaye played drums on ``Please Mr. Postman," one of Motown's million selling records.

Gordy wrote ``Do You Love Me?" by the Contours and he almost released it as his own song. He decided to give it to the Temptations, but when the time came to record they couldn't be found (they were in church watching other singers), so the Contours got it.


\section{Atlantic Soul}

A NY recording label. They had success with ``sweet soul" in the early 60s, with the Drifters, the Coasters, and Ben E. King. ``Sweet soul" tended to be elegant and restrained like ``Stand By Me." ``Southern Soul" tended to feature a more openly enthusiastic emotional expression often found in gospel music.

\subsection{Atlantix-Stax Partnership}

Stax was founded in 1960 by Jim Stewart and his sister Estelle Axtion in Memphis, TN. They converted an old theater into a recording studio and called it ``Soulsville." A licensing agreement with Atlantic, licensing shares in exchange for production and distribution occured - considered a win-win. They had a studio band like Motown: Booker T. and the MG's - example song ``Green Onions"


The relationship soured and Atlantic looked elsewhere. Big artists at Stax include: Issac Hayes, Sam and Dave, and Otis Redding.


\subsubsection{Otis Redding}

Famous song ``Sitting' on the Dock of the Bay" which hit \#1 in 1968, after Redding died in a plane crash in late 1967.


\subsection{Aretha Franklin - ``Respect" 1967}

In the song she is the one main singer with supporting backup singers for repitition. 

\subsection{FriedLander's Theories}

The \Emph{Rook WIndow Method} prizes assessing the context of a song to make meaning of it.


Most people experience music first on an emotional or visceral level. We listen and without thinking we have an unconscious emotional response. Alternatively, we can take an anlytical approach.


\begin{eg}{Emotional ``Respect"}{}
    Possible emotional reactions to Aretha Franklin's ``Respect" are ``Right on!", ``Tell it, sister!", ``Wow"
\end{eg}

\begin{eg}{Analytical ``Respect"}{}
    What instruments are being played and what beats are emphasized? Tries to identify the genre. Does Aretha's gospel background factor into her vocal delivery? Trying to consider the impact of the artist's musical background on the song. Is this about Aretha's personal history or current lifestyle? Trying to make sense of the artist in a broader context as a person living in a society.
\end{eg}

\section{Friedlanger's ``Rock Window" Method}

\begin{enumerate}
    \item[I.] Music \begin{enumerate}
            \item Ensembles: what instruments are present?
            \item Rhythmic Emphasis: what is the dominant beat? What instrument or instruments carry the beat?
            \item Vocal Style: What words describe the vocal delivery? What previous styles influence this vocal delivery?
            \item Instrumental solo: Is there a solo? What style is it derived from?
            \item Harmonic Structure: what chords are present?
    \end{enumerate}
    \item[II.] Lyrics \begin{enumerate}
            \item What are the song's major themes? Does it tell a story? Suggested topical classifications: romantic love, sex, alienation, justice, intraspection, music itself, others.
            \item Is there an explicit or underlying political or cultural message?
    \end{enumerate}
    \item[III.] Artist History \begin{enumerate}
            \item What are important elements of the artists personal history nad career that enhance your understanding of the music?
            \item Psychological, soical, and economic conditions during youth?
            \item Musical history?
            \item Important career landmark?
    \end{enumerate}
    \item[IV.] Societal Context \begin{enumerate}
            \item How did the surrounding political and cultural climates influence the artists and their work?
            \item Youth culture and its relationship to society?
            \item Cultural and political movements including the struggle for civil and human rights for minorities, peace and antiwar movements, and establishment of counterculture alternatives?
            \item The music industry and its current point of development?
    \end{enumerate}
    \item[V.] Stance \begin{enumerate}
            \item Which elements of the artist's live performances and public actions or behavious provide us with a clearer understanding of the music itself?
    \end{enumerate}
\end{enumerate}


\subsection{``Respect" Background}

Otis Redding released ``Respect" in 1965, with it becoming a crossover hit. It was perceived as less polished but more heartfelt than Motown. The performance is faster than Franklin's with different emphasis placements and sax/trumpet sections.


\section{Muscle Shoals}

After the partnership with Stax fell through, Atlantic looked towards Muscle Shoals, a small town in Alabama, with \Emph{Fume Studios} operated by Rick Hall. The first main success was \Emph{Percy Sledge}'s ``When a Man Loves a Woman" (1966), licensed by Atlantic executive \Emph{Jerry Wexler}. Note Atlantic is a NY label, but this is happening during the civil rights era in the south. A big part of the story is black and white people working together inside the studio while outside racist policies were the norm.

\subsection{Fume Studios Excerpt}

George Wallis was making sure no black people went to school at the University of Alabama. In the studio you just worked together (a small escape from the segregation outside their door). Music helped to progress civil rights movements.


\subsection{Pickett}

Wexler brought \Emph{Wilson Pickett} to Muscle Shoals, and recorded ``Land of 1000 Dances" (1966) and ``Mustang Sally" (1966). Pickett worked with the House Band the \Emph{Swampers}. Pickett had a quick temper, and was likely to beat the drummer.


\subsection{Aretha at Muscle Shoals}

Aretha Franklin released several records with Columbia (a NY label), but failed commercially. When her contract expired Wexler signed her to Atlantic. He took her down to Muscle SHoals and with the Swampers she found her new sound, as seen in the recording ``I Never Loved a Man (The Way I Love You)" (1967)

She finished the album in NY with the Muscle Shoals' Swampers.

The recording process was characterized by ``headsessions;" no real music for it, just go by feel. Aretha says that ``Comin' to Muscle Shoals was the turning point."


\begin{qst}{}{}
    What makes music sound ``black" or ``white"?
\end{qst}

Music can change how we think about each other in society.


\section{James Brown}

He began his career as a stand in for Little Richard. he was a fan of Big Bands with influences such as \Emph{Lewis Jordan}. He started out by touring ``chitlin" circuits in the 50s and had a hit with ``Please Please Please" in 1956. His early style was doo wop esque, singing lead supported by backup vocalists. 

``Think" (1960) marks a change driven by the rhythm section and horns, and aggressive singing style - less R\&B like.

This is the idea of ``groove" where the focus is more than lyrics and harmony. He was known for his live shows, dancing, and antics. He tries to capture this with a recording of one of his live shows at the Apollo ``Live at the Apollo: in 1962, which goes \#2 in pop album charts in 1963.

He would have a crossover with the T.A.M.I show (Teenage Awards Music International). It featured rising stars such as the Beach Boys, British Invasion artists (such as the Rolling Stones), and James Brown. It was shot like a movie. Initially Brown was upset that he wasn't scheduled as the closing act.


James Brown's ``Papa's Got a Brand New Bag" had the ingredients for what would later be called \Emph{funk}. Even more apparent in ``Cold Sweat" (1967). Another very popular song of his was ``I Got You (I Feel Good)." He wrote and produced almost everything he did. He made his own business decisions and was a millionare by the mid 60s. He was also very controlling; known for fining musicians for making mistakes on stage (flashed his hands for the amount). He also fired and replaced is band for asking for a raise.


James Brown also performed on the Ed Sullivan show.


\subsection{James Brown Influence Excerpt}


Michael Jackson and Prince can be seen to be musical children of Brown. His music was very popular in the hip hop and break dance communities. Funky Drummer was used by many hip hop artists - their one failed single. Public Enemy was the main act that used it for their songs.

Brown also inspired Kanye, and the Brown scream was very popular: soulfully gospel.




\section{Horn Section in a Motown Band}

Music express horn section: lead trumpet player Dan, second trumpet player Larry, trombone player Nick, sax and woodwind specialist Don. 

The trumpets role in Motown is to enhance the music with riffs and such, as well as longer notes to harmonize with chords and create ``color." Provide harmony and make it more exciting.

A lot of bands had two or more trumpet players. The second trumpet provides some additional punch and harmony, following the lead player.

Dan started playing trumpet at 11 and Larry started trumpet at 10 (with piano before that). Dan toured with Munford and Suns, and Kayne West. Larry played lots of broadway musicals. 


The role of a trombone player is as part of the horn section to give it power. It usually plays in octaves with the trumpet. 


Saxophone for solos were somewhat a tradition in the beginning of Motown. In a Motown section the two main saxophones would be a Berry and a Tenor. The Berry is the ``bottom" of the section, giving the punch and supplying support. Tenor usually plays the harmonics of the chords, playing similar to what the trombone/trumpets are playing to beef it up. 


\section{Can We Recreate the Motown Sound?}

Tamla records was founded in Detroit by Berry Gordy in 1959, and was changed to Motown Records a year later. Motown became an assembly line for hit records, running nearly 24/7, with it being inspired by Gordy's work in automotive factories in Detroit. Between 1959 to 1972, The Funk Brothers (the Motown House Band) played on more hit records than the Beatles, the Beach Boys, and Elvis combined.


\subsection{The Studio}

The live room was a small rectangular room with cables hanging down from the ceiling (giving it the nickname the Snake Pit). The walls in Motown's room were covered in peg board. 


The Echo chamber had a speaker at one end of the attic and a mic at the other, to produce the echo effect on vocals ran through the speaker.


\subsection{Drums and Percussion}

The drum kit was a piece made kit of several brands. Blankets were sometimes placed in the bass drum to experiment, and the snares and other heads were sometimes taped to the bass head. 

Many Motown songs emphasize beats 2 and 4 heavily, and this would be added by other auxillary precussion such as tambourines.


\subsection{Piano and Keyboards}

The piano often dictated the harmonic structure of motown songs. Often multiple keyboards would be playing the same part to get an orchestral feel. 

The piano in Motown was an 1800s steinway. 

\subsection{Bass and Guitar}

The Motown bass was a p-bass strung with flat wounds and a foam piece at the base. 


\subsection{Electric Guitar}

Most Motown tracks have multiple electronic guitars. The Motown guitar sound had zero amplifiers. They went into a DI box, went into a mixer, and came out a speaker in the live room which played all the instruments together.


\subsection{Strings, Horns, and Overdubs}

Strings, horns, and winds are a huge part of the Motown sound. The would often record the strings section in a different studio as an overdub over the rhythm section.

\subsection{Vocals}

Early on the vocalists would often sing in the room with the rhythm section, or in an isolated room. Later they found that recording the vocalists separately and doing an overdubbing made for a cleaner recording.

After the vocals were recorded they would use a feder to edit it. Another trick they would do (for dynamics and character) is they would duplicate it to two channels, leaving one along and compressing the other quite heavily with a high end.

Oftentimes there were distortion on the vocals, caused by hitting the tape machine two hard (singing too loud). 


\subsection{Mixing}

Initially motown had 8 tracks, though they later upgraded to 16. 


\section{``Where Did Our Love Go" Stereogum article}

The Supremes were Motown in microcosm: three girls from Detroit who forced their way into music through sheer charm and determination and who then made themselves over. They came from the same Detroit housing project. They started out as the Primettes, a sister group to a male singing group the Primes. Berry Gordy initially turned them down as they were still in high school, but they stuck around. 

When they got their shot, none of their singles landed at the start. Their nickname was the ``No-Hit Supremes" at Motown. The story goes that ``Where Did Our Love Go" was initially written by Eddie Holland, Lamont Dozier, and Brian Holland for the Marvelettes. The Marvelettes hated it, and the Supremes didn't much like it either, but they couldn't refuse. 

Then ``Where Did Our Love Go" hit number 1 only a few weeks after Lyndon Johnson signed the 1964 Civil Rights Act, with it hitting number 1 on August 22, 1964, and staying there for two weeks. ``Where Did Our Love Go" is a song for dancing. 


\section{``My Girl" Stereogum article}

The Temptations' ``My Girl" hit number 1 on March 6, 1965, staying there for 1 week.


The Temptations formed when members of two rival Detroit singing groups joined forces. They also weren't scoring any hits early on in the Motown system. The song was written by Smokey Robinson and Ronald White for Temptations replacement David Ruffin and his raspy voice. 

``My Girl" was really the start for the Temptations.




\chapter{The Psychedelic 60s}


\section{Up-State: The Psychedelic 60s (1966-1969)}

\begin{rec}{}{}
    James Brown was controlling of everyone that worked for him.
\end{rec}

With psychadelic 60s and the normalizing of recreational drug use, Brown indulged in ``Orange Sunshine" (a popular form of LSD). Frankie, Chicken, Chopper, Bootsie, etc., would crush up orange sunshine into what they were drinking before shows. James was originally anti drugs, but was given OS by his bandmates - do to this he held a moon position for 15+ mins during the show - tripping out.


\section{Social Justice}

\begin{rmk}{}{}
    The Vietnam War and the assissination of JFK were major setbacks for the civil rights movement.
\end{rmk}

Policies don't change racism, so many of the injustices mentioned in JFK's 1963 address remained to this day. It seemed in 1960 that things were changing for the better. In the mid-60s president Lyndon Johnson passed the Civil Rights Act. In the heat of an election campaign Johnson made the decision to increase US involvement in the escalating situation in Vietnam. 

Often the 60s in the US are represented by hippies and psychedelia, but there is much more to the decade.

\section{Psychedelia in Music}

Music was never the same after the late 60s, as evidenced by the music of the Beatles and the Beach Boys. Music to enhance a drug trip (e.g. the Grateful Dead in San Francisco and Pink Floyd in London) Music as the trip (e.g. the Doors and the Beatles' music of the mid-to-late 60s)


\section{Friendly Competition}

Late 60s: when the Beatles and the Beach Boys did their best work. Bob Dylan introduced the Beatles to weed in 64, influencing the Beatles' \Emph{Rubber Soul} (1965): In this the recording studio is being conceptualized as a musical instrument. They are not the first to do this, but they alongisde Brian Wilson and Jimi Hendrix aided in popularizing it. A studio is a place to ``make" music not just to ``capture" it. They effectively camped out in the studio for the entire making of Rubber Soul.

\subsection{Excerpt: Norwegian Wood (1965)}

Acoustic instruments and folk style: guitar, mando, banjo, etc. The Beach Boys' Pet Sounds (1966) and then the Beatles' Revolver (1966) shows the back and forth. The last song ``Tomorrow Never Knows" features tape loops of random sounds and backwards guitar parts in Revolver. ``Tomorrow Never Knows" (1966) has a psychedelic feel with reversed guitar parts.





The Beach Boys' Good Vibrations (1966), The Beatles' ``Penny Lane/Strawberry Fields Forever" (1967), The Beach Boys' SMiLE (never released), and the Beatles' Sgt. Pepper's Lonely Hearts Club Band (1967).

\subsection{Excerpt: Good Vibrations (1966)}

Was 3.5 mins long, recorded in 5 studios over 17 sessions in about 3 months. Larry Levine recording engineer. Wilson didn't have a clear vision at first. Wilson's mother is where he got the idea of ``vibrations." Mike came up with the piano hook for the record. Brian employed his revolutionary modular recording style for the first time on this track. In total it took 7 months to get Good Vibrations - part pop record part modern symphony.

\subsection{Excerpt: SMiLE}

SMiLE was supposed to follow this but was never finished. But a version was released in 2004. Wilson was smoking Hashish when they began. They were laying on the floor putting the mics on their heads. They were listening to ``Fire" tapes and Brian had fire equipment and burning wood brought in. There was actually a fire down the street and they thought they started it due to all the drugs. All the dregs messed Brian up, especially the LSD.

\begin{rec}{}{}
    Pet Sounds is said to be the impetus for Sgt. Pepper
\end{rec}

Wilson doesn't think they are very alike, except for in creativity, and McCartney thinks Pet Sounds was very creative.


\subsection{Excerpt: Yesterday and George Martin (Beatles' producer)}

Every record company turned down the Beatles before they met George Martin. Ringo said they loved George because he accepted them. Martin was an arranger and musician and helped the Beatles in that way. Paul dreampt about yesterday and brought it to George to check if it didn't already exist. George then suggested adding a string quartet. Initially Paul was unsure/intimidated by classical music, but later accepted George's suggestion. Later they brought Martin Eleanor Rigby, in which the Beatles didn't play any instruments - just a string octet, organized by Martin.

\section{Sgt. Pepper}

Released in 1967, Sgt. Pepper is claim to be Rolling Stone Magazine's \#1 album of all time. Considered the first concept album, or at least popularized the form. An elaborate album outlay/photgraph. Lyrics on the back cover, indicating their importance and centrality to the music - meant to be read and contemplated. Stylistically merges rock and roll with elements of chamber music, Indian music, and avant garde classical musical.

\subsection{Excerpt: A Day in the Life}

John sung while playing acoustic guitar, Paul was on the piano, George on maracas, and Ringo on bongos. Ringo's drum fills the background. The song was initially John's but it was incomplete. Paul had a song he was working on and they put it in middle, and initially added some empty bars around it. Originally the pianist counted them back in with a clock going off at 10, before also adding a massive orchestral build up.


\begin{qst}{}{}
    What is a Counter-Culture?
\end{qst}

In order for a counter culture to exist their must be a mainstream dominant culture that adheres to a certain set of values.

\textbf{Quote:} ``Turn on, tune in, and drop out" - Ex-Harvard prof and LSD advocate Timothy Leary. 

Originally thought of it as a therapy but later suggested it for the general populace.


\section{LSD}

Accidentally developed by Swiss scientist Albert Hoffmann, in 1943 while working on a cure for migrain headaches. Used by the CIA in the 50s as a truth serum, and they performed unethical experiments with it to determine its effects. In the mid-to-late 60s LSD was viewed as a drug leading to ``higher consciousness" by Leary and Ken Keasy.


\subsection{Excerpt: Ken Keasy}

Tests if you can get through an acid test. Eight Mile High was the first public appearance of a psychedelic scene. Kesey believed it was a way of getting closer to sensing god. An acid test was a way of having people go through hellish experiences. The music played was R\&B by Jerry Gossey's band The Grateful Dead. The bus has ``Further" painted on it.


The main psychedelics scene was occurring in San Francisco. 

\subsection{Excerpt: White Rabbit (1967)}

Jefferson Airplane: About ``pills" (drugs) and their effects - smoking caterpillar and pulsing colours with psychedelic imagery such as mushrooms. 


They rebranded to Jefferson Starship, and then just Starship, with hits such as ``We Built This City" (1985).


Like the Grateful Dead in the US, Pink Floyd served as a house band in London for people's acid trips. They bought there instruments/technology mostly from a grant. They didn't improvise with chord sequences, unlike other psychedelia groups. Pink Floyd later signed with EMI.


\section{Pink Floyd}

Interested in exploratory noise making. Unlike Grateful Dead who were interested in the blues in their improvizations. Long improvizations (going on 30 mins). Also wrote radio-friendly songs, but refused to play them live. Syd Barrett, leader and guitarist, was replaced by David Gilmour (due to mental illness).

\subsection{Excerpt: Syd Barrett}

Syd would just stand there for videos, not moving at all - ``going crazy." 


1967 was known as the \Emph{summer of love} featuring the Monterey Pop Festival. Jimi Hendrix smashes and burns his guitar to signal the ``end of the old guard" in favour of the counter culture.


\subsection{Excertp: Monterey Festival}

The first major pop festival. Hendrix ended his performance but burning his guitar - a signal of the counter culture.


\section{Guitar Heroes}

This period (1966-1969) witnesses the emergence of rock guitar innovators, most notably \Emph{Eric Clapton (Cream)} and \Emph{Jimi Hendrix}. Known for their distinctive ``hooks," ``riffs," or ``licks" (all mean the same thing, a central guitar figure). Soloing virtuosity is more appreciated, inspiring the creation of interesting sounds and timbres.


\subsection{Excerpt: Hendrix}

Played initially on Little Richard's band, and late played with Cream - upset the ``applecart" of guitar playing in London with his style and technique. Hendrix started out playing in soul bands. Little Richard got rid of Hendrix as he tried to upstage him.

Hendrix was known for playing wildly such as with his teeth and behind the back of his head. He knows the Masters: Albert, Freddy, and B.B.

Hendrix was also a psychedelic hippy, experimented with stereo fazing, and made albums like sonic fantasies.


Conversely, the Beatles played with eastern instrumentation.

After Hendrix the guitar became the face of rock. He used bends to subvert the expectations of the listeners; subtle bends and wild heavy bends to give blues accents. Red House features a number of bends of different speeds.

In Axis: Bold As Love, ``Spanish Castle Magic" features mixes fast bands, hammer-ons, and pull-offs. ``Bold As Love" also showcases Hendrix's rhythm guitar skills. Hendrix often broke up his chords to play with them, most evidently seen in ``Little Wing." He also mixes trills and chords, and fretted his lower strings with his thumb, adding bass notes to his flair. He was also known to use a whammy bar and wap peddle, such as in ``Voodoo Child." He also used them live, and used them to make the sound of dive bombs and exploisions in his playing of Star Spangled Banner. He uses it again in Machine Gun to a similar effect.

``All Along the Watch Tower" takes a classic Bob Dylan song and makes something revolutionary with three different guitar solos, with one short one filled with runs and bends. 


In the late 1960s audiences had become obsessed with guitar solos.


\section{LA-The Doors}

Breathrough hit ``Light My Fire" (1967). They explored the idea of the ``bad trip" such as in their song ``Break on Through" (1967). Singer Jim Morrison cultivated an image of being non-conformist and avertly sexual, taking on the alter ego the \Emph{Lizard King}.

The lengthening of songs was a trend of psychedelics, such as in Iron Butterfly's ``In-a-Gadda-Da-Vida" (1967).

There are also more dynamic songs, that is ones featuring contrasts between loud and soft.

\begin{eg}{}{}
    You Keep Me Hanging On: cover by Vanilla Fudge. Mixing loud and soft, and a look into the heavy metal to come.

    They did a lot of drugs, but did not fit with the American psychedelia scene, being in upstate New York.
\end{eg}


\section{Americana and The Band}

The Band backed Bob Dylan on his world tour in 1965 and 1966. They were often booed due to the transition of Dylan from folk to electric. They separated with Dylan, making their own albums: Music from Big Pink (1968) and The Band (1969). Their sound would come to be known as roots music or Americana. The Band consisted of 4 Canadians and 1 American (Levon Helm), which played a large role in redefining American Rock - all were virtuosos. 

Similar concept to the British Invasion, but with Canadians, introducing American audiences to another aspect of their musical roots.

\subsection{Excerpt: The Band}

Emphasizes coming together and becoming greater than the sum of their parts. Clapton would often run away from bands, so was in awe of their brotherhood. They drew upon Country, Irish, Welsh, and Mississippi Delta sounds. Characterized by a powerful sense of Brotherhood. They inspired Gang Starr's ``Beyond Comprehension" (1991).


Also Creedence Clearwater Revival - between rock and roots - with their ``Fortunate Son" which takes a shot at wealthy families that could keep their sons out of the Vietnam war.


\section{Woodstock}

Embodied the ideas of the counter culture. Held from Aust 15 to August 17 1969. At least 400,000 attended. Featured the Grateful Dead, Jimi Hendrix, Jefferson Airplane, Janis Joplin, the Who, Suntana, Crosby, Stills \& Nash, and Sly and the Family Stone.






\chapter{A to Zep}

\section{The 1970s}

Counter culture begins to go mainstream. Civil rights movements such as the ``Woman's Liberation Movement" or ``Woman's Lib" - feminism.

\subsection{Excerpt: Woman's Lib}

The ``Age of the pill" and Betty (Free) Dan as their leader. Challenged all examples of male dominance including the church. Jacqui Ceballos became a leader on this front - ``God created women in her own image"

Annual beauty pageant seen as an offense to feminists. They threw bras, girdles, cosmetics, etc. in the trash - they didn't do any burning because they weren't allowed. They protested about equal pay and opportunity in the job market. 


In 1972 Helen Reddy's ``I Am Woman" wins grammy. Progress for other marginalized groups also increased - example: 1973 APA removes homosexuality from its list of diseases.

Woman's Lib movement influenced many things. For example, Virginia Slims marketing slogan ``You've Come a long way, baby," ``This is the one cigarette made just for women. They're slimmer than the fat cigarettes men smoke with full rich Virginia flavour women like"


\section{TV: All in the Family}

Main character: Archie Bunker, is a middle-aged poorly educated white bigot. Michael: his liberal college-educated son in law (calls him meathead). Gloria: his woman's lib inspired daughter. Edith: his big-hearted (but sometimes dim-witted) wife - serves as a comedic buffer. Debuted in 1971, addresses social issues of the times. Takes place in working-class Queens.

\subsection{Episode}

Michael brings in an abstract art sculpture (no one gets what he sees in it). Archie mad that ``spades" are moving into the neighborhood (black people). Michael brings studies and facts but Archie just ignores it - calls black people ``c**ns," and starts spouting racial stereotypes to a backing laugh track.

Lynol (Michael's black friend) is the one who bought the house.


\section{Music on TV}

\begin{itemize}
    \item 1971: Soul Train (continued until 2006)
    \item 1972: In Concert (Later renamed Rock Concert)
    \item 1973: The Midnight Special
    \item 1975: Saturday Night Live
\end{itemize}


\section{Musical Trends}

Influenced by psychedelic 60s. A shift from a concentration on the single to the album. Album's provided a listening experience. Range of styles within rock changes by incorporating classical, electronic, and jazz. The recording studio increasingly seen as a creative tool.

\section{The Hippie Aesthetic}

Also known as the aesthetic of ambition. In other words taking music seriously - rock becomes a serious artistic endeavour. Instrumental virtuosity became highly valued, this is similar to jazz where individual prowess on an instrument is highly valued. This concept ties early 1970s rock to psychedelia: by guitar heroes like Hendrix and Clapton.

\section{Blues-Based British Rock}

The Rolling Stones' string of number \#1 albums: \begin{itemize}
    \item Sticky Fingers (1971)
    \item Exile on Main Street (1972)
    \item Goat's Head Soup (1973)
\end{itemize}

\subsection{Excerpt: Brown Sugar}

Recorded at Muscle Shoals. Difficult to decipher lyrics, recorded in 1969. Jagger wrote the song in Australia while starring in the film Ned Kelly (released in 1970). The first song released on Rolling Stones records, a subsidiary of Atlantic Records. The lyrics are racist, sexual, mysogynistic, etc.

\section{Recall the Yardbirds ...}

A band that featured Eric Clapton, Jeff Beck, and Jimmy Page as guitarists in different points in time - 3 best UK guitarists. By 1969, Page was the only remaining member and he recruited bassist John Paul Jones, drummer John Banham, and singer Robert Plant to form the New Yardbirds, soon to be changed to Led Zeppelin.

\subsection{Excerpt: John Bonham}

Used his hands on his drums at times. Bonham often followed the music of the other instruments in the band. Bob accents the beats with triplets - breaks up the rhythm. Influenced by Jazz musicians such as Jean Crupa and Buddy Rich. Laid back rhythm following along with the other musicians; inspired by funk music, especially James Brown.


Ron Everson said the essence of Zeppelin was Bonham following along with the guitar player. 


In Wayne's World we see a reference to Zeppelin with a no Stairway to Heaven song. 


\section{Stairway to Heaven}

Gradually builds instrumentation - song adds layers as it progresses. First just guitar, vocals and recorders. Adds electric twelve string, electric piano, bass, and drums - about 4 mins 20 s into the song.

Dynamics build throughout the song; a contrast between faint and loud. The live performance is also quite faithful to the studio recording.


\section{Led Zeppelin}

Eigth \#1 albums in the UK (six in the US). Fourth album (usually referred to as Led Zeppelin IV) is the 5th highest selling album in US sales history (23 million)


\section{Deep Purple}

Blended rock and classical influences. Most famous album ``Machine Head" (1971). ``Highway Star" features keyboard viruasity of Jon Lord and guitar virtuosity of Ritchie Blackmore. ``Highway Star" seen as an early metal influence. Their most famous song was ``Smoke on the Water"

\subsection{Excerpt: Smoke on the Water riff}

Pete Townsman said to keep the riff simple so it sticks with the audience. Ritchie finger picks the two notes to give it that ``big sound."


\section{Black Sabbath}

By most accounts the original metal band. Consisted of guitarist Tony Iommi, bassist Geezer Butler, drummer Bill Ward, and singer John ``Ozzy" Osbourne. Began as a blues band named ``Earth," then switched to ``Black Sabbath" named after a horror film. The dark texture of their music is achieved by playing variants of the same riff simultaneously. For example in ``Iron Man" from Paranoid (1970)


\section{The Who}

English band: guitarist Pete Townshend, singer Roger Daltrey, bassist John Entwistle, and drummer Keith Moon. Tommy (1969) a concept album about a boy who is deaf, dumb, and blind, who gains spiritual enlightenment through playing pinball. When Tommy is cured he is cast as a guru possessing the great wisdom of ages - full force hippie aesthetic. Who's Next (1971) album, and Quadrophenia (1973) album are two of their other hits.

\subsection{Excerpt: 1967 Smothers Brothers}

The Who were a rebellious bunch. Destroyed their instruments live and blew up their drum set.

The Who were quite chaotic, with Keith Moon having a very unorthodox approach to drumming.

\subsection{Excerpt: Keith Moon}

Would interpret the entire piece, rather than just play the beat. Almost lyrical in his drumming, kicking the whole thing up.


Pete was the creative driver of the band. An early adopter of synthesizer and recording technologies. Had a patented wind mill guitar strum.

\section{``Baba O' Riley" (1971)}

About the teenagers at Woodstock being wasted on acid. Meher Baba: Indian Spiritualist - did not speak from 1925 to his death in 1969. Terry Riley: minimalist composer - pioneered using tape loops: example ``A Rainbow in Curved Air" (1969). Similar sonic arrangements in ``Baba O' Riley"


\section{Pink Floyd}

Recall ties to psychedelia in London. Long improvizations and experimentation. 70s lineup: bassist Roger Waters, keyboardist Rick Wright, drummer Nick Mason, and guitarist David Gilmour. Multiple \#1 albums: The Dark Side of the Moon (1973), Wish You Were Here (1975), and The Wall (1979).


\subsection{The Dark Side of the Moon}

Recorded at Abbey Road studios (same facility used by the Beatles) in 1972 and 73. Each side of the album is continuous (5 songs each side with no breaks in between) and each track represents a stage of life. Song ``Money" uses tape loops and a 7/8 meter. Note most rock songs are 4/4. 

\subsubsection{Excerpt: ``Money"}

Robert Rogers said it was initially meant to be a blues song; trans atlantic bluesy twan initially. Dick Parray on sax (did his solo in 7/8): ``Part of the Cambridge mafia." Changes to a 4/4 tempo for Gilmour's guitar solo.

\section{American Blues and Southern Rock}


The Allman Brothers Band (est. 1969), guitarist Duane Allman was a Muscle Shoals guitarist in the late 60s at Fame Studios. Best known for his improvizations and use of a slide (aka ``bottle neck playing"): see ``Layla" by Derek and the Dominoes (1970) featuring Duane Allman and Eric Clapton. Clapton referred to Duane as ``the musical brother I wished I had but never did." Duane recorded 3 albums with the Allman Brothers before dying in 1971 (motercycle accident)

\subsection{Excertp: ``Layla"}

Mutual respect and admiration between Duane and Clapton. Played notes ``off the top of the instrument"


\section{Lynyrd Skynyrd}

More radio friendly than the Allman Brothers. Made more direct connections with the South: ``Sweet Home Alabama" (1974) and ``Free Bird" (1976). Late 1977: plane crash kills 3 band members, including singer Ronnie Van Zant.


\section{Other ``Southern" Rock}

Santana: started by Carlos Santana, born in Mexico. Emerged from the San Francisco psychedelic scene. Played at Woodstock, and blends jazz and blues with latino rhythm and accompianment: popularized fusion of latin styles and rock, example being ``Evil Ways" (1970).  ZZ Top, a texas based band. Example Tush (1975), a twelve note type of blues-like song.


\section{Other American Blues}

Stepponwolf with ``Born to Be Wild" and ``Magic carpet Ride" (1968). Three Dog Night with ``Mama Told Me (Not to Come)" and ``Joy to the World" (1970). Grand Funk Railroad with ``Some Kind of Wonderful" (1975). Aerosmith with ``Sweet Emotion" (1975).


\section{Progressive ROck}


Also known as \Emph{Prog Rock}. The album as a self contained artistic statement was a central theme. The album art was considered as a critical component of the experience. Lyrics of ten centered on non-romantic topics such as religion, spritiuality, politics, power, technology and existential angst. Often incorportated classical styles. 

\begin{rmk}{}{}
    Peter Gabriel of Genesis often dressed up in concert costumes.
\end{rmk}

Gabriel eventually went solo, and drummer Phil Collins became the lead singer - more of a straight ahead pop rock sound. 

\subsection{Rick Wakeman of Yes}

Keyboardist. Often wore a cape. Keyboard soloist and keyboard ``king." Example ``Roundabout" (1971).

\subsection{Excerpt: King Crimson - ``21st Century Schizoid Man" (1969)}

sampled by Kanye West - ``Power" (2010)

\begin{rmk}{}{}
    In the mid-79s drummer Neil Peart joined the Canadian band Rush - pushed them in a more progressive direction.
\end{rmk}

They became very adept multiple instrument virtuosos. Passed away in 2020. 

\begin{eg}{}{}
    ``Tom Sawyer" (1976)
\end{eg}

\section{Jazz Influenced}

Frank Zappa: his music features complexity and satire, and is a virtuoso guitarist.


Miles Davis: jazz trumpeter fused rock elements, with ``Bitches Brew" (1970)


Horn Bands: \begin{itemize}
    \item Blood, Sweat, and Tears - ``You've Made Me So Very Happy" (1969)
    \item Chicago - ``Does Anybody Really Know What Time It Is" (1970)
\end{itemize}

\section{Glam}

Theatrical rock. More prominent in the UK in the early 70s than in the US. David Bowie: most famous glam rocker in the US; pushed gender boundaries with his androgynous character; breakout hit ``Space Oddity" (1969); created personas for himself such as ``Ziggy Stardust," as seen in ``Suffragette City" (1972)


\subsection{Authentic?}

\textbf{Quote:} ``I kind of, you know, tried to do my own version of my interpretation of that kind of music and it wasn't authentic at all. I think my stage movements kind of resembled a rather spindly wooden toy, I think was kind of my version of James Brown movement."

Bowie wanted to tap into a philadelphia sound, so he recorded at Sigma Sound Studios. 

Alice Cooper: On stage antics included ending shows by being hanged, being executed in an electric chair, being beheaded by a guillotine. Also pushed the boundaries of sexual and gender identities. Most famous song ``School's Out" (1972)


KISS: wore elaborate costumes with characters ``The Starman, the Demon, the Spaceman, and the Catman"

\begin{qst}{}{}
    What is Authenticity in Music?
\end{qst}

\section{Singer Songwriters}

Antithesis to Alice Cooper and David Bowie. Instead of playing characters they played themselves. Wrote confessional style lyrics. Both Bob Dylan and John Lennon continued to have successful careers as ``themselves" into the 1970s: i.e. Lennon's ``Imagine" (1971)


\subsection{Emerging Singer Songwriters}

James Tayler: Sweet Baby James (1970), Mud Slide Jim (1971), and One Man Dog (1972). Carole King (previously worked as a songwriter for other musicians): Tapestry (1970), Music (1972), and Rhymes and Reasons (1972).

Album covers are often just the musicians.

\begin{eg}{}{}
    Paul Simon (from Simon and Garfunkel) - ``Mother and Child Reunion" (1972)
\end{eg}

Recorded in Jamaica with Jamaican musicians, as he was a fan of Regae.


\section{Elton John}


Wrote the songs but not necessarily the lyrics. A string of successful albums, all \#1 in both the US and UK: Honky Chateau (1972), Shoot Me I'm Only the Piano Player (1973), Goodbye Yellow Brick Road (1973), and Caribou (1974). Singer songwriter and front runner - wore extravagant clothing, but was himself.

\section{Joni Mitchell and Neil Young (Canadians)}

Jonie Mitchell: enjoyed considerable critical success for her songwriting abilities and stylistic choices musically; biggest commercial success ``Help Me" (1974)


Neil Young: A member of Buffalo Springfield; on-again, off-again member of Crosby, Stills, Nash, and Young; Harvest (1972) \#1 album in US and UK, featuring ``Heart of Gold" (1972)








\section{``Brown Sugar" Stereogum article}

Hit number one on the charts on May 29, 1971, for two weeks. Jagger wanted to call it ``Black Pussy." He changed it when he decided that it was ``too direct." The Rolling Stones were so devoted to rhythm and blues that they didn't like being called a rock and roll band for a long time. ``Brown Sugar" is a song about white men having sex with black women, and it's a song where Jagger ties his own impulses to those of the white slavers who were raping black women hundreds of years earlier. 

Keith Richards insisted that ``scarred-up slaver" was ``skydog slaver," referencing Duane Allman. Jagger wrote the ``Brown Sugar" riff while he was filming his part in the 1969 movie Ned Kelly. 

They'd recorded the song at Alabama's Muscle Shoals Sound Studio, and two days after they got done with that session, they debuted it onstage at Altamont, the same show where their Hells Angels security stabbed a black fan to death.

Biz Markie performs a race-flipped version called ``Snow Flake."


\section{``Heart Of Gold" Stereogum Article}

Hit number one on March 18, 1972, and stayed there for one week. Young's album Harvest was created by Young and a ragtag group of Nashville session guys, including Mazer drummer Kenny Buttrey. 

When Young move to Toronto he joined the Mynah Birds, a band led by Rick James, and the band later signed to Motown. The band fell apart when James was arrested for being AWL from the US Navy, so Young and Bruce Palmer drove to LA and started the Buffalo Springfield. When that band broke up Young released a self-titled singer-songwritery solo debut, formed Crazy Horse, and dropped the absurdly important 1969 fuzz-rock masterpiece Everybody Knows This Is Nowhere. 

Young briefly joined Crosby, Still, Nash and Young and played Woodstock with them. ``Heart of Gold" can be thought of as a song about getting tired of groupies and wanting a stable relationship.

Young didn't like the way vast commercial success affected his life, so he went left, again and again. He went so left that a decade after Harvest, Geffen Records, Young's label, sued him for making music that was ``unrepresentative" and anti-commercial.


\section{``Fame" Stereogum Article}

Hit number one on September 20, 1975, and remained there for two weeks. 

In the middle of his Diamond Dogs tour, Bowie had booked a recording session at Philadelphia's Sigma Sound studio, the home base of Gamble and Huff and their Philadelphia International label. Bowie endeavored to figure out a sound that he called ``plastic soul," a distant and mutated and maybe quasi-ironic take on the sound that he'd loved.

``Fame" was recorded in New York's Electric Lady Studios. Lennon and Bowie wrote the song together, inspired by a Carlos Alomar riff. 



\chapter{Black Pop, Funk, and the Roots of Rap}


Riots in the black communities of Watts, LA (1965), and in New York New Jersey and Detroit (1967) signaled the rate of progress for civil rights was too slow.


\section{Excerpt: Riots}


White counter culture celbrating the summer of love. Black americans began identifying more with African cultures. Transition from being ``blacks" to being ``Afro-americans," especially college students: ``I'm black and I'm proud." Black Panthers burst on the scene in Oakland, founded by Bobby Seale and Huey Newton, who was a law student and his advisor Edwin Meese went on to work with Reagan. Huey was convicted for stabbing someone, then later returned to finish law school. Oakland was known for their racist police. Black middle class increased in this time, but black lower class also increased in both size and poverty. Black Panthers fought for and utilized this population. 


Black communities in other cities burst out in rage: Watts in 1965, New York NJ in 1967, and Detroit in 1967. In 3 days 100+ blocks were engulfed in chaos. Johnson sent out almost 5000 troops, without bullets, to stop the riots. At the end there were 43 dead, 33 of which were black. Hoover tried to convince Johnson that the riots were a communist plot.


James Brown's ``Say it Loud - I'm Black and I'm Proud" became an anthem for the black counter culture. Brown was very vocal with his views.

\section{Excerpt: Black Dignity TV Show}

``We got to get our image, our identity." Brown cut his hair and went natural - an afro. He was known for his hair before the change, so giving it up showed support for his people. He made ``Say It Loud: seeing all the turmoil and in fighting in Black communities. ``I'd rather be a black man `cause that is where I have my identity." ``If you stand up in the states they say you're a militant, but I say you're a man." Also addresses colourism against dark skin black people, especially women.


\section{Funk}

Brown is the godfather of this style - ``Say It Loud" is considered funk. Emphasis on rhythm: interlocking guitar, bass, and drums - seems like a single unit. An emphasis on the first beat: \Emph{ONE-two-three-four}. Concept of ``groove" is foundational - shouldn't have to count, rather feel the emphasis.

\begin{eg}{}{}
    ``Cold Sweat" (1967) (often celebrated as the first funk song). Sax player repeated riff from Miles Davis' ``So What." Also important guitar parts. ``Cold Sweat" signals Brown's own new style: really tight and loose at the same time, like jazz. Maceo Parker, saxophonist, for Brown. Clyde Stubblefield, drummer for Brown: bouncy left hand, grace notes on his snare (gospel style), never took lessons - played from his heart and soul. Jazz was 2 and 4, while funk was 1 and 3 (pick you up and drop it back down). Funk = comination of jazz and R\&B.
\end{eg}


\subsection{Tiger Roholt: Groove (2014)}

\textbf{Quote:} Drummers, other musicians, vocalists as well, go to great lengths to accurately perform rhythms in such a way that they acquire various qualities of groove, specific qualities of ``pushing," ``pulling," ``leaning forward," being ``laid-back," being ``in the pocket," and so on. Musicians achieve this by playing certain notes ever-so-slightly early or ever-so-slightly late."

Loosely speaking, a froove is the feel of the rhythm.


\subsection{The Funk cont.}

Pre-70s funk success can mostly be attributed to James Brown. This success continued into the early 70s: ``Get Up (I Feel Like Being a) Sex Machine" (1970), ``Super Bad" (1970), and ``Hot Pants" (1971). All of these songs are very much dance oriented. Funk itself is very much a dance music.


\section{White Rock}

No dancing expectation: a spectator sport. By the end of the 1970s, fans of mainstream rock were overwhelmingly white. Same goes for the musicians and others involved with making, performing, and selling rock. 


\section{Sly and the Family Stone}

Early example of funk. Heavily influenced by James Brown. Part of San Francisco psychedelia scene. One of few racially and sexually integrated/diverse bands and this was reflected in their music. ``Dance to the Music" (1968), the first of many crossover singles. Sons have a ``groove," that is, a repeated rhythmic element. Inspired a generation of funk and pop musicians. Sly was an artist and producer - embraced the ``do it yourself" approach to recording (common in present but rare then).

\subsection{Excerpt: Sly}

Black artists using the studio in their own unique ways and empowering themselves. Took a stand politically and musically - was his own boss. Put his own spin on song craftmanship and funk. Did his 5th record ``There's a Riot Going On" by himself due to issues with the Family Stone. Portrayed early hip hop styles.


\section{Funk Followers}

The funk elements of Sly and the Family Stone's music paved the way for other dance-oriented bands: Ohio Players with ``Love Rollercoaster" (1975), Kool and the Gang with ``Jungle Boogie" (1973) (very dance oriented music), Earth, Wind, and Fire with ``September" (1978), and Tower of Power featuring black, white, and latino musicians.



\section{War: Eric Burdon Goes Funk}

War: group found by Eric Burdon (from the Animals). First album: Eric Burdon Declares War. Burdon left the group after the first album. Notable songs include: ``Spill the Wine" (1970), ``The World is a Ghetto" (1973), and ``Low Rider" (1975). ``Spill the Wine" had a distinct groove, with possibly drug inspired lyrics.


\section{Motown}

Embraced the funk trend. Motown's flagship group, the Temptations, absorbed the psychedelic and funk influences of SLy and the Family Stone. ``Papa Was a Rollin' Stone" (1972) addressed problems with black urban life. It was nearly 12 minutes long with no vocals until about the 4 minute mark. Musically darker tone, with haunting string arrangement to set the mood. Song is built on a single repeated bass line (``groove").

\subsection{Motown: The Commodores}

One of the most successful black pop bands of the 70s. Started out as a party band before signing to Motown. Gained exposure opening for the Jackson 5. Early roots in funk evident in ``Brick House" (1977). Greatest success came from Lionel Ritchie's penned ballads such as ``Easy" (1977) and ``Three Time a Lady" (1978). Ritchie would go on to have a successful solo career in the 1980s. The group also performed at American 1978 music awards. 

\begin{rmk}{}{}
    Jackson 5 is an example of Motown produced black pop.
\end{rmk}

``I Want You Back" was their first number 1 hit. The Osman's (a white group) copies Jackson 5's style.

\subsection{Motown: Marvin Gaye}

What's Going On? (1971) - one of the first concept albums in black pop. Multiple crossover hits: ``What's Goin' On," ``Mercy Mercy Me (The Ecology)," and ``Inner City Blues (Make Me Wanna Holler)." Confronted many issues: black urban life, environmentalism, and the Vietnam war. Followed up What's Going On (1971) with Let's Get It Out (1973). What's Goin' On was an intellectual song - the lyrics were meant to be listened to, not just danced to. It was a response to global and national events, and contrasted with Motown's usual apolitical model (Berry was against it).

\subsubsection{Excerpt: What's Going On}

Gaye started really noticing what's going on in Vietnam and the world in 1969/1970, changing his viewpoint. In Motown you had ``freedom within restriction." ``What's Goin' On"'s lyrics, instruments, and backing sounds increase as the song progresses. Motown was afraid of addressing all the issues Marvin was seeing. Marvin had a brother in Vietnam.


\subsection{Motown: Stevie Wonder}

Was a child star with Motown, startin at 11. On his 21st birthday he was given complete artistic control over his records. Often wrote, produced, and played many of the instruments on his albums. Clearly influenced by album oriented rock as evidenced by his experiments with new sounds and timbres, especially with the synthesizer.

\subsubsection{Excerpt: Stevie Wonder}

Motown artist styles and performances cultivated to not offend mainstream white sensibilities. Stevie was fed up with it at 21. Malcolm and Bob had to figure out what was going on in Stevie's head, and then use their technology to translate it into reality. Stevie embraced electronics in his music. His music featured: complex arrangements, an intense vocal style, inventive musical material, and topical lyrics. 


Stevie recorded his first \#1 album at 12. At 21 Stevie's contract with Motown expired. In a 5-year period he released 5 albums, 8 top singles, and obtained 12 Grammy's: known as the Classic Period. Almost no creative control in Motown initially. The new deal gave Stevie full control, as well as a 14\% royalties. Music of My Mind: Wonder plays everything except guitar. Malcolm Cecil and Robert Margouleff, of Tonto's Expanding Head Band, caught Stevie's eye (named after Tonto's synthesizer, invented by Malcolm).

Stevie covers had orange, red, and yellow as the central palette. Jeff Beck wrote the guitar line for ``Superstition," which won two Grammy's. Wonder took a break in 1976 for the first time since 1966. This leads up to the concept album ``Songs in the Key of Life" - the first album he produced on his own. Love defines the theme of much of Wonder's career. He then took three year's off, signaling the end of his Classic Period.


\section{Blaxploitation}

Film genre that took a take on funk and soul music. Example: the 1971 film Shaft: oscar-winning soundtrack by Isaac Hayes. Hayes worked as a writer and producer of Stax for artists such as Sam and Dave. Had a series of \#1 albums on the soul charts as a solo artist as well. His approach to vocals, part spoken and part sung, became a model for other artists. Prominently features the wah-wah guitar effect. He was known for singing low. Conversely, Curtis Mayfield was known for singing in the higher range, and did the soundtrack for Superfly (1972).


\section{George Clinton}

Wrote songs under two groups: Funkadelic and Parliament. The first few Funkadelic albums blended psychadelic rock with soul, showing influences of Hendrix, Sly and the Family Stone, and James Brown. He used Brown's musicians: Maceo Parker (sax), Fred Wesley (trombone), and Bootsy Collins (bass). Breakthough album by Parliament was Mothership Connection (1976). He often performed as ``Dr. Funkenstein"

\subsection{Excerpt: ``Tear the Roof Off"}

Dr. Funkenstein decends from space to give the funk.

\subsection{Bam on Funk}

Direct connection between hip hop and funk. ``All hip hop music is based on the funk." ``Hip Hope was born in the Bronx" - Tony Fletcher. NY consists of 5 burroughs, with the Bronx in the north.


\section{Hip-Hop}


Typically said to be made up of 4 elements: \begin{itemize}
    \item DJing
    \item MCing
    \item B-boying 
    \item Graffiti
\end{itemize}

\textbf{Quote:} ``People talk about the four hip hop elements: DJing, B-Boying, MCing, and Graffiti. I think that there are far more than those: the way you walk, the way you talk, the way you look, the way you communicate." - DJ Kool Herc


\textbf{Quote:} ``To me, hip hop says, ``Come as you are." We are a family. It ain't about security. It ain'st about bling-bling. It ain't about how much your gun can shoot. It ain't about \$200 sneakers. It is not about me being better than you or you being better than me. It's about you and me, connecting one to one. That's why it has universal appeal. It has given young people a way to understand their world, whether they are from the suburbs or the city or wherever."

\subsection{Excerpt: Chuck D}

Hip hop is the culture of whatever black people create. It is in essence black creativity.

\section{The Bronx}

In 1940 the Bronx had 1.4 million people. 33\% were in the South Bronx, 90\% of which were in rentals. The old poor (Jewish people and German people) moved out, as the new poor (black people from the South and West Indies, Puerto Rican people) moved in. Over time 170,000 black people and latinx people were displaced by Manhattan slum clearance and went to the Bronx. South Bronx lost 60,000 manufacturing jobs - 40\% of the sector disappeared. By the mid 70s average per capita income dropped to \$2430, half of the NYC average.

\subsection{The Bronx in Decline}

Cross-Bronx Express was constructed in the late 1940s and early 1950s. Cut through 113 streets, 7 parkways, 8 expressways, 9 railways and subway lines, relocating 60,000 from a dozen neighborhoods. Le Corbusier housing also started: 12,500 apartments in 96 buildings or 5 housing projects. All designated as low income housing: no gardens, stores, movie theatres, etc. Managed by slumlords: landlords who negledct buildings because they are not profitable.

\subsection{The Bronx on Fire}

Landlords turn to arson to collect insurance money: they start buying buildings to burn them down. NYC gave burned out tenants priority for public housing and replacement money for old furnishings, so tenants became arsonists too. Firefighters would sow up to families fully dressed and suitcases packed. Between '73 and '77: 30,000 fires were set. By about 77/78: 43,000 housing units were lost

\subsection{1968: Heroin Influx}

Proliferates the poorer neighborhoods of NYC: LES, Harlem, and the Bronx. Polic involvement in the drug trade. Crime escalates: 1650\% through the 1960s in the Bronx, with a rise of gangs.

\subsubsection{Excerpt: South Bronx}

Cross-Bronx expressway killed the Bronx. Residents kept moving, but wherever they moved was called the South Bronx - it was an idea. Wearing the wrong stuff in the wrong territory got you beat up. Gang peace meeting in 1973 to end the violence.


\section{DJ Kool Herc}

Clive Campbell, aka DJ Kool Herc. Emigrated from Jamaica in 1967, was oused at the Grand Concourse hotel after being burned out on East 168th street. The Grand Concourse housed the Tunnel Plaza discotheque which Herc frequented. In Jamaica, his family spent some time living in Trenchtown where PA culture thrived. ADj Kool Herc Party - a back to school jam for his sister's Birthday was his first hip hop DJing scene.


\subsection{Herc's Method}

Played Jamaican music, but also James Brown, Isley Brothers, and Rare Earth. He announced himself Jamaican MC style using a mic and echo chamber. Observed that the instrumental breaks were party goers' favourite times to dance. His speakers became known as the ``Herculoids" - very loud. He played music you didn't hear on the radio. Soaked off the record labels to preserve originality and prevent people from copying him.

\subsubsection{Excerpt: Herc}


All word of mouth advertising. Hosted block parties. He was considered the godfather of hip hop. Used record store bins for new material. The Incredible Bongo Band was a major group sampled by Herc - in particular, Apache was the main song he sampled from. Many famous artists covered this song including the Beatles. It inspired Herc's merrygoround style, in particular the drum beat portion. He used light pole's energy as a source of power for his speakers. MC's were employed to pump up the crowd, leading to rap with them adding rhymes on top. 


Scratch was invented, by accident, while DJing the Bongo Band. 


The blackout of 1977 spanned two days and resulted in mass looting.

\section{The Jeffersons}

Had 11 seasons from 1975 to 85. Spin off of All in the Family - George and Louise Jefferson were Archie Bunker's neighbors. The Jeffersons move from Queens to Manhattan thanks to the success of their dry cleaning business.

\subsection{Episode: ``The Blackout" (1978)}

Jefferson's mugged and their store is robbed, before he is arrested for robbing from his own store. Most people in jail are black, while the officers are white. Man rob's an appliance store to sell for food for his family. Jefferson decides to not close down after initially wanting to due to his distrust in the people in the neighborhood.

\section{Flash: DJing as Composition}


\textbf{Quote:} ``As for [Herc's] skills, I continued to pay close attention to what I saw him do. I still had issues with the lack of precision, but I was noticing patterns and whythms to how Herc connected the dots between his songs."

\textbf{Quote:} ``To Herc, a DJ set was one continuous piece of music. My man was composing something. And if he was a composer, that went for me too... I'd hear a piece of one song and a piece of another and would imagine the two pieces playing together over each other.

People say Flash usurped Hurc with his seemless transitions between breaks.

\subsection{Excerpt: Flash}

Hurc would play records at random, not really on time. Flash switched on the beat to keep the dancing going. Used the ``breaks" of the records (5sec) and extended them for dancing.

\subsection{Quick-Mix Theory}

\textbf{Quote:} ``I found a way to start the first record with my hand physically on the vinyl itself. The platter would turn but the music wouldn't play because the needle wouldn't be traveling through the groove. However, when I took my hand off the record...BAM! The music started right where I wanted it. It was that simple, but I had jsut made my first discovery"

Touching the vinyl was seen as poor practice or taboo.

\subsection{Flash's Clock Theory}

\textbf{Quote:} ``I took a grease pencil and drew a big line on the label of the record, pointing right to the first beat in the break-now I knew exactly where I needed the needle to hit the beat. Then I let the grease pencil trace a line around the record as the break played-now I knew exactly where the break started and ended."


\textbf{Quote:} ``Now I had a formula: \begin{itemize}
    \item Flip the fader as soon as the end marker hits the needle
    \item Wind the other record back while the first one plays
    \item Wait for the break to finish
    \item BAM! Throw the fader on turntable numebr one and start all over again"
\end{itemize}



\subsubsection{Excerpt: Clock}

Had complete control using his fingers. Used a circular mark with a crayon to mark where the break is.

\subsection{Flash on Record}


``The Adventures of Grand Master Flash on the Wheels of Steel" (1981). Live DJ Mix (7+ mins long): 3 turntables and 2 mixers, 3 hours (and 10-15 takes/attempts), and no edits - done in 1 single live take.


\subsubsection{Excerpt: Break Mix}

Need something to reduce drag on the record, so he put felt or paper below the record.


\section{Joseph Schloss}

\textbf{Quote:} ``Hip hop's idiosyncrasies were designed to represent the spirit and intelligence and individuality of its many creators, in a world that would have preferred to ignore them. So when we focus on the seemingly minor artistic and practical choices that go into hip-hop production...we are not minimizing hip-hop's social or political significance. We are celebrating its humanity."

\section{Afrika Bambaataa}

\textbf{Quote:} (On seeing Zulu) ``That just blew my mind, because at that time we was [c**ns], coloreds, [n-words], everything degrading. We was busy watching Heckyl and Jechyl, Tarzan - a white guy who is king of the jungle. Then I see this movie come out showing Africans fighting for a land that was theirs against the British imperialists. To see these black people fight for their freedom and their land just stuck in my mind. I said when I get older I'm gonna have me a group called the Zulu Nation."


He traveled to Africa in 1975. His cousin and best friend was shot and killed by police in 1975. He formed the Zulu Nation, replacing the ``gang" with organization. Preached the elements of hip hop and made a fifth: knowledge. Also added: ``Peace, Love, Unity and having fun." He did graffiti tags: BAM 117, BOM 117, BAMBAATAA. DJ'd an electric mix, for example featuring: James Brown, Parliament, Rolling Stones, and the Monkees. Bam and his crews made up of DJs, MCs, B-boys, and security wore jackets (like in gangs). In June 2016 Bam was removed as a leader of Zulu Nation following sexual abuse allegations.

\begin{rmk}{}{}
    Hip-hop music elements all contributed to the idea of ``pay attention to me"
\end{rmk}

Flash Dance Film: features scene with b-boying to hip hop music.

David Letterman hosted two b-boys, Crazy Legs and Ken Swift.

\begin{note}{}{}
    Graffiti is the most clear example of garnering attention in hip hop.
\end{note}

\subsection{Excerpt: Graffiti}

``Lok at me, I'm famous" mentality. Sign your name and underline it. They wrote it our before hand, then went and expanded onto the trains. Had to do it over and over so it wouldn't dissapear. They would often steal spray paint using jacket sleaves. ``Art is anything you can get away with" - Warhol. THe end goal was to see their artwork on display in public.


\section{Towards Rap...}

Initial function to support the DJ in hip hop.

\subsection{Excerpt: Rap}

The MC was the guy who shouted out the DJ and his future parties. Examples: Lovebug Starsk (MC/rapper) and Eddie Cheba (MC/rapper). DK Hollywood  popularized rhyming in MCing. In Wild Style, we see a rap battle scene on the basketball court. 

\begin{rmk}{}{}
    Hip-hop started out of the idea: ``look what I can do, I matter"
\end{rmk}










\section{``Everyday People" Stereogum Article}

Hit \#1 on February 15, 1969, and stayed there for four weeks. The Family Stone had black members and white members, men and women, everyone playing different rolls. This band brought back together R\&B and Rock `n' roll.

Sly Stone, born Sylvester Stewart, was making gospel records at the age of four, and by high school he was in a racially integrated doo-wop band called the Viscaynes. In the mid-60s Stone was in the San Fran rock scene producing records for garage-rock bands and working as a DJ on R\&B radio, playing rock and soul records alongside each other. 

The Family Stone was initially a bar band called the Stoners. They had already released four albums prior to Stand! which yielded ``Everyday People," their first \#1. The song is a world vision of people from different races and different walks of life learning how to come together and respect each other's differences. The vocals stay buried in the mix, a reflection of a Bay Area acid-rock scene where the vocals were almost an afterthought. (Scooby-doo)



\section{``I Want You Back" Stereogum Article}

Hit \#1 on January 31, 1970, and stayed there for 1 week. With ``I Want You Back" the first nationally available single from the Jackson 5, Michael Jackson became the first singer to top the Hot 100 who was younger than the Hot 100 itself. Jackson was 11 when he recorded the lead vocal on the song.

There is debate on whether James Jamerson or Wilton Felder played the bass line on the song.

Joe Jackson their father pushed them out onto the stage at an early age, hitting Indiana Talent shows, chitlin' circuit venues and even strip clubs. 

Initially Motown turned the Jacksons down twice. When Gordy signed the Jacksons in 1969 he put all of Motown's resources into the group. Gordy marketed them as a group discovered by Diana Ross. Gordy had also just moved Motown from Detroit to Hollywood, with the song being the first hit from the new production unit the Corportation.

The Jackson 5 were the first group to see their first four singles all hit \#1.


\section{``Theme From Shaft" Stereogum Article}

Hit \#1 on November 20, 1971 and stayed there for 2 weeks. The blaxploitation movies of the 1970s was a result of film executives pandering to a specific market as cheaply as they could. The are best known for their music. 

Shaft was possibly the first case of a major studio figuring out how to make an action movie for a black audience. Hayes, who made the sountrack for the film, grew up dirt poor in Tennessee. His father left, and his mother died young, so his sharecropper grandparents raised him. When Hayes won the Best Original Song Oscar for ``Theme From Shaft" - the first black non-actor to win any Oscar - he brought his grandmother to the ceremony and dedicated the award to her. 

Hayes started out singing in church, and that's where he taught himself to play piano, organ, flute, and saxophone. Hayes eventually went to work for Stax Records. Hayes started out as a session musician and he soon became a producer and a songwriter, cranking out a ton of hard-squelching hits with his regular collaborator Dave Porter.

After Otis Redding's death in 1967, Hayes became a solo artist for Stax. Hayes would take well-known songs, and he would turn them into existential deep-funk symphonies, spending entire record-sides intoning deeply over chicken-scratch guitars and swirling strings.

Hayes later bounced into a full-on acting career.

``Theme From Shaft" is the only \#1 single that the blaxploitation funk genre ever yielded.



\section{``Papa Was a Rollin' Stone" Stereogum Article}

Hit \#1 on December 2, 1972, and stayed there for 1 week. By 1972 the classic Motown sound was dead. Instead the music had turned lush and psychedelic and orchestral. Motown had moved most of its operations to LA, and the label had its own soul auteurs which gained more artistic control. On the other hand the Temptations still recorded in Detroit, with the Funk Brothers. 

Before splitting ways, Whitfield and the Temptations would pull off one last job, ``Papa Was a Rollin' Stone." Originally Norman Whitfield and Barrett Strong wrote the song for the Motown trio Undisputed Truth, and they recorded in 1972. But, their version was only a minor hit on R\&B radio.

The song was 12-minutes on the album, and even the single was seven minutes long. The guitars from Funk Brothers' Melvin ``Wah Wah Watson" Ragin and Paul Warren carry through the song. The song is about a terrible father - a man who may have seen himself as a romantic devil-may-care outlaw type - who is now dead and goes unmourned by his family.

There is no lead singer on the song, with the Temptations all trading off lead vocals. The Temptations weren't happy about the song, thinking that it was more of a Norman Whitfield record than a Temptations record.



\section{``Superstition" Stereogum Article}

Hit \#1 on January 27, 1973, and stayed there for 1 week. Stevie Wonder was 11 when he signed with Motown. For a full decade Wonder worked as a cog within the Motown machine. Wonder first hit \#1 with the electric 1963 single ``Fingertips (Pt. II)." Stevie started writing and producing his own records, learned new instruments, and got married, all before turning 21. 

When Stevie Wonder re-signed with Motown, he wanted complete artistic control, his own publishing, and a better royalty rate. During the 70s Stevie Wonder cranked out Where I'm Coming From, Music of My Mind, Talking Book, Innervisions, Fulfillingness' First Finale, and Songs In The Key of Life. He won album of the year Grammyes in 1974, 1975, and 1977. 

Jeff Beck wrote ``Superstition" alongside Wonder, and originally Beck was going to cover and release the single but Berry Gordy didn't want that. Wonder wrote, co-produced, and played every instrument on the song except the horns. The song is still a soul song, with its perfectly timed horn stabs and its squechy low end. Wonder is singing about how we shouldn't believe passed-down wisdom. 


\end{document}


%%%%%% END %%%%%%%%%%%%%
