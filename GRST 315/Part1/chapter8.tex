%%%%%%%%%%%%%%%%%%%%% chapter.tex %%%%%%%%%%%%%%%%%%%%%%%%%%%%%%%%%
%
% sample chapter
%
% Use this file as a template for your own input.
%
%%%%%%%%%%%%%%%%%%%%%%$% Springer-Verlag %%%%%%%%%%%%%%%%%%%%%%%%%%
%\motto{Use the template \emph{chapter.tex} to style the various elements of your chapter content.}
\chapter{The Venus Pudica: uncovering art history's `hidden agendas' and pernicious pedigrees}
\label{venusPudica} % Always give a unique label
% use \chaptermark{}
% to alter or adjust the chapter heading in the running head


%%% Questions to think about
%The \textbf{Thesis} or general sense of the article is ...

%The \textbf{method} the author uses to argue their point is ...

%In their \textbf{analysis} the author uses tools such as ...
% How do they look at the evidence? Do they place it in some theoretical framework? (i.e. gender studies, music studies, etc.)

%Additionally they conclude ...
% How does this compare to others throughout time? What is the societal context?

%What connections does the author portray with regard to \textbf{space}, \textbf{relationships}, \textbf{occupation}, and \textbf{religion}.


\abstract{}

\section{Questions and Remarks}
\label{sec:QR8}

\begin{qst}
    What is the \textbf{Venus Pudica}?
\end{qst}

\begin{qst}
    Who is \textbf{Praxiteles}?
\end{qst}

\begin{qst}
    What is the \textbf{Knidian} Aphrodite?
\end{qst}


\begin{qst}
    What is the \textbf{Pudica}?
\end{qst}




\section{First Reading}
\label{sec:FirRead8}



To write history, and perhaps particularly the history of cultural objects, is to engage in creating coherences. 

\begin{rmk}
    Within traditional art history, no more satisfying explanatory connection can be made for a work of art than to link it to classical antiquity.
\end{rmk}

Ever since the Italian Renaissance, the infinitely elevated regard for classical works of art has been matched only by the elevated intellectual gratification produced by the historians coherent narratives, which revel in unending reaffirmations of `classicism' as the adherent stuff of western cultures `high' history.


\begin{rmk}
    Female nudes fashioned as covering their pubises were and continue to be a most favoured subject/pose/gesture in the art of the western world.
\end{rmk}

The pose was first mainstreamed into western culture by fourth-century Greek sculptor \textbf{Praxiteles}. The endemic presence of this pose has become so normalized, so `natural', that it is made invisible or transparent.


\subsection{The Knidian Aphrodite by Praxiteles}

Praxiteles' monumental sculture, usually called the \textbf{Knidian Aphrodite}, was produced in the volatile period around 350 BCE. It is the very first monumental cult statue of a goddess to be represented completely nude. Moreover, it is the first monumental female nude sculpture to be positioned with her hand over her pubis, which at some undetermined moment in ancient times was given the name `\textbf{pudica}', or so-called \textbf{modest pose}.

\begin{nte}
    Its popularity was expressed not only in accolades of ancient writers but also in the countless Hellenistic and Roman copies, adaptations and derivations `inspired' by Praxiteles' concept.
\end{nte}

Praxiteles' introduction of the monumental female nude occurred at least three centuries after the introduction of its counterpart, the monumental male nude statue.

\begin{rmk}
    A survey of Greek monumental sculpture of men and women in teh sixth and fifth centuries readily reveals the strong differentation along gender lines already inherent in their definition. 
\end{rmk}

In the archaic period the \textbf{kouroi} (athletic male youths) are fabrications of an idealized humanity defined as male, youthful, and heroically nude. The correspondence female \textbf{korai} are consistently draped. The male anatomy continued in fifth-century classicism to be the form in which primary creative energy was invested. Its treatment is ever more precisely scientifically informed, culminating in \textbf{Polykleitas\ Doryphoros}.

\begin{quotation}
    The male figure is portrayed as coherent and rational from within; the female figure is portrayed as attractive from without; the male body is dynamically explored as an internally logical, organic unity; the female body is treated as an external surface for decoration.
\end{quotation}

The asymmetrical treatment of the nude male and clothed female in archaic and classical Greek art can be matched with the by now well-known social and legal inequities between men and women in ancient Athens. 

\begin{nte}
    In the formation of the \textbf{polis} or city-state, women were legally positioned somewhere between slaves and citizens, and under the law they fell closer to slaves than to citizens.
\end{nte}

The artistic practice coincided with the differentiated social practices of the polis, where young men exercised in the nude, while women in public places were always discreetly covered. The practice of preserving an idealized concept of youthful nudity exclusively for the masculine subject had a strong historical relationship with the Greek definition of beauty, which was defined specifically as a male attribute and ultimately with Greek homoerotic desire. 

\begin{nte}
    Homoerotic impulses were considered natural in ancient Greece, and that socially legitimate desire contributed to the forming of the male nude as an ideal.
\end{nte}

The male sexual organs are presented like any other body part, having no special claim to our attention. This is not the case in the monumental female nude introduced into Greek sculpture.

\begin{quotation}
    Praxiteles' Aphrodite is in the condition of both complete nudity and self-conscious nakedness.
\end{quotation}

The Knidia is the starting point of a history that privileges the female over the male nude. Further, it is a history that sexually defines the represented woman by her pubis and, on that account, keeps her in a perpetual state of vulnerability.

\begin{quotation}
    ...the artistic codes of female nudity as fetishized
\end{quotation}

The issue of whether Aphrodite in the Knidia points to herself as to her powers of fertility, or whether she is covering herself before the eyes of an intruder can never be resolved. We are, in either case, directed to her pubis, which we are not permitted to see. Woman, thus fashioned, is reduced to her sexuality.

Praxiteles was renowned for naturalizing the gods, making them more human and life-like than ever. The conditions of desirability presented in Praxiteles' creation shed light on its enduring popularity as a benchmark for the construction of woman as perpetual rape victim in western European art. 

\begin{rmk}
    The Knidia is portrayed holding drapery in her left hand above a vase. Iconically, this type of image recalls Aphrodite's connection with water as she was born from the sea. On the level of narrative, it communicates that she, as a grown woman, was in the process of bathing.
\end{rmk}


The gesture of the right hand over the pubis constructs a sexual narrative of protective fear that is conveyed by her body language as a whole.

The word \textbf{pudica} is etymologically related to \textbf{pudenda} a word that simultaneously means both shame and genitalia.This goes back to the double meaning of the Greek root word \textbf{aidos}. In a description by the \textbf{Pseudo-Lucian}, he says it is the Knidia's \textbf{aidos} that she covers with her right hand

\begin{nte}
    The etymological connection situates those `things about which one must have \textbf{pudor}, modesty, shame, and respect' with sexual demeanour. 
\end{nte}

For the Greeks, \textbf{aidos} is a virtue to be taught as part of a young boy's education between the ages of fourteen and twenty to balance out his natural tendency to \textbf{hubris} or arrogance. Moreover, \textbf{aidos} is related to the all important Greek notion of \textbf{s phrosyne}, meaning soundness of mind, sobriety and self-control, the trait which allows one to master ones desires by exerting rational control. Feminine \textbf{s phrosyne} `always includes, and is frequently no more than, chastity'. Even when it does come to be used as chastity for both men and women, `masculine chastity derives from self-control, feminine chastity from obedience.' This is an idea championed by \textbf{Aristotle}. 

\begin{quotation}
    For the man, control comes from within, for the woman, since she cannot control herself, it must be exerted from the outside.
\end{quotation}


\begin{quotation}
    Man, as his image, is constructed as managed internally, woman, as her image, is constructed as managed externally.
\end{quotation}

The pose constructs the female as the opposite of the aggressive unseen male. According to Foucault and Dover, Greek sexual relations are always conceived of as 

\begin{quotation}
    being of the same type as the relationship between a superior and a subordinate, an individual who dominates, and one who is dominated, one who commands and one who complies, one who vanquishes and one who is vanquished.
\end{quotation}

While such sexual practices were apparently equally operative in the love of boys and women, in monumental Greek sculpture they find expression only in the female form.

\subsection{The Pudica in the Christian Era/Venus as Eve}


The Christian period capitalized on the connections between the pudica pose and its narrative implications. The word `pudica' is never applied in traditional art istory to Eve. Nevertheless, the pudica pose is the one classical trope which is maintained without break throughout the medieval period. The ancient form of female nudity `fits' the Christian disdain for the human body, especially the female body, so well. The hand-to-genitals gesture is represented as a normal way to hide nakedness, not because the Bible describes it that way. 

\begin{rmk}
    Within the medieval visual scheme, man has been subjected to the worst form of humiliation: by being defined as a pudica, he has been feminized.
\end{rmk}

In the \textbf{Expulsion from Paradise}, Adam covers his face keeping his emotional expression of grief and shame from the viewers gaze; Eve covers her breasts and pubis. Once again, while the torture of Adam's shame is an emotional internal affair, Eve's is indexed by reference to her primary and secondary sexual organs.

\subsection{True Renaissance Connections}

The culture of early fifteenth-century \textbf{Florence} is the initial seat of the Italian Renaissance. It is there that the human nude is no longer exclusively relegated to the shameful Adam and Eve. A Renaissance meant a revival of classical concepts and values that were deeply homocentric, misogynist, classist, and racist.

The celebration of the male nude precedes that of the female nude, as in ancient Greece. \textbf{Donatello}'s bronze sculpture of David is the case in point. This influenced \textbf{Michelangelo}'s statue of the same subject. Janson writes that we must take into account Donatello's `reputation as a homosexual', and further, that we should recall the creation of the work coincides with the publication of \textbf{Hermaphroditus}. The homoerotic aspects of Michelangelo's David are well known and have been discussed in the scholarship.

The reintroduction fo Venus proper into the western European artistic tradition was predictably as a pudica. Botticelli's \textbf{Birth of Venus}, about fifty years after David, is acclaimed as  a `first' in the narratives of the traditional historical canon as Praxiteles' and Donatello's work.

\begin{quotation}
    The Birth of Venus, in fact, contains the first monumental image since Roman times of the nude goddess in a pose derived from classical statues of Venus.
\end{quotation}

The pose's intrinsic work in constructin female sexuality could be accessed by acknowledging its shared use with the representation of Eve, as in \textbf{Masaccio}, or in its difference from Donatello's \textbf{David}, as two examples. The Botticelli's Venus and the ancient source share a vacuous `unknowing' look. They gesture as if in a trance or through some agency outside their own volition. Again the gesture is divorced from a narrative reading of a particular figure or moment and thus free to work as an essentialist definition of woman in general through this all-telling attribute.

\subsection{The Supine Pudica}

The sixteenth-century \textbf{Venetians}, \textbf{Giorgione} and \textbf{Titian}, popularized the gesture in a reclining figure in their mythological paintings of recumbent Venus. In the \textbf{Sleeping Venus} and \textbf{Venus of Urbino}, her one act is to draw her hand to her pubis, again both directing attention there while blocking its full view. Accent on the pubis is further abetted by the formal technique of tipping that part of the female body up and presenting it forward so that it is fully exhibited.

\begin{rmk}
    The pictures' message ultimately conveys a form of licensed voyeurism and ownership.
\end{rmk}

The proliferation of the artistic nude female pudica from the early sixteenth century onwards is in a proportionally inverse relationship to that of the nude male. The pudica as a form that culturally promotes and instigagtes a certain kind of heterosexual desire can be seen as a reaction against the homosexual erotics carried by the artistic male nude.

For the counter-reformation writers Michelangelo's love of male nudity and its open display in teh \textbf{Last Judgement} in 1541 signified all that was lewd and its destruction was contemplated even before it was unveiled.

In his letter \textbf{Aretino} asserts that Michelangelo's figures are more suitable to a bath house than to the highest chapel in the world. He invokes the `modesty' displayed `even by the Ancients', although significantly he can cite only sculputes of female deities: Diana clothed and Venus where they were `careful that the chaste gesture of her hand should replace her vestment'. Further, Aretino recommends that Michelangelo follow the example of the modest Florentines, who have covered the genitalia of his David with leaves.

\begin{nte}
    The censorial practice of mutiliating and then covering with fig-leaves the genital area of ancient and classcizing male figures became a commonplace during the counter-reformation.
\end{nte}

The pudica, unlike the fig-leaf, is presented as part of the volition of the figure herself. It is designed as both the narrative and inner character of ideal femininity. 

\begin{quotation}
    The fig-leaf is seen as a social imposition; the pudica gesture is seen as a personal condition.
\end{quotation}



\subsection{By Way of a Conclusion}

The Venus pudica's initial historically significant appearance occurred at the crucial moment when the citizen/slave structure of the ancient polis gave way to the far more complex social order and division of the Hellenistic empire. The vulnerable, sexualized female nude is the culturally fabricated site and teh public display of heterosexual desire for that male bonding ritual. The representation of `pudicated' women therewith allowed for the diversification of the western male population into power hierarchies by providing them all with a common `natural' and `essentially manly' site of mastery.

\begin{rmk}
    The forced sense of male heterosexual desire allows for the practice of homosocial bonding without the stigma of homosexual innuendo.
\end{rmk}





\section{Notes on Analysis and Societal Context}
\label{sec:SocCont8}


\begin{nte}
    The author argues that from origin to copy, teacher to student, generation to generation, the history of western man is made to cohere along classical values presented as rational, logical, and universalist.
\end{nte}

The author brings into question ``how logical and reassuring" is it? or more accurately, for whom is it such?

\begin{quotation}
    history-writing actively participates in generating ideology.
\end{quotation}

The author tracks the incredibly durable set of power relationships structured on gender difference and defined as sexual which are figured by the so-called \textbf{Venus Pudica}, the depiction of an idealized female nude who covers her \textbf{pubis} with her hand.


The author argues that we can rethink the conditions of traditional art history as the terms which testify to the advent and continuation of certain shared, culturally constructed exxpressions of power hierarchies.

\begin{nte}
    The author aims to reinstate to vision the political significance of this subject/pose/gesture in its endless permutations in western art; to denaturalize it and underscore its configuration as idealogical artiface.
\end{nte}

The author argues what is at stake in the Knidia is fundamental to our understanding of ourselves and our images of self as a sexual, deployed `other' through the conditioning of culture.


\section{Terms}
\label{sec:terms8}

\begin{enumerate}
	\item
\end{enumerate}

%
% \begin{acknowledgement}
% If you want to include acknowledgments of assistance and the like at the end of an individual chapter please use the \verb|acknowledgement| environment -- it will automatically render Springer's preferred layout.
% \end{acknowledgement}
%
% \section*{Appendix}
% \addcontentsline{toc}{section}{Appendix}
%


% Problems or Exercises should be sorted chapterwise
\section*{Problems}
\addcontentsline{toc}{section}{Problems}
%
% Use the following environment.
% Don't forget to label each problem;
% the label is needed for the solutions' environment
\begin{prob}
\label{prob1}
A given problem or Excercise is described here. The
problem is described here. The problem is described here.
\end{prob}

% \begin{prob}
% \label{prob2}
% \textbf{Problem Heading}\\
% (a) The first part of the problem is described here.\\
% (b) The second part of the problem is described here.
% \end{prob}

%%%%%%%%%%%%%%%%%%%%%%%% referenc.tex %%%%%%%%%%%%%%%%%%%%%%%%%%%%%%
% sample references
% %
% Use this file as a template for your own input.
%
%%%%%%%%%%%%%%%%%%%%%%%% Springer-Verlag %%%%%%%%%%%%%%%%%%%%%%%%%%
%
% BibTeX users please use
% \bibliographystyle{}
% \bibliography{}
%


% \begin{thebibliography}{99.}%
% and use \bibitem to create references.
%
% Use the following syntax and markup for your references if 
% the subject of your book is from the field 
% "Mathematics, Physics, Statistics, Computer Science"
%
% Contribution 
% \bibitem{science-contrib} Broy, M.: Software engineering --- from auxiliary to key technologies. In: Broy, M., Dener, E. (eds.) Software Pioneers, pp. 10-13. Springer, Heidelberg (2002)
% %
% Online Document

% \end{thebibliography}

