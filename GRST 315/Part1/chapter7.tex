%%%%%%%%%%%%%%%%%%%%% chapter.tex %%%%%%%%%%%%%%%%%%%%%%%%%%%%%%%%%
%
% sample chapter
%
% Use this file as a template for your own input.
%
%%%%%%%%%%%%%%%%%%%%%%$% Springer-Verlag %%%%%%%%%%%%%%%%%%%%%%%%%%
%\motto{Use the template \emph{chapter.tex} to style the various elements of your chapter content.}
\chapter{How to Kill an Amazon}
\label{howToAmaz} % Always give a unique label
% use \chaptermark{}
% to alter or adjust the chapter heading in the running head


%%% Questions to think about
%The \textbf{Thesis} or general sense of the article is ...

%The \textbf{method} the author uses to argue their point is ...

%In their \textbf{analysis} the author uses tools such as ...
% How do they look at the evidence? Do they place it in some theoretical framework? (i.e. gender studies, music studies, etc.)

%Additionally they conclude ...
% How does this compare to others throughout time? What is the societal context?

%What connections does the author portray with regard to \textbf{space}, \textbf{relationships}, \textbf{occupation}, and \textbf{religion}.


\abstract{}

\section{Questions and Remarks}
\label{sec:QR7}


\begin{qst}
    Who is Euphronius?
\end{qst}


\begin{qst}
    Where were the Amazon's located?
\end{qst}


\begin{qst}
    How is Hercules' related to the Amazons?
\end{qst}





\section{First Reading}
\label{sec:FirRead7}


Greek \textbf{Heracles} is not only the great civilizer, but also the most bestial of heroes. His exploits are fueled by a violent animal energy that gives him a special kinship with antisocial monsters and simultaneously enables him to conquer them.

\begin{nte}
    Heracles is prone to murderous rages, voracious gluttony for food and drink, and rampant sexual promiscuity towards women and men.
\end{nte}

\begin{nte}
    Heracles has two wives, \textbf{Deianeira} and \textbf{Megara}, as well as \textbf{Iole}, a princess he destroys a city to adbuct. \textbf{Euripides} shows him briefly as a loving husband and father, but in the tradition as a whole the principal function of \textbf{Megara} and her children is to highlight the horror of their deaths at Heracles' hands, while \textbf{Deianeira}'s most memorable appearance is as the agent of his own horrific demise as a result of his destructive passion for \textbf{Iole}.
\end{nte}

There is also \textbf{Omphale}, to whom he is temporarily enslaved in female costume---a gender reversal that serves to enhance his hypermasculinity.

\begin{rmk}
    In 1994 Heracles was reborn in the person of Kevin Sorbo, star of the television show \textbf{Hercules: The Legendary Journeys} (HLJ). The telemovies present Hercules as a monogamous, strictly heterosexual family man, utterly devoted to his wife and children.
\end{rmk}

In the course of Greek tradition, Heracles' inner need to perform his exploits came to replace the external compulsion of the gods. But in the telemovies he needs a woman's permission. To be sure, the hypermasculine Greek Heracles had been cleaned up long before the 1990s, in a tradition that had its beginnings in archaic Greece, became entrenched in Roman Stoicism, and swept on through the Middle Ages and Renaissance to the present day. Heracles became a figure of \textbf{perfect moral virtue}.

Heracles' popularity in modern American mass culture began in 1959 with \textbf{Pietro Francisci}'s movie Hercules. Greek Heracles' promiscuity is erased, with the only love interest appearing being \textbf{Iole}, while there is no mention of \textbf{Megara}, \textbf{Deianeira}, or \textbf{Omphale}.


In HLJ Hercules is portrayed as trying to use reason before resorting to muscle power. He repeatedly concerns himself with righting social wrongs, often through persuasion. Hercules was explicitly envisaged as a role model who would send the ``right message," especially to children. Thus the ferocious Greek Heracles reinvented as a model of \textbf{conventional, respectable, middle-class American values}, a bourgeois fantasy hero of late twentieth-century popular culture.

Nearly all the female figures whom the Greek Heracles had relationships do appear, but they are \textbf{radically} rewritten. In \textbf{Underworld} Deianeira becomes the great love of his life, while Iole is a seductress who is unable to seduce Heracles. Omphale appears in \textbf{Lost Kingdom} as a queen to whom Hercules voluntarily enslaves himself for just one day, in order to further his own schemes. (\textbf{no cross-dressing appears})


In the weekly series that picked up where the Action Pack movies left off, the same kind of revisionism was seen. At the beginning of the first season Deianeira and the children are killed by Hera, through no fault of Hercules' own, in order to enable him to follow a life of adventure without becoming a neglectful ``absent" husband and father. This allows him to elide the tension between heroic and domestic roles, which is dramatized by Euripides. Eliminating it HLJ allows Sorbo's Hercules to evade contemporary anxieties regarding ``fractured families."

In HJL he remains devoted to the dead Deianeira. Though not entirely celibate, he is exceedingly slow to seize the countless opportunities for sexual adventure thrust upon him by eager women. By nature he is portrayed as monogamous. In Season 3 he meets and marries his second wife, Serena, but before doing so visits Deianeira in the Underworld to obtain her blessing.

\begin{rmk}
    Most of Hercules' suitors, all of whom are female, are portrayed as sexual predators in pursuit of his virtue. The fifty virgin daughters of THespius with whom the greek Heracles had sex in a single night at their father's behest, became a flock of fifty numphomaniacs from whom Hercules flees in terror.
\end{rmk}

The most innocent mythological women like Iole, Arachne, and the Thestiads have become aggressors, often eroticized in ways that locate both their power and the threat they pose in their sexual desire. What of the ``real" female aggressors? The \textbf{Hydra}, the second of the traditional Twelve Labors, provides Hercules with his first adventure in \textbf{Hercules and the Amazon Women} (HAW). The ``\textbf{treacherous power of the feminine}": the Hydra lures in her victims by posing as a helpless and ``adorable" sad little girl, before metamorphising into the ``ultimate phallic woman".

\begin{rmk}
    Innocent females, including Heracles' own victims, are reimagined as threats to his monogamous domesticity, while the truly monstrous female lowers his guard by turning into the kind of innocent girl-child who would naturally touch the heart of a dovoted family man.
\end{rmk}

The agenda not only for the movie but the series as a whole is set: Hercules' struggle against evil as an engagement with the monstrous female. Even his male enemies are usually pawns of the goddess Hera. 

\begin{nte}
    Sorbo's Hercules is a direct descendant of the fifth-century BCE sophist \textbf{Prodicus}'s influential Heracles. When faced with a fork in the road, this Heracles chooses virtue and eschews pleasure. HLJ's Hercules shows similar self-restraint, and in HAW he warns \textbf{Iolaus} against recklessness in battle.
\end{nte}

This reinvention takes a form that reflects our culture's efforts to evade the disturbing implications of extraordinary human performance. For example, witness the ubiquitous but bizarre notion that sports ``heroes" should serve as exemplars of moral perfection, when everything about their acculturation would seem to predispose them, like Heracles, towards antisocial violence, and abundant evidence corroborates that expectation.

\begin{rmk}
    HLJ asks us to believe that the most physically powerful hero of them all is more likely to be a victim of sexual assault than a perpetrator.
\end{rmk}


The ninth of Heracles' traditional Twelve Labors is to steal the war-belt of the Amazon queen \textbf{Hippolyta}, a feat that not so subtly betokens both military and sexual conquest. In the process he slaughters numerous Amazons. This encounter, radically revised, is the central subject of \textbf{Hercules and the Amazon Women}.

\begin{nte}
    In defiance of the strictly gender ideology of ancient Athens, they are, at their core, ``a female people who fight."
\end{nte}

The earliest references call them ``equivalent to men," yet in the agonistic, zero-sum terms of Greek culture, such equivalence is tantamount to hostility.

\begin{rmk}
    In a vase by \textbf{Euphronius} We see Heracles and male warriors fighting Amazons. Although the Amazons are losing, they are positioned equal in size to Heracles and the men.
\end{rmk}

Amazons are \textbf{radically alien} in virtue of their rejection of conventional gender norms. The fighting equipment of Amazons is portrayed as ``barbarian" equipment of various kinds. After the Persian Wars the Amazons start to be identified with the Persians as the barbarian other. At all periods it is vital that they be shown as defeated or dying at the hands of heroic Greek males. 

\begin{nte}
    As a threat to the ``civilized" social order, the Amazons conquest came to be seen as part of Heracles' civilizing mission.
\end{nte}

The Amazons ``masculinity" and separatism are in turn made possible by a lack of any sexual need for the male beyond an occasional roll in the reproductive hay. In HLJ the aggressive Amazons undergo a transformation in which they are reimagined in much the same fashion as Heracles himself. The opening of the move situates Hercules as an antimarriage misogynist. By the ned his meeting with Hippolyta and her Amazons has transformed him into a fervent devotee of heterosexual romance.

In contrast to Greek representations, none of these modern Amazons is armed like a conventional male warrior. They are monkeylike, jungle-dwelling tomboys who swing down from trees to attack. ``Primitive" animal masks betoken both savagery and ``feminine" deviousness.

Hercules neither kills any Amazons on the way to the Amazon queen, nor the queen herself even when she has challenged him to finish her off. Hercules allows himself to be captured by the Amazons. After making him wash her feet, Hippolyta informs him that women want to be respected and explains that the roots of his misogyny lie in his childhood acculturation. Hercules picks up on this and asks ``what if I tried to change?", and he does. He is instantly transformed into a ``sensitive" late-twentieth-century middle class male.

\begin{rmk}
    Hippolyta's psychotherapeutic strategy and its success reflect the enormous influence of the self-help movement of the late-twentieth century.
\end{rmk}

Hercules, the therapeutic ``client," will learn more than she meant to teach, turning the tables to become a better psychotherapist than his teacher. Newly wise in the ways of women, he informs Hippolyta that her subjects ``feel an emptiness they cannot explain." Like their forebears in Greek art, the Amazons are svelte, beautiful, and eroticized for the presumptively male gaze. In the end the militantly ``feminist" leader Hippolyta is seduced away from her feminist separatism by the newly sensitive Hercules. This is configured as courage on Hippolyta's part for resisting Hera, who was enslaving the Amazons' wills and teaching them to hate all men.

\begin{rmk}
    Hecules plays the role of those late-twentieth-century counselors who took it upon themselves to ``cure" women of the ``disease" of independence from men and the ``infection" of feminism.
\end{rmk}

Like late-twntieth-century American career women, the Amazons are assured that they will be really happy only if they give up their independent lives and devote themselves to maternity and ``nurturance." The means by which Hercules achieves this transformation include not only an antifeminist discourse of personal fulfillment, but also, a related feminist rhetoric of self-determinationn.

\begin{rmk}
    The Amazons' embrace of conventional heterosocial domesticity is represented as a newfound ``freedom".
\end{rmk}

The movie tells women that ``they must choose between a womanly existence and an independent one...if they [give] up the unnatural struggle for self-determination, they [can] regain their natural femininity."

\begin{rmk}
    The rhetoric of the natural became more loaded than ever in the 1990s, as biological determinism, especially regarding sex roles, swept triumphantly through the popular media in a trend that shows no sign of abating (link to \textbf{Dean-Jones and Flemming}).
\end{rmk}

Far from being feminist in its essence, the kind of ``respect" that Herculs advocates is the restoration of an antiquated romantic ideal, whereby such gestures as flowers and love songs are the price men pay for female domestic labor. The men will persist in their romantic gestures, but these serve to underline the traditional hierarchical division of labor, not to challenge it. The monster of \textbf{female separatism} is contained by our hero.

\begin{nte}
    As different as he is from the early Greek Heracles, Kevin Sorbo's Hercules too is an agent of ``civilization," as defined by the patriarchal ideology of his own time.
\end{nte}

The Greek Amazons are domesticated in HLJ much as Heracles is, by the use of therapeutic discourses characteristic of the late-twentieth century to incorporate them into a bourgeois, pseudo-enlightened model of the household, an incorporation that only echoes Heracles' own domestication but enacts it under his therapeutic regime. The result is a taming of the Amazons more insidious than the unapologetic Greek representations of slaughter.

\begin{rmk}
    Despite their physical prowess and self-sufficiency, waht proves effective in the end is the passive female power exercised in Aristophanes' \textbf{Lysistrata}.
\end{rmk}

In the telemovie the rejection of domesticity is fueled by a desire for a better domesticity. 

\begin{nte}
    The Greek Amazons exist in order to define the ``civilized" normative male (and female) through the reversal of culturally coded gender expectations; the televisual Amazons perform the same function by embracing those norms.
\end{nte}

Female power is shown to serve, in the end, only the interests of Bourgeois domesticity with which female interests are held to be identical. And the therapeutic discourse that has been embraced predominantly by women, as a means of improving both themselves and their men, is used to reinscribe the gender roles that many women, in their quest for ``self-fulfillment," have attempted to escape.

This outcome may be designed to appeal to one type of heterosexual male viewer, allowing him to fantasize that despite his own evident limitations, women of the caliber of the Amazons are available to him as mates, if he just adds a dash of ``sensitivity." Kevin Sorbo's Hercules is a kinder gentler action hero, intended to appeal to women as well as men. This can be seen in the products used to advertise the initial showing of \textbf{Underworld} (November 1994). In other words, the antifeminist ideology of HAW, along with its pseudo-feminist veil, was directed not just at men but at women, who were presumably expected to embrace this assertion of their ``essential" needs as women.

Both the Greek myth of the Amazons and the treatment of the Amazons in HAW serve to obfuscate and naturalize the tensions and contradictions inherent in the institution of marriage, but the ways in which they do so are tailored to their cultural contexts.

\begin{rmk}
    Classical Greek culture so dreaded the prospect of female power and the threat it presents to such structures that it kept its women in an explicitly subordinate position based on their supposed inferiority to and need for men.
\end{rmk}

In the ideology of contemporary popular culture the patriarchal family's ability to reproduce itself is likewise threatened by female autonomy and power. In contrast to classical Greece, howerver, modern American popular culture purports to respect the powerful female and to give a voice to ``feminine" needs even within the bourgeois family. The ``mythic" strategy for neutralizing the Amazons adopted by HAW therefore diverges somewhat from its Greek sources.

For the Greeks the Amazons' ``masculine" independence is itself the problem, and must be erased in the interests of ``civilization." A culture that purports to respect female independence must maintain that while women do indeed have the capacity for autonomy, this is not what they really wantm since it militates against their ``natural" domesticity.

\begin{rmk}
    For the women, it is death if they are Greek Amazons; but if they are modern Americans, it is a denial of their self-fulfillment. It is for this reason that is is unnecessary for our hero actually to kill any of these Amazons. As far as HAW is concerned, the Greek Amazons are already dead.
\end{rmk}

The ``preferred" ideology of HAW is clearly patriarchal, flavored with just enough lip service to women's ``needs" to veil its coercive definition of those needs.  

Hercules is often sexually objectified throughout the series, often explicitly through women's eyes. In HAW he is bound, caged, and humiliated for the satisfaction of the female (Amazon) gaze.

In the final moments of the movie Zeus reverses time (so that various corpses will be restored to life), andn the whole story is undone and started over again. This time Hercules gives the messenger fro mGargarencia the Horn with which Hippolyta controls her Amazons. In doing so it picks up on a theme from the beginning of the movie, where a good deal is made of the fact that Hercules carries no weapons.

The Greek Heracles' signature weapons, the bow and the club, symbolize opposed modes of heroism, since the club requires brutal violence at close quarters, while the more sophisticated bow, which allows one to kill from a distance, betokens caution and ingenuity, if not actual cowardice. The television Hercules' masculine completeness is demosntrated by his lack of weapons, his ability to rely on the self-sufficiency of the male body.


When Hercules resotres Hippolyta's ``horn" to the Gargarencia men at the movie's close, what viewer can doubt that these women will reappropriate it from these men in no time flat?


\section{Notes on Analysis and Societal Context}
\label{sec:SocCont7}

\begin{rmk}
    The author argues that depictions in telemovies such as HLJ killed the Amazons of classical Greek myth, domesticating them. On the other hand, \textbf{Xena: Warrior Princess} telemovies brought the Amazons back to life.
\end{rmk}




\section{Terms}
\label{sec:terms7}

\begin{enumerate}
	\item
\end{enumerate}

%
% \begin{acknowledgement}
% If you want to include acknowledgments of assistance and the like at the end of an individual chapter please use the \verb|acknowledgement| environment -- it will automatically render Springer's preferred layout.
% \end{acknowledgement}
%
% \section*{Appendix}
% \addcontentsline{toc}{section}{Appendix}
%


% Problems or Exercises should be sorted chapterwise
\section*{Problems}
\addcontentsline{toc}{section}{Problems}
%
% Use the following environment.
% Don't forget to label each problem;
% the label is needed for the solutions' environment
\begin{prob}
\label{prob1}
A given problem or Excercise is described here. The
problem is described here. The problem is described here.
\end{prob}

% \begin{prob}
% \label{prob2}
% \textbf{Problem Heading}\\
% (a) The first part of the problem is described here.\\
% (b) The second part of the problem is described here.
% \end{prob}

%%%%%%%%%%%%%%%%%%%%%%%% referenc.tex %%%%%%%%%%%%%%%%%%%%%%%%%%%%%%
% sample references
% %
% Use this file as a template for your own input.
%
%%%%%%%%%%%%%%%%%%%%%%%% Springer-Verlag %%%%%%%%%%%%%%%%%%%%%%%%%%
%
% BibTeX users please use
% \bibliographystyle{}
% \bibliography{}
%


% \begin{thebibliography}{99.}%
% and use \bibitem to create references.
%
% Use the following syntax and markup for your references if 
% the subject of your book is from the field 
% "Mathematics, Physics, Statistics, Computer Science"
%
% Contribution 
% \bibitem{science-contrib} Broy, M.: Software engineering --- from auxiliary to key technologies. In: Broy, M., Dener, E. (eds.) Software Pioneers, pp. 10-13. Springer, Heidelberg (2002)
% %
% Online Document

% \end{thebibliography}

