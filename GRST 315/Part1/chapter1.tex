%%%%%%%%%%%%%%%%%%%%% chapter.tex %%%%%%%%%%%%%%%%%%%%%%%%%%%%%%%%%
%
% sample chapter
%
% Use this file as a template for your own input.
%
%%%%%%%%%%%%%%%%%%%%%%$% Springer-Verlag %%%%%%%%%%%%%%%%%%%%%%%%%%
%\motto{Use the template \emph{chapter.tex} to style the various elements of your chapter content.}
\chapter{Sappho's Private World}
\label{SapphoPriv} % Always give a unique label
% use \chaptermark{}
% to alter or adjust the chapter heading in the running head


%%% Questions to think about
%The \textbf{Thesis} or general sense of the article is ...

%The \textbf{method} the author uses to argue their point is ...

%In their \textbf{analysis} the author uses tools such as ... 

%Additionally they conclude ...

%What connections does the author portray with regard to \textbf{space}, \textbf{relationships}, \textbf{occupation}, and \textbf{religion}.


\abstract{The author of this article seeks to portray the unique and powerful qualities of Sappho's poetry and compare and contrast it with the male lyric poets of the time. To do this the author analyzes a number of poems and poem fragments from male lyric poets and Sappho. In particular, the focus throughout is on the theme of love in the lyric poems, and how this theme is portrayed by Sappho versus male lyric poets. The author contrasts Sappho's depiction and relation with figures such as Aphrodite and Eros to that of the male lyric poets. Sappho is often on ``good terms" with the depiction of Aphrodite or Eros, while being helpless in the face of the ``love object," opposite to the perspective of the male lyric poets. An important conclusion the author draws is the relation Sappho's poetry has to \textbf{space}, and its contrast with male poets. Sappho's poems often presuppose a ``private space", containing the two women. This relies on the \textbf{homesexual relationship} - the fact that two women can mirror each other. The dynamic of mutual erotic attraction becomes an invisible bond, or impenetrable enclosure. It is a metaphor for \textbf{emotional openness}. In some poems this space is physically realized, but in all poems it is the poem itself.}

\section{Questions and Remarks}
\label{sec:QR1}

\begin{rmk}
    Throughout the article the author mentions the role of biology plays in the nature of the lesbian relationship and placement of women. How much of this is based in evidence based fact versus social bias? Is this a location bias? Is it due to definitions and societal standards of the time? (the article was written in 1981).
\end{rmk}

\section{First Reading}
\label{sec:FirRead1}

\begin{rmk}
    Sappho was a lyric poet who wrote love poetry in Greek between about 700 and 500 BC. 
\end{rmk}

The other refers to the unique quality of Sappho's poetry her \textbf{romanticism}.

\begin{defn}
    By \textbf{romanticism} the author seeks to evoke literary associations, essentially a yearning for escape from the isolation of the self and affirmation of the yearning in the face of knowledge that escape is impossible. It is a self-conscious easthetic attitude that paradoxically distances the individual from that into which they would lose themself.
\end{defn}

\begin{rmk}
    Hymns, poems of praise and blame, re-tellings of myth with culturally normative motives, and love poems are all traditional types found in Sappho's work.
\end{rmk}

The theme of love in Sappho's work showed a distinct structure of narrative from that of any other lyric poetry. Sappho faced the problem of presenting the female persona as an erotic subject. Sappho's solution to direct the erotic impulse toward other women was in fact a traditional one.

Sappho is fundamentally different from the male lyric poets because she explores what a woman might desire and might offer erotically and how these interact.

\begin{nte}
    The love poems of male lyric poets see a pattern. The man is helpless, prostrate, stricken by the power of Eros (concept of sensual or passionate love in greek philosophy) or Aphrodite, but toward the particular boy or girl who attracts him the man is confident and prepared to seduce.

    Note Eros or Aphrodite is the universal, eternal sexual longing which can never be mastered, while the individual provoking it is only a temporary focus of the longing, the prey or prize which loses its allure once the man has captured it.
\end{nte}

A short poem of Ibycus, a poet about sixty years after Sappho, illustrates this:

\begin{quotation}
    Eros again looking at me meltingly with his eyes under dark eylids, flips me with manifold charms into the inescapable nets of Aphrodite. In truth I termble at his coming as a yoke-bearing, prize-winning horse, nearing old age, unwillingly goes with his quick chariot into the fray.
\end{quotation}

\textbf{Analysis:}
\begin{enumerate}
    \item ``Eros has driven the narrator into the nets of Aphrodite"; he is like a trapped and helpless wild animal.
    \item Towards the object of the narrators love; he is an old prize-winning horse who returns again to the contest. Now he is active, competitive, and the boy will be his prize. The ``prize-winning" in the epithet hints that the narrator has won the individual boy before, yet this has previous victory has not secured him respite from the power of Aphrodite.
    \item The boy's eyes are treated as the momentary location of Eros, but when the narrator has won this boy for himself, Eros will laughingly skip off elsewhere.
\end{enumerate}

Anacreaon too though that the contest was with Eros, stating \begin{quotation}
    The dice of Eros are madnesses and uproars.
\end{quotation}
Several fragments of Archilochus proclaim that the narrator is being overwhelmed by desire, for instance: \begin{quotation}
    Miserable I lie wrapped in longing, soulless, pierced through the bones by harsh griefs from the gods.
\end{quotation}

In a longer fragment, the \textbf{Cologne Epode}, Archilochus describes the seduction of a virgin girl. The narrator acts, despite his admitted haste, with a graceful, gentle masterfulness that bespeaks control, experience, self-assurance. THe implied impotence of the first fragment is not imagined as impeding action where opportunity presents itself.

We see the same pattern fro mAnacreon. In one fragment the narrator complains that Eros like a bronze-smith again hit him with a hammer and dipped him in a wintry river. But elsewhere he has the speaker boast to a skittish girl: 

\begin{quotation}
    Thracian filly, why do you glance at me askance and flee pitilessly? Do you think I have no art? Know, then, neatly could I throw on the bridle and holding the reins steer you around the course. At the moment you pasture in meadows and play lightly prancing, for you do not have an adroit experienced rider.
\end{quotation}

\begin{nte}
    It is essential to the male lyric poet that the object of his passion be vulnerable to seduction, but unseduced.
\end{nte}


Anacreon gives expression to the allure of innocence in a brief poem:

\begin{quotation}
    Oh child, virgin glancing, I seek you, but you do not hear, not knowing that you are the charioteer of my soul.
\end{quotation}

\begin{rmk}
    We see a pattern of longing for the ver uncapturable essence of Eros and excitement at discovering its momentary embodiment in a vulnerable, innocent figure, is the poetic rhythm of the male lyric poets.
\end{rmk}

This pattern Sappho could not use. Had Sappho portrayed herself as an active seeker after the virginity of a succession of girls, even, she might have presented a figure too close to that of a lusty woman for cultural acceptance or aesthetic appreciation among her contemporaries. 

\begin{nte}
    The poetic reason for the inappropriateness of the male pattern to Sappho is that the implicit metaphors of recurrent prostration, domination, and release are based on male sexual psychology, the man's sense of his action in sexual encounter.
\end{nte}

Sappho had to find patters that allow her to express romantic longing, fulfillment, and struggle with the mystery of sexuality, with truth to her emotional and bodily sense of them. The only definitely complete poem we have of Sappho goes as thus:

\begin{quotation}
    Richly-throned immortal Aphrodite, daughter of Zeus, weaver of wiles, I pray to you: break not my spirit, Lady, with heartache or anguish;

    But hither come, if ever in the past you heard my cry from afar, and marked it, and came, leaving your father's house,

    Your golden chariot yoked: sparrows beautiful and swift conveyed you, with rapid wings a-flutter, above the dark earth from heaven through the mid-air;

    And soon they were come, and you, Fortunate, with a smile on your immortal face, asked what ails me now, and why I am calling now,

    And what in my heart's madness I most desire to have: Whom now must I persuade to join your friendship's ranks? Who wrong you, Sappho?

    For if she flees, she shall soon pursue; and if she recieves not gifts, yet shall she give; and if she love not, she shall son love even against her will.

    Come to me now also, and deliver me from cruel anxieties; fulfil all that my heart desires to fulfil, and be yourself my comrade-in-arms.
\end{quotation}

This is the only poem of Sappho's in which the narrator expresses an adversary relationship with the love-object. The narrator is on good terms with Aphrodite but helpless in the face of the love-object, the opposite of the male pattern. 

\begin{nte}
    Sappho does not portray herself as a woman skilled in seduction, nor does she claim the potential to master the other or to ``win." On ther other hand, Aphrodite is not the capricious, impersonal force that she and Eros are for the male poets. Far from trembling at her approach, Sappho calls her for help. Aphrodite here is a cosmic affirmation of Sappho's own eroticism, the source of terrible pain but also of loveliness and joy, of contact with the divine, of heightened self-awareness, as the vivid sensuousness of the poem bears witness.
\end{nte}

Though Aphrodite may be her ally Sappho does not ask her to make the other girl submit, and Aphrodite offers only to have the other girl suffer too. Connection with Aphrodite grants Sappho a way of making manifest her own eroticism, which will perhaps, via the epiphany itself, draw the other girl closer to her. Only in this fashion can the other girl be won; the other girl's response to Sappho must be spontaneous. Thus unlike the innocent beloved of the male poets, the other girl's envisioned role here is to turn to Sappho out of her own longing. She must come independently to want Sappho before either woman can find intimacy satisfying.

\begin{rmk}
    The two women must be equals, each understanding the other from insight into herself.
\end{rmk}

In a fragment Sappho says:\begin{quotation}
    I loved you, atthis, long ago. . . . You seemed to me to be a small and graceless child.
\end{quotation}
The name Atthis recurs in other of Sappho's poems as a companion, so one can imagine the narrator telling atthis about her previous attraction only after the two have become intimate, when Atthis is in a position to appreciate it.

Another fragment of a poem of Sappho reads:

\begin{quotation}
    Honestly, I wish I were dead. Weeping she left me With many tears, and said ``Oh what unhappiness is ours; Sappho, I vow, against my will I leave you."

    And this answer I made to her: `Go, and fare well, and remember me; you know how we cared for you.

    If not, yet I would remind you ... of our past happiness. Many wreaths of violets and roses and ... you put around you at my side,

    And many woven garlands, fashioned of flowers, ... round your soft neck,

    And... with perfume of flowers, fit for a queen, you anointed ...
    
    And on soft beds ... you would satisfy your longing ... And no ... holy, no ... was there, from which we were away.
\end{quotation}


\textbf{Analysis:}
\begin{enumerate}
    \item Sappho recalls a whole range of shared experience, including but not limited to the erotic.
    \item In Archilochus' Cologne Epode the girl presses for a verbal statement, perhaps an offer of marraige, rather than sexual contact, and the seducer must cut off the conversation in order to further the seduction. Sexual intimacy and verbal understanding inhibit one another. In Sappho's poem the conversation is a continuation and confirmation of erotic intimacy, an attempt to perpetuate it. 
    \item In Archilochus the conversation is manipulative, she negotiating, he trying to disarm her. In Sappho the persona also  takes over the dialogue, but uses it to banish impending separation between herself and the other woman. Sappho's method of recreating the intimacy verbally to the girl whom she comforts is to reflect the girl's past happiness back to her. Thus Sappho dramatizes her absorption with the other woman, the lapse of her separate self-consciousness as she is cought up in the other's sensuousness. The tension is Sappho's poem is between the friends and the outside forces that are requiring them to separate.
\end{enumerate}

Sappho's poem resupposes a protected place containing the two women in perfect understanding.

\begin{rmk}
    Sappho's poetic problem was to find a pattern consistent with female experience of love within which to express her romantic sensibility. The pattern of love in Sappho's poetry, of mutuality rather than domination and subjection, of intimacy based on comprehending the other out of the self, is the ideal characteristic of lesbian love.
\end{rmk}

Simone de Beauvoir confirms a felt identity of two lovers which erases the distinction between ``self" and ``other," at least ideally, so that Sappho's concentration on the other womean can be seen as a poetic equivalent of erotic fulfillment.

\begin{nte}
    Sappho's method to create a romantic posture was to pick three aspects of love to dramatize. \begin{enumerate}
        \item the appeal to Aphrodite, who displaces the desired girl in Sappho's attention in the first quotation. Through this Sappho claims that erotic desire, if cherished above release or calm, opens a path to divinity and absolute beauty, that through intensity of longing comes transcendence.

            For the erotic impulse as Sappho projected it is less a matter of loving another individual than of finding in love a form of intensification and grandeur which must be ever renewed.
        \item the loss of the beloved by parting. Poems of longing for one absent imply that bliss would come with the numinous presence of that individual. Sappho uses the moment of parting as the frame for the picture of intimacy, for intimacy seems most previous, union most complete in the face of imminent loneliness. Sappho here romanticizes what is in fact the dominant experience of women in love-making and in child-birth---intimacy followed by withdrawal.
        \item Sappho shapes her pattern to romanticism is in the creation of a private world. In this she depends specifically on the fact that two women can mirror one another. For to come together each woman must spontaneously wish to be close to the other; the act of love requires communication between the two, since it has no other outward manifestation. The dynamic of mutual erotic attraction, the interplay between two women, becomes an invisible bond, or in Sappho' formulation a single enclosure, impenetrable by others, in which the two are so open to another that they feel united. It is the poetic equivalent of Simone de Beauvoir's phrase, ``duality becomes mutuality."

            The beautiful robes and ornaments and flowers with which Sappho decorates her poems are the furnishings of this poetically-created private space. The private space exists, in fact, only in the poem.
    \end{enumerate}
\end{nte}

The existence of a private space created by the poem, counter-balances the focus on loss in Sappho's poetry. The continuation of the private space is asserted in the face of the loss of the loved woman.

\begin{rmk}
    The private space is the most important metaphor for love in the poetry of Sappho. Itself based on a contradiction, that the inward can become outward, that solitude can be replaced by perfect intimacy, the claim of a private world can mediate the contradictions of Sappho's romanticism.

    Powerful erotic drive can coexist with a biological role as non-aggressor, Sappho's self-involvement can coexist with intimacy with another, and lament for the loss of another can coexist with the certainty that the unity of the two still exists.
\end{rmk}

In conclusion, \textbf{Sappho's romanticism devoles on the creation of a poetic metaphor that both affirms and transcends the inward, self-contained nature of woman's love}.


%
% \begin{acknowledgement}
% If you want to include acknowledgments of assistance and the like at the end of an individual chapter please use the \verb|acknowledgement| environment -- it will automatically render Springer's preferred layout.
% \end{acknowledgement}
%
% \section*{Appendix}
% \addcontentsline{toc}{section}{Appendix}
%


% Problems or Exercises should be sorted chapterwise
\section*{Problems}
\addcontentsline{toc}{section}{Problems}
%
% Use the following environment.
% Don't forget to label each problem;
% the label is needed for the solutions' environment
\begin{prob}
\label{prob1}
A given problem or Excercise is described here. The
problem is described here. The problem is described here.
\end{prob}

% \begin{prob}
% \label{prob2}
% \textbf{Problem Heading}\\
% (a) The first part of the problem is described here.\\
% (b) The second part of the problem is described here.
% \end{prob}

%%%%%%%%%%%%%%%%%%%%%%%% referenc.tex %%%%%%%%%%%%%%%%%%%%%%%%%%%%%%
% sample references
% %
% Use this file as a template for your own input.
%
%%%%%%%%%%%%%%%%%%%%%%%% Springer-Verlag %%%%%%%%%%%%%%%%%%%%%%%%%%
%
% BibTeX users please use
% \bibliographystyle{}
% \bibliography{}
%


% \begin{thebibliography}{99.}%
% and use \bibitem to create references.
%
% Use the following syntax and markup for your references if 
% the subject of your book is from the field 
% "Mathematics, Physics, Statistics, Computer Science"
%
% Contribution 
% \bibitem{science-contrib} Broy, M.: Software engineering --- from auxiliary to key technologies. In: Broy, M., Dener, E. (eds.) Software Pioneers, pp. 10-13. Springer, Heidelberg (2002)
% %
% Online Document

% \end{thebibliography}

