%%%%%%%%%%%%%%%%%%%%% chapter.tex %%%%%%%%%%%%%%%%%%%%%%%%%%%%%%%%%
%
% sample chapter
%
% Use this file as a template for your own input.
%
%%%%%%%%%%%%%%%%%%%%%%$% Springer-Verlag %%%%%%%%%%%%%%%%%%%%%%%%%%
%\motto{Use the template \emph{chapter.tex} to style the various elements of your chapter content.}
\chapter{The Horror of the Terrifying and the Hilarity of the Grotesque: Daimonic Spaces and Emotions in Ancient Greek Literature}
\label{DaimonicSpaces} % Always give a unique label
% use \chaptermark{}
% to alter or adjust the chapter heading in the running head


%%% Questions to think about
%The \textbf{Thesis} or general sense of the article is ...

%The \textbf{method} the author uses to argue their point is ...

%In their \textbf{analysis} the author uses tools such as ...
% How do they look at the evidence? Do they place it in some theoretical framework? (i.e. gender studies, music studies, etc.)

%Additionally they conclude ...
% How does this compare to others throughout time? What is the societal context?

%What connections does the author portray with regard to \textbf{space}, \textbf{relationships}, \textbf{occupation}, and \textbf{religion}.


\abstract{}

\section{Questions and Remarks}
\label{sec:QR13}

\begin{qst}
    Who/what were the \textbf{Daimones}?
\end{qst}

\begin{qst}
    Who/what was the \textbf{Apollonius of Tyana}?
\end{qst}





\section{First Reading}
\label{sec:FirRead13}


\begin{defn}
    \textbf{Gorgoneion}: a Gorgon mask
\end{defn}

Vernant says the Gorgon's mask ``expresses and maintains the radical otherness, the alterity of the world of the dead, which no living person may approach." This otherness comprises a network of ideas that associate the realm of the dead and night with some particular qualities of the female and monstrosity.

The Gorgon is an offspring of \textbf{Phorkys} and \textbf{Keto}; she and her siblings play the role of ``watchmen, even bogeys, who bar the way to forbidden places."

\begin{defn}
    The term \textbf{daimones} is used to refer to the dead, and it may also capture these creatures' close relationship even identification with, the goddess \textbf{Hekate}.
\end{defn}


\subsection{Case Study One: Daimones and Comedy}

In both comedy and tragedy we have cases of asking wearing masks called \textbf{mormolykeia} (``mormo-goblins"), and these were hung around Dionysus's precinct, either as dedications or as part of the announcement of new productions. Their apperance was fashioned to create the emotion of \textbf{kataplexis}, a term that evokes not just fear but a horrified almost frozen fixation on a dreadful vision.

\begin{rmk}
    A number of sources indicate that Mormo and her fellow creatures were the subjects of stories told by parents to frighten their children into behaving.
\end{rmk}

The scholiast to Aristides reports that Mormo was a Corinthian woman who purposefully ate her own children and then flew away. The tale of Mormo is about the radical transformation of mother to monster: her changing face not only symbolizes this switch it also evokes the very moment of horror itself, perhaps especially from the viewpoint of a child hearing the sory of another child's fate. (p.216)


The \textbf{Lamia} has similar associations.

\begin{nte}
    The Lamia has a variety of meanings and alludes to a range of different mythic figures.
\end{nte}

The term \textbf{lamiai} was use not only of phantoms, but also of gluttonous people, and of fish (p.216). Lamia could refer both to monsters and an ancient Libyan woman of this name. The Libyan Lamia is the one associated with the loss and killing of children. Lamia was a beautiful queen in Libya. Zeus's interest in her makes Hera angry, and she destroys all the queen's children. Lamia's grief deforms her; she seizes the children of other women and either tears them apart or eats them.

\textbf{Diodorus Siculus} describes Lamia as driven to madness by the loss of her own beauty.

In comedy Lamia's uncivilised characteristics were the focus. Aristophanes suggests that Lamia had testicles, while Krates' allusion to her possession of a \textbf{skytale} may indicate that she was imagined as having a penis.

\begin{rmk}
    In contrast, \textbf{Empousa} does not seem to be linked to a story of mothers killing children. Rather, we find an association with Hekate and with more obscure ritual activities.
\end{rmk}

The Empousa is defined as ``A daimonic ghost sent by Hekate and appearing to the ill-fated. [Something] which seems to change into many forms." (p.218) Empousa is connected to rituals of initiation into the mysteries. She is also described as having a single leg, which is either of bronze, an ass' leg, or made of donkey excrement.

\begin{nte}
    In the \textbf{Frogs}, Empousa is described both as lovely and as having a face blazing with fire. (p.219)
\end{nte}


\subsubsection{``Uncivilized Space and Time"}

All of Empousa, Lamia, and Mormo inhabit spaces outside the civic sphere, spaces characterized by night, darkness, and the wild. The disturbing appearances of these creatures is linked to morally unacceptable behaviour. (p.220)

\begin{rmk}
    These monstrous names were mor usually found attached to a very different social group: \textbf{hetairai}. Sources indicate that hetairai might well have adopted such names for themselves in the real world.
\end{rmk}

The transgressive quality of the monster was not, or not always, straightforwardly concerned with a horrific appearance. Empousa was considered rampant, and her lust was not simply titillating but it also crossed behavioural boundaries and transgressed social norms. It is in this crossing of boundaries that audiences could experience hilarity in horror and horror in their hilarity (p. 221-222)


\subsection{Case Study Two: Daimones in Disguise: Apollonius of Tyana}

The biography of Apollonius of Tyana is a confection, written 120 years after the death of Apollonius (98 CE) in order to satisfy the empress Julia Domna by presenting Apollonius as a philosophical sage like Pythagoras and Empedokles.

The empousa are characterized by the marginal space at the end of the world and, is itself, also used to mark it.

\begin{rmk}
    \textbf{Corinth} has a particular link to these creatures.
\end{rmk}


\subsubsection{Changing Spaces}

The darkness in Aristophanes' work is not only spatial but also social: the creatures inhabit or are used to characterize those who are somehow marginal of uncivilized. (p. 225)

\begin{rmk}
    In the \textbf{Frogs}, the presence of Empousa reinforces the alterity and horror of the underworld; in the Life of Apollonius, the appearance of an empousa on the road at night indicates how far the travellers are from civilization and how dangerous is their quest, while implying the initiatory nature of this part of their journey. 
\end{rmk}

These daimones were used to provide a consistent commentary on the transgressive and dangerous sexuality of women. (p. 226)

In the classical period these creatures emphasise the ``otherness" of women, especially those whose focus is sexual desire. (p.227)

\begin{nte}
    The classical evidence puts particular emphasis on the role of children as the targets of these monsters.
\end{nte}


\subsection{A Shift in Perception}

Gasparro argues that daimonology was a more or less homogenous and articulated set of ideas and beliefs, sometimes associated with ritual practice, relating to the category of the divine with the Greeks from the time of Homer. The daimon began to offer a way of thinking about or expressing the turbulence of the Hellenistic age. The term daimon could be used to represent a graduated conception of the divine, which both distanced divinity, but also mediated between mand and the divine. (p.228)


\subsection{Boundaries: The Uncanny and the Abject...}

These creatures were in part horrific because they revealed the presence of ``the other" emergin from the familiar everyday world. The instantiaition of these figures through time and genre reveals a repeated cultural identifiaction of what is ``other" and its rejection. (p. 230)


\begin{defn}
    The \textbf{abject} is ``a sudden and massive emergence of uncanniness"; abjactness is elaborated through a failure to recognize its kin; nothig is familar, not even the shadow of a memory.
\end{defn}

\begin{rmk}
    The repeated depiction of these daimones and their sexual appetites can be described as a cultural abjection of female sexuality.
\end{rmk}









\section{Notes on Analysis and Societal Context}
\label{sec:SocCont13}

The author focuses on the features of childhood terror of the \textbf{Empousa}, the \textbf{Gello}, and the \textbf{Mormo}. Their appearance is in primarly oral rather than literary, accounts.

The author analyzes these creatures using the ``space" they inhabit. The author argues that there is a significance in the spaces that these creatures were perceived to inhabit, which supports their identification as daimones. This also involves an examination of how physical space relates to social space and temporal space, including times of day or night and the human lifecycle.

\begin{rmk}
    These creatures were themselves ``spaces" of meaning, available for claims and counter-claims about the natures of the unnatural and the supernatural, and man's relationship to them.
\end{rmk}

The author argues this implies that the physical embodiment of each daimon is significant. The author also explores the conceptions of cultural space in or from which these creatures emerged.

The author aims to examine those representations for their intention to provoke the emotions of eithe rand/or both hilarity and horror and reflects on the question of the relationship between the two. (p.213)


\section{Terms}
\label{sec:terms13}

\begin{enumerate}
	\item
\end{enumerate}

%
% \begin{acknowledgement}
% If you want to include acknowledgments of assistance and the like at the end of an individual chapter please use the \verb|acknowledgement| environment -- it will automatically render Springer's preferred layout.
% \end{acknowledgement}
%
% \section*{Appendix}
% \addcontentsline{toc}{section}{Appendix}
%


% Problems or Exercises should be sorted chapterwise
\section*{Problems}
\addcontentsline{toc}{section}{Problems}
%
% Use the following environment.
% Don't forget to label each problem;
% the label is needed for the solutions' environment
\begin{prob}
\label{prob1}
A given problem or Excercise is described here. The
problem is described here. The problem is described here.
\end{prob}

% \begin{prob}
% \label{prob2}
% \textbf{Problem Heading}\\
% (a) The first part of the problem is described here.\\
% (b) The second part of the problem is described here.
% \end{prob}

%%%%%%%%%%%%%%%%%%%%%%%% referenc.tex %%%%%%%%%%%%%%%%%%%%%%%%%%%%%%
% sample references
% %
% Use this file as a template for your own input.
%
%%%%%%%%%%%%%%%%%%%%%%%% Springer-Verlag %%%%%%%%%%%%%%%%%%%%%%%%%%
%
% BibTeX users please use
% \bibliographystyle{}
% \bibliography{}
%


% \begin{thebibliography}{99.}%
% and use \bibitem to create references.
%
% Use the following syntax and markup for your references if 
% the subject of your book is from the field 
% "Mathematics, Physics, Statistics, Computer Science"
%
% Contribution 
% \bibitem{science-contrib} Broy, M.: Software engineering --- from auxiliary to key technologies. In: Broy, M., Dener, E. (eds.) Software Pioneers, pp. 10-13. Springer, Heidelberg (2002)
% %
% Online Document

% \end{thebibliography}

