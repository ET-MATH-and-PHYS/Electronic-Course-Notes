%%%%%%%%%%%%%%%%%%%%% chapter.tex %%%%%%%%%%%%%%%%%%%%%%%%%%%%%%%%%
%
% sample chapter
%
% Use this file as a template for your own input.
%
%%%%%%%%%%%%%%%%%%%%%%$% Springer-Verlag %%%%%%%%%%%%%%%%%%%%%%%%%%
%\motto{Use the template \emph{chapter.tex} to style the various elements of your chapter content.}
\chapter{Revisiting Roman Sexuality}
\label{RevRomSex} % Always give a unique label
% use \chaptermark{}
% to alter or adjust the chapter heading in the running head


%%% Questions to think about
%The \textbf{Thesis} or general sense of the article is ...

%The \textbf{method} the author uses to argue their point is ...

%In their \textbf{analysis} the author uses tools such as ...
% How do they look at the evidence? Do they place it in some theoretical framework? (i.e. gender studies, music studies, etc.)

%Additionally they conclude ...
% How does this compare to others throughout time? What is the societal context?

%What connections does the author portray with regard to \textbf{space}, \textbf{relationships}, \textbf{occupation}, and \textbf{religion}.


\abstract{}

\section{Questions and Remarks}
\label{sec:QR4}

\begin{qst}
    What is the penetration model?
\end{qst}
The penetration model is a model of ancient Roman society which places partaking in sexual acts into two categories: either penetrating or penetrated.


\begin{qst}
    What is the \textbf{irrumatus}?
\end{qst}


\begin{qst}
    What is the \textbf{pedicatus/fututus}?
\end{qst}


\begin{qst}
    What is the \textbf{fellator}?
\end{qst}


\begin{qst}
    What is the \textbf{cinaedus/pathicus}?
\end{qst}




\section{First Reading}
\label{sec:FirRead4}


\begin{rmk}
    To the Romans, penetrating was associated with freeborn status, masculinity, and social dominance, whereas being penetrated was associated with servility, femininity, and social inferiority. This is sometimes called the \textbf{``penetration paradigm"}.
\end{rmk}


Recent scholarship has proposed nuances for this model, pointing out that some types of sexual behaviours cannot be understood, or understood alone, through the framework of penetration. The author argues that in addition to the primary conceptual axis of penetration, the Romans further envisioned a secondary axis of agency (activity versus passivity) in the sexual act.

\subsection{Agency and the penetration model}


The author argues that penetration is conflated with agency with the use of ``active" and ``passive" in the literature, which in general is an oversimplification at best. We now propose the use of ``active" and ``passive" based on the model of Latin grammar. As \textbf{Charisius}, a fourth-century CE grammarian explains:

\begin{quotation}
    Active is that which will indicate doing something, like ``I say," indicating a motion either of the body, like ``I mount," or of the mind, like ``I foresee" ... Passive is the opposite of active, [and is] that which indicates enduring something, like ``I am burned."
\end{quotation}

Thus ``activity" should refer to more than just penetration; it should encompass, for example, performing a sex act, moving one's body during sex, or moving one's soul.

The author focuses on the literary and \textbf{epigraphic} (written matter recorded on hard or durable material) use of the nouns \textbf{cinaedus} and \textbf{pathicus} (``sexually penetrated male"), as well as the following verbs and their associated participles: 
\begin{enumerate}
    \item \textbf{irrumare} and \textbf{irrumari} (``to face-fuck" and ``to be face-fucked")
    \item \textbf{fellare} (``to suck cock"),
    \item \textbf{pedicare} and \textbf{pedicari} (``to ass-fuck" and ``to be ass-fucked")
    \item \textbf{futui} (``to be fucked")
    \item \textbf{cevere} (``to waggle the buttocks")
\end{enumerate}

\begin{nte}
    The author pays carefull attention to grammar (e.g., active versus passive voice, subjects versus objects), as well as descriptions of agency, movement, and desire. Through these close readings, we argue that some penetrated males (the \textbf{irrumatus} and the \textbf{pedicatus/fututus}) were conceptualized as passive, while others (the \textbf{fellator} and the \textbf{cinaedus/pathicus}) were characterized as active.
\end{nte}


\subsection{Passive Penetrated Males}

\subsubsection{The irrumatus}

Because irrumatio, ``face-fucking," was a violent act---namely, oral rape---its victim was conceptualized as an unwilling or inactive participant, in contrast to those who were said to perform fellatio (fellare). In both literature and graffiti, irrumatio often appears in implicit or explicit threats.

\begin{eg}
    One graffito from Pompeii reads, ``L(ucius) Habonius sauciat / irrumat Caesum / Felic(e)m", ``Lucius Habonius wounds, face--fucks Caesus Felix."
\end{eg}

\begin{eg}
    A set of graffiti from Rome: ``ir(r)uno te Sexte," ``I face-fuk you, Sextus."
\end{eg}

\textbf{Irrumatio} is used as a threat in the epigrams of the poet Martial as well as poems of Catullus. Other examples come from the \textbf{Priapic Corpus}. These poems are written from the perspective of the woodland deity Priapus, guardian of gardens, who threatens thieves with rape by his oversized phallus. In none of these cases does the threatened individual want to engage in oral sex; rather, the implication is that he is an unwilling, passive party in his penetration.

\subsubsection{The pedicatus/fututus}

Anally penetrated males who did not seek their penetration could be described with passive forms of the verb \textbf{pedicare}, ``to ass-fuck", or as objects of pedicare. The terms appeared often in graffito paired with names.

\begin{eg}
    The following is graffito from Northern Italy which focuses on the actions and agency of the penetrator: ``Antioc(h)us pedicatus / ego qui feci non / nego," ``Antiochus [has been] ass-fucked. I, the one who did it, don't deny it."
\end{eg}

As with irrumare, pedicare was often used in threats. In addition to being used against thieves, pedicatio (``ass-fucking") also features in threats against nosy and envious individuals.

Forms of pedicare were also used in humorous threats against readers. At its simplest we find ``ego qui lego pedicor," ``I who read am ass-fucked" in a graffito from Gaul.

In sum, irrumati, pedicati, and fututi routinely show a lack of sexual agency, movement, and desire for being penetrated.


\subsection{Active penetrated males}


\subsubsection{The fellator}

The subjects of the active verb fellare, ``to suck cock," are conceptualized as agents in their oral penetration. Literature often portrays fellatio as willingly performed. 

\begin{rmk}
    \textbf{Martial} plays on the common trope that oral sex pollutes the mouth.
\end{rmk}

The active desire for the act of fellatio is very present in epigrams and graffito. The agency of performers of fellatio is further suggested by the word fellator, ``cock sucker," an agent noun composed of the root fell- and the agentive suffix -tor.

Fellatores were characterized by their habitual behaviour. For example, a certain Vacerra is said by Martial to be an informer, a slanderer, a swindler, a dealer, a gladiator-trainer, and a cock sucker. Just as a slanderer or a cobbler performs his defining actions repeatedly, so too does the fellator habitually perform fellatio.

Note the distinction between the agency of the fellator and the passivity of the irrumatus.

\subsubsection{The cinaedus/pathicus}

The Latin word cinaedus comes from Greek and is of uncertain origin. The ancient lexicographers proposed various etymologies, including derivation from ``empty of shame," and ``to move one's shameful parts." Although in early Latin cinaedus was used primarily to designate dancers, with an emphasis on their bodily movements, in time it came to have a broader semantic range: most often it referred to a male who desired to be penetrated anally or was effeminate, but it could also designate a male who was lustful in general. Thus, a cinaedus might also perform oral sex on males and females and even penetrate females and (albeit rarely) males.

We focus on the active role of the cinaedus in his own penetration. Cinaedi and males of a similar ilk are depicted as wanting to be penetrated. We even see a man who pays to be penetrated in a graffito from Pompeii.

As with the man who ``bends over of his own accord," sometimes the agency of cinaedi is made manifest through their enthusiastic bodily movements, whether in attracting men or in the act of sex.

In other instances, verbs such as ceveo, ``to waggle the buttocks," are used to describe the sexual movement of cinaedi and those with similar characteristics.


The term \textbf{pathicus} shows up considerably less frequently than cinaedus in Latin literature and graffiti. The word is related to the verb \textbf{pati}, ``to endure"; in the case of males, it indicates enduring anal penetration. Despite this, the near-interchangeability of the terms pathicus and cinaedus hints that at least some pathici were thought to be agents.


\subsection{Further repercussions}

This distinction has further repercussions for our conceptual map of Roman sexuality, providing nuances to models set forth by other scholars. Parker's teratogenic grid (1997) provides a visualization of the penetration paradigm, although in grouping together all penetrated individuals under the label of ``passive," this model obscures the potential agency of some of these individuals.

\begin{table}[H]
    \centering
    \caption{Teratogenic grid}
    \begin{tabular}{cccc}
        \hline 
            & Vagina & Anus & Mouth \\ \hline
            Active & & & \\
            Activity & futuere & pedicare & irrumare \\
            Person & fututor & pedicator/pedico & irrumator \\
            Passive & & & \\
            Activity & futui & pedicari & irrumari/fellari \\
            Person & & & \\
            Male & cunnilinctor & cinaedus/pathicus & fellator \\
            Female & femina/puella & pathica & fellatrix \\ \hline
    \end{tabular}
    \label{tab:teratogenic}
\end{table}


Williams's chart of sexual verbs on the other hand allows for precisely this agency, as well as eschewing the terms ``active" and ``passive" as glosses for one's role in penetration.

\begin{table}[H]
    \centering
    \caption{Chart of sexual verbs}
    \begin{tabular}{ccc}
        \hline
        & Insertive & Receptive \\ \hline
        Vaginal & futuere & crisare \\
        Anal & pedicare & cevere \\
        Oral & irrumare & fellare \\ \hline
    \end{tabular}
    \label{tab:sexverb}
\end{table}

Neither model, however, encompasses both active and passive penetrated individuals, or fully represents the complex relationship between penetration and agency. While all penetrated men were non-normative objects of scorn, the Romans at the same time drew distinctions between active and passive. For example, accusations or insuations that someone was not only penetrated but also an agent in his own penetration were particularly defamatory. At times, this difference could have legal consequences too. Being willingly penetrated could result in civic restrictions, whereas being an unwilling party did not.

\begin{table}[H]
    \centering
    \caption{Penetration-agency model for male sexuality}
    \begin{tabular}{cccc}
        \hline 
            & & Orifice & \\ \hline
            & Vagina & Anus & Mouth \\ 
            Penetrating & & & \\
            Verb & futuere & pedicare & irrumare \\
            Person & fututor & pedicator/pedico & irrumator \\
            Penetrated & & & \\
            Verb & futui & pedicari & irrumari/fellari \\
            Person & & & \\
            Male (passive) & - & pedicatus/fututs & irrumatus \\
            Male (active) & - & cinaedus/pathicus(?) & fellator \\
            Female & femina/puella & pathica & fellatrix \\ \hline
    \end{tabular}
    \label{tab:penAgModel}
\end{table}





\section{Notes on Analysis and Societal Context}
\label{sec:SocCont4}



\section{Terms}
\label{sec:terms4}

\begin{enumerate}
    \item
\end{enumerate}


%
% \begin{acknowledgement}
% If you want to include acknowledgments of assistance and the like at the end of an individual chapter please use the \verb|acknowledgement| environment -- it will automatically render Springer's preferred layout.
% \end{acknowledgement}
%
% \section*{Appendix}
% \addcontentsline{toc}{section}{Appendix}
%


% Problems or Exercises should be sorted chapterwise
\section*{Problems}
\addcontentsline{toc}{section}{Problems}
%
% Use the following environment.
% Don't forget to label each problem;
% the label is needed for the solutions' environment
\begin{prob}
\label{prob1}
A given problem or Excercise is described here. The
problem is described here. The problem is described here.
\end{prob}

% \begin{prob}
% \label{prob2}
% \textbf{Problem Heading}\\
% (a) The first part of the problem is described here.\\
% (b) The second part of the problem is described here.
% \end{prob}

%%%%%%%%%%%%%%%%%%%%%%%% referenc.tex %%%%%%%%%%%%%%%%%%%%%%%%%%%%%%
% sample references
% %
% Use this file as a template for your own input.
%
%%%%%%%%%%%%%%%%%%%%%%%% Springer-Verlag %%%%%%%%%%%%%%%%%%%%%%%%%%
%
% BibTeX users please use
% \bibliographystyle{}
% \bibliography{}
%


% \begin{thebibliography}{99.}%
% and use \bibitem to create references.
%
% Use the following syntax and markup for your references if 
% the subject of your book is from the field 
% "Mathematics, Physics, Statistics, Computer Science"
%
% Contribution 
% \bibitem{science-contrib} Broy, M.: Software engineering --- from auxiliary to key technologies. In: Broy, M., Dener, E. (eds.) Software Pioneers, pp. 10-13. Springer, Heidelberg (2002)
% %
% Online Document

% \end{thebibliography}

