%%%%%%%%%%%%%%%%%%%%% chapter.tex %%%%%%%%%%%%%%%%%%%%%%%%%%%%%%%%%
%
% sample chapter
%
% Use this file as a template for your own input.
%
%%%%%%%%%%%%%%%%%%%%%%$% Springer-Verlag %%%%%%%%%%%%%%%%%%%%%%%%%%
%\motto{Use the template \emph{chapter.tex} to style the various elements of your chapter content.}
\chapter{The Cultural Construct of the Female Body in Classical Greek Science}
\label{CultConstrFemBod} % Always give a unique label
% use \chaptermark{}
% to alter or adjust the chapter heading in the running head


%%% Questions to think about
%The \textbf{Thesis} or general sense of the article is ...

%The \textbf{method} the author uses to argue their point is ...

%In their \textbf{analysis} the author uses tools such as ...
% How do they look at the evidence? Do they place it in some theoretical framework? (i.e. gender studies, music studies, etc.)

%Additionally they conclude ...
% How does this compare to others throughout time? What is the societal context?

%What connections does the author portray with regard to \textbf{space}, \textbf{relationships}, \textbf{occupation}, and \textbf{religion}.


\abstract{}

\section{Questions and Remarks}
\label{sec:QR5}




\section{First Reading}
\label{sec:FirRead5}

In most culturees an individual is ascribed to one sex or another at birth on the evidence of external genitalia and this categorization is taken to predict their physical and mental development and capabilities, which in turn support the differentation between the sexes in the home, workplace, religion, and the law. 

\begin{rmk}
    Cultures support the division of sexes by claiming that there are mental and emotional trats, outside the physical genetalia, which naturally differentiate the sexes.
\end{rmk}

Challenges to cultural constructs of male and female are neutralized by claiming that they are exceptions to a \textbf{natural law}.

\begin{nte}
    The belief in a natural law of the disjunction of the sexes can be found initially in the expression of mythology or religion.
\end{nte}

The deeply implanted cultural belief that men and women are radically different can condition the interpretation of empirical evidence so that science, in its turn, supports the belief that perceived differences between men and women are a result of biology rather than \textbf{social conditioning}.

\begin{rmk}
    In ancient Greece the polarization of sexual roles was far more marked than in our own society. This disjunction was expressed in Greek myth by separate origins of the sexes. \begin{enumerate}
        \item Hesiod portrays man as already existing when woman, a later manufactured product of the gods, was given to him.
    \end{enumerate}
\end{rmk}

After the beginnings of \textbf{natural philosophy} in \textbf{ionia} in the \textbf{sixth century BCE} mythology was no longer universally accepted as giving a true explanation of the world.

\begin{nte}
    The Greeks did not dissect the human body, and so had only the vaguest understanding of the internal reproductive organs; nor could they have any knowledge of genetics or endocrinology. The strict biological polarization of the sexes was thus even more dependent on external sexual characteristics than our own society.
\end{nte}

The \textbf{cultural paradigm of masculinity and femininity} had to be supported by demonstrating that typical male or female observable characteristics were evidence of a more perfectly male or female invisible nature (\textbf{physis}). 

\begin{rmk}
    Once the cultural archetype was shown to be grounded in nature, a man or woman who deviated from this norm could be viewed as aberrant---lacking in something essentially masculine or feminine---rather than as a challenge to what it was to be male or female, and the traditional polarization of the sexual roles could claim a scientific foundation.
\end{rmk}

Note that \begin{enumerate}
    \item Men were thought to be best suited to dealing with matters outside the home, the \textbf{polis}, and
    \item women with the concerns of the household, the \textbf{oikos}
\end{enumerate}

Although the female role in managing the oikos was recognized as important, a woman was still considered inferior and subordinate to her husband. On the other hand, although a man could not bear or nurse a child, he was not thought incapable of performing female tasks in the same way; rather the management of the household was considered beneath him.

\begin{rmk}
    \textbf{Hesoid} account of the first women as a gift (albeit \textbf{malicious}) of the gods to men.
\end{rmk}

Many scientific theories attempted to justify the subordination of female to the male. Note that little explicit reference is made to female anatomy or physiology in the majority of Greek literature, besides the two sources mentioned previously. 

\begin{defn}
    The \textbf{Hippocratic Corpus} was a collection of theoretical and therapeutical treatises written between the last quarter of the fifth and the middle of the fourth century BCE. These were written by several different authors.
\end{defn}

\begin{defn}
    The \textbf{biology of Aristotole}, primarily \textbf{History of Animals, Parts of Animals, Generation of Animals}, was written around the third quarter of the fourth century BCE.
\end{defn}

Both of these texts functioned in the same culture. 

\begin{rmk}
    The Hippocratic theory is the product of different physicians in different generations concerned, primarily, with pathology. On the other hand Aristotle was a single philosopher of nature, interested more in normative physiology and in developing a thoroughgoing theory of the female which could explain the similarities as well as the differences between the male and the female.
\end{rmk}

Both constructs were similar in that they were shaped by their \textbf{cultural assumption} that the female body was inherently inferior to that of the male. Both Hippocratics and Aristotle also argued that the fundamental differentation between the sexes did not become apparent until puberty.

\begin{nte}
    According to Aristotle, at puberty a man's body changes more drastically than a woman's, until then the two sexes are very similar.
\end{nte}


\subsection{Diseases of Women 1.1}

This work atributes menstruation to the very nature of a woman's flesh, which at puberty becomes ``loose and spongy," causing her body to soak up excess blood from her stomach.

If a man should have any excess moisture in his body after exercise, it is absorbed by his glands, which are especially constructed for this purpose. The author of \textbf{Glands 1} described their nature as spongy, porous, and plump, language very similar to that which is used in \textbf{Diseases of Women} to describe the female body in general. The author likens the glands to wool and emphasizes how much they differ from the rest of the body, but for woman the body is one big gland, and hence similar to that of a male body which functioned only after a man had evacuated or used up most of his excess fluid through vigorous activity.

\begin{rmk}
    The implication is that a truly feminine woman would be incapable of developing the sort of flesh that would enable her to perform the same tasks as a man.
\end{rmk}

The difference in the size of male and femal breasts was used as another indication of the extent to which a woman's body is ``looser than a man's". 

In \textbf{Epidemics 2.6.19}, it is stated that a large vein runs to each breast and that these are the seat of the greatest part of consciousness. That women would always be more susceptible to having more blood in their breasts than men, would give a ``scientific" basis to the belief that women were always closer to the irrational than men.

\subsection{Value Judgement}

Underlying the Hippocratic characterization of male and female flesh is a value judgement: firm and compact is good, while loose and spongy is bad. The man was thought to work much harder than a woman and thereby use up all his nourishment in building a stronger body. On the other hand a woman soaks up moisture through inactivity.

\textbf{Regimen 1.34} says that women are colder and moister than men in part because they use a more frivolous regimen. In \textbf{Diseases 4.45}, it is stated that less work does not simply result in a different type of body, but a sicker body; characterizing the result of idleness as \textbf{kakon} shows that the change was looked upon as a deterioration. 

\begin{qst}
    Could a woman change her body type and cease to menstruate if she led a strenuous life?
\end{qst}
The Hippocratic Corpus suggests nowhere that a woman could overcome her inherently inferior physis to the extent that she could cease to menstruate altogether. Despite the frequency of menstrual cycles which must have lasted longer than the canonical month, the Hippocratics assumed that an absence of menses for longer than a month meant that the blood was trapped in a woman's body, not that there was no excess blood to be evacuated.

\begin{nte}
    By the second century CE, when perhaps partly as a result of dissection, Soranus expressed the opinion that excessively active women did cease to menstruate.
\end{nte}

Because the Hippocratics believed that the difference between men and women was to be explained primarily by biology rather than by their socially allotted ways of life, they did not believe the female could ever assimdate to the male in this way or, thus, could ever expect to live more like a man. The converse, however, seemed quite possible. in \textbf{Airs, Waters, Places 20-22} the description of Scythian men shows that if a man pursues a sedentary lifestyle, his body becomes loose, flabby, and mosit and therefore more like a woman's.

The science of Hippocratics and Aristotle used menstruation to construct a female body inherently weak and capable of exerting influence on her emtions and intellect, thereby buttressing her subordinate and restricted position in society. 

\begin{rmk}
    For the Hippocratics the weakness of a woman's body (her porous flesh) caused menstruation; for Aristotle, menstruation caused her physical weakness.
\end{rmk}

\textbf{Menstruation and Health:}
\begin{enumerate}
    \item In the Hippocratic theory the release of excess matter in menstrual blood once a month prevented a woman's body from becoming diseased.
    \item Aristotle attributes the production of menstrual blood as what forced women away from the ideal of male health.
\end{enumerate}


\subsection{Theories of the Womb}

\textbf{On the Nature of the Child 15} says that the drawing of the blood from the woman's body into her womb happens all at once each month when she is not pregnant; this is perhaps to account for symptoms women report before menstruation each month. The passage for blood to flow was believed to be blocked in young girls, for whom the best way to remove the impediment is to be married as soon as possible. The Hippocratics believed that an imperforate membrane could stretch across the vagina, but they viewed this as an unusual pathological symptom, not as a natural hymen common to all women.

\begin{rmk}
    In Plato's \textbf{Timaeus} the womb is portrayed as an animal travelling around the body of a woman seeking satisfaction in sexual intercourse and pregnancy. The Hippocratics explain the ``movements of the womb through a woman's body" as it not being anchored in place by pregnancy or if not being kept moist by intercourse, it becomes dry and is attracted to the moister organs of the heart, the liver, the brain, and sometimes to the bladder and the rectum.
\end{rmk}

\begin{rmk}
    The womb could prolapse completely and issue from the vulva as a result of intercourse too soon after childbirth or a difficult birth.
\end{rmk}

A prolapsed uterus is recognized as a medical conditino today, and it has been suggested that it was this which gave rise to the belief that the womb could wander; however, it is simply a falling downard of the organ through the vagina. \textbf{Hanson} remarks that as men's bodies held no uterus, the human body had no special place for it to reside. That ``rational medicine" did not reject such a strange idea out of hand suggests that it fulfilled an important role in characterizing the female sex.

\begin{rmk}
    The wandering womb was believed to account for the suffocating sensation some women experienced in the chest and for various other pains dispersed throughout the body. 
\end{rmk}

In their explanations of womb movements the Hippocratics were rationalizing the theories, not of women themselves, but of a culture which needed to promote, and yet at the same time wished to maintain control over, women's power of procreation. 

\begin{rmk}
    The wandering womb deprived a woman of independent control of her own sexuality.
\end{rmk}

The Hippocratics retain the model of the womb as a separate animal within the woman which, without the intervention of a man, is in danger of subjugating the woman's own life force. Odors were used to repel the womb from one end ofthe body and attract it the the other. \textbf{Manuli} points out that employing perfumes in attracting the womb parallels the use of incense in invoking a god, an entity with a very definite mind of its own which is not easy for even a man to control.

\textbf{King} asserts that the idea of the womb as an independent animal would not suggest itself in the Hippocratic texts if we were not reading back from \textbf{Timaeus}. But even so we have to ask why the Hippocratics expended so much effort explicating a traditional belief which seems to us to have such little basis in reality. However, although the Hippocratics may have attempted to deny that the womb had any desires, the use of foul and sweet-smelling substances to draw it back contradicts the idea that their system was totally mechanical. 

Aristotle asserted that the womb was held in place just like the wombs of other animals and like the seminal passages in the male. Nevertheless, even he thought that when the womb was empty it could be pushed upwards and cause a stifling sensation.

\begin{rmk}
    In \textbf{History of Animals 582b22-26} Aristotle explains a prolapsed womb as a result of lack of sexual intercourse. A prolapsed uterus is one of the rare female conditions for which Hippocratics recommend abstinence from intercourse. 
\end{rmk}

Although Aristotle may have been more rigorously ``scientific" in observing anatomical and physiological phenomena, he to some extent was more bound by his cultural assumptions than the Hippocratic doctors.

According to \textbf{King}, the nostrils and vagina of a woman were thought to be connected by one long hollow tube giving the womb free passage from the top to the bottom of the body. Hence a favored method for deciding whether a woman could conceive was to sit her over something strong-smelling and see if it could be smelled through her mouth.

Aristotle also used the smell of pessaries through the mouth to showing if the passages in the body have closed over. However, he believed that the seminal secretion originates in the area of the diaphragm, and just as this passes down to genitalia, any movement set up in that area passes back to the chest, such that it is from here that the scent becomes perceptible on the breath.

\begin{nte}
    The Hippocratics frequently refer to the human womb in the plural, and Aristotle explicitly says that it is \textbf{double}.The occasional birth of twins probably confirmed this belief. It may also be from the observation of other mammalian uteri, such as that of a pig.
\end{nte}

\subsection{Conception of the Vagina}

Because they were compiling a pathology rather than a physiology, the Hippocratics did not describe in detail every part of the female anatomy of which they were aware. They explicitly differentiate this from the urethra, and often advise inserting pessaries into the vagina without any directions for steps to avoid obstructing the flow of uring, which again suggests that they viewed the vagina solely as the passage to the womb and completely separate from the urethra.

Aristotle failed to make the distinction between the vagina and the urethra. This is a consequence of one of his founding principles: that the female is less representative of the human form than the male. Further, on the principle that men are naturally superior to women, Aristotle claims that men have more teeth, which he associates with a longer life-span.


\subsection{Issus of Gender}

The physical difference that men are on the whole harier than women is credited to a man's greater volume and agitation of semen by the author of \textbf{On the Nature of the Child 20}. Hair, he claims, needs moisture (primarily semen) to grow, and that the semen is stored in the human head and that is where the epidermis is most porous. Thus it is believed women have some semen, but not as much as males, and the genital area is the only palce secondary body hair grows.

Aristotle thought hair grew when moisture was able to seep through the skin and then evaporated, leaving an earthy precipitate behind. People had most hair on their headas because the brain was the moistest part of the body and the sutures in the skull would allow the fluid to seep through. Again men go bald at the front of their heads because this is where semen is stored. The growth of hair is associated by Aristotle to hot fluid in the body. Aristotle has difficulty in attaining consistency in his theory of hair growth because adult men produce more but also lose more, and he wants both to be indications of male superiority.

\subsection{Conclusion}

The sexual differnita of menstruation, breasts, and womb are all accounted for in Hippocratic theory by the nature of female flesh. They are utlized in procreation, but they are the result of a difference between men and women which does not have sexual generation as its prime purpose. As a man produces more seed it becomes more agitated and he produces more hair. Thus in Hippocratic theory there are two fundamental causes for the observable difference in male and female physiology
\begin{nte}
    the difference between male and female flesh dictates a woman's incapacity to perform in a man's world, rather than differenes in reproductive fluids.
\end{nte}

Aristotle's theory ties all differences to a man's naturally greater heat, which allows him to concoct nourishment ot a greater degree for the purposes of sexual reproduction.

\begin{rmk}
    Aristotle considered women to be less ``other" and more like men than Hippocratics, but he could only maintain this general thoery while adhering to the principle of male superiority in every feature at the loss of some consistency and the neglect of some observable anatomical realities.
\end{rmk}

\begin{rmk}
    Because they thought woman was a completely different creature and not simply a substandard man, the Hippocratics did not have to look for a correspondence between all male and female body parts. THey felt woman was inferior, of course, but her ``otherness" allowed her body to be defined more by its own parameters. However, because they thought a woman was so different these parameters sometimes spread a little too widely.
\end{rmk}






\section{Notes on Analysis and Societal Context}
\label{sec:SocCont5}

This article was written in 2003, and hence still well-before gender studies was commonly studied in the scholarship.

\begin{nte}
    The author analyzes the two main sources in Greek literature which reference the female anatomy, which are discussed in the gynecology of the \textbf{Hippocratic Corpus}, and the biology of \textbf{Aristotle}. The differentiation in the female body that appears at puberty, including mentruation and breasts, is assigned by oth Hippocratics and Aristotle as a manifestation of a ``female nature" that makes it difficult for women to perform in male spheres.
\end{nte}


The contents of the Corpus range from Hippocrates' time in circa 460-375 BCE to many centuries later. Note that Aristotle is shortly after the time of Hippocrates, being born in 384 BCE and dying in 322 BCE.


\begin{rmk}
    The author seeks to demonstrate how Greek scientific theories of female anatomy and physiology were conditioned by cultural assumptions of female nature: specifically, how Greek scientists used menstruation, breasts, womb, and lack of body hair to define female physical nature as fundamentally different from and inferior to the physical nature of the male, and how, on occasion, their assumptions led them to misinterpret or overlook data which could have challenged their theories.
\end{rmk}

The author conclude that the culture's unwavering belief in female inferiority constrained the theories of Greek scientists.


\section{Terms}
\label{sec:terms5}

\begin{enumerate}
    \item
\end{enumerate}


%
% \begin{acknowledgement}
% If you want to include acknowledgments of assistance and the like at the end of an individual chapter please use the \verb|acknowledgement| environment -- it will automatically render Springer's preferred layout.
% \end{acknowledgement}
%
% \section*{Appendix}
% \addcontentsline{toc}{section}{Appendix}
%


% Problems or Exercises should be sorted chapterwise
\section*{Problems}
\addcontentsline{toc}{section}{Problems}
%
% Use the following environment.
% Don't forget to label each problem;
% the label is needed for the solutions' environment
\begin{prob}
\label{prob1}
A given problem or Excercise is described here. The
problem is described here. The problem is described here.
\end{prob}

% \begin{prob}
% \label{prob2}
% \textbf{Problem Heading}\\
% (a) The first part of the problem is described here.\\
% (b) The second part of the problem is described here.
% \end{prob}

%%%%%%%%%%%%%%%%%%%%%%%% referenc.tex %%%%%%%%%%%%%%%%%%%%%%%%%%%%%%
% sample references
% %
% Use this file as a template for your own input.
%
%%%%%%%%%%%%%%%%%%%%%%%% Springer-Verlag %%%%%%%%%%%%%%%%%%%%%%%%%%
%
% BibTeX users please use
% \bibliographystyle{}
% \bibliography{}
%


% \begin{thebibliography}{99.}%
% and use \bibitem to create references.
%
% Use the following syntax and markup for your references if 
% the subject of your book is from the field 
% "Mathematics, Physics, Statistics, Computer Science"
%
% Contribution 
% \bibitem{science-contrib} Broy, M.: Software engineering --- from auxiliary to key technologies. In: Broy, M., Dener, E. (eds.) Software Pioneers, pp. 10-13. Springer, Heidelberg (2002)
% %
% Online Document

% \end{thebibliography}

