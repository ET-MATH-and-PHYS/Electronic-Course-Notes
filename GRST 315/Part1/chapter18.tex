%%%%%%%%%%%%%%%%%%%%% chapter.tex %%%%%%%%%%%%%%%%%%%%%%%%%%%%%%%%%
%
% sample chapter
%
% Use this file as a template for your own input.
%
%%%%%%%%%%%%%%%%%%%%%%$% Springer-Verlag %%%%%%%%%%%%%%%%%%%%%%%%%%
%\motto{Use the template \emph{chapter.tex} to style the various elements of your chapter content.}
\chapter{Modern Tourists, Ancient Sexualities: Looking at Looking in Pompeii's Brothel and the Secret Cabinet}
\label{pompeiiBrothel} % Always give a unique label
% use \chaptermark{}
% to alter or adjust the chapter heading in the running head


%%% Questions to think about
%The \textbf{Thesis} or general sense of the article is ...

%The \textbf{method} the author uses to argue their point is ...

%In their \textbf{analysis} the author uses tools such as ...
% How do they look at the evidence? Do they place it in some theoretical framework? (i.e. gender studies, music studies, etc.)

%Additionally they conclude ...
% How does this compare to others throughout time? What is the societal context?

%What connections does the author portray with regard to \textbf{space}, \textbf{relationships}, \textbf{occupation}, and \textbf{religion}.


\abstract{}

\section{Questions and Remarks}
\label{sec:QR18}


\begin{qst}
    What is the secret cabinet?
\end{qst}


\begin{qst}
    What are \textbf{Bourbon excavations}?
\end{qst}


\begin{qst}
    What is \textbf{sex essentialism}, \textbf{sex negativity}, \textbf{sex hierarchy}, and \textbf{lascivious treatment of sexuality}?
\end{qst}

\begin{qst}
    What was the \textbf{Lupanar}?
\end{qst}


\begin{qst}
    What was the \textbf{Gabinetto Segreto}?
\end{qst}


\begin{qst}
    What is the \textbf{Museo Archeologico}?
\end{qst}







\section{First Reading}
\label{sec:FirRead18}


\begin{defn}
    \textbf{Sex negativity}: the belief that sexual material reflects or contributes to moral corruption.
\end{defn}


\begin{defn}
    \textbf{Sex hierarchy}: the privileging of hetero-erotic sexuality
\end{defn}


\begin{defn}
    \textbf{Sex essentialism}: the belief that sex and sexuality are ahistorical concepts needing no explanation.
\end{defn}


\subsection{Looking at Looking in the Brothel}

\begin{defn}
    \textbf{Lupanare}: a brothel (p. 319)
\end{defn}

There is a structure in the area to the east of Pompeii's forum which is the only structure at Pompeii that unambiguously meets scholars' criteria for an ancient brothel. The brothel was restored and opened to the public in 2006. Tour guides often present a `sound-bite' version of sexuality that is titillating yet not offensive (p. 320). Included in most explanations of the brothel's frescoes are two popular theories: that the erotic frescoes were the `Greco-Roman Kama Sutra', or a sex menu for the clients.

\begin{rmk}
    The brothel's frescoes illustrated Hellenistic sex manuals. Modeled on these manuals, frescoes depicting couples in various sexual positions, often called \textbf{figurae veneris} (positions of love), could be found in brothels, baths, taverns, and houses.
\end{rmk}

The menu theory isn't as well backed up, as it suggests each door and fresco were in one-to-one correspondence, which is not the case. Further, one of the frescoes by the doors doesn't even show a sexual position.

\begin{nte}
    Almost all of the brothel's graffiti were written in Latin, suggesting that clients and prostitutes did speak a common language, and would not need to point or grunt at pictures.
\end{nte}

The frescoes also only depict a narrow range of sexual activities: male-female genital sex. Regardless of its veracity, the `sex menu' theory is often repeated by guides and tourists alike.

On the other hand, the guidebook and audioguide both caution against such theories and hypotheses about the past. However these are infrequently used resources. Most tourists learn about ancient sexuality through the stories told by their guides, and few are exposed to the broader information contained in the site's audio and visual materials.

\begin{rmk}
    Other sexual acts, including oral and anal sex and involving same sex-pairs and groups, were part of the Pompeians' sexual repertoire. Many of these acts are depicted in the frescoes of the Suburban Baths at Pompeii.
\end{rmk}

However, the epigraphic elements of the brothel (i.e. graffiti) are rather inaccessible to tourists. Additionally, none of the site's didactic material raises the possibility of non-genital sexual activities or homoerotic activity. The sex manual's, according to Holt Parker, serve to create a normative intercourse and to reassure the male initiate that he will meet with nothing unexpected ... the fear of sexual contact with the Other is removed, not only by advance familiarity with certain physiological facts but by the construction of a carefully delimited sexual cosmos. (p.323)


\subsection{Looking at Looking in the Secret Cabinet}

The \textbf{Gabinetto Segreto} is currently stored on the first floor of the National Archaeological Museum of Naples. In the twenty-first century the Secret Cabinet is made completely accessible to the public. Frescoes dominate the collection, with a smattering of statuary in tehe garden and street sections, and small finds displayed in glass cabinets in a few sections of the exhibit.

The role of tour guides as mediators for this collection is much smaller than at the brothel, although it still represents a majority of interactions.

\begin{rmk}
    The lascivious treatment of sexuality and sex essentialism fail to explain the sculpture of Pan, and perhaps the sculpture even calls uncomfortable attention to these two tactics as exactly what they are---ways of rationalizing ancient sexuality. (p. 329)
\end{rmk}

\subsection{Conclusions}

Sex negativity in the nineteent hand twentieth centuries resulted in restrictions on who could enter the brothel or Gabinetto Segreto. Modern-day tourists will likely encounter, and themselves contribute to, sex essentialism and the lascivious treatment of sexuality. The belief that sex and sexuality have remained unchanged throughout time and a titllating view of erotic material are strong themes both in guided tours and in tourists comments at the Lupanar and Gabinetto Segreto. The privileging of heteroerotic acts over homoerotic or otherwise marginalized sexual behaviours, combined with the belief that sexuality is unchanging, leads to a blanched, surprisingly modern-seeming picture of ancient sexuality. (p.330)

\begin{qst}
    What does the public construction of sexuality reflect about what we want ancient sexuality to be?
\end{qst}

From the comments of tour guides and tourists, we can infer that the public represented here wants ancient sexuality to be fun, a little risque, but ultimately normative and familiar.



\section{Notes on Analysis and Societal Context}
\label{sec:SocCont18}


Author examines tourists' experiences at two sites that often form the basis for popular conceptions of ancient Pompeian sexuality: the brothel (Lupanar) at Pompeii, and the Secret Cabinet (Gabinetto Segreto) in the National Archaeological Museum of Naples.

The author argues that the tourists' perceptions of ancient sexuality at Pompeii reflect not only the lascivious treatment of sexuality, but also the privileging of hetero-erotic sexuality (sex hierarchy) and the belief that sex and sexuality are ahistorical concepts needing no explanation (sex essentialism). The recognition of historical and current roadblocks to popular understandings of sexuality can give insights into the selective appropriation of the past.


The analysis uses observations of tourists in March 2007 and April 2009, supplemented by surveys filled out by university students studying at the Intercollegiate Center for Classical Studies during the spring of 2007.

Details are most often drawn from English-speaking tourists. The findings reflect the attitudes of a particular sector of the increasingly global Pompeian tourist audience.


Author argues the perceptions of tourists reflect not only the lascivious treatment of sexuality, the privileging of hetero-erotic sexuality, and the belief that sex and sexuality are ahistorical concepts needing no explanation.

\section{Terms}
\label{sec:terms18}

\begin{enumerate}
	\item
\end{enumerate}


%
% \begin{acknowledgement}
% If you want to include acknowledgments of assistance and the like at the end of an individual chapter please use the \verb|acknowledgement| environment -- it will automatically render Springer's preferred layout.
% \end{acknowledgement}
%
% \section*{Appendix}
% \addcontentsline{toc}{section}{Appendix}
%


% Problems or Exercises should be sorted chapterwise
\section*{Problems}
\addcontentsline{toc}{section}{Problems}
%
% Use the following environment.
% Don't forget to label each problem;
% the label is needed for the solutions' environment
\begin{prob}
\label{prob1}
A given problem or Excercise is described here. The
problem is described here. The problem is described here.
\end{prob}

% \begin{prob}
% \label{prob2}
% \textbf{Problem Heading}\\
% (a) The first part of the problem is described here.\\
% (b) The second part of the problem is described here.
% \end{prob}

%%%%%%%%%%%%%%%%%%%%%%%% referenc.tex %%%%%%%%%%%%%%%%%%%%%%%%%%%%%%
% sample references
% %
% Use this file as a template for your own input.
%
%%%%%%%%%%%%%%%%%%%%%%%% Springer-Verlag %%%%%%%%%%%%%%%%%%%%%%%%%%
%
% BibTeX users please use
% \bibliographystyle{}
% \bibliography{}
%


% \begin{thebibliography}{99.}%
% and use \bibitem to create references.
%
% Use the following syntax and markup for your references if 
% the subject of your book is from the field 
% "Mathematics, Physics, Statistics, Computer Science"
%
% Contribution 
% \bibitem{science-contrib} Broy, M.: Software engineering --- from auxiliary to key technologies. In: Broy, M., Dener, E. (eds.) Software Pioneers, pp. 10-13. Springer, Heidelberg (2002)
% %
% Online Document

% \end{thebibliography}

