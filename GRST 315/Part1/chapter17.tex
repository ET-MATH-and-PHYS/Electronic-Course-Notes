%%%%%%%%%%%%%%%%%%%%% chapter.tex %%%%%%%%%%%%%%%%%%%%%%%%%%%%%%%%%
%
% sample chapter
%
% Use this file as a template for your own input.
%
%%%%%%%%%%%%%%%%%%%%%%$% Springer-Verlag %%%%%%%%%%%%%%%%%%%%%%%%%%
%\motto{Use the template \emph{chapter.tex} to style the various elements of your chapter content.}
\chapter{Prostitutes, Women, and Gender in Ancient Greece}
\label{chap:ProstWomGen} % Always give a unique label
% use \chaptermark{}
% to alter or adjust the chapter heading in the running head


%%% Questions to think about
%The \textbf{Thesis} or general sense of the article is ...

%The \textbf{method} the author uses to argue their point is ...

%In their \textbf{analysis} the author uses tools such as ...
% How do they look at the evidence? Do they place it in some theoretical framework? (i.e. gender studies, music studies, etc.)

%Additionally they conclude ...
% How does this compare to others throughout time? What is the societal context?

%What connections does the author portray with regard to \textbf{space}, \textbf{relationships}, \textbf{occupation}, and \textbf{religion}.


\abstract{}

\section{Questions and Remarks}
\label{sec:QR17}




\section{First Reading}
\label{sec:FirRead17}


A history of ancient sex laborers is difficult to write given the lack of evidence from the perspective of such workers themselves. Actual anecdotes about these working women, penned by male writers, cannot be trusted, since an accurate portrayal of the prostitutes themselves is not the motivation behind these accounts. (p. 703)


\subsection{Preconceptions and Terminology}


The focus of the literature on the case of hetaira and sacred prostitution has limited the study of the variety and variability of sexual labor in ancient Greece. 

\begin{rmk}
    The impression of \textbf{hetaira} as beautiful, educated, and witty is based on anecdotes, such as \textbf{Pausanias} and \textbf{Athenaeus}, written at least a few hundred years after any such woman likely lived and have little bearing on the reality of the hetaira in classical Greece. (p.703)
\end{rmk}

The focus of prostitution in Corinth has remained temple prostitution, neglecting the many references to prostitution at Corinth in general. 

\begin{nte}
    Any conclusions about a class of female sex laborers serving the goddess Aphrodite and working in her sanctuaries is highly problematic and controversial. (p.703)
\end{nte}

The result of ignoring the importance of context and prioritizing one type of evidence over another is an idealization of the hetaira and the sacred prostitute (p.703-704)

\begin{nte}
    Common terms from the classical period for such female laborers were \textbf{porne}, \textbf{hetaira}, and \textbf{pallake}.
\end{nte}

Nicknames and slang terms such as One Obol (\textbf{Obole}), Twelve Obols (\textbf{Didrachmon}) and penny whore (\textbf{khalkiditis}) emphasize the material nature of the prostitute-client relationship, the low cost of such women, and thus emphasize their communal accessibility. (p.704)

\begin{rmk}
    Modern scholarship genearlly assumes that \textbf{pornai} worked for a fee in brothels and were of slave status, while \textbf{hetairai}, freed or free, were longer-term companions of one man and often paid in kind rather than in cash.
\end{rmk}


While \textbf{pallakai} might be of slave status, they differ from \textbf{hetairai} and \textbf{pornai} in that they live in semi-permanent arrangements with their lovers.

According to \textbf{Plutarch}, \textbf{hetaira} was simply an Athenian euphemism for \textbf{porne}. It is a mistake to impose a kind of taxonomy of prostitution with porne at the bottom and hetaira at the top.


Prostitutes could ply their trade anywhere throughout the city. Some ancient clients referred to these women and girls as ``polluted ones."

\begin{nte}
    Free, slave, ex-slave, and citizens in addition to foreigners openly practiced prostitution.
\end{nte}
Access to prostitutes was easy, even for slaves.


\subsection{Living the Life: Three Case Studies}


Women working in porneia were likely of slave status and subject to a pimp, who was frequently female, the \textbf{pornoboskousa}. The presence of loom weights in such establishements suggest that the women likely also worked at looms to maximize the profit from their labor. 

\begin{rmk}
    We see statuettes of foreign goddesses, such as \textbf{Astarte} and \textbf{Cybele}, suggesting the women were non-Greeks and actually live in the complex.
\end{rmk}

Hosts of symposia hired female dancers, harp players, and flute players (\textbf{auletrides}) (p. 705). 

The story of \textbf{Alke} hints that some such women were able to use their trade to their own advantage and move up the socio-economic ladder (p.706). The reverse was also a possibility.

\begin{rmk}
    Slave women, unlike a wife, were able to accompany men to dinners with their male companions. (p.706)
\end{rmk}

The story of the unnamed \textbf{pallake} reveals the vulnerability of sex slaves as marginal members of society. (p.706)


\begin{nte}
    Women who were of foreign status and originally slaves were not legally allowed to marry Athenians or have legitimate children.
\end{nte}

\begin{rmk}
    In the story of \textbf{Nikarete} it is suggested that status, more than looks, affected the prices of sex workers.
\end{rmk}

The texts attest to the use of child prostitutes and an elite culture that lavished gifts and money in support of the sex trade. The slave status and young age of prostitutes argues against any agency and autonomy in their sexual relations, but the texts do suggest that some girls might eventually achieve freedom and better their socio-economic position through the practice of prostitution. Managing sex laborers was one way for women to earn a living and perhaps even improve their standard of living.Instability perhaps best characterizes the lives of prostitutes in classical Athens. (p.708)


\subsection{Prostitutes and Gender}

All three case studies are associated with the speaker's opponent, and their characterizations depend on negative female stereotypes and demonstrate how gender was used to construct the prostitute body. \textbf{Apollodoros} claims that prostitutes have a particular sexual nature that predestined them for their profession. 

\begin{rmk}
    Whereas males might engage in prostitution, women who did the same took on the identity of prostitute. (p. 708)
\end{rmk}

The distinction in terminology for male and female prostitutes at Athens reinforces the attitude that women are prostitutes by nature and that being a prostitute is more than simply a way to make a living. The ancients, at least in the case of women, did not view prostitution as a trade, but treated it as an identity. (p. 708)

\begin{nte}
    Sexual virtue was the most important quality for female citizens in classical Athens and Greek culture more broadly.
\end{nte}

\textbf{Demosthenes} states 
\begin{quotation}
    We have hetairai for pleasure, and pallakai for the daily services of our bodies, but wives, for the production of legitimate offspring and to have [a] reliable guardian of our household
\end{quotation}

Wives and daughters are characterized by the \textbf{sophron} (modest) behaviour required of them (sexual virtue, prudence, and moderation).

Such examplsee reveal how the existence of prostitution and prostitutes could work as a form of social control on female sexual behaviour more generally. The identity of prostitute was an insult, but brought one's status as wife and thus the legitimate status of her children into question. (p. 709)


\subsection{Conclusion}

While prostitution was accepted, those practicing prostitution might be devalued and denigrated. In Athens, it was easy to obscure the relationship between a male and female and suggest it was prostitution, since women lacked a public persona.


\section{Notes on Analysis and Societal Context}
\label{sec:SocCont17}

The author attempts to consider the possible environment and livelihood of female prostitutes in classical Greece by looking at the narratives of three women identified in the literature as sex laborers: \textbf{Alke}, an unnamed slave, and \textbf{Neaira}. These accounts appear in law court speeches. By focusing on this type of evidence, the author hopes to reveal what we can and cannot know about these women, while also showing how negative female stereotypes centered on the prostitute body.


\section{Terms}
\label{sec:terms17}

\begin{enumerate}
    \item \textbf{hetaira} (sexual companion commonly translated as courtesan) (p.703)
    \item \textbf{sacred prostitution} (sex purchased in honor of the goddess Aphrodite at Corinth) (p.703)
    \item \textbf{Porne} likely comes from the verb to sell (\textbf{pernemi}) (p.704) 
    \item \textbf{Pallake} are frequently identified as foreign women in long-term relationships which might even result in semi-legitimate children.
    \item \textbf{Megalomisthoi} refer to independent, high-priced prostitutes.
    \item \textbf{Pornoboskos} is a pimp and \textbf{pornoboskousa} is a procuress
    \item \textbf{Khamaitupe}: ground beater
    \item \textbf{Spodesilaura, peripolis, and dromas}: streetwalker
    \item \textbf{gephuris}: bridge-girls
    \item \textbf{Musachne}: polluted one
    \item \textbf{Porneia}: brothels 
\end{enumerate}


%
% \begin{acknowledgement}
% If you want to include acknowledgments of assistance and the like at the end of an individual chapter please use the \verb|acknowledgement| environment -- it will automatically render Springer's preferred layout.
% \end{acknowledgement}
%
% \section*{Appendix}
% \addcontentsline{toc}{section}{Appendix}
%


% Problems or Exercises should be sorted chapterwise
\section*{Problems}
\addcontentsline{toc}{section}{Problems}
%
% Use the following environment.
% Don't forget to label each problem;
% the label is needed for the solutions' environment
\begin{prob}
\label{prob1}
A given problem or Excercise is described here. The
problem is described here. The problem is described here.
\end{prob}

% \begin{prob}
% \label{prob2}
% \textbf{Problem Heading}\\
% (a) The first part of the problem is described here.\\
% (b) The second part of the problem is described here.
% \end{prob}

%%%%%%%%%%%%%%%%%%%%%%%% referenc.tex %%%%%%%%%%%%%%%%%%%%%%%%%%%%%%
% sample references
% %
% Use this file as a template for your own input.
%
%%%%%%%%%%%%%%%%%%%%%%%% Springer-Verlag %%%%%%%%%%%%%%%%%%%%%%%%%%
%
% BibTeX users please use
% \bibliographystyle{}
% \bibliography{}
%


% \begin{thebibliography}{99.}%
% and use \bibitem to create references.
%
% Use the following syntax and markup for your references if 
% the subject of your book is from the field 
% "Mathematics, Physics, Statistics, Computer Science"
%
% Contribution 
% \bibitem{science-contrib} Broy, M.: Software engineering --- from auxiliary to key technologies. In: Broy, M., Dener, E. (eds.) Software Pioneers, pp. 10-13. Springer, Heidelberg (2002)
% %
% Online Document

% \end{thebibliography}

