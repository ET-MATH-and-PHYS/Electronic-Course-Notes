%%%%%%%%%%%%%%%%%%%%% chapter.tex %%%%%%%%%%%%%%%%%%%%%%%%%%%%%%%%%
%
% sample chapter
%
% Use this file as a template for your own input.
%
%%%%%%%%%%%%%%%%%%%%%%$% Springer-Verlag %%%%%%%%%%%%%%%%%%%%%%%%%%
%\motto{Use the template \emph{chapter.tex} to style the various elements of your chapter content.}
\chapter{The Tale of Salmakis}
\label{salm} % Always give a unique label
% use \chaptermark{}
% to alter or adjust the chapter heading in the running head


%%% Questions to think about
%The \textbf{Thesis} or general sense of the article is ...

%The \textbf{method} the author uses to argue their point is ...

%In their \textbf{analysis} the author uses tools such as ...
% How do they look at the evidence? Do they place it in some theoretical framework? (i.e. gender studies, music studies, etc.)

%Additionally they conclude ...
% How does this compare to others throughout time? What is the societal context?

%What connections does the author portray with regard to \textbf{space}, \textbf{relationships}, \textbf{occupation}, and \textbf{religion}.


\abstract{}

\section{Questions and Remarks}
\label{sec:QRHerm}

\begin{qst}
    Who was \textbf{Hermaphroditus}? How did they cross between the realms of woman and man?
\end{qst}


\begin{qst}
    How does dress place Hermaphroditus in a \textbf{liminal/transitional} status?
\end{qst}

\begin{qst}
    How do dress and the body figure into representations of Hermaphroditus?
\end{qst}



\section{First Reading}
\label{sec:FirReadHerm}


\textbf{Salmakis} (Salmacis) was the \textbf{Naiad}-nymph of a spring of the town of \textbf{Halikarnassos} (Halicarnassus) in \textbf{Karia} (Caria). She fell in love with the handsome youth \textbf{Hermaphroditos} and prayed the gods be united with him forever. Their forms were merged as one to create the first \textbf{hermaphrodite}. 

\begin{rmk}
    Salmakis' namesake fountain was believed to make men who bathed in its waters effeminate.
\end{rmk}

With regard to the fountain, \textbf{Strabo} says 
\begin{quotation}
    It seems that the effeminacy of man is laid to the charge of the air or of the water; yet it is not these, but rather riches and wanton living, that are the cause of effeminacy.
\end{quotation}

Ovid says Salmacis is the only of the Naiads unkown to swift Diana (\textbf{Artemis}). Upon seeing Hermaphroditos the Naiad please for the boys affection. The Nympha pleaded, begged, besought at least a sister's kiss, and made to throw her arms around his ivory neck. `Enough!' he cried `Have done! Or I shall quit this place--and you.' The Nympha at first conceded, and after she left the boy stripped off his clothes to enter the pool. Seeing this while in hiding the Nympha could no longer hold it in and plunged into the pool and grappled the boy, and caressed him as he fought to escape her hold. Atlantiades (\textbf{Hermaphroditus}) fought back, denied the Nympha her joy; the Nympha cried `Ye Gods ordain no day shall ever dawn to part us twain!', and her prayer found gods to hear; both bodies merged in one, both blended in one form and face. Ovid describes the union as `one body then that neither seemed and both.'

Raising his hand Hermaphroditus cried, `Dear father [\textbf{Hermes}] and dear mother [\textbf{Aphrodite}], both of whose names I bear, grant me, you child, that whoso in these waters bathes a man emerge half woman, weakened instantly.'

Ovid says both Hermes and Aphrodite heard their bi-sexed son, and drugged the bright water with that power impure.



\subsection{Art, Pride, and Ranbow Flag}

By Dr. Bryan C. Keene, David Bardeen, Dr. David Brafman, Sarah Cooper, Dr. Mazie Harris, Arpad Kovacs, Casey Lee, Pietro Rigolo, and David Saunders \url{https://smarthistory.org/art-pride-and-the-rainbow-flag/}

\textbf{Turquoise: Magic}

In the Chemical Wedding of Hermes and Aphrodite, Michaelis Majeri depicts Hermaphrodite with two heads being roasted over a fire.In the original myth, Hermaphroditus is the offspring of the god Hermes and goddess Aphrodite, born with ``male" and ``female" sex organs. In alchemy the figure is adapted into a symbol for creating the illusion of gold.

\begin{rmk}
    Hermes, the Roman Mercury, stands for the element mercury. Aphrodite comes to be the symbol for copper. The island of Cyprus, her birthplace, was heavily mined in antiquity for copper. One takes mercury, copper, and a sprinkle of genuine gold flake, and heats the mixture, the mercury bonds the gold to the surface of the copper, making it \textbf{look like} gold. \textbf{An illusion}.
\end{rmk}

Alchemical imagery often expresses transformation by metaphorically toying with the ambiguity of gender as a physical illusion. ``Gender's a transitory illusion. True identity's an embedded secret, waiting to be empowered." - David Brafman


\subsection{Theoi: Hermaphroditos}

\url{https://www.theoi.com/Ouranios/ErosHermaphroditos.html}. 

\textbf{Hermaphroditos} was the \textbf{god of hermaphrodites and of effeminates}. They were numbered amongst the winged love-gods known as \textbf{Erotes}. Hermaphroditos was the child of Hermes and APhrodite, the gods of male and female sexuality.

According to some, they were once a handsome youth who attracted the love of a Naiad nymphe Salmakis. She prayed to be united with Hermaphroditos, and the gods answered her prayers, merging their two forms. At the same time the spring acquired the property of making men who bathed in its waters soft and effeminate.

\begin{rmk}
    Hermaphroditos was depicted as a winged youth with both male and female features---usually female thighs, breasts, and style of hair, and male genitalia.
\end{rmk}

The name Hermaphroditus is the compound of Hermes and Aphrodite, and is synonymous with \textbf{androgunes}, \textbf{gunandros}, \textbf{hemiandros}. Hermaphroditus was brought up by the nymphs of Mount Ida.


\begin{quotation}
    ``Some say that this Hermaphroditos is a god and appears at certain times among men, and that ze is born with a physical body which is a combination of that of a man and that of a woman, in that ze has a body which is beautiful and delicate like that of a woman, but has the masculine quality and viour of a man. But there are some who declare that such creatures of two sexes are monstrosities, and coming rarely into the world as they do they have the quality of presaging the future, sometimes for evil and sometimes for good." - \textbf{Diodorus Siculus} (1st century BCE)
\end{quotation}



\section{SENSING HERMAPHRODITUS IN THE DIONYSIAN THEATRE GARDEN}

The author notes that Ovid's narrative of Hermaphroditus and Salmacis in Metamorphoses was possibly staged for theatrical performances in Pompeian domestic garden settings. Dionysus' ancient Mediterranean presence had a connection with fertility festivals and mystery rites, which in its early stages included men, women, and children, who engaged in rites under the influence of wine and dance (p.68).

\begin{rmk}
    Visual depictions of Hermaphroditus date from as early as the fourth century BCE, coinciding with the emergence of both the display of the female nude body in Aphrodite Knidos and the iconographic transformation of the ffeminate beardless Dionysus. (p. 71-72)
\end{rmk}

Hermaphroditus is not introduced into Dionysus' entourage until the second century BCE (p.72). Members of Dionysus' entourage who frequently accompany Hermaphroditus include satyrs/Pan, Silenus, and maenads. Hermaprhoditus is associated with Dyonisiac performance.

Hermaphroditus is predominantly seen in Pompeii-ing depictions with a \textbf{tympanum} (tambourine) played by a follower of Dionysus or on the ground near Hermaphroditus (p.72 and 74). We also see the \textbf{kithara} (lyre).

\textbf{Saffron yellow} was perceived as a feminine colour in antiquity in part because of its associations with young unmarried women, seduction, female ritual, and by extension, a change of identity (p.75). Saffron had semantic ties to the theatre, cult of Dionysus, and effeminacy. Hermaphroditus' appearance with saffron yellow hence links hir with Dionysus.

Often Hermaphroditus is portrayed either sleeping or wrestling as a satyr/pan figure `discovers/awakens' hir. Ovid's likening of Hermaphroditus' blushing red cheeks to apples in an orchard has been linked to Sappho's fragment. Apples were used as gifts for marriage and as aphrodisiac symbols to elicit sexual desire. The apple is linked to Aphrodite, the mother of Hermaphroditus. This simile also links Hermaphroditus to Dionysus, due to his patronage being extended to various fruits, gardens, and orchards. Apples were also a reference for female breasts in Greek and Roman literature.

A third century BCE comedy by Poseidippus entitled Hermaphroditos suggests that Hermaphroditus' images were considered appropriate marvels for Pompey the Great's theatre in Rome. (p.80) Fragmentary remains of Hermaphroditus/satyr groups have also been found in Hellenistic/Roman theatres at Daphne.



\section{Notes on Analysis and Societal Context}
\label{sec:SocContHerm}



\section{Terms}
\label{sec:termsHerm}

\begin{enumerate}
	\item
\end{enumerate}

%
% \begin{acknowledgement}
% If you want to include acknowledgments of assistance and the like at the end of an individual chapter please use the \verb|acknowledgement| environment -- it will automatically render Springer's preferred layout.
% \end{acknowledgement}
%
% \section*{Appendix}
% \addcontentsline{toc}{section}{Appendix}
%


% Problems or Exercises should be sorted chapterwise
\section*{Problems}
\addcontentsline{toc}{section}{Problems}
%
% Use the following environment.
% Don't forget to label each problem;
% the label is needed for the solutions' environment
\begin{prob}
\label{prob1}
A given problem or Excercise is described here. The
problem is described here. The problem is described here.
\end{prob}

% \begin{prob}
% \label{prob2}
% \textbf{Problem Heading}\\
% (a) The first part of the problem is described here.\\
% (b) The second part of the problem is described here.
% \end{prob}

%%%%%%%%%%%%%%%%%%%%%%%% referenc.tex %%%%%%%%%%%%%%%%%%%%%%%%%%%%%%
% sample references
% %
% Use this file as a template for your own input.
%
%%%%%%%%%%%%%%%%%%%%%%%% Springer-Verlag %%%%%%%%%%%%%%%%%%%%%%%%%%
%
% BibTeX users please use
% \bibliographystyle{}
% \bibliography{}
%


% \begin{thebibliography}{99.}%
% and use \bibitem to create references.
%
% Use the following syntax and markup for your references if 
% the subject of your book is from the field 
% "Mathematics, Physics, Statistics, Computer Science"
%
% Contribution 
% \bibitem{science-contrib} Broy, M.: Software engineering --- from auxiliary to key technologies. In: Broy, M., Dener, E. (eds.) Software Pioneers, pp. 10-13. Springer, Heidelberg (2002)
% %
% Online Document

% \end{thebibliography}

