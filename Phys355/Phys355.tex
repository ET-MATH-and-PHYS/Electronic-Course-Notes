\documentclass[12pt]{report}
\usepackage[utf8]{inputenc}

%%%%%%%%%%%%%%%%% Book Formatting Comments:

%%%%%%%%%%%%%%%%%%%%%%%%%%%%%%%%%%%%% for Part

%%%%%%%%%%%%%%%%%%%%%% for chapter

%%%%%%%%%%%%%%%%%%%% for section








%%%%%% PACKAGES %%%%%%%
\usepackage{hyperref}
\hypersetup{
    colorlinks,
    citecolor=black,
    filecolor=black,
    linkcolor=black,
    urlcolor=black
}
\usepackage{amsmath} % Math display options
\usepackage{amssymb} % Math symbols
\usepackage{amsfonts} % Math fonts
\usepackage{amsthm}
\usepackage{mathtools} % General math tools
\usepackage{array} % Allows you to write arrays
\usepackage{empheq} % For boxing equations
\usepackage{mathabx}
\usepackage{mathrsfs}
\usepackage{nameref}

\usepackage{soul}
\usepackage[normalem]{ulem}

\usepackage{txfonts}
\usepackage{cancel}
\usepackage[toc, page]{appendix}
\usepackage{titletoc,tocloft}
\setlength{\cftchapindent}{1em}
\setlength{\cftsecindent}{2em}
\setlength{\cftsubsecindent}{3em}
\setlength{\cftsubsubsecindent}{4em}
\usepackage{titlesec}

\titleformat{\section}
  {\normalfont\fontsize{25}{15}\bfseries}{\thesection}{1em}{}
\titleformat{\section}
  {\normalfont\fontsize{20}{15}\bfseries}{\thesubsection}{1em}{}
\setcounter{secnumdepth}{1}  
  
  

\newcommand\numberthis{\refstepcounter{equation}\tag{\theequation}} % For equation labelling
\usepackage[framemethod=tikz]{mdframed}

\usepackage{tikz} % For drawing commutative diagrams
\usetikzlibrary{cd}
\usetikzlibrary{calc}
\tikzset{every picture/.style={line width=0.75pt}} %set default line width to 0.75p

\usepackage{datetime}
\usepackage[margin=1in]{geometry}
\setlength{\parskip}{1em}
\usepackage{graphicx}
\usepackage{float}
\usepackage{fancyhdr}
\setlength{\headheight}{15pt} 
\pagestyle{fancy}
\lhead[\leftmark]{}
\rhead[]{\leftmark}

\usepackage{enumitem}

\usepackage{url}
\allowdisplaybreaks

%%%%%% ENVIRONMENTS %%%
\definecolor{purp}{rgb}{0.29, 0, 0.51}
\definecolor{bloo}{rgb}{0, 0.13, 0.80}



%%\newtheoremstyle{note}% hnamei
%{3pt}% hSpace above
%{3pt}% hSpace belowi
%{}% hBody fonti
%{}% hIndent amounti
%{\itshape}% hTheorem head fonti
%{:}% hPunctuation after theorem headi
%{.5em}% hSpace after theorem headi
%{}% hTheorem head spec (can be left empty, meaning ‘normal’)i


%%%%%%%%%%%%% THEOREM STYLES

\newtheoremstyle{BigTheorem}
{20pt}
{20pt}
{\slshape}
{}
{\Large\color{purp}\bfseries}
{.}
{\newline}
{\thmname{#1}\thmnumber{ #2}\thmnote{ (#3)}}



\newtheoremstyle{TheoremClassic}
{15pt}
{15pt}
{\slshape}
{}
{\bfseries}
{.}
{.5em}
{}

\newtheoremstyle{Definitions}
{15pt}
{15pt}
{\slshape}
{}
{\bfseries}
{.}
{.5em}
{\thmname{#1}\thmnumber{ #2}\thmnote{ (#3)}}


\newtheoremstyle{Remarks}
{10pt}
{10pt}
{\upshape}
{}
{\bfseries}
{.}
{.5em}
{}

\newtheoremstyle{Examples}
{10pt}
{10pt}
{\upshape}
{}
{\bfseries}
{.}
{.5em}
{}


%%%%%%%%%%%%% THEOREM DEFINITIONS

\theoremstyle{BigTheorem}
\newtheorem{namthm}{Theorem}
\newtheorem{conj}[namthm]{Conjecture}

\theoremstyle{TheoremClassic}
\newtheorem{thm}{Theorem}[section]
\newtheorem*{thm*}{Theorem}
\newtheorem{lem}[thm]{Lemma}
\newtheorem{cor}[thm]{Corollary}
\newtheorem{prop}[thm]{Proposition}
\newtheorem{claim}[thm]{Claim}


\theoremstyle{Definitions}
\newtheorem{defn}{Definition}[section]
\newtheorem{axi}[defn]{Axiom}
\newtheorem{cust}[defn]{}
\newtheorem{cons}[defn]{Construction}
\newtheorem{props}[defn]{Properties}
\newtheorem{proc}[defn]{Process}
\newtheorem*{law}{Law}


\theoremstyle{Examples}
\newtheorem{eg}{Example}[section]
\newtheorem{noneg}[eg]{Non-Example}
\newtheorem{xca}[eg]{Exercise}


\theoremstyle{Remarks}
\newtheorem{rmk}{Remark}[section]
\newtheorem{qst}[rmk]{Question}
\newtheorem*{ans}{Answer}
\newtheorem{obs}[rmk]{Observation}
\newtheorem{rec}[rmk]{Recall}
\newtheorem{summ}[rmk]{Summary}
\newtheorem{nota}[rmk]{Notation}
\newtheorem{note}[rmk]{Note}



\renewcommand{\qedsymbol}{$\blacksquare$}


\numberwithin{equation}{section}

\newenvironment{qest}{
    \begin{center}
        \em
    }
    {
    \end{center}
    }

%%%%%% MACROS %%%%%%%%%
%% New Commands
\newcommand{\ip}[1]{\langle#1\rangle} %%% Inner product
\newcommand{\abs}[1]{\lvert#1\rvert} %%% Modulus
\newcommand\diag{\operatorname{diag}} %%% diag matrix
\newcommand\tr{\mbox{tr}\.} %%% trace
\newcommand\C{\mathbb C} %%% Complex numbers
\newcommand\R{\mathbb R} %%% Real numbers
\newcommand\Z{\mathbb Z} %%% Integers
\newcommand\Q{\mathbb Q} %%% Rationals
\newcommand\N{\mathbb N} %%% Naturals
\newcommand\F{\mathbb F} %%% An arbitrary field
\newcommand\ste{\operatorname{St}} %%% Steinberg Representation
\newcommand\GL{\mathbf{GL}} %%% General Linear group
\newcommand\SL{\mathbf{SL}} %%% Special linear group
\newcommand\gl{\mathfrak{gl}} %%% General linear algebra
\newcommand\G{\mathbf{G}} %%% connected reductive group
\newcommand\g{\mathfrak{g}} %%% Lie algebra of G
\newcommand\Hbf{\mathbf{H}} %%% Theta fixed points of G
\newcommand\X{\mathbf{X}} %%% Symmetric space X
\newcommand{\catname}[1]{\normalfont\textbf{#1}}
\newcommand{\Set}{\catname{Set}} %%% Category set
\newcommand{\Grp}{\catname{Grp}} %%% Category group
\newcommand{\Rmod}{\catname{R-Mod}} %%% Category r-modules
\newcommand{\Mon}{\catname{Mon}} %%% Category monoid
\newcommand{\Ring}{\catname{Ring}} %%% Category ring
\newcommand{\Topp}{\catname{Top}} %%% Category Topological spaces
\newcommand{\Vect}{\catname{Vect}_{k}} %%% category vector spaces'
\newcommand\Hom{\mathbf{Hom}} %%% Arrows

\newcommand{\map}[2]{\begin{array}{c} #1 \\ #2 \end{array}}

\newcommand{\Emph}[1]{\textbf{\ul{\emph{#1}}}}

\newcommand{\mapsfrom}{\mathrel{\reflectbox{\ensuremath{\mapsto}}}}


%% Math operators
\DeclareMathOperator{\ran}{Im} %%% image
\DeclareMathOperator{\aut}{Aut} %%% Automorphisms
\DeclareMathOperator{\spn}{span} %%% span
\DeclareMathOperator{\ann}{Ann} %%% annihilator
\DeclareMathOperator{\rank}{rank} %%% Rank
\DeclareMathOperator{\ch}{char} %%% characteristic
\DeclareMathOperator{\ev}{\bf{ev}} %%% evaluation
\DeclareMathOperator{\sgn}{sign} %%% sign
\DeclareMathOperator{\id}{Id} %%% identity
\DeclareMathOperator{\supp}{Supp} %%% support
\DeclareMathOperator{\inn}{Inn} %%% Inner aut
\DeclareMathOperator{\en}{End} %%% Endomorphisms
\DeclareMathOperator{\sym}{Sym} %%% Group of symmetries


%% Diagram Environments
\iffalse
\begin{center}
    \begin{tikzpicture}[baseline= (a).base]
        \node[scale=1] (a) at (0,0){
          \begin{tikzcd}
           
          \end{tikzcd}
        };
    \end{tikzpicture}
\end{center}
\fi




\newdateformat{monthdayyeardate}{%
    \monthname[\THEMONTH]~\THEDAY, \THEYEAR}
%%%%%%%%%%%%%%%%%%%%%%%



%%%%%% BEGIN %%%%%%%%%%


\begin{document}

%%%%%% TITLE PAGE %%%%%

\begin{titlepage}
    \centering
    \scshape
    \vspace*{\baselineskip}
    \rule{\textwidth}{1.6pt}\vspace*{-\baselineskip}\vspace*{2pt}
    \rule{\textwidth}{0.4pt}
    
    \vspace{0.75\baselineskip}
    
    {\LARGE Electromagnetism 1}
    
    \vspace{0.75\baselineskip}
    
    \rule{\textwidth}{0.4pt}\vspace*{-\baselineskip}\vspace{3.2pt}
    \rule{\textwidth}{1.6pt}
    
    \vspace{2\baselineskip}
    Phys 355: E\& M1 \\
    \vspace*{3\baselineskip}
    \monthdayyeardate\today \\
    \vspace*{5.0\baselineskip}
    
    {\scshape\Large Elijah Thompson, \\ Physics and Math Honors\\}
    
    \vspace{1.0\baselineskip}
    \textit{Recap}
\end{titlepage}

%%%%%%%%%%%%%%%%%%%%%%%
\tableofcontents


%%%%%%%%%%%%%%%%%%%%%%% - Chapter 1
\chapter{Electric Fields}

\section{Point Charges}

\begin{defn}{Charge}{}
	An \Emph{electric charge} is an intrinsic characteristic of fundamental particles. The unit of charge is the \Emph{Coulomb}, $C$, and is defined as an ampere second \begin{equation}
		i = \frac{dq}{dt} \implies dq = idt \implies 1\;C = 1\;A\cdot 1\;s
	\end{equation}
\end{defn}


\begin{defn}{Coulomb's Law}{}
        \Emph{Coulomb's Law} states that for two charged particles of charge $q_1$ and $q_2$, with a relative position vector $\vec{r}$ from 1 to 2 we have that the force on $q_2$ by $q_1$ is \begin{equation}
                \vec{F}_{on2by1} = \frac{1}{4\pi\varepsilon_0}\frac{q_1q_2}{r}\hat{r}
        \end{equation}
\end{defn}


\begin{defn}{Electric Field}{}
        We define the \Emph{electric field} at a location in space and time by \begin{equation}
                \vec{E}(x,y,z,t) = \frac{\vec{F}(x,y,z,t)}{q}
        \end{equation}
        where $q$ is some non-zero test charge.
\end{defn}


\begin{rmk}{}{}
        Note that by definition, electric field lines point in the direction positive charges feel a force. That is positive charges will accelerate along field lines while negative charges will not.
\end{rmk}


\begin{cor}{}{}
        By the definition of electric field, we find that the electric field generated by a point charge $Q$ is \begin{equation}
                \vec{E} = \frac{1}{4\pi\varepsilon_0}\frac{Q}{|\vec{r}|^2}\hat{r}
        \end{equation}
\end{cor}


\begin{cor}{Superposition Principle}{}
        Suppose $N$ point charges exist in some region, each creating an electric field $\vec{E}_{A_i}$ at a point $A$. Then the net electric field at $A$ is \begin{equation}
                \vec{E}_A = \sum_{i=1}^N\vec{E}_{A_i}
        \end{equation}
\end{cor}

\section{Dipoles}

\begin{defn}{Electric Dipole}{}
        An \Emph{electric dipole} is a set of two equal and opposite charges separated by a small distance $s$.
\end{defn}


\begin{defn}{Dipole Moment}{}
        We define $\vec{p} := qs\hat{r}$ as the \Emph{dipole moment} of the dipole, where $\hat{r}$ points from the negative charge to the positive charge.
\end{defn}

\begin{prop}{Perpendicular Axis}{}
        It can be shown that the electric field along a line perpendicular to the charge axis and along the center of the charges is \begin{equation}
                \vec{E}_{net,\perp} = \frac{-k_e\vec{p}}{\left[d^2+\left(\frac{s}{2}\right)^2\right]}
        \end{equation}
        where $d$ is the perpendicular distance to the observation point. If $d >> s$ we can approximate and write \begin{equation}
                \vec{E}_{net,\perp} \approx \frac{-k_e\vec{p}}{d^3}
        \end{equation}
\end{prop}

\begin{prop}{Charge Axis}{}
        For the electric field along the charge axis we, let $y$ denote the distance (magnitude) from the center of the dipole, with $y > \frac{s}{2}$:  \begin{equation}
                \vec{E}_{net,axis} = \frac{2k_ey}{\left(y-\frac{s}{2}\right)^2\left(y+\frac{s}{2}\right)^2}\vec{p}
        \end{equation}
        If $0 \leq y < \frac{s}{2}$ then the above expression must be multiplied by a negative. For $y >> \frac{s}{2}$ we have the approximation \begin{equation}
                \frac{2k_e}{y^3}\vec{p}
        \end{equation}
\end{prop}




\section{Particles In Matter}

\begin{thm}{Conservation of Charge}{}
        The net charge of a system plus its surroundings cannot change. Pairwise creation and annihilation of charges is possible, but the net charge will remain the same.
\end{thm}


\subsection{Conductors and Insulators}

\begin{defn}{Conductor}{}
        In a \Emph{conductor} charges can move freely, and there are a large number of mobile charged particles. In equilibrium excess charge resides on the surface and the inner material is an equipotential.
\end{defn}


\begin{defn}{Insulator}{}
        In an \Emph{insulator} charges cannot move freely. In equilibrium excess charge is uniformly distributed throughout the volume as polarized dipole moments of the constituent molecules. In general, the inside of an insulator is not an equipotential.
\end{defn}

\begin{defn}{Semiconductor}{}
        In a \Emph{semi-conductor} charges can move freely but there are less of them than in conductors.
\end{defn}


\begin{defn}{Superconductors}{}
        \Emph{Superconductors} are materials that are perfect conductors once a certain temperature has been reached. allowing charge to flow without hindrance.
\end{defn}


\subsection{Polarization}

\begin{defn}{}{}
        A neutral (and usually symmetrical) atom is polarized by an external charge. Two opposite charges with a slight separation between them form an \Emph{electric dipole}
\end{defn}

\begin{defn}{Polarizability}{}
        We quantify the amunt of polarization induced in a material by its resulting dipole moment $\vec{p}$. In general the polarized molecule has a dipole moment \begin{equation}
                \vec{p} = \alpha \vec{E}
        \end{equation}
        where $\alpha$ is the \Emph{polarizability} of the material.
\end{defn}


\begin{defn}{}{}
        In conductors the sea of electrons is mobile, so the electrons will move when the conductor is polarized creating a \Emph{drift speed}\begin{equation}
                \overline{v} = uE_{net}
        \end{equation}
        where $u$ is the mobility of charge in the conductor.
\end{defn}

\begin{defn}{Drude Model}{}
        The \Emph{drude model} is a classical model for the motion of electrons in conductors in the presence of an electric field, and is quantified by \begin{equation}
                \frac{\Delta \vec{p}}{\Delta t} = \vec{F}_{net}
        \end{equation}
        Then we can write \begin{equation}
                \overline{v} = \frac{eE_{net}\overline{\Delta t}}{m}
        \end{equation}
        where $m$ is the mass of the charge,= and $\overline{\Delta t}$ is the average time between collisions. From this we find the mobility of charges to be \begin{equation}
                u = \frac{e\overline{\Delta t}}{m}
        \end{equation}
\end{defn}

\section{Distributed Charges}

\begin{rmk}{}{}
        To find the electric field due to a continuous charge distribution at a point $A$, first choose an arbitrary charge element $dq$ on the material. Determine an expression for the direction and magnitude of the field due to $dq$ at $P$ using Coulomb's Law for a point charge. Break you field expression into $dE_x$, $dE_y$, and $dE_z$. Consider symmetry to reduce certain components to zero before calculation. Then, integrate over the material to obtain each component of the net field. In particular, for $\alpha \in \{x,y,z\}$ integrate \begin{equation}
                E_{\alpha} = \int\lambda ds,\;\;E_{\alpha} = \int\int\eta dA,\;or\;E_{\alpha} = \int\int\int\rho dV
        \end{equation}
        defpending on the object, where $dq = \lambda ds, \eta dA, \rho dV$ depending if we have a curve, surface, or volume.
\end{rmk}


\begin{prop}{Line of charge}{}
        The electric field about a perpendicular axis through the center of a line of charge, say the $y$ axis, is \begin{equation}
                \vec{E} = \frac{2k_e\lambda}{y\sqrt{\frac{4y^2}{L^2} + 1}}\hat{y}
        \end{equation}
        where $\lambda = \frac{Q}{L}$, $Q$ is the charge of the line, and $L$ is its length. Taking the limit as $L$ goes to infinity we have \begin{equation}
                \vec{E} = \frac{2k_e\lambda}{y}\hat{y}
        \end{equation}
\end{prop}


\begin{prop}{Ring of Charge}{}
        The electric field about a perpendicular axis through the center of a ring of charge, say the $z$ axis, is \begin{equation}
                \vec{E} = \frac{k_eQz}{(R^2+z^2)^{3/2}}\hat{z}
        \end{equation}
        where $Q$ is the charge and $R$ the radius of the ring.
\end{prop}

\begin{prop}{Disk of Charge}{}
        The electric field about a perpendicular axis through the center of a disk of charge, say the $z$ axis, is \begin{equation}
                \vec{E} = \frac{\eta}{2\varepsilon_0}\left[1 - \frac{z}{\sqrt{R^2+z^2}}\right]\hat{z}
        \end{equation}
        where $\eta$ is the surface charge density of the disk. 
\end{prop}

\begin{prop}{Infinite Plane}{}
        Extending the last example to that of an infinite plane by taking $R \rightarrow \infty$ we have that \begin{equation}
                \vec{E} = \frac{\eta}{2\varepsilon_0}\hat{z}
        \end{equation}
\end{prop}

\begin{prop}{Capacitor}{}
        Using the infinite plane electric field we can approximate the electric field between a parallel plate capacitor by \begin{equation}
                E_{capacitor} = \frac{\eta}{\varepsilon_0}
        \end{equation}
\end{prop}


\begin{prop}{Thin Spherical Shell}{}
        Given a thin spherical shell (insulating or conducting) we have that \begin{equation}
                E = \left\{\begin{array}{ll} E_{inside} = 0 &\text{if $R > r$} \\ E_{outside} = \frac{k_eQ}{r^2} & \text{if $R < r$} \end{array}\right.
        \end{equation}
        if $R$ is the radius of the shell.
\end{prop}


\begin{prop}{Solid Sphere}{}
        For a conducting sphere, the excess charge must remain on the sphere's surface giving the same electric field as a spherical shell. For an insulating sphere the charge is distributed uniformly through the volume giving: \begin{equation}
                E = \left\{\begin{array}{ll} E_{inside} = \frac{k_eQ}{R^3}r &\text{if $R > r$} \\ E_{outside} = \frac{k_eQ}{r^2} & \text{if $R < r$} \end{array}\right.
        \end{equation}
\end{prop}






%%%%%%%%%%%%%%%%%%%%%%% - Chapter 2
\chapter{Electric Potentials}

\section{Potential Energy}

\begin{defn}{}{}
        Note that the Coulomb force is conservative force so we can define a potential function, called the \Emph{electric potential energy}, by \begin{equation}
                \Delta U_E = -\int_A^B\vec{F}_E \cdot d\vec{r}
        \end{equation}
        so that $\vec{F}_E = -\nabla U_E$. In particular, we have that $\Delta U_E = -W_C$ where $W_C$ is the work done by the coulomb force.
\end{defn}


\begin{defn}{Electric Potential}{}
        We define the \Emph{electric potential} associated with an electric potential energy by \begin{equation}
                \Delta V = \frac{\Delta U}{q} = -\int_A^B\vec{E}_E \cdot d\vec{r}
        \end{equation}
        In particular, we define $V_{\infty} = 0$ so we can express $V$ at any point in space as \begin{equation}
                V(B) = -\int_{\infty}^B\vec{E}_E \cdot d\vec{r}
        \end{equation}
        The units of $V$ are volts, $V = J/C$.
\end{defn}

\begin{rmk}{}{}
        The electric potential decreases in the direction of the existing electric field $\vec{E}$, while it increases in the opposite direction.
\end{rmk}


\begin{defn}{Equipotentials}{}
        We define an equipotential surface to be a surface in which $V = $ constant, so $\Delta V = 0$, $\Delta U = 0$, and $W_C = 0$.
\end{defn}


\begin{prop}{Infinite Wire}{}
        The potential difference between points at perpendicular distances $r_A$ and $r_B$ from an infinite line of charge is \begin{equation}
                \Delta V(A\rightarrow B) = 2k_e\lambda\ln\left|\frac{r_A}{r_B}\right|
        \end{equation}
\end{prop}

\begin{rmk}{}{}
        The electric potential at any point in space (relative to infinity) due to $N$ point charges is \begin{equation}
                V_{net} = \sum_{i=1}^NV_i = \sum_{i=1}^N\frac{k_eq_i}{r_i}
        \end{equation}
\end{rmk}


\begin{prop}{Ring of Charge}{}
        The potential difference between two points $A$ and $B$ along the line perpendicular to and through the center of a ring of charge is \begin{equation}
                \Delta V(A\rightarrow B) = k_eq\left[\frac{1}{\sqrt{R^2+z_B^2}} - \frac{1}{\sqrt{R^2+z_A^2}}\right]
        \end{equation}
        Taking the limit as $A$ goes to infinity we have \begin{equation}
                V_B = \frac{k_eq}{\sqrt{R^2+z_B^2}}
        \end{equation}
\end{prop}

\begin{rmk}{}{}
        One can determine the electric potential at a point in space using its definition as a line integral, or if the surface generating is the electric field is nice and known, then \begin{equation}
                V = \int dV = \int \frac{k_e}{r}dq
        \end{equation}
\end{rmk}


\begin{prop}{Disk of Charge}{}
        Above a charged disk (along a line perpendicular to the disk and through its center) we have the potential \begin{equation}
                \Delta V(A\rightarrow B) = -\frac{\eta}{2\varepsilon_0}\left[(z_B - z_A) - (\sqrt{z_B^2+R^2} - \sqrt{z_A^2 + R^2})\right]
        \end{equation}
        and relative to infinity \begin{equation}
                V = \frac{\eta}{2\varepsilon_0}\left[\sqrt{R^2+z^2} - |z|\right]
        \end{equation}
\end{prop}


\begin{prop}{Conducting Sphere}{}
        For a charged conducting sphere we have that \begin{equation}
                \Delta V(A\rightarrow B) = \frac{k_eQ}{r_B} - \frac{k_eQ}{r_A}
        \end{equation}
        if $r_B,r_A \geq R$, and \begin{equation}
                V_{outside} = \frac{k_eQ}{r}
        \end{equation}
        Inside the sphere we have an equipotential with \begin{equation}
                V_{inside} = \frac{k_eQ}{R}
        \end{equation}
\end{prop}


\begin{prop}{Insulating Sphere}{}
        Outside a charged insulating sphere we have\begin{equation}
                \Delta V(A\rightarrow B) = \frac{k_eQ}{r_B} - \frac{k_eQ}{r_A}
        \end{equation}
        if $r_B,r_A \geq R$, and \begin{equation}
                V_{outside} = \frac{k_eQ}{r}
        \end{equation}
        Inside the insulating sphere we do not have an equipotential and \begin{equation}
                V_A = \frac{k_eQ}{2R}\left[3 -  \frac{r_A^2}{R^2}\right]
        \end{equation}
        for $r_A < R$.
\end{prop}


\section{Energy Density and Insulators}

\begin{defn}{}{}
        An applied field polarizes an insulator. Due to this the net electric field inside the insulator is reduced due the opposing electric fields caused by the superposition of the induced dipoles, causing \begin{equation}
                \vec{E}_{ins} = \frac{\vec{E}_{app}}{\kappa}
        \end{equation}
        where $\kappa$ is the dielectric constant of the material, and \begin{equation}
                \Delta V_{ins} = \frac{\Delta V_{app}}{\kappa}
        \end{equation}
\end{defn}

\begin{defn}{Energy Density}{}
        The \Emph{energy density} in a region of space is given by \begin{equation}
                \frac{\Delta U}{\Delta Vol} = \frac{1}{2}\varepsilon_0 E^2
        \end{equation}
\end{defn}








%%%%%%%%%%%%%%%%%%%%%%% - Chapter 3
\chapter{Magnetic Fields}


\section{Detecting Magnetic Fields and Electron Currents}

\begin{defn}{Electron Current}{}
        The \Emph{electron current} $i$ is the number of electrons per second that enter a section of a conductor. As electrons drift through the wire their collisions with atomic cores heat up the wire, preventing them from speeding up continuously. 
\end{defn}

\begin{defn}{Detecting Magnetic Fields}{}
        We can use the needle of a compass to detect and measure magnetic fields, and in particular the Earth's magnetic field. If we measure the deflection $\theta$ from the direction of the Earth's magnetic field due to a perpendicular known, generated magnetic field $B_{wire}$ we have that \begin{equation}
                \tan(\theta) = \frac{B_{wire}}{B_{Earth}}
        \end{equation}
\end{defn}


\section{Biot Savart Law}

\begin{defn}{Biot-Savart Law}{}
        A moving charge generates a magnetic field. In particular Biot-Savart's Law states that \begin{equation}
                \vec{B}_{pt,charge} = \frac{\mu_0}{4\pi}\frac{q\vec{v}\times \hat{r}}{r^2}
        \end{equation}
        Biot-Savart's Law is only correct for $v << c$.
\end{defn}


\section{Conduction and Currents}

\begin{defn}{Model of Conduction}{}
        For conduction of electrons through a wire we model the electron current by \begin{equation}
                i_e = n_e A v_d = n_e A \frac{e\tau}{m}E
        \end{equation}
        where $n_e$ is the electron density and $\tau$ is the average time between collisions.
\end{defn}



\begin{defn}{Current}{}
        For general \Emph{charge carriers} in a conductor, of charge $q$, we have that the current in the conductor is \begin{equation}
                I = |q|nAv_d
        \end{equation}
        where $n$ is the charge carrier density, $A$ is the cross-sectional area, and $v_d$ is the drift speed superimposed on the random motion of the charge carriers.
\end{defn}

\begin{prop}{Superposition Principle}{}
        Magnetic fields obey the superposition principle, so given $N$ magnetic fields at a point $P$ we have that the net magnetic field at $P$ is \begin{equation}
                \vec{B}_{net} = \sum_{i=1}^N\vec{B}_i
        \end{equation}
\end{prop}


\begin{defn}{Magnetic Field due to Currents}{}
        Given a current carrying wire with wire segment $d\vec{s}$, pointing in the direction of conventional current, we have that the magnetic field at a position $\vec{r}$ relative to the segment due to the current in that segment is \begin{equation}
                d\vec{B} = \frac{\mu_0}{4\pi} \frac{Id\vec{s}\times \hat{r}}{r^2}
        \end{equation}
        We then integrate over the wire to get the net magnetic field.
\end{defn}

\begin{prop}{Straight Long Wire}{}
        Along a line perpendicular to a wire with current running through it, the magnitude of the magnetic field is given by \begin{equation}
                B_{wire} = \frac{\mu_0IL}{2\pi d\sqrt{L^2+d^2}}
        \end{equation}
        for length $L$ and distance $d$. For $L >> d$ we have that \begin{equation}
                B_{wire} \approx \frac{\mu_0I}{2\pi d}
        \end{equation}
\end{prop}


\begin{prop}{Loop}{}
        The magnitude of the magnetic field along a line perpendicular to a circular loop carrying current is \begin{equation}
                B_{loop} = \frac{\mu_0IR^2}{2(R^2+z^2)^{3/2}}
        \end{equation}
        If the location is at the center of the loop ($z = 0$) then we have \begin{equation}
                B_{loop} = \frac{\mu_0I}{2R}
        \end{equation}
        Conversely, if $z >> R$ then we have \begin{equation}
                B_{loop} = \frac{\mu_0IR^2}{2z^3}
        \end{equation}
\end{prop}

\begin{defn}{Magnetic Moment}{}
        We define the magnetic moment of a closed loop carrying current as \begin{equation}
                \vec{\mu} = I\vec{A}
        \end{equation}
        where $\vec{A}$ points in the direction of the generated magnetic field, so \begin{equation}
                \vec{B}_{axis} = \frac{\mu_0}{4\pi}\frac{2\vec{\mu}}{z^3}
        \end{equation}
        in the case of a circular loop.
\end{defn}


\subsection{Atomic Structures}

\begin{defn}{Electron Orbits}{}
        We define the magnetic dipole moment of an electron orbiting a nucleus to be \begin{equation}
                \mu_{e^-} = \frac{1}{2}eRv
        \end{equation}
        Then the angular momentum is $L = Rmv$ since it is in circular motion, so $Rv = L/m$ and we have \begin{equation}
                \mu_{e^-} = \frac{1}{2}e\frac{L}{m}
        \end{equation}
        Note that the angular momentum is quantized such that $L = N\bar{h}$ for $N \in \N$.
\end{defn}





%%%%%%%%%%%%%%%%%%%%%%% - Chapter 4
\chapter{Electric Fields and Circuitry}

\section{Circuits}

\begin{defn}{}{}
        A circuit is said to be in a \Emph{steady state} if the charges are moving ($\overline{v} \neq 0$), but their drift velocities at any location do not vary in time, and there is no change in the deposits of excess charge in the circuit.
\end{defn}

\begin{thm}{Kirchhoff Node Rule}{}
        The \Emph{Law of Conservation of Current} states that the current is the same at all points in a current carrying wire with no junctions. Moreover, \Emph{Kirchhoff's Node Rule} states that at a junction \begin{equation}
                \sum_i I_{in,i} = \sum_j I_{out,j}
        \end{equation}
\end{thm}

\begin{rec}{}{}
        Recall that electron current is equal to $i_e = nA\overline{v}$, and by the Law of Conservation of Current, even if one of the variables change, $i_e$ must remain constant.
\end{rec}


\begin{rmk}{}{}
        The electric field which points along the wire and causes current is due to a build of charge along the wire caused by the presence of a battery or some other potential difference. Moreover, this charge build-up changes as a gradient through the wire. Additionally, there is no current in an open circuit, but once closed it will take a few nanoseconds for charges to redistribute at the point of contact generating the desired electric field through the wire, causing current to flow.
\end{rmk}


\begin{defn}{Feedback}{}
        Feedback ensures equalization of current, and can be summed up as follows: surface charge will rearrange itself in such a way as to ensure that the electric field points along the wire and has appropriate magnitude to drive the appropriate amount of steady state current. If the electric field is not along the wire charges will be pushed into the wires walls, redistributing the charge and adjusting the electric field so that it doesn't point towards any walls, and instead along the wire.
\end{defn}


\begin{rmk}{}{}
        In general, we can express current as $I = nuEA$, which will be constant along a wire without any junctions in a steady-state. Hence, if the parameters change at some type of boundary we must have that $n_1u_1E_1A_1 = n_2u_2E_2A_2$.
\end{rmk}


\begin{thm}{Kirchhoff's Loop Rule}{}
        \Emph{Kirchhoff's Loop Rule} states that the sum of the potential differences across all elements around any closed circuit loop must be zero:\begin{equation}
                \sum_i\Delta V_i = 0
        \end{equation}
        This follows from the conservation of energy principle, so $\Delta U(A \rightarrow A) = 0$. Recall that potential increases when moving against the electric field and decreases when moving with the electric field.
\end{thm}


\begin{defn}{}{}
        The rate of energy generation (power) due to a battery with potential difference $\epsilon$ is \begin{equation}
                \frac{dU}{dt} = P = i\epsilon
        \end{equation}
        Similarly, the rate of energy dissipation across a resistor with potential difference $V$ is $P = iV$.
\end{defn}



\section{Currents and Current Density}

\begin{defn}{Electric Current}{}
        We define electric current by \begin{equation}
                i = \frac{dq}{dt} \implies q = \int dt = \int_0^ti(t)dt
        \end{equation}
\end{defn}


\begin{defn}{Current Density}{}
        We define the current density vector $\vec{J}$ such that it is in the direction of the moving charges. Then the current is \begin{equation}
                i = \int\vec{J} \cdot d\vec{A}
        \end{equation}
        If the current is uniform and parallel to $A$ then $i = JA$, so $J  =\frac{i}{A}$.
\end{defn}

\begin{cor}{Drift Speed}{}
        Recall that $i = nAev_d$. It then follows that \begin{equation}
                J = nev_d
        \end{equation}
\end{cor}


\begin{rmk}{}{}
        The charge carrier density $n$ can be expressed by \begin{equation}
                n = density\times \frac{N_Az}{\text{Molar mass}}
        \end{equation}
        where $N_A$ is Avogadro's number.
\end{rmk}


\section{Resistance and Resistivity}


\begin{defn}{}{}
        The resistance of an object is defined as \begin{equation}
                R = \frac{V}{i}
        \end{equation}
        Note that the resistance depends on both the material and geometry of the object. Equivalently the resistence can be written as \begin{equation}
                R = \rho \frac{L}{A}
        \end{equation}
        where $\rho$ is the \Emph{resistivity} of the material, and is equal to $\rho = \frac{E}{J}$ (it is a property of the material). We can also define \Emph{conductivity} which is $\sigma = \frac{1}{\rho}$.
\end{defn}

\begin{thm}{Ohm's Law}{}
        \Emph{Ohm's Law} asserts that the resistance in \Emph{ohmic materials} is approximately constant so the current depends linearly on the potential difference across a circuit element. Conducting materials are ohmic. Note that microscopically we can write \begin{equation}
                \rho = \frac{m}{e^2n\tau}
        \end{equation}
        where $\tau$ is the average time between collisions.
\end{thm}

\begin{rmk}{}{}
        The resistivity of a material is a function of the temperature, and in particular \begin{equation}
                \rho = \rho_0(1+\alpha(T - T_0))
        \end{equation}
        For metals we have $\alpha > 0$ while for semi-conductors $\alpha < 0$.
\end{rmk}


\section{Superconductors}

\begin{defn}{Superconductors}{}
        \Emph{Superconductors} are a class of metals and compounds whose resistance decreases to zero once a certain critical temperature $T_C$ is reached. $T_C$ is sensitive to chemical composition, pressure, and molecular structure. A \Emph{type 1 superconductor} is characterized by \begin{enumerate}
                \item Sharp transition to superconducting state
                \item Requires extremely cold temperatures to become superconducting
                \item Perfect diamagnetism - (magnets kill the superconducting effect)
                \item BCS theory - electrons team up in ``Cooper pairs"
                \item Soft superconductors
        \end{enumerate}
        A \Emph{type 2 superconductor} is characterized by \begin{enumerate}
                \item Except for Vanadium, Technetium and Niobium, metallic compounds and alloys and perovskites
                \item Higher $T_C$ than type 1
                \item Mechanism not completely understood
                \item Meissner effect
                \item Hard superconductors
        \end{enumerate}
\end{defn}

\section{Batteries and Circuit Calculations}

\begin{defn}{EMF}{}
        The \Emph{electromotive force}, emf, is the energy per unit charge that is converted reversibly from chemical, mechanical, or other forms of energy into electrical energy in a battery. Real batteries have internel resistence $r$, so the voltage provided to the circuit is $\epsilon -ir$, not simply the emf $\epsilon$.
\end{defn}


\begin{prop}{Equivalent Resistence}{}
        Series wiring means that the circuit elements are connected in such a way that there is the same current running through each device. If resistors are connected in series we have the equivalent resistence \begin{equation}
                R_{eq} = \sum_{i=1}^NR_i
        \end{equation}
        Next, parallel wiring means that the devices are connected in such a way that the same voltage is applied across each device. In this case the equivalent resistence for resistors connected in parallel is \begin{equation}
                \frac{1}{R_{eq}} = \sum_{i=1}^N\frac{1}{R_i}
        \end{equation}
\end{prop}


\section{Capacitance}

\begin{defn}{Capacitance}{}
        The charge on a capacitor plate and the potential difference across the capacitor are proportional. The proportionality constant is the \Emph{capacitance}, $C$, of the capacitor, such that \begin{equation}
                C = \frac{q}{\Delta V}
        \end{equation}
        The unit of capacitance is farads, $F = C/V$
\end{defn}


\begin{rmk}{}{}
        To calculate the capacitance of a capacitor we determine the charge on the surface using $Q = \eta A$, then we calculate the potential difference between the capacitor plates before taking the ration $Q/\Delta V$.
\end{rmk}


\begin{prop}{Parallel Plate}{}
        The capacitance of a parallel plate capacitor is \begin{equation}
                C = \frac{\varepsilon_0 A}{d}
        \end{equation}
\end{prop}

\begin{prop}{Cylindrical}{}
        The capacitance of a cylindrical capacitor with inner cylinder of radius $a$ and an outer cylinder of radius $b$ is \begin{equation}
                C = \frac{2\pi L\varepsilon_0}{\ln\left|\frac{b}{a}\right|}
        \end{equation}
\end{prop}


\begin{prop}{Spherical}{}
        The capacitance of a spherical capacitor with inner radius $a$ and an outer radius $b$ is \begin{equation}
                C = \frac{4\pi \varepsilon_0}{\frac{1}{a} - \frac{1}{b}}
        \end{equation}
\end{prop}


\begin{thm}{}{}
        The equivalent capacitance of capacitors connected in series is \begin{equation}
                \frac{1}{C_{eq}} = \sum_{i=1}^N\frac{1}{C_i}
        \end{equation}
        The equivalent capacitance of capacitors connected in parallel is \begin{equation}
                C_{eq} = \sum_{i=1}^NC_i
        \end{equation}
\end{thm}



\begin{defn}{Energy Stored in an Electric Field}{}
        The potential energy stored in the capacitor upon charging (due to external forces) is \begin{equation}
                U = \frac{1}{2}CV^2
        \end{equation}
        The energy density is defined to be \begin{equation}
                u = \frac{U}{Ad} = \frac{CV^2}{2Ad}
        \end{equation}
        For a parallel plate capacitor we have \begin{equation}
                u = \frac{1}{2}\varepsilon_0 E^2
        \end{equation}
\end{defn}


\begin{defn}{Dielectrics}{}
        If a dielectric $\kappa$ is inserted between the plates of a capacitor while a battery is connected the voltage across the capacitor will initially decrease before increasing back to its original value, giving \begin{equation}
                C' = \kappa \frac{q}{\Delta V} = \kappa C
        \end{equation}
        Due to this, the potential energy stored in the capacitor becomes \begin{equation}
                U' = \frac{1}{2}C'\Delta V^2 = \kappa \frac{1}{2}C\Delta V^2 = \kappa U
        \end{equation}
        If no battery is connected the potential difference will not increase back to its original energy and the potential energy stored in the capacitor becomes \begin{equation}
                U' = \frac{1}{2}Q\Delta V' = \frac{1}{\kappa} \frac{1}{2}Q\Delta V = \frac{U}{\kappa}
        \end{equation}
\end{defn}


\begin{rmk}{}{}
        Capacitors with different dielectrics side by side act like capacitors in parallel each with a different dielectric (and a different area). Capacitors with dielectrics stacked on top of each other act like capacitors in series each with a different dielectric in it (and a different width).
\end{rmk}


\section{RC Circuits}


\begin{defn}{}{}
        During the charging phase of the capacitor we have that \begin{equation}
                q(t) = \epsilon C\left(1-e^{-t/RC}\right)
        \end{equation}
        and \begin{equation}
                V_C(t) = \epsilon\left(1-e^{-t/RC}\right)
        \end{equation}
        During discharge we have \begin{equation}
                q(t) =\epsilon C e^{-t/RC}
        \end{equation}
        and \begin{equation}
                V_C(t) = \epsilon e^{-t/RC}
        \end{equation}
\end{defn}









%%%%%%%%%%%%%%%%%%%%%%% - Chapter 5
\chapter{Magnetic Force}


\section{Force on a particle}

\begin{defn}{Magnetic Force}{}
        The magnetic force on a point charge $q$ in the presence of a magnetic field $\vec{B}$ is \begin{equation}
                \vec{F}_B = q\vec{v} \times \vec{B}
        \end{equation}
\end{defn}

\begin{rmk}{}{}
        In the presence of a uniform magnetic field a moving charge will often exhibit helical or circular motion. If $\vec{v}\perp\vec{B}$ then $qvB = \frac{mv^2}{r}$ so we have a radius of orbit of \begin{equation}
                r = \frac{mv}{qB}
        \end{equation}
        and period \begin{equation}
                T = \frac{2\pi r}{v} = \frac{2\pi m}{qB}
        \end{equation}
        In general if $angle(\vec{v},\vec{B}) = \phi$ then \begin{equation}
                r = \frac{mv\sin(\phi)}{qB}
        \end{equation}
        but still \begin{equation}
                T = \frac{2\pi m}{qB}
        \end{equation}
\end{rmk}


\begin{defn}{Relativistic}{}
        Relativistic momentum is given by $p = \gamma mv$. Classically we have \begin{equation}
                \left|\frac{d\vec{p}}{dt}\right| = p\omega
        \end{equation}
        with $\omega = \frac{v}{R}$. Relativistically \begin{equation}
                \left|\frac{d\vec{p}}{dt}\right| = \omega\gamma m v = |q| vB\sin(\phi)
        \end{equation}
        so we get $\omega = \frac{|q|B}{\gamma m}$ so $T = \frac{2\pi}{\omega} = \frac{2\pi\omega m}{qB}$
\end{defn}


\begin{defn}{Lorentz Force}{}
        The \Emph{Lorentz Force} on a point charge in teh presenc of an electric and magnetic field is \begin{equation}
                \vec{F} = \vec{F}_E +\vec{F}_B = q\vec{E}  + q\vec{v}\times \vec{B}
        \end{equation}
\end{defn}


\begin{rmk}{}{}
        In a velocity selector we have that the Lorentz Force is zero.
\end{rmk}


\begin{rmk}{Cyclotrons}{}
        In a cyclotron a charged particle is accelerated while going in cicular-like paths at a fixed frequency $f = f_{osc}$. Recall $f = \frac{\omega}{2\pi} = \frac{|q|B}{2\pi m}$, and $E_{K,final} = q\Delta v\cdot n$ where $n$ is the number of passes through the accelerating gap.
\end{rmk}


\section{Magnetic Force on Wires}

\begin{defn}{}{}
        The magnetic force on a segment of current carrying wire is \begin{equation}
                d\vec{F} = id\vec{L}\times \vec{B}
        \end{equation}
        where $d\vec{L}$ points in the direction of conventional current. We then integrate along the wire to get the total force. For a straight current carrying wire in a uniform magnetic field \begin{equation}
                \vec{F} = I\vec{L}\times \vec{B}
        \end{equation}
\end{defn}


\section{Hall Effect}


\begin{claim}{Hall Effect}{}
        The \Emph{Hall Effect} asserts that if a current carrying wire is subject to a uniform perpendicular magnetic field, the charge carriers will experience a magnetic force causing a potential difference to be created across the width of the wire. With width $w$, this \Emph{hall voltage} is \begin{equation}
                V_H = wv_dB = \frac{IB}{dne}
        \end{equation}
        where $v_d = \frac{I}{Ane} = \frac{I}{wdne}$.
\end{claim}

\subsection{Motional EMF}

\begin{prop}{}{}
        For a conductor of length $L$ moving velocity $\vec{v}$ perpendicular to a uniform magnetic field $\vec{B}$, Hall's Effect predicts a potential difference $\Delta V = EL$ to be created along the length of the conductor where $E = vB$. It follows that the induced motional emf is \begin{equation}
                \epsilon = \Delta V = vBL
        \end{equation}
\end{prop}

\begin{rmk}{}{}
        If the conductor in the previous proposition is connected to a conducting wire, creating a circuit that changes size, current is induced. Moreover, this current causes a magnetic force $F_B = ILB$ on the bar in the opposite direction of its motion. Note that the current induced is \begin{equation}
                I = \frac{\Delta V}{R} = \frac{vBL}{R}
        \end{equation}
        Then, the acceleration of the bar due to the magnetic force is \begin{equation}
                a = \frac{ILB}{m} = \frac{L^2B^2v}{mR} = \frac{dv}{dt}
        \end{equation}
        Note that the rate of change of the flux through the circuit and the accompying emf across the bar are proportional.
\end{rmk}


\begin{rmk}{}{}
        If a force $F = ILB$ is applied to the bar to keep it moving at constant velocity, the power delivered to the circuit is \begin{equation}
                P = Fv = \frac{L^2B^2v}{R}v = \frac{(LBv)^2}{R} = \frac{\Delta V^2}{R} = I\Delta V
        \end{equation}
        (Note that the power dissipated by the circuit is $P = IV$, so this checks out)
\end{rmk}


\section{Flux}


\begin{defn}{}{}
        The \Emph{magnetic flux} through a surface is defined by the surface integral \begin{equation}
                \Phi_B = \int\int \vec{B}\cdot d\vec{A}
        \end{equation}
\end{defn}

\subsection{Induction}

\begin{thm}{Faraday's Law of Induction}{}
        \Emph{Faraday's Law of Induction} state that the emf induced in a circuit is given by \begin{equation}
                \epsilon = -N\frac{d\Phi_B}{dt}
        \end{equation}
        where $N$ is the number of loops in the circuit.
\end{thm}


\begin{thm}{Lenz' Law}{}
        \Emph{Lenz' Law} states that current induced in a circuit by a changing flux will be such that it opposes the change.
\end{thm}


\begin{thm}{}{}
        The electric field induced around a loop of radius $r$ in a changing magnetic field is \begin{equation}
                E = -\frac{dB}{dt}\frac{r}{2}
        \end{equation}
\end{thm}

\section{Magnetic Torque}

\begin{defn}{}{}
        The \Emph{magnetic torque} on a current carrying loop is defined to be \begin{equation}
                \vec{\tau} = \vec{\mu}\times \vec{B}
        \end{equation}
        where as before $\vec{\mu} = I\vec{A}$, which points in the direction of the generated magnetic field.
\end{defn}

\begin{defn}{}{}
        The work needed to rotate a current carrying loop from an angle $\theta_i$ to $\theta_f$ is \begin{equation}
                W = \Delta U_m = \int_{\theta_i}^{\theta_f} \tau d\theta = -\mu B\cos(\theta)\rvert_{\theta_i}^{\theta_f}
        \end{equation}
        Hence, in general we define the potentional energy for a magnetic dipole subject to a torque by \begin{equation}
                U_m = -\vec{\mu}\cdot \vec{B}
        \end{equation}
\end{defn}








%%%%%%%%%%%%%%%%%%%%%%% - Chapter 6
\chapter{Field Patterns in Space and Maxwell's Equations}


\section{Gauss' Law}

\begin{defn}{}{}
        The \Emph{electric flux} through a surface is defined by the surface integral \begin{equation}
                \Phi_E = \int\int \vec{E} \cdot d\vec{A}
        \end{equation}
\end{defn}


\begin{thm}{Gauss' Law}{}
        \Emph{Gauss' Law} states that the electric flux through a closed surface is equal to the charge enclosed divided by the permittivity of free space: \begin{equation}
                \oiint\limits \vec{E}\cdot d\vec{A} = \frac{q_{enc}}{\varepsilon_0}
        \end{equation}
\end{thm}


\begin{rmk}{}{}
        If $\vec{E}$ and $\vec{A}$ are parallel, perpendicular, or in general constant over some closed surface, we can use Gauss' Law to determine $\vec{E}$.
\end{rmk}


\begin{thm}{Gauss' Law for Magnetism}{}
        For magnetism Gauss' Law is a statement of the fact that there are no magnetic monopoles \begin{equation}
                \oiint\limits \vec{B} \cdot d\vec{A} = 0
        \end{equation}
\end{thm}

\section{Ampere's Law}


\begin{thm}{Ampere's Law}{}
        \Emph{Ampere's Law} states that given any closed loop $\mathcal{C}$, we have \begin{equation}
                \oint\limits_{\mathcal{C}} \vec{B}\cdot d\vec{s} = \mu_0I_{enc}
        \end{equation}
\end{thm}

\section{Divergence Forms}

\begin{defn}{Gauss' Law Divergence}{}
        Gauss' Law can be expressed in terms of the divergence of $\vec{E}$ using the divergence theorem for surface integrals, giving \begin{equation}
                \nabla \cdot \vec{E} = \frac{\rho}{\varepsilon_0}
        \end{equation}
        and sicne $\vec{E} = -\nabla V$ we have \begin{equation}
                \nabla^2V = -\frac{\rho}{\varepsilon_0}
        \end{equation}
        where $\rho$ is the charge density in the region.
\end{defn}


\begin{defn}{Ampere's Law Curl}{}
        Ampere's Law can be expressed in terms of the curl of $\vec{B}$ using Stoke's Theorem for line integrals, which gives \begin{equation}
                \nabla \times \vec{B} = \mu_0 \vec{J}
        \end{equation}
        where $\vec{J}$ is the charge density.
\end{defn}

\begin{defn}{Ampere-Maxwell Law}{}
        For non-continuous conduction currents (i.e. through capacitors), Ampere's Law can be extended to the \Emph{Ampere-Maxwell Law} \begin{equation}
                \oint\limits \vec{B}\cdot d\vec{l} = \mu_0 i_{enc} + \mu_c\varepsilon_0\frac{d\Phi_E}{dt}
        \end{equation}
        where the second term is known as the \Emph{displacement current}.
\end{defn}


\begin{defn}{Faraday's Law of Induction Integral}{}
        Faraday's Law of Induction can be stated as the closed loop integral \begin{equation}
                \oint\limits\vec{E}\cdot d\vec{s} = emf = -N\frac{d\Phi_B}{dt}
        \end{equation}
        where for an electric field in space without any conducting loops present \begin{equation}
                \oint\limits \vec{E}\cdot d\vec{s} = -\frac{d\Phi_B}{dt}
        \end{equation}
\end{defn}


\section{Inductors}

\begin{defn}{}{}
        An inductor (solenoid) is used to produce a desired uniform magnetic field $\vec{B}$. The \Emph{inductance} associated with an inductor is defined to be \begin{equation}
                L := \frac{N\Phi_B}{i}
        \end{equation}
        where $i$ is the current in the solenoid and $N$ is the number of loops. In particular if we consider the length $l$ near the middle of a solenoid, and write $n = N/l$ for the loop density, we have $B = \mu_0in$ so \begin{equation}
                L = \mu_0n^2lA
        \end{equation}
\end{defn}

\begin{thm}{Self-Inductance}{}
        If two coils (inductors) are near each other, a current $i$ in one produces a magnetic flux through the second. If $i$ changes the induced emf will appear in the second coil, which causes a current, and hence a changing flux in the first, which produces another emf, this time in the first coil. This process is called \Emph{self-induction}. For any inductor $Li  = N\Phi_B$, so by Faraday's Law \begin{equation}
                \epsilon_L = -N\frac{d\Phi_B}{dt} = -L\frac{di}{dt}
        \end{equation}
\end{thm}


\begin{rmk}{}{}
        Inductors oppose the change of current in a circuit. That is, if current increases due to a battery, the inductor will oppose the battery, while if current is decreasing the inductor will preserve the current.
\end{rmk}



\begin{rmk}{}{}
        The rate at which energy is stored in the magnetic field of the inductor is \begin{equation}
                \frac{dU_B}{dt} = Li\frac{di}{dt}
        \end{equation}
        Then, the potential energy stored is \begin{equation}
                U_B = \frac{1}{2}Li^2
        \end{equation}
        Moreover, the energy density is \begin{equation}
                u_B = \frac{U_B}{Al} = \frac{Li^2}{2lA}
        \end{equation}
        where $L/l = \mu_0 n^2A$, so $u_B = \frac{1}{2}\mu_0n^2i^2$, which implies that for a solenoid with $B = \mu_0in$, $u_B = \frac{B^2}{2\mu_0}$.
\end{rmk}

\begin{rmk}{LC Circuit}{}
        Consider a circuit with a capacitor and inductor, starting with the capacitor fully charged. By Kirchhoff's Loop rule we have that \begin{equation}
                \Delta V_C + \Delta V_L = \frac{q}{C} - L \frac{di}{dt} = 0
        \end{equation}
        Substituting $i = -\frac{dq}{dt}$ (since the capacitor is discharging) we get \begin{equation}
                \frac{q}{C} = -L\frac{d^2q}{dt^2}
        \end{equation}
        so \begin{equation}
                \frac{d^2q}{dt^2} = \frac{-1}{LC}q
        \end{equation}
        which is the form of a harmonic oscillator. With initial condition $q(0) = Q_0$ we have that \begin{equation}
                q(t) = Q_0\cos(\omega t + \phi_0)
        \end{equation}
        where evidentally $\phi_0 = 0$ by the initial condition. Differentiating and substituting $i = -\frac{dq}{dt}$ we obtain \begin{equation}
                i(t) = i_{max}\sin(\omega t + \phi_0)
        \end{equation}
        where $i_{max} = Q_0\omega$.
\end{rmk}


\begin{rmk}{RLC Circuit}{}
        An RLC circuit acts like a damping harmonic oscillator. The current is the same through each circuit component \begin{equation}
                i(t) = i_0\sin(\omega t+\phi_0)
        \end{equation}
        where $\omega^2 = \frac{1}{LC}$. Moreover, the voltage across the resistor \begin{equation}
                V_R(t) = Ri(t)
        \end{equation}
        is in phase with the current, while the voltage across the inductor \begin{equation}
                V_L = -L\frac{di}{dt} = -Li_0\omega\cos(\omega t+\phi_0)
        \end{equation}
        is shifted, along with the voltage across the capacitor \begin{equation}
                V_C = \frac{q}{C} = \frac{i_0}{\omega}\cos(\omega t+\phi_0)
        \end{equation}
\end{rmk}






%%%%%%%%%%%%%%%%%%%%%%% - Chapter 7
\chapter{Electromagnetic Radiation}


\section{Accelerated Charges}

\begin{defn}{}{}
        If a charge $q$ is accelerated briefly by $\vec{a}$, then at an angle $\theta$ around the particle relative to the acceleration vector, we have electric radiation of \begin{equation}
                \vec{E}_{rad} = k_e\frac{-q\vec{a}_{\perp}}{c^2r}
        \end{equation}
        where $a_{\perp} = a\sin(\theta)$ and points in the direction $\vec{a} - proj_{\vec{r}}(\vec{a})$ (making a right triangle with one angle equal to $\theta$).
\end{defn}







\end{document}


%%%%%% END %%%%%%%%%%%%%
