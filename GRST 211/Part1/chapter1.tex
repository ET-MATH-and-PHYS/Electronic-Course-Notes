%%%%%%%%%%%%%%%%%%%%% chapter.tex %%%%%%%%%%%%%%%%%%%%%%%%%%%%%%%%%
%
% sample chapter
%
% Use this file as a template for your own input.
%
%%%%%%%%%%%%%%%%%%%%%%$% Springer-Verlag %%%%%%%%%%%%%%%%%%%%%%%%%%
%\motto{Use the template \emph{chapter.tex} to style the various elements of your chapter content.}
\chapter{Breaking Down Medical Terms, Building Up Medical Definitions}
\label{BreakDown} % Always give a unique label
% use \chaptermark{}
% to alter or adjust the chapter heading in the running head


%%% Questions to think about
%The \textbf{Thesis} or general sense of the article is ...

%The \textbf{method} the author uses to argue their point is ...

%In their \textbf{analysis} the author uses tools such as ... 

%Additionally they conclude ...

%What connections does the author portray with regard to \textbf{space}, \textbf{relationships}, \textbf{occupation}, and \textbf{religion}.


\abstract{}


\section{Notes}
\label{sec:NOTE1}

\textbf{Types of medical terminology} (roughly):

\begin{itemize}
    \item[i)] Greek and Latin terms that have entered the English language in an anglicized form. Some of them for example sperm, artery, and nerve, were incorporated so long ago that we have ceased to think of them as foreign.
    \item[ii)] Terms that have entered the English language in their original form. Some terms such as ganglion are Greek, but the majority are Latin terms used in anatomy, such as sacrum, vena cava, and fossa ovalis.
    \item[iii)] Compound terms that were systematically devised. Many utilize Greek base words, as in oligomenorrhea, since the Greek language is particularly suited to forming compounds. However, Latin compounds, such as labiogingival, do occur, as do hybrid terms such as neonatal that mix Latin and Greek elements.
\end{itemize}


\begin{rmk}
    \textbf{First objective:} breaking compound medical terms down into plain English that we can understand.
\end{rmk}


\subsection{Breaking Down Medical Terms}

\begin{defn}{Base}
    The \textbf{base} carries the basic meaning and sense of a word. Bases always make some sort of sense on their own, since they are modified by \textbf{nouns} (`things'), \textbf{adjectives} (`describing' words), or \textbf{verbs} (`doing' words), but their endings are missing. A term can include more than one base, as in psychosomatic, where the bases are `psych' and `som,' meaning `mind' and `body,' respectively.
\end{defn}


\begin{defn}{Suffix}
    The \textbf{suffix} is added to the end of the base to make meaningful sense. It can be as little as one letter, often a few letters, sometimes more. The suffix usually makes no sense on its own, but added to the end of the base it forms a complete noun, adjective, or verb. Occasionally, a word might have two suffixes following each other.
\end{defn}


\begin{defn}{Prefix}
    A \textbf{prefix} can be added to the front of the base. It can be as little as one letter, often a few letters, sometimes more. Not all medical terms include a prefix. The prefix does not make sense on its own; it modifies or adds extra information about the base, telling us how, where, or to what degree something occurs. Prefixes are derived from Greek and Latin adverbs, orprepositions.
\end{defn}

\begin{longtable}{ c p{0.2\textwidth} p{0.2\textwidth} p{0.2\textwidth}}
    \caption{Word components.}
    \hline
        & prefix & base & suffix \\ \hline
        Position in word & beginning & middle & end \\ \hline
        Is it essential? & no & yes & yes \\ \hline 
        More than one? & hardly ever & often & sometimes \\ \hline
        Function & adds extra information about the base---often how, where, or to what degree & carries the basic meaning---a modified noun, adjective, or verb with a bit missing & completes the sense of the base---in combination, the suffix and base make a noun, adjective, or verb \\ \hline
    \label{tab:parts}
\end{longtable}


\begin{defn}{Combining Vowel}
    A \textbf{combining vowel} is added to the end of the base (before the suffix) to make pronunciation easier. It adds nothing at all to the meaning. The combining vowel is always considered as added to the end of the base, not the beginning of the suffix.
\end{defn}

\begin{rmk}{Strategy}
    When faced with a compound medical term, what do you do? \begin{itemize}
        \item[i)] Identify all the parts. It is a good idea to write the term out, so that you can mark the parts as you identify them.
        \item[ii)] You know that there will at least be one base and a suffix, so find them first. There may be a combining vowel between the base and suffix.
        \item[iii)] Still have something left over? There probably is not a second suffix, but is there a second base? There may be a combining vowel between bases. Is there a prefix? Mark everything.
        \item[iv)] Make sure that nothing is left over.
        \item[v)] Write down the meaning of each individual part. Then, go on to build up the definition.
    \end{itemize}
\end{rmk}

\begin{defn}{Order}
    The \textbf{definition order} for a term is \textbf{suffix-prefix-base}.
\end{ddefn}




\subsection{Building Up Medical Definitions}

\begin{rmk}{Strategy for Building}
    Once everything is marked and sorted proceed as follows: \begin{itemize}
        \item[i)] In all cases, \textbf{BEGIN WITH THE SUFFIX}. This is an important point, and it gets the definition off to the right start. It will tell you whether the whole medical term is a noun, an adjective, or a verb.
        \item[ii)] If you only have a base and a suffix, then the base comes next.
        \item[iii)] If you have a base, a suffix, and a prefix, then the prefix, since it modifies the base, usually comes next, then the base last of all.
    \end{itemize}
\end{rmk}

\textbf{NOTE:} You might need to add in little words, such as `the' and `of,' just to make the definition sound right.



\section{Questions and Remarks}
\label{sec:QR1}






%
% \begin{acknowledgement}
% If you want to include acknowledgments of assistance and the like at the end of an individual chapter please use the \verb|acknowledgement| environment -- it will automatically render Springer's preferred layout.
% \end{acknowledgement}
%
% \section*{Appendix}
% \addcontentsline{toc}{section}{Appendix}
%


% Problems or Exercises should be sorted chapterwise
\section*{Problems}
\addcontentsline{toc}{section}{Problems}
%
% Use the following environment.
% Don't forget to label each problem;
% the label is needed for the solutions' environment
\begin{prob}
\label{prob1}
A given problem or Excercise is described here. The
problem is described here. The problem is described here.
\end{prob}

% \begin{prob}
% \label{prob2}
% \textbf{Problem Heading}\\
% (a) The first part of the problem is described here.\\
% (b) The second part of the problem is described here.
% \end{prob}

%%%%%%%%%%%%%%%%%%%%%%%% referenc.tex %%%%%%%%%%%%%%%%%%%%%%%%%%%%%%
% sample references
% %
% Use this file as a template for your own input.
%
%%%%%%%%%%%%%%%%%%%%%%%% Springer-Verlag %%%%%%%%%%%%%%%%%%%%%%%%%%
%
% BibTeX users please use
% \bibliographystyle{}
% \bibliography{}
%


% \begin{thebibliography}{99.}%
% and use \bibitem to create references.
%
% Use the following syntax and markup for your references if 
% the subject of your book is from the field 
% "Mathematics, Physics, Statistics, Computer Science"
%
% Contribution 
% \bibitem{science-contrib} Broy, M.: Software engineering --- from auxiliary to key technologies. In: Broy, M., Dener, E. (eds.) Software Pioneers, pp. 10-13. Springer, Heidelberg (2002)
% %
% Online Document

% \end{thebibliography}

