%%%%%%%%%%%%%%%%%%%%% chapter.tex %%%%%%%%%%%%%%%%%%%%%%%%%%%%%%%%%
%
% sample chapter
%
% Use this file as a template for your own input.
%
%%%%%%%%%%%%%%%%%%%%%%$% Springer-Verlag %%%%%%%%%%%%%%%%%%%%%%%%%%
%\motto{Use the template \emph{chapter.tex} to style the various elements of your chapter content.}
\chapter{Nose}
\label{Nose} % Always give a unique label
% use \chaptermark{}
% to alter or adjust the chapter heading in the running head


%%% Questions to think about
%The \textbf{Thesis} or general sense of the article is ...

%The \textbf{method} the author uses to argue their point is ...

%In their \textbf{analysis} the author uses tools such as ... 

%Additionally they conclude ...

%What connections does the author portray with regard to \textbf{space}, \textbf{relationships}, \textbf{occupation}, and \textbf{religion}.


\abstract{}


\section{Notes}
\label{sec:NOTE7}


\subsection{Prefixes}

\begin{longtable}{c | p{0.4\textwidth} | p{0.4\textwidth}}
    \caption{Prefixes for the Nose.}
    \hline
    Prefix & Meaning(s) & Example(s) \\ \hline
        mal- & `bad,' `inadequate' & malignant, malnutrition \\
    \label{tab:Ch7Prefix}
\end{longtable}


\subsection{Suffixes}

\begin{longtable}{c | p{0.4\textwidth} | p{0.4\textwidth}}
    \caption{Suffixes for the nose.}
    \hline
    Suffix & Meaning(s) & Example(s) \\ \hline
        -ant & `pertaining to & malignant \\
        -or & `person who (does...),' `thing that (does...)' & \\
        -plasm & `formed substance,' `growth' & ectoplasm \\
        -tome & `instrument used to cut' & \\
    \label{tab:Ch7Suffix}
\end{longtable}


\subsection{Bases}


\begin{longtable}{c | p{0.4\textwidth} | p{0.4\textwidth}}
    \caption{Bases for the nose.}
    \hline
    Base & Meaning(s) & Example(s) \\ \hline
        OLFACT- & `smell,' `sense of smell' & olfactor (OLFACT-or) - thing that has a sense of smell, olfaction (OLFACT-ion) - the act of smelling, olfactory (OLFACT-ory) - having the function of smelling \\
        OSM- & `smell,' `sense of smell' & anosmia (an-OSM-ia) - condition of without a sense of smell, osmatic (OSM-atic) - pertaining to the sense of smell, parosmia (par-OSM-ia) - condition of abnormal sense of smell \\
        OSPHRES-, OSPHRESI- & `smell,' `sense of smell' & hyperosphresia (hyper-OSPHRES-ia) - condition of more than normal sense of smell, osphresiophobia (OSPHRESI-O-phobia) - abnormal fear of (bad) smells \\
        ODOR- & `odor,' `smell' & malodorous (mal-ODOR-ous) - full of bad odor, deodorant (de-ODOR-ant) - pertaining to without odor, odorimetry (ODOR-I-metry) - process of measuring odor \\
        NAS- & `nose' & nasal (NAS-al) - pertaining to the nose, nasoliciary (NAS-O-CILI-ary) - pertaining to the eyelids and nose, nasoscope (NAS-O-scope) - instrument used to examine the nose \\
        RHIN-, -RRHIN- & `nose' & rhinaesthesia (RHIN-AESTHE-sia) - condition of sensation in the nose, rhinalgia (RHIN-algia) - painful condition of the nose, rhinotome (RHIN-O-tome) - instrument used to cut the nose, rhinoplasm (RHIN-O-plasm) - formed substance for the nose (i.e. repair material for reconstruction of the nose, although the term is most often seen as an outdated alternative to rhinoplasty) \\
        NAR- & `nostril' & internarial (inter-NAR-ial) - pertaining to between the nostrils, nariform (NAR-I-form) - having the form of nostrils \\
        AL- & `wing' & alar (AL-ar) - pertaining to a wing or winged structure, alinasal (AL-I-NAS-al) - pertaining to the wings of the nose \\
        SEPT- & (i) `dividing wall': (ii) `seven' & septal (SEPT-al) - pertaining to a dividing wall \\
        MUC- & `mucus' & mucous (MUC-ous) - pertaining to mucus, mucocutaneous (MUC-O-CUTANE-ous) - pertaining to the skin and mucous membrane \\
        MUCOS- & `mucosa,' (mucus-producing membrane that lines certain bodily structures) `mucous membrane' & mucosal (MUCOS-al) - pertaining to a mucosa \\
        BLENN- & `mucus' & blennophthalmia (BLENN-OPHTHALM-ia) - condition of the eyes (caused by) mucus, blennogenic (BLENN-O-genic) - producing mucus \\
        MYX- & `mucus' & myxoid (MYX-oid) - resembling mucus, myxocyte (MYX-O-cyte) - cell in mucus (producing) tissue \\
        PHLEGM- & `phlegm,' (the abnormal amount of mucus that is produced by an inflamed respiratory system, especially if it is voided through the mouth) `inflammation' & phlegmasia (PHLEGM-A-sia) - condition of inflammation, phlegmy (PHLEGM-y) - (having the) quality of phlegm \\
        SIN- & `curve,' `cavity' & sinuate (SIN-U-ate) - having a curved (shape) \\
        SIN-, SINUS- & `sinus' (a hollow cavity or channel leading to a cavity) & perisinuous (peri-SIN-U-ous) - pertaining to around a sinus, sinusitis (SINUS-itis) - inflammation of the sinuses, rhinosinusopathy (RHIN-O-SINUS-O-pathy) - disease of the sinuses and nose \\
        PLAS-, PLAST- & `to form,' `to mold' & hyperplasia (hyper-PLAS-ia) - condition of more than normal formation (of tissue), anaplasty (ana-PLAST-y) - act of again formation (i.e. restoring or reforming body parts), rhinoplasty (RHIN-O-PLAST-y) - act of forming the nose (i.e. reconstructive surgery of the nose), theleplasty (THEL-E-PLAST-y) - act of forming the nipple (i.e. reconstructive surgery of the nipple) \\
        CENTE- & `to puncture' & ophthalmocentesis (OPTHALM-O-CENTE-sis) - process of puncturing the eye, tympanocentesis (TYMPAN-O-CENTE-sis) - process of puncturing the tympanic membrane, craniocentesis (CRANI-O-CENTE-sis) - process of puncturing the cranium \\
        TOM- & `to cut,' `to slice,' `section' & tomography (TOM-O-graphy) - process of recording sections (of the body), tomophobia (TOM-O-phobia) - abnormal fear of being cut, ototomy (OT-O-TOM-y) - process of cutting the ear \\
        ECTOM- & `to cut out,' `to cut away' & adipectomy (ADIP-ECTOM-y) - act of cutting out fat, blepharectomy (BLEPHAR-ECTOM-y) - act of cutting away the eyelid, craniectomy (CRANI-ECTOM-y) - act of cuttin out (part of) the cranium \\
        STOM-, STOMAT- & `mouth,' `opening' & dacryocystostomy (DACRY-O-CYST-O-STOM-y) - act of (creating) an opening in the sac related to tears, anastomosis (ana-STOM-osis) - process of up-opening, stomatodynia (STOMAT-ODYN-ia/STOMAT-odynia) - condition of pain of the mouth, stomatitis (STOMAT-itis) - inflammation of the mouth \\
        STOM- & `stoma' (opening or pore in the body, including artificial openings) & stomal (STOM-al) - pertaining to a stoma \\
        CLEI-, CLEIST- & `to close,' `close' & corecleisis (CORE-CLEI-sis) - condition of closure of the pupil of the eye, cleistophobia (CLEIST-O-phobia) - abnormal fear of closed (spaces) \\
        CLAS- & `to break,' `fragment' & aclasia (a-CLAS-ia) - condition of without a break (i.e. continuity between normal and abnormal body tissues), aclassis (a-CLAS-sis) - condition of without a break, histoclastic (HIST-O-CLAS-tic) - pertaining to breaking (down) of tissue, trichoclasis (TRICH-O-CLA(S)-sis) - condition of breackage of the hair \\
        DE- & `to bind' & syndesis (syn-DE-sis) - condition of together binding \\
        PEX- & `to fasten' `to fix' & adipopexia (ADIP-O-PEX-ia) - state of fixing fat, retinopexy (RETIN-O-PEX-y) - act of fastening the retina \\
        LY- & `to loosen,' `dissolving' & lipolysis (LIP-O-LY-sis) - process of dissolving fat, dermatolysis (DERMAT-O-LY-sis) - process of loosening of the skin, dialysis (dia-LY-sis) - process of apart dissolving \\
        CLY- & `to wash' & clysis (CLY-sis) - process of washing \\
        TRIB-, TRIPS- & `to rub,' `to crush' & tribometer (TRIB-O-meter) - instrument used to measure rubbing (friction), cephalotripsy (CEPHAL-O-TRIPS-y) - act of crushing the head \\
        RHAPH-, -RRHAPH- & `to stitch,' `to suture' & blepharorrpaphy (BLEPHAR-O-RRHAPH-y) - act of sutuing the eyelid, rhinorrhaphy (RHIN-O-RRHAP-y) - act of sutuing the nose \\
        RHE-, -RRH- & `to flow' & rheoencephalography (RHE-O-ENCEPHAL-O-graphy) - process of recording brain (blood) flow, cryptorrhetic (CRYPT-O-RRH-etic) - pertaining to flow that is hidden, catarrhal (cata-RRH-al) - pertaining to down flow (i.e. pertaining to catarrh, the increased flow of mucus) \\
    \label{tab:Ch7Base}
\end{longtable}



\subsection{Compound Suffixes}


\begin{longtable}{c | p{0.4\textwidth} | p{0.4\textwidth}}
    \caption{Compound suffixes for the nose.}
    \hline
    Suffix & Meaning(s) & Example(s) \\ \hline
        -centesis & `surgical puncturing,' `puncturing' & \\
        -clasia, clasis & `surgical fracture,' `breaking,' `rupture' & \\
        -cleisis & `surgical closure,' `closure' & \\
        -clysis & `therapeutic influsion of liguid,' `irragation' & \\
        -desis & `surgical fusion,' `binding' & \\
        -dialysis & `surgical dissolving,' `surgical separation,' `filtration' & \\
        -ectomy & `surgical removal,' `removal' & \\
        -lysis & `surgical dissolving,' `surgical separation,' `loosening' & \\
        -pexis, -pexy & `surgical fastening,' `fixing' & \\
        -plasia, -plasty & `surgical reshaping,' `formation' & \\
        -stomy & `making a surgical opening,' `making an opening' & \\
        -tripsy & `surgical crushing,' `crushing' & \\
        -tomy & `surgical cutting,' `cutting' & \\
        -osmia & `condition of sense of smell' & \\
        -rrhage & `excessive flow,' `excessive discharge' & \\
        -rrhag-ia & `excessive flow,' `excessive discharge' & \\
        -rrhaphy & `surgical suture' & \\
        -rrhea & `flow,' `discharge' & \\
        -rrhexis & `rupture' & \\
    \label{tab:Ch7Suffix2}
\end{longtable}



\section{Questions and Remarks}
\label{sec:QR7}






%
% \begin{acknowledgement}
% If you want to include acknowledgments of assistance and the like at the end of an individual chapter please use the \verb|acknowledgement| environment -- it will automatically render Springer's preferred layout.
% \end{acknowledgement}
%
% \section*{Appendix}
% \addcontentsline{toc}{section}{Appendix}
%


% Problems or Exercises should be sorted chapterwise
\section*{Problems}
\addcontentsline{toc}{section}{Problems}
%
% Use the following environment.
% Don't forget to label each problem;
% the label is needed for the solutions' environment
\begin{prob}
\label{prob1}
A given problem or Excercise is described here. The
problem is described here. The problem is described here.
\end{prob}

% \begin{prob}
% \label{prob2}
% \textbf{Problem Heading}\\
% (a) The first part of the problem is described here.\\
% (b) The second part of the problem is described here.
% \end{prob}

%%%%%%%%%%%%%%%%%%%%%%%% referenc.tex %%%%%%%%%%%%%%%%%%%%%%%%%%%%%%
% sample references
% %
% Use this file as a template for your own input.
%
%%%%%%%%%%%%%%%%%%%%%%%% Springer-Verlag %%%%%%%%%%%%%%%%%%%%%%%%%%
%
% BibTeX users please use
% \bibliographystyle{}
% \bibliography{}
%


% \begin{thebibliography}{99.}%
% and use \bibitem to create references.
%
% Use the following syntax and markup for your references if 
% the subject of your book is from the field 
% "Mathematics, Physics, Statistics, Computer Science"
%
% Contribution 
% \bibitem{science-contrib} Broy, M.: Software engineering --- from auxiliary to key technologies. In: Broy, M., Dener, E. (eds.) Software Pioneers, pp. 10-13. Springer, Heidelberg (2002)
% %
% Online Document

% \end{thebibliography}

