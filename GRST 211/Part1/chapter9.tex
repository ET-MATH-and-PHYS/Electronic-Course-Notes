%%%%%%%%%%%%%%%%%%%%% chapter.tex %%%%%%%%%%%%%%%%%%%%%%%%%%%%%%%%%
%
% sample chapter
%
% Use this file as a template for your own input.
%
%%%%%%%%%%%%%%%%%%%%%%$% Springer-Verlag %%%%%%%%%%%%%%%%%%%%%%%%%%
%\motto{Use the template \emph{chapter.tex} to style the various elements of your chapter content.}
\chapter{Throat and Neck}
\label{Throat} % Always give a unique label
% use \chaptermark{}
% to alter or adjust the chapter heading in the running head


%%% Questions to think about
%The \textbf{Thesis} or general sense of the article is ...

%The \textbf{method} the author uses to argue their point is ...

%In their \textbf{analysis} the author uses tools such as ... 

%Additionally they conclude ...

%What connections does the author portray with regard to \textbf{space}, \textbf{relationships}, \textbf{occupation}, and \textbf{religion}.


\abstract{}


\section{Notes}
\label{sec:NOTE9}


\subsection{Suffixes}


\begin{longtable}{c | p{0.4\textwidth} | p{0.4\textwidth}}
    \caption{Suffixes for the Throat and Neck.}
    \hline
    Suffix & Meaning(s) & Example(s) \\ \hline
        -acious & `tending to,' `inclined to' & spacious \\
        -ence & `state of' & \\
        -ian & `pertaining to'
    \label{tab:Ch9Suffix}
\end{longtable}


\subsection{Bases}


\begin{longtable}{c | p{0.4\textwidth} | p{0.4\textwidth}}
    \caption{Bases for the throat and neck.}
    \hline
    Base & Meaning(s) & Example(s) \\ \hline
        CERVIC- & `neck' & cervicalgia (CERVIC-algia) - painful condition of the neck, cervical (CERVIC-al) - pertaining to the neck, cervicobuccal (CERVIC-O-BUCC-al) - pertaining to the cheek (side) of the neck (of a tooth), cervicooccipital (CERVIC-O-OCCIPIT-al) - pertaining to the occiput and the neck \\
        TRACHEL- & `neck' & trachelodynia (TRACHEL-odynia) - painful condition of the neck, tracheleal (TRACHEL-eal) - pertaining to the neck, trachelectomy (TRACHEL-ectomy) - surgical removal of the neck (of the uterus), trachelooccipital (TRACHEL-O-OCCIPIT-al) - pertaining to the occiput and the neck \\
        COLL- & `neck' & decollate (de-COLL-ate) - to (take) away from the neck (i.e. to decapitate, to behead) \\
        CLEID- & `key,' `hook,'  `collar-bone' & cleidagra (CLEID-agra) - painful seizure in the collar-bone, cleidocranial (CLEID-O-CRANI-al) - pertaining to the cranium and collar-bone, cleidotripsy (CLEID-O-TRIPS-y) - act of crushing the collar-bone \\
        CLAV-, CLAVICUL- & `key,' `clavicle,' `collar-bone' & clavicular (CLAVICUL-ar) - pertaining to the clavicle, subclavian (sub-CLAV-ian) - pertaining to below the clavicle \\
        JUGUL- & `neck,' `throat' & jugular (JUGUL-ar) - pertaining to the neck or throat \\
        GUTTUR- & `throat' & guttural (GUTTUR-al) - pertaining to the throat, gutturonasal (GUTTUR-O-NAS-al) - pertaining to (sound produced from) nose and throat \\
        PHARYNG- & `throat,' `pharynx' & pharyngeal (PHARYNG-eal) - pertaining to the pharynx, pharyngectomy (PHARYNG-ectomy) - surgical removal of the pharynx, pharyngoscope (PHARYNG-O-scope) - instrument used to examine the pharynx, oropharyngeal (OR-O-PHARYNG-eal) - pertaining to the oropharynx \\
        TONSILL- & `tonsil' & `tonsillitis (TONSILL-itis) - inflammation of the tonsil, tonsillotome (TONSILL-O-tome) - instrument used to cut the tonsil, peritonsillar (peri-TONSILL-ar) - pertaining to around the tonsil \\
        ADEN- & `gland' & polyadenous (POLY-ADEN-ous) - having glands - many of them, adenomegaly (ADEN-O-megaly) - enlargement of a gland, adenoid (ADEN-oid) - resembling a gland \\
        LARYNG- & `voice-box,' `larynx' & laryngeal (LARYNG-eal) - pertaining to the larynx, laryngorrhea (LAYRNG-O-rrhea) - (excessive) discharge from the larynx, laryngostenosis (LARYNG-O-STEN-osis) - abnormal condition of narrowing of the larynx \\
        GLOTT- & (i) `tongue,' `language'; (ii) `glottis,' `mouth of the windpipe,' & glottitis (GLOTT-itis) - inflammation of the glottis, infraglottic (infra-GLOTT-ic) - pertaining to below the glottis \\
        THYR- & `oblong shield' & thyroid (THYR-oid) - resembling an oblong shield \\
        THYR-, THYROID- & (i) `thyroid gland,': (ii) `thyroid cartilage' & thyromegaly (THYR-O-megaly) - enlargement of the thyroid gland, thyrotomy (THYR-O-tomy) - surgical cutting of the thyroid gland or the thyroid cartilage, thyroiditis (THYROID-itis) - inflammation of the thyroid gland \\
        HY- & `Greek letter upsilon' ($\upsilon, \Upsilon$) & hyoid (HY-oid) - resembling the greek letter upsilon \\
        HY- & `hyoid bone' (the $\upsilon$-shaped bone that supports the tongue) &  hyoglossal (HY-O-GLOSS-al) - pertaining to the tongue and the hyoid bone, hyopharyngeal (HY-O-PHARYNG-eal) - pertaining to the pharynx and the hyoid bone \\
        PHAS- & `speech,' `to talk' & dysphasia (dys-PHAS-ia) - condition of abnormal or difficult speech, cryptophasic (CRYPT-O-PHAS-ic) - pertaining to speech that is hidden (i.e. pertaining to speech that is unintelligible to outsiders) \\
        PHRAS- & `speech,' `to talk' & polyphrasia (POLY-PHRAS-ia) - condition of speech - much of it, hypophrasic (hypo-PHRAS-ic) - pertaining to less than normal speech \\
        PHEM- & `speech,' `to talk' & aphemic (a-PHEM-ic) - pertaining to without speech, paraphemia (para-PHEM-ia) - condition of abnormal speech \\
        LAL- & `speech,' `to talk' & alalia (a-LAL-ia) - condition of without speech, laliatry (LAL-iatry) - medical treatment of speech, lalopathy (LAL-O-pathy) - disease or disorder (affecting) speech \\
        LOQU- & `speech,' `to talk' & multiloquacious (MULTI-LOQU-acious) - tending to speak much, sialoquence (SIA(L)-LOQU-ence) - state of speaking and (producing) saliva) \\
        VOC- & `speech,' `voice,' `to talk' & vocal (VOC-al) - pertaining to the voice \\
        PHON- & `speech,' `voice,' `sound' & aphonic (a-PHON-ic) - pertaining to without voice, euphonous (eu-PHON-ous) - having a good sound \\
        LOG- & `speech,' `word' & dyslogia (dys-LOG-ia) - condition of abnormal speech, logorrhea (LOG-O-rrhea) - flow of words, logamnesia (LOG-a-MNE-sia) - condition of without memory of words, adenology (ADEN-O-LOG-y) - act of speech about glands \\
        BRADY- & `slow' & bradylalia (BRADY-LAL-ia) - condition of speech that is slow, bradylogia (BRADY-LOG-ia) - condition of speech that is slow, bradyphasia (BRADY-PHAS-ia) - condition of speech that is slow, bradyphrasia (BRADY-PHRAS-ia) - condition of speech that is slow \\
        TACHY- & `fast' & tachylalia (TACHY-LAL-ia) - condition of speech that is fast, tachylogia (TACHY-LOG-ia) - condition of speech that is fast, tachyphasia (TACHY-PHAS-ia) - condition of speech that is fast, tachyphrasia (TACHY-PHRAS-ia) - condition of speech that is fast \\
        COPR- & `dung,' `filth,' `feces' & coprolalia (COPR-O-LAL-ia) - condition of speech that is filthy, coprophasia (COPR-O-PHAS-ia) - condition of speech that is filthy, coprophemia (COPR-O-PHEM-ia) - condition of speech that is filthy, coprophrasia (COPR-O-PHRAS-ia) - condition of speech that is filthy \\
        CAC- & `bad' & cacolalia (CAC-O-LAL-ia) - condition of speech that is bad \\
        ECH- & `returned sound,' `repetition' & echolalia (ECH-O-LAL-ia) - condition of speech repetition, echophrasia (ECH-O-PHRAS-ia) - condition of speech repetition \\
        MON- & `one,' `single' & monesthetic (MON-ESTHE-tic) - pertaining to a sensation - a single one \\
        UN- & `one,' `single' & uniocular (UN-I-OCUL-ar) - pertaining to the eye - one of them \\
        HAPL- & `single' & haplopia (HAPL-OP-ia) - condition of sight (that forms a) single (image) \\
        PROT- & `first,' `original,' `primitive' & protomorphic (PROT-O-MORPH-ic) - pertaining to the shape that is (the most) primitive \\
        PRIM- & `first' & primigeneal (PRIM-I-GEN-eal) - pertaining to produced first, primigravida (PRIM-I-GRAVIDA) - a woman who has been pregnant for the first time, primipara (PRIM-I-PARA) - a woman who has given birth for the first time \\
        DI- & `two,' `twice' & dicephaly (DI-CEPHAL-y) - condition of the head - two of them \\
        DICH- & `in two' & dichotomy (DICH-O-tomy) - cutting into two \\
        BI- & (i) `two,' `twice,' `double': (ii) `life,' `living' & bilabial (BI-LAB-ial) - pertaining to the lips - two of them \\
        BIN- & `double,' `pair' & binotic (BIN-OT-ic) - pertaining to the ears - a pair of them \\
        DEUT-, DEUTER- & `second' & deutogenic (DEUT-O-GEN-ic) - pertaining to produced second, deuteropathy (DEUTER-O-pathy) - disease that is second(ary to the intitial disease) \\
        SECOND-, SECUND- & `second,' `following' & secondary (SECOND-ary) - pertaining to second, secundigravida (SECUND-I-gravida) - a woman who has been pregnant a second time,secundipara (SECUND-I-para) - a woman who has given birth a second time \\
        GEMIN-, GEMELL- & `twin,' `paired,' `born at the same time' & geminate (GEMIN-ate) - having paired (structures), gemellipara (GEMELL-I-para) - a woman who has given birth to twins \\
        TRI- & `three' & trisulcate (TRI-SULC-ate) - having grooves - three of them, trigeminal (TRI-GEMIN-al) - pertaining to born at the same time - three \\
        TERTI- & `third' & tertian (TERTI-an) - pertaining to the third, tertigravida (TERTI-gravida) - a woman who has been pregnant a third time \\
        TERN- & `three each' & ternary (TERN-ary) - pertaining to three each \\
        TETR- & `four' & tetramastous (TETR-A-MAST-ous) - having breasts - four of them \\
        QUADR- & `four' & quadritubercular (QUADR-I-TUBERCUL-ar) - pertaining to tubercles - four of them \\
        QUART- & `fourth' & quartan (QUART-an) - pertaining to the fourth, quartigravida (QUART-I-gravida) - a woman who has been pregnant a fourth time \\
        QUATERN- & `four each' & quaternary (QUATERN-ary) - pertaining to four each \\
        PENT- & `five' & pentapterous (PENT-A-PTER-ous) - having wings - five of them \\
        QUINQUE- & `five' & quinquetubercular (QUINQUE-TUBERCUL-ar) - pertaining to tubercles - five of them \\
        QUINT- & `fifth' & quintan (QUINT-an) - pertaining to the fifth, quintipara (QUINT-I-para) - a woman who has given birth a fifth time \\
        QUIN- & `five each' & quinary (QUIN-ary) - pertaining to five each \\
        HEX- & `six' & hexagon (HEX-A-gon) - (geometrical figure with) six angles \\
        SEX- & `six' & sexdigitate (SEX-DIGIT-ate) - having digits - six \\
        SEXT- & `sixth' & sextan (SEXT-an) - pertaining to the sixth, sextigravida (SEXT-I-gravida) - a woman who has been pregnant a sixth time \\
        HEPT- & `seven' & heptachromic (HEPT-A-CHROM-ic) - pertaining to colors - seven of them \\
        SEPT- & (i) `dividing wall': (ii) `seven' & septan (SEPT-an) - pertaining to seven, septigravida (SEPT-I-gravida) - a woman who has been pregnant seven times \\
        SEPTIM- & `seventh' & septimal (SEPTIM-al) - pertaining to the seventh \\
        OCT- & `eight' & octan (OCT-an) - pertaining to eight, octipara (OCT-I-para) - a woman who has given birth eight times \\
        OCTAV- & `eighth' &\\
        ENNE- & `nine' & enneagon (ENNE-A-gon) - (geometrical figure with) nine angles \\
        NOVEM- & `nine' & November \\
        NON- & `ninth' & nonipara (NON-I-para) - a woman who has given birth a ninth time \\
        DEC- & `ten' & decagon (DEC-A-gon) - (geometrical figure with) ten angles \\
        DECEM- & `ten' & December \\
        DEC-, DECIM- & `tenth' & decigravida (DEC-I-gravida) - a woman who has been pregnant a tenth time \\
    \label{tab:Ch9Base}
\end{longtable}

The term `neck' is applied to many body parts that have a constricted or narrowed portion. The base CERVIC- may indicate the neck of the uterus, bladder, or tooth, as well as the area between the head and the torso. TRACHEL- also has some ambiguity as it may be used about the neck of the uterus as well as the area between the head and torso (rarely the bladder or tooth).


\subsection{Compound Suffixes}


\begin{longtable}{c | p{0.4\textwidth} | p{0.4\textwidth}}
    \caption{Compound suffixes for the throat and neck.}
    \hline
    Suffix & Meaning(s) & Example(s) \\ \hline
        -logist & `one who studies' & archeologist \\
        -logy & `study of' & psychology \\
        -gravida & `woman who is, or has been, pregnant' & \\
        -para & `woman who has given birth' & \\
    \label{tab:Ch9Suffix2}
\end{longtable}


\section{Questions and Remarks}
\label{sec:QR9}






%
% \begin{acknowledgement}
% If you want to include acknowledgments of assistance and the like at the end of an individual chapter please use the \verb|acknowledgement| environment -- it will automatically render Springer's preferred layout.
% \end{acknowledgement}
%
% \section*{Appendix}
% \addcontentsline{toc}{section}{Appendix}
%


% Problems or Exercises should be sorted chapterwise
\section*{Problems}
\addcontentsline{toc}{section}{Problems}
%
% Use the following environment.
% Don't forget to label each problem;
% the label is needed for the solutions' environment
\begin{prob}
\label{prob1}
A given problem or Excercise is described here. The
problem is described here. The problem is described here.
\end{prob}

% \begin{prob}
% \label{prob2}
% \textbf{Problem Heading}\\
% (a) The first part of the problem is described here.\\
% (b) The second part of the problem is described here.
% \end{prob}

%%%%%%%%%%%%%%%%%%%%%%%% referenc.tex %%%%%%%%%%%%%%%%%%%%%%%%%%%%%%
% sample references
% %
% Use this file as a template for your own input.
%
%%%%%%%%%%%%%%%%%%%%%%%% Springer-Verlag %%%%%%%%%%%%%%%%%%%%%%%%%%
%
% BibTeX users please use
% \bibliographystyle{}
% \bibliography{}
%


% \begin{thebibliography}{99.}%
% and use \bibitem to create references.
%
% Use the following syntax and markup for your references if 
% the subject of your book is from the field 
% "Mathematics, Physics, Statistics, Computer Science"
%
% Contribution 
% \bibitem{science-contrib} Broy, M.: Software engineering --- from auxiliary to key technologies. In: Broy, M., Dener, E. (eds.) Software Pioneers, pp. 10-13. Springer, Heidelberg (2002)
% %
% Online Document

% \end{thebibliography}

