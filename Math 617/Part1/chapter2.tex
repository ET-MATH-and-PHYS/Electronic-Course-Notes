%%%%%%%%%%%%%%%%%%%%% chapter.tex %%%%%%%%%%%%%%%%%%%%%%%%%%%%%%%%%
%
% sample chapter
%
% Use this file as a template for your own input.
%
%%%%%%%%%%%%%%%%%%%%%%%% Springer-Verlag %%%%%%%%%%%%%%%%%%%%%%%%%%
%\motto{Use the template \emph{chapter.tex} to style the various elements of your chapter content.}
\chapter{Operators on Hilbert Spaces}
\label{OpHilb} % Always give a unique label
% use \chaptermark{}
% to alter or adjust the chapter heading in the running head



\abstract{Summary of material in chapter (to be completed after chapter)}

\section{Elementary Properties}
\label{sec:Hilb2}

\begin{prop}
    Let $\mathscr{H}$ and $\mathscr{K}$ be Hilbert spaces and $A:\mathscr{H}\rightarrow \mathscr{K}$ a linear operator. The following are equivalent: \begin{enumerate}
        \item $A$ is continuous
        \item $A$ is continuous at $0$
        \item $A$ is continuous at some point
        \item There is a constant $c > 0$ such that $\norm{Ah} \leq c\norm{h}$ for all $h \in \mathscr{H}$
    \end{enumerate}
\end{prop}

The proof of this proposition is identical to the one for linear functionals seen in the previous chapter. We recall the equivalent definitions of the operator norm: \begin{align*}
    \norm{A} &= \sup\{\norm{Ah}:h \in \mathscr{H},\norm{h} \leq 1\} \\
    &= \sup\{\norm{Ah}:\norm{h} = 1\} \\
    &= \sup\{\norm{Ah}/\norm{h}:h\neq 0\} \\
    &= \inf\{c > 0:\norm{Ah} \leq c\norm{h},h \in \mathscr{H}\}
\end{align*}
Also, $\norm{Ah}\leq \norm{A}\norm{h}$. Let $\mathscr{B}(\mathscr{H},\mathscr{K})$ denote the set of bounded linear transformations from $\mathscr{H}$ into $\mathscr{K}$.

\begin{prop}
    \begin{enumerate}
        \item[(a)] If $A,B \in \mathscr{B}(\mathscr{H},\mathscr{K})$, then $A+B \in \mathscr{B}(\mathscr{H},\mathscr{K})$, and $\norm{A+B}\leq \norm{A}+\norm{B}$
        \item[(b)] If $\alpha \in \F$ and $A \in \mathscr{B}(\mathscr{H},\mathscr{K})$, then $\alpha A \in \mathscr{B}(\mathscr{H},\mathscr{K})$ and $\norm{\alpha A} = |\alpha|\norm{A}$
        \item[(c)] If $A \in \mathscr{B}(\mathscr{H},\mathscr{K})$ and $B \in \mathscr{B}(\mathscr{K},\mathscr{L})$, then $BA \in \mathscr{B}(\mathscr{H},\mathscr{L})$, and $\norm{BA} \leq \norm{B}\norm{A}$.
    \end{enumerate}
\end{prop}
\begin{proof}
    For (a), observe that $\norm{(A+B)h} = \norm{Ah+Bh} \leq \norm{Ah}+\norm{Bh} \leq \norm{A}\norm{h}+\norm{B}\norm{h}$. As $A,B$ are bounded, $\norm{A},\norm{B} < \infty$, so $\norm{A}+\norm{B} < \infty$. Thus $A+B$ is bounded, and $\norm{A+B} \leq \norm{A}+\norm{B}$ by definition of the infimum.

    For (b), $\norm{\alpha Ah} = |\alpha|\norm{Ah} \leq |\alpha|\norm{A}\norm{h}$. Thus $\alpha A$ is bounded and $\norm{\alpha A} \leq |\alpha|\norm{A}$. If $\alpha = 0$, $\alpha A = 0$ so $\norm{\alpha A} = 0 = |\alpha|\norm{A}$. Otherwise, we can write $\norm{A} \leq \frac{1}{|\alpha|}\norm{\alpha A}$, so $|\alpha|\norm{A}\leq \norm{\alpha A}$. Hence $\norm{\alpha A} = |\alpha|\norm{A}$.

    Finally, for (c) we have $\norm{BAh} \leq \norm{B}\norm{Ah} \leq \norm{B}\norm{A}\norm{h}$, so as $B$ and $A$ are both bounded, $\norm{B}\norm{A} < \infty$, and so $BA$ is bounded with $\norm{BA} \leq \norm{B}\norm{A}$.
\end{proof}


Thus the operator norm is indeed a norm on the vector space of bounded linear operators, and so we can define a metric $d(A,B) = \norm{A-B}$.

\begin{eg}
    If $\dim \mathscr{H} = n <\infty$ and $\dim \mathscr{K} = m < \infty$, let $\{e_1,...,e_n\}$ be an orthonormal basis for $\mathscr{H}$ and let $\{\epsilon_1,...,\epsilon_m\}$ be an orthonormal basis for $\mathscr{K}$. Let $A$ be a linear transformation between these spaces, with associated matrix $(a_{i,j})$, with $a := \max\{|a_{i,j}|\}$. Then we have for $x = \sum_ix_ie_i$, \begin{align*}
        \norm{Ax}^2 &= \norm{\sum_i\sum_ja_{i,j}x_i\epsilon_j}^2 \\
        &= \norm{\sum_j\left(\sum_ia_{i,j}x_i\right)\epsilon_j}^2 \\
        &= \sum_j\left|\left(\sum_ia_{i,j}x_i\right)\right|\norm{\epsilon_j}^2 \\
        &\leq \sum_j\sum_i|a_{i,j}|^2|x_i|^2\norm{\epsilon_j}^2 \\
        &\leq \sum_j\sum_ia^2|x_i|^2c^2 \tag{$c = \max\{\norm{\epsilon_j}\}$} \\
        &= mac^2\sum_i|x_i|^2 \\
        &\leq mac^2\sum_i|x_i|^2\frac{\norm{e_i}^2}{d^2}\tag{$d = \min\{\norm{e_i}\}$} \\
        &= \frac{mac^2}{d}\sum_i\norm{x_ie_i}^2 = \frac{mac^2}{d}\norm{x} 
    \end{align*}
    Thus $A$ is bounded. Observe that $a_{i,j} = \inner{Ae_j}{\epsilon_i}$.
\end{eg}

\begin{eg}
    let $l^2 = l^2(\N)$ and let $e_1,e_2,...$ be its usual basis. If $A \in \mathscr{B}(l^2)$, form $\alpha_{i,j} = \inner{Ae_j}{e_i}$. The infinite matrix $(\alpha_{i,j})$ represents $A$ as finite matrices represent operators on finite dimensional spaces. However, this representation has limited value unless the matrix has a special form.
\end{eg}

\begin{thm}
    Let $(X,\Omega,\mu)$ be a $\sigma$-finite measure space and put $\mathscr{H} = L^2(X,\Omega,\mu) = L^2(\mu)$. If $\phi \in L^{\infty}(\mu)$, define $M_{\phi}:L^2(\mu)\rightarrow L^2(\mu)$ by $M_{\phi}f = \phi f$. Then $M_{\phi} \in \mathscr{B}(L^2(\mu))$, and $\norm{M_{\phi}} = \norm{\phi}_{\infty}$.
\end{thm}
\begin{proof}
    Here $\norm{\phi}_{\infty}$ is the $\mu$-essential supremum norm. Recall \begin{align*}
        \norm{\phi}_{\infty} &:= \inf\{\sup\{|\phi(x)|:x \in N\}:N \in \Omega:\mu(N) = 0\} \\
        &= \inf\{c > 0:\mu(\{x\in X:|\phi(x)| > c\}) = 0\}
    \end{align*}
    Thus $\norm{\phi}_{\infty}$ is the infimum of all $c > 0$ such that $|\phi(x)| \leq c$ almost everywhere $[\mu]$ and, moreover, $|\phi(x)| \leq \norm{\phi}_{\infty}$ almost everywhere $[\mu]$. As $L^{\infty}(\mu)$ are equivalence classes of functions, we can assume that $\phi$ is a bounded measurable function and $|\phi(x)| \leq \norm{\phi}_{\infty}$ for all $x$. So if $f \in L^2(\mu)$, then $$\int_X|\phi f|^2d\mu \leq \norm{\phi}_{\infty}^2\int_X|f|^2d\mu$$

    That is $M_{\phi} \in \mathscr{B}(L^2(\mu))$, and $\norm{M}_{\phi} \leq \norm{\phi}_{\infty}$. If $\epsilon > 0$, the $\sigma$-finiteness of the measure space implies that there is a set $\Delta$ in $\Omega$, $0 < \mu(\Delta) < \infty$, such that $|\phi(x)| \leq \norm{\phi}_{\infty} - \epsilon$ on $\Delta$. If $f = (\mu(\delta))^{-1/2}\chi_{\Delta}$, then $f \in L^2(\mu)$ and $\norm{f}_2 = 1$. So $$\norm{M_{\phi}}^2 \geq \norm{\phi f}_2^2 = (\mu(\Delta))^{-1}\int_{\delta}|\phi|^2d\mu \geq (\norm{\phi}_{\infty}-\epsilon)^2$$
    Letting $\epsilon \rightarrow 0$, we get that $\norm{M_{\phi}}\geq \norm{\phi}_{\infty}$.
\end{proof}

The operator $M_{\phi}$ is called a \textbf{multiplication operator}.

If the measure space $(X,\Omega,\mu)$ is not $\sigma$-finite, then the conclusion of the theorem need not hold.

\begin{thm}
    Let $(X,\Omega,\mu)$ be a measure space and suppose $k:X\times X\rightarrow \F$ is an $\Omega\times \Omega$-measurable function for which there are constants $c_1$ and $c_2$ such that \begin{align*}
        &\int_X|k(x,y)|d\mu(y) \leq c_1,\;\text{a.e.}[\mu] \\
        &\int_X|k(x,y)|d\mu(x) \leq c_2,\;\text{a.e.}[\mu]
    \end{align*}
    If $K:L^2(\mu)\rightarrow L^2(\mu)$ is defined by $$(Kf)(x) = \int_Xk(x,y)f(y)d\mu(y)$$
    then $K$ is a bounded linear operator and $\norm{K} \leq (c_1c_2)^{1/2}$.
\end{thm}
\begin{proof}
    If $f \in L^2(\mu)$, observe that \begin{align*}
        |Kf(x)| &\leq \int_X|k(x,y)||f(y)|d\mu(y) \\
        &= \int_X|k(x,y)|^{1/2}|k(x,y)|^{1/2}|f(y)|d\mu(y) \\
        &\leq \left[\int_X|k(x,y)|d\mu(y)\right]^{1/2}\left[\int_X|k(x,y)||f(y)|^2d\mu(y)\right]^{1/2} \tag{by H\"{o}lder's inequality} \\
        &\leq c_1^{1/2}\left[\int_X|k(x,y)||f(y)|^2d\mu(y)\right]^{1/2}
    \end{align*}
    Hence, \begin{align*}
        \int_X|Kf(x)|^2 d\mu(x) &\leq c_1 \int_X\int_X|k(x,y)||f(y)|^2d\mu(y)d\mu(x) \\
        &= c_1\int_X|f(y)|^2\int_X|k(x,y)|d\mu(x)d\mu(y) \tag{by Fubini's Theorem} \\
        &\leq c_1c_2\norm{f}^2 < \infty
    \end{align*}
    This shows that $Kf \in L^2(\mu)$, the formula defining $Kf$ is finite a.e. $[\mu]$, and $\norm{Kf}^2 \leq c_1c_2\norm{f}^2$, so $\norm{K} \leq \sqrt{c_1c_2}$.
\end{proof}

The operator described above is called an \textbf{integral operator} and the function $k$ is called its \textbf{kernel}.

\begin{eg}
    Let $k:[0,1]\times [0,1]\rightarrow \R$ be the characteristic function of $\{(x,y): y < x\}$. The corresponding operator $V:L^2(0,1)\rightarrow L^2(0,1)$ defined by $Vf(x) = \int_0^1k(x,y)f(y)dy$ is called the \textbf{Volterra operator}. Note that $$Vf(x) = \int_0^xf(y)dy$$
\end{eg}

Note that any isometry is a bounded operator with norm $1$.


\section{The Adjoint of an Operator}
\label{sec:adj}

\begin{defn}\index{Sesquilinear form}
    If $\mathscr{H}$ and $\mathscr{K}$ are Hilbert spaces, a function $u:\mathscr{H}\times \mathscr{K}\rightarrow \F$ is a \textbf{sesquilinear form} if for $h,g \in \mathscr{H}$, $k,f \in \mathscr{K}$, and $\alpha, \beta \in \F$, \begin{enumerate}
        \item[(a)] $u(\alpha h+\beta g,k) = \alpha u(h,k) + \beta u(g,k)$ 
        \item[(b)] $u(h,\alpha k+\beta f) = \overline{\alpha}u(h,k)+\overline{\beta}u(h,f)$
    \end{enumerate}
\end{defn}

The prefix ``sesqui" is used becuase the function is linear in one variable but only conjugate linear in the other.

A sesquilinear form is \textbf{bounded} if there is a constant $M$ such that $|u(h,k)| \leq M\norm{h}\norm{k}$ for all $h \in \mathscr{H}$ and $k \in \mathscr{K}$. The constant $M$ is called a \textbf{bound for $u$}.

If $A \in \mathscr{B}(\mathscr{H},\mathscr{K}$ and $B \in \mathscr{B}(\mathscr{K}(\mathscr{H})$, then both $\inner{Ah}{k}$ and $\inner{h}{Bk}$ are bounded sesquilinear forms.

\begin{thm}
    If $u:\mathscr{H}\times \mathscr{K}\rightarrow \F$ is a bounded sesquilinear form with bound $M$, then there are unique operators $A \in \mathscr{B}(\mathscr{H},\mathscr{K})$ and $B \in \mathscr{B}(\mathscr{K},\mathscr{H})$ such that $$u(h,k) = \inner{Ah}{k} = \inner{h}{Bk}$$
    for all $h \in \mathscr{H}, k \in \mathscr{K}$, and $\norm{A},\norm{B} \leq M$.
\end{thm}
\begin{proof}
    For each $h \in \mathscr{H}$, define $L_h:\mathscr{K}\rightarrow \F$ by $L_h(k) = \overline{u(h,k)}$. Then $L_h$ is linear and $L_h(k) \leq M\norm{h}\norm{k}$. By the Riesz Representation THeorem there is a unique vector $f \in \mathscr{K}$ such that $\inner{k}{f} = L_h(k) = \overline{u(h,k)}$ and $\norm{f} \leq M\norm{h}$. Let $Ah = f$. By the uniqueness part of the Riesz Theorem $A$ is linear. Also, $\inner{Ah}{k} = \overline{\inner{k}{Ah}} = \overline{k}{f} = u(h,k)$.

    The proof for $B$ is similar. If $A_1 \in \mathscr{B}(\mathscr{H},\mathscr{K})$, and $u(h,k) = \inner{A_1h}{k}$, then $\inner{Ah-A_1h}{k} = 0$ for all $k$, so $Ah-A_1h$ for all $h$. Thus, $A$ is unique.
\end{proof}

\begin{defn}
    If $A \in \mathscr{B}(\mathscr{H},\mathscr{K})$, then the unique operator $B \in \mathscr{B}(\mathscr{K},\mathscr{H})$ satisfying the equality $\inner{Ah}{k} = \inner{h}{Bk}$ for all $h,k$ is called the \textbf{adjoint} of $A$, and is denoted by $B = A^*$.
\end{defn}

\begin{prop}
    If $U \in \mathscr{B}(\mathscr{H},\mathscr{K})$, then $U$ is an isomorphism if and only if $U$ is invertible and $U^{-1} = U^*$.
\end{prop}
\begin{proof}
     If $U$ is an isomorphism we know that $U$ is invertible and $\inner{Uh}{Uh'} = \inner{h}{h'}$ for all $h,h' \in \mathscr{H}$. Then we have that $\inner{h}{U^*Uh'} = \inner{h}{h'}$ for all $h,h'$, so $U^*U = \text{id}_{\mathscr{H}}$. As $U$ is invertible $U^*$ must be its inverse by uniqueness.

     The reverse implication is immediate from the observation $\inner{Uh}{Uh'} = \inner{h}{U^*Uh'} = \inner{h}{h'}$.
\end{proof}


\begin{prop}
    If $A,B \in \mathscr{B}(\mathscr{H})$, and $\alpha \in \F$, then: \begin{enumerate}
        \item[(a)] $(\alpha A+B)^* = \overline{\alpha}A^*+B^*$
        \item[(b)] $(AB)^* = B^*A^*$
        \item[(c)] $A^{**} = (A^*)^* = A$.
        \item[(d)] If $A$ is invertible in $\mathscr{B}(\mathscr{H})$ and $A^{-1}$ is its inverse, then $A^*$ is invertible and $(A^*)^{-1} = (A^{-1})^*$.
    \end{enumerate}
\end{prop}
\begin{proof}
    For (a), observe that $$\inner{(\alpha A+B)h}{k} = \alpha \inner{Ah}{k}+\inner{Bh}{k} = \alpha \inner{h}{A^*k} + \inner{h}{B^*k} = \inner{h}{(\overline{\alpha}A^*+B^*)k}$$
    By uniqueness $(\alpha A+B)^* = \overline{\alpha}A^*+B^*$.

    For (b), observe $$\inner{ABh}{k} = \inner{Bh}{A^*k} = \inner{h}{B^*A^*k}$$
    so by uniqueness $(AB)^* = B^*A^*$.

    For (c), $$\inner{A^*h}{k} = \overline{\inner{k}{A^*h}} = \overline{\inner{Ak}{h}} = \inner{h}{Ak}$$
    so $(A^*)^* = A$ by uniqueness.

    Finally, for (d) we have $$\inner{(A^{-1})^*A^*h}{k} = \inner{A^*h}{A^{-1}k} = \inner{h}{AA^{-1}k} = \inner{h}{k}$$
    so by uniqueness $(A^{-1})^*A^* = \text{id}_{\mathscr{H}}$. Similarly, we have $$\inner{A^*(A^{-1})^*h}{k} = \inner{(A^{-1})^*h}{Ak} = \inner{h}{A^{-1}Ak} = \inner{h}{k}$$
    so $A^*(A^{-1})^* = \text{id}_{\mathscr{H}}$. Thus $A^*$ is invertible and $(A^*)^{-1} = (A^{-1})^*$.
\end{proof}

\begin{prop}
    If $A \in \mathscr{B}(\mathscr{H})$, $\norm{A} = \norm{A^*} = \norm{A^*A}^{1/2}$.
\end{prop}
\begin{proof}
    For $h \in \mathscr{H}$, $\norm{h} \leq 1$, $$\norm{Ah}^2 = \inner{Ah}{Ah} = \inner{A^*Ah,h} \leq \norm{A^*Ah}\norm{h} \leq \norm{A^*A} \leq \norm{A^*}\norm{A}$$
    Hence $\norm{A}^2 \leq \norm{A^*A}\leq \norm{A^*}\norm{A}$. Using the two ends of this string of inequalities $\norm{A} \leq \norm{A^*}$. But $A = (A^*)^*$, and so if $A^*$ is substituted for $A$ we get $\norm{A} = \norm{A^*}$. Thus the string of inequalities becomes a string of equalities and the proof is complete.
\end{proof}

\begin{eg}
    Let $(X,\Omega,\mu)$ be a $\sigma$-finite measure space and let $M_{\phi}$ be the multiplication operator with symbol $\phi$. Then $M_{\phi}^*$ is $M_{\overline{\phi}}$, the multiplication operator with symbol $\overline{\phi}$.
\end{eg}

If an operator on $\F^d$ is presented by a matrix, then its adjoint is represented by the conjugate transpose of the matrix.

\begin{eg}
    If $K$ is the integral operator with kernel $k$, then $K^*$ is the integral operator with kernel $k^*(x,y) = \overline{k(x,y)}$.
\end{eg}

\begin{prop}
    IF $S:l^2\rightarrow l^2$ is defined by $S(\alpha_1,\alpha_2,...) = (0,\alpha_1,\alpha_2,...)$, then $S$ is an isomotry and $S^*(\alpha_1,\alpha_2,...) = (\alpha_2,\alpha_3,...)$.
\end{prop}
\begin{proof}
    We have already shown that $S$ is an isometry. For $(\alpha_n),(\beta_n) \in l^2$, $$\inner{S^*(\alpha_n)}{(\beta_n)} = \inner{(\alpha_n),S(\beta_n)} = \alpha_2\overline{\beta}_1+\alpha_3\overline{\beta}_2 +\cdots = \inner{(\alpha_2,\alpha_3,...)}{(\beta_n)}$$
    and so the result follows by uniqueness.
\end{proof}

The operator $S$ is called the \textbf{unilateral shift} and the operator $S^*$ is called the \textbf{backward shift}.

\begin{defn}
    If $A \in \mathscr{B}(\mathscr{H})$, then \begin{enumerate}
        \item[(a)] is \textbf{hermitian} or \textbf{self-adjoint} if $A^* = A$;
        \item[(b)] $A$ is \textbf{normal} if $A^*A = AA^*$.
    \end{enumerate}
\end{defn}

Notice that hermitian and unitary operators are normal. In light of the previous examples, every multiplication operator $M_{\phi}$ is normal; $M_{\phi}$ is hermitian if and only if $\phi$ is real-valued; $M_{\phi}$ is unitary if and only if $|\phi| \equiv 1$ almost everywhere $[\mu]$. Additionally, an integral operator $K$ with kernel $k$ is hermitian if and only if $k(x,y) = \overline{k(y,x)}$ almost everywhere $[\mu\times \mu]$.

\begin{prop}
    If $\mathscr{H}$ is a $\C$-Hilbert space and $A \in \mathscr{B}(\mathscr{H})$, then $A$ is hermitian if and only if $\inner{Ah}{h} \in \R$ for all $h \in \mathscr{H}$.
\end{prop}
\begin{proof}
    If $A= A^*$, then $\inner{Ah}{h} = \inner{h}{Ah} = \overline{\inner{Ah}{h}}$; hence $\inner{Ah}{h} \in \R$.

    For the converse assume $\inner{Ah}{h}$ is real for every $h \in \mathscr{H}$. If $\alpha \in \C$ and $h,g \in \mathscr{H}$, then $$\inner{A(h+\alpha g)}{h+\alpha g} = \inner{Ah,h} + \overline{\alpha}\inner{Ah}{g} + \alpha \inner{Ag}{h}+|\alpha|^2\inner{Ag}{g} \in \R$$
    So this expression equals its complex conjugate. Using the fact that $\inner{Ah}{h}$ and $\inner{Ag}{g} \in \R$ yields \begin{align*}
        \alpha \inner{Ag}{h}+\overline{\alpha}\inner{Ah}{g} &= \overline{\alpha}\inner{h}{Ag}+\alpha\inner{g}{Ah} \\
        &= \overline{\alpha}\inner{A^*h}{g} + \alpha\inner{A^*g}{h}
    \end{align*}
    By first taking $\alpha =1$ and then $\alpha = i$, we obtain \begin{align*}
        \inner{Ag}{h} + \inner{Ah}{g} &= \inner{A^*h}{g} +\inner{A^*g}{h} \\
        i\inner{Ag}{h} - i\inner{Ah}{g} &= -i\inner{A^*h}{g}+i\inner{A^*g}{h}
    \end{align*}
    Multiplying the second equation by $i$ and adding the two we obtain \begin{equation*}
        2\inner{Ah}{g} = 2\inner{A^*h}{g}
    \end{equation*}
    As this holds for all $h,g$, we find $A = A^*$.
\end{proof}

The preceding proposition is false if it is only assumed that $\mathscr{H}$ is an $\R$-Hilbert space.

\begin{prop}
    If $A = A^*$ then $$\norm{A} = \sup\{|\inner{Ah}{h}|:\norm{h} = 1\}$$
\end{prop}
\begin{proof}
    Put $M = \sup\{|\inner{Ah}{h}|:\norm{h} = 1\}$. If $\norm{h} = 1$, then $|\inner{Ah}{h}| \leq \norm{Ah}\norm{h} \leq \norm{A}$; hence $M \leq \norm{A}$. On the other hand, if $\norm{h} = \norm{g} = 1$, then \begin{align*}
        \inner{A(h\pm g)}{h\pm g} &= \inner{Ah}{h}\pm \inner{Ah}{g}\pm\inner{Ag}{h}+\inner{Ag}{g} \\
        &= \inner{Ah}{h}\pm \inner{Ah}{g}\pm\inner{g}{Ah}+\inner{Ag}{g} \tag{since $A= A^*$} \\
        &= \inner{Ah}{h}\pm 2\text{Re}\inner{Ah}{g} + \inner{Ag}{g}
    \end{align*}
    Subtracting one of these equations from the other gives $$4\text{Re}\inner{Ah}{g} = \inner{A(h+g)}{h+g} -\inner{A(h-g)}{h-g}$$
    Now it is follows by linearity that $|\inner{Af}{f}| \leq M\norm{f}^2$ for any $f \in \mathscr{H}$. Hence, using the parallelogram law we get \begin{align*}
        4\text{Re}\inner{Ah}{g} &\leq M(\norm{h+g}^2+\norm{h-g}^2) \\
        &= 2M(\norm{h}^2+\norm{g}^2) \\
        &= 4M
    \end{align*}
    since $h$ and $g$ are unit vectors. Now suppose $\inner{Ah}{g} = e^{i\theta}|\inner{Ah}{g}|$. Replacing $h$ in the inequality above with $e^{-i\theta}h$ gives $|\inner{Ah}{g}| \leq M$ if $\norm{h} = \norm{g} = 1$. Taking the supremum over all $g$ gives $\norm{Ah} \leq M$ when $\norm{h} = 1$. Thus $\norm{A} \leq M$.
\end{proof}

\begin{cor}
    If $A = A^*$ and $\inner{Ah}{h} = 0$ for all $h$, then $A = 0$.
\end{cor}

The preceding corollary is not true unless $A = A^*$. However, if a complex Hilbert space is present, this hypothesis can be deleted.

\begin{prop}
    If $\mathscr{H}$ is a $\C$-Hilbert space and $A \in \mathscr{B}(\mathscr{H})$, such that $\inner{Ah}{h} = 0$ for all $h \in \mathscr{H}$, then $A = 0$.
\end{prop}
\begin{proof}
    Suppose $\inner{Ah}{h} = 0$ for all $h$. Then for $h,g$, $$0= \inner{A(h\pm ig)}{h\pm ig} = \inner{Ah}{h}\pm i\inner{Ag}{h} \mp i\inner{Ah}{g}-\inner{Ag}{g} = \pm i\inner{Ag}{h} \mp i\inner{Ah}{g}$$
    Thus $\pm i\inner{Ag}{h} = \pm i \inner{Ah}{g}$. Hence $\inner{Ag}{h} = \inner{Ah}{g}$. Further, $$0 = \inner{A(h\pm g)}{h\pm g} = \pm\inner{Ag}{h}\pm\inner{Ah}{g}$$
    so $\inner{Ag}{h} = -\inner{Ah}{g} = -\inner{Ag}{h}$, so $\inner{Ag}{h} = 0$. As this holds for all $g,h$, $A = 0$.
\end{proof}

If $\mathscr{H}$ is a $\C$-Hilbert space and $A \in \mathscr{B}(\mathscr{H})$, then $B = (A+A^*)/2$ and $C = (A-A^*)/2i$ are self-adjoint and $A = B+iC$. The operators $B$ and $C$ are called the \textbf{real and imaginary parts of $A$}.

\begin{prop}
    If $A \in \mathscr{B}(\mathscr{H})$, the following statements are equivalent. \begin{enumerate}
        \item[(a)] $A$ is normal.
        \item[(b)] $\norm{Ah} = \norm{A^*h}$ for all $h$.
        \item[(c)] If $\mathscr{H}$ is also a $\C$-Hilbert space, these are equivalent to the real and imaginary parts of $A$ commuting.
    \end{enumerate}
\end{prop}
\begin{proof}
    If $h \in \mathscr{H}$, then $$\norm{Ah}^2 - \norm{A^*h}^2 = \inner{Ah}{Ah} - \inner{A^*h}{A^*h} = \inner{(A^*A-AA^*)h}{h}$$
    Since $A^*A-AA^*$ is hermitian, the equivalence of (a) and (b) follow from the corollary.

    If $B,C$ are the real and imaginary parts of $A$, then a calculation yields \begin{align*}
        A^*A &= B^2-iCB + iBC + C^2 \\
        AA^* &= B^2 + iCB - iBC + C^2
    \end{align*}
    Hence $A^*A = AA^*$ if and only if $2iCB = 2iBC$, if and only if $CB = BC$.
\end{proof}

\begin{prop}
    If $A \in \mathscr{B}(\mathscr{H})$, the following statements are equivalent. \begin{enumerate}
        \item[(a)] $A$ is an isometry.
        \item[(b)] $A^*A = I$
        \item[(c)] $\inner{Ah}{Ag} = \inner{h}{g}$ for all $h,g \in \mathscr{H}$
    \end{enumerate}
\end{prop}
\begin{proof}
    The equivalence of (a) and (c) was shown in Chapter 1. Note that if $h,g \in \mathscr{H}$, then $\inner{A^*Ah}{g} = \inner{Ah}{Ag}$. Hence (b) and (c) are equivalent.
\end{proof}

\begin{prop}
    If $A \in \mathscr{B}(\mathscr{H})$, then the following statements are equivalent. \begin{enumerate}
        \item[(a)] $A^*A = AA^* = I$
        \item[(b)] $A$ is unitary
        \item[(c)] $A$ is a normal isometry
    \end{enumerate}
\end{prop}
\begin{proof}
    (a) implies $A$ is invertible and an isometry by the previous result, so we have (b). For (b) implies (c) observe that $A^*A = I$ since $A$ is an isometry. But, $A$ is invertible so by uniqueness of the inverse $A^* = A^{-1}$ and $A^*A = AA^* = I$. Thus $A$ is normal.

    Finally, for (c) implies (a), $A^*A = I$ since $A$ is an isometry, and as $A$ is also normal $AA^* = A^*A = I$.
\end{proof}

\begin{thm}
    If $A \in \mathscr{B}(\mathscr{H})$, then $\ker A = (\ran A^*)^{\perp}$.
\end{thm}
\begin{proof}
    If $h \in \ker A$ and $g \in \mathscr{H}$, then $\inner{h}{A^*g} = \inner{Ah}{g} = 0$, so $\ker A \subseteq (\ran A^*)^{\perp}$. On the other hand, if $h \perp \ran A^*$ and $g \in \mathscr{H}$, then $$\inner{Ah}{g} = \inner{h}{A^*g} = 0$$
    so $(\ran A^*)^{\perp} \subseteq \ker A$.
\end{proof}
Since $A^{**} = A$, it also holds that $\ker A^* = (\ran A)^{\perp}$. Second, $(\ker A)^{\perp} \neq \ran A^*$ in general, since $\ran A^*$ may not be closed. All that can be said is $(\ker A)^{\perp} = \text{cl}(\ran A^*)$ and $(\ker A^*)^{\perp} = \text{cl}(\ran A)$.


\section{Projections and Idempotents}
\label{sec:ProjId}

\begin{defn}\index{Idempotent}
    An \textbf{idempotent} on $\mathscr{H}$ is a bounded linear operator $E$ on $\mathscr{H}$ such that $E^2= E$. A \textbf{projection} is an idempotent $P$ such that $\ker P = (\ran P)^{\perp}$.
\end{defn}
If $\mathcal{M} \leq \mathscr{H}$, then $P_{\mathcal{M}}$ is a projection.

Let $E$ be any idempotent and set $\mathcal{M} = \ran E$, and $\mathcal{N} = \ker E$. Since $E$ is continuous, $\mathcal{N}$ is a closed subspace of $\mathscr{H}$. Notice that $(I-E)^2 = I-2E+E^2 = I-2E+E = I-E$; thus $I-E$ is also an idempotent. Also $0 = (I-E)h = h-Eh$ if and only if $Eh = h$, so $\ran E \supseteq \ker(I-E)$. On the other hand, if $h \in \ran E$, $h = Eg$, and so $Eh = E^2g = Eg =h$; hence $\ran E = \ker(I-E)$. Similarly, $\ran(I-E) = \ker E$.

\begin{prop}
    \begin{enumerate}
        \item[(a)] $E$ is an idempotent if and only if $I-E$ is an idempotent
        \item[(b)] $\ran E = \ker(I-E), \ker E = \ran(I-E)$, and both $\ran E$ and $\ker E$ are closed linear subspaces of $\mathscr{H}$.
        \item[(c)] If $\mathcal{M} = \ran E$ and $\mathcal{N} = \ker E$, then $\mathcal{M}\cap \mathcal{N} = (0)$ and $\mathcal{M}+\mathcal{N} = \mathscr{H}$.
    \end{enumerate}
\end{prop}

If $\mathcal{M},\mathcal{N} \leq \mathscr{H},\mathcal{M}\cap\mathcal{N} = (0)$, and $\mathcal{M}+\mathcal{N} = \mathscr{H}$, then there is an idempotent $E$ such that $\mathcal{M} = \ran E$ and $\mathcal{N} = \ker E$; moreover, $E$ is unique.

\begin{prop}
    If $E$ is an idempotent on $\mathscr{H}$ and $E \neq 0$, the following statements are equivalent.
    \begin{enumerate}
        \item[(a)] $E$ is a projection
        \item[(b)] $E$ is the orthogonal projection of $\mathscr{H}$ onto $\ran E$
        \item[(c)] $\norm{E} = 1$
        \item[(d)] $E$ is hermitian
        \item[(e)] $E$ is normal
        \item[(f)] $\inner{Eh}{h} \geq 0$ for all $h \in \mathscr{H}$.
    \end{enumerate}
\end{prop}
\begin{proof}
    For (a) implies (b) let $\mathcal{M} = \ran E$ and $P = P_{\mathcal{M}}$. If $h \in \mathscr{H}$, $Ph = $ the unique vector in $\mathcal{M}$ such that $h-Ph \in \mathcal{M}^{\perp} = (\ran E)^{\perp} = \ker E$ by (a). But $h-Eh = (I-E)h \in \ker E$.Hence $Eh = Ph$ by uniqueness.

    For (b) implies (c) we have $\norm{E} \leq 1$. But $Eh = h$ for $h \in \ran E$, so $\norm{E} = 1$.

    For (c) implies (a) let $h \in (\ker E)^{\perp}$. Now $\ran(I-E) = \ker E$, so $h-Eh \in \ker E$. Hence $0 = \inner{h-Eh}{h} = \norm{h}^2-\inner{Eh}{h}$. Hence $\norm{h}^2 = \inner{Eh}{h} \leq \norm{Eh}\norm{h} \leq \norm{h}^2$. So for $h \in (\ker E)^{\perp}$, $\norm{Eh} = \norm{h} = \inner{Eh}{h}^{1/2}$. But then for $h \in (\ker E)^{\perp}$, $$\norm{h-Eh}^2 = \norm{h}^2-2\text{Re}\inner{Eh}{h}+\norm{Eh}^2 = 0$$
    That is $(\ker E)^{\perp} \subseteq \ker(I-E) = \ran E$. On the other hand, if $g \in \ran E$, $g = g_1+g_2$, where $g_1 \in \ker E$ and $g_2 \in (\ker E)^{\perp}$. Thus $g = Eg=Eg_2 = g_2$; that is, $\ran E \subseteq (\ker E)^{\perp}$. Therefore $\ran E = (\ker E)^{\perp}$ and $E$ is a projection.

    For (b) implies (f) If $h \in \mathscr{H}$, write $h = h_1+h_2$, $h_1 \in \ran E$ and $h_2 \in \ker E = (\ran E)^{\perp}$. Hence $$\inner{Eh}{h} = \inner{E(h_1+h_2)}{h_1+h_2} = \inner{Eh_1}{h_1} = \inner{h_1}{h_1} = \norm{h_1}^2 \geq 0$$.

    For (f) implies (a) let $h_1 \in \ran E$ and $h_2 \in \ker E$. Then $$0 \leq \inner{E(h_1+h_2)}{h_1+h_2} = \inner{h_1}{h_1}+\inner{h_1}{h_2}$$
    Hence $-\norm{h_1}^2 \leq \inner{h_1}{h_2}$. If there are such $h_1,h_2$ with $\inner{h_1}{h_2} = \overline{\alpha} \neq 0$, then substituting $k_2 = -2\alpha^{-1}\norm{h_1}^2h_2$ for $h_2$ in this inequality, we obtain $-\norm{h_1}^2 \leq -2\norm{h_1}^2$, a contradiction. Hence $\inner{h_1}{h_2} = 0$ whenever $h_1 \in \ran E$ and $h_2 \in \ker E$. That is, $E$ is a projection.


    For (a) implies (d) let $h,g \in \mathscr{H}$ and put $h = h_1+h_2$ and $g = g_1+g_2$, where $h_1,h_2 \in \ran E$ and $h_2,g_2 \in \ker E = (\ran E)^{\perp}$. Hence $\inner{Eh}{g} = \inner{h_1}{g_1}$. Also, $\inner{E^*h}{g} = \inner{h}{Eg} = \inner{h_1,g_1} = \inner{Eh}{g}$. Thus $E = E^*$.

    (d) implies (e) is always true for a hermitian operator.

    Finally, for (e) implies (a), recall that $E$ being normal implies $\norm{Eh} = \norm{E^*h}$ for all $h$. Hence $\ker E = \ker E^*$. But $\ker E^* = (\ran E)^{\perp}$, so $E$ is a projection.
\end{proof}

\begin{defn}
    If $\{\mathscr{M}_i\}$ is a collection of pairwise orthogonal subspaces of $\mathscr{H}$, then $$\bigoplus_i\mathscr{M}_i := \land_i\mathscr{M}_i$$
    If $\mathscr{M},\mathscr{N}$ are two closed linear subspaces of $\mathscr{H}$, then $$\mathscr{M}\ominus\mathscr{N} := \mathscr{M}\cap \mathscr{N}^{\perp}$$
    THis is called the \textbf{orthogonal difference} of $\mathscr{M}$ and $\mathscr{N}$.
\end{defn}

Note that if $\mathscr{M},\mathscr{N}\leq \mathscr{H}$ and $\mathscr{M}\perp\mathscr{N}$, then $\mathscr{M}+\mathscr{N}$ is closed.

\begin{defn}
    If $A \in \mathscr{B}(\mathscr{H})$ and $\mathscr{M} \leq \mathscr{H}$, say that $\mathscr{M}$ is an \textbf{invariant subspace} for $A$ if $Ah \in \mathscr{M}$ whenever $h \in \mathscr{M}$. In other words, if $A\mathscr{M} \subseteq \mathscr{M}$. Say that $\mathscr{M}$ is a \textbf{reducing subspace} for $A$ if $A\mathscr{M}\subseteq \mathscr{M}$ and $A\mathscr{M}^{\perp} \subseteq \mathscr{M}^{\perp}$.
\end{defn}

If $\mathscr{M} \leq \mathscr{H}$ then $\mathscr{H} = \mathscr{M} \oplus \mathscr{M}^{\perp}$. If $A \in \mathscr{B}(\mathscr{H})$, then $A$ can be written as a $2\times 2$ matrix with operator entries $$A = \begin{bmatrix} W & X \\ Y & Z \end{bmatrix}$$
where $W \in \mathscr{B}(\mathscr{M}), X\in \mathscr{B}(\mathscr{M}^{\perp},\mathscr{M}),Y\in\mathscr{B}(\mathscr{M},\mathscr{M}^{\perp})$, and $Z \in \mathscr{B}(\mathscr{M}^{\perp})$.

\begin{prop}
    If $A \in \mathscr{B}(\mathscr{H}), \mathscr{M} \leq \mathscr{H}$ and $P = P_{\mathscr{M}}$, then statements (a) though (c) are equivalent.
    \begin{enumerate}
        \item[(a)] $\mathscr{M}$ is invariant for $A$
        \item[(b)] $PAP = AP$
        \item[(c)] In the above matrix $Y = 0$.
    \end{enumerate}
    Also, statements (d) through (g) are equivalent.
    \begin{enumerate}
        \item[(d)] $\mathscr{M}$ reduces $A$
        \item[(e)] $PA = AP$.
        \item[(f)] In the above matrix $Y$ and $X$ are $0$.
        \item[(g)] $\mathscr{M}$ is invariant for both $A$ and $A^*$.
    \end{enumerate}
\end{prop}
\begin{proof}
    For (a) implies (b), if $h \in \mathscr{H}$, $Ph \in \mathscr{M}$. So $APh \in \mathscr{M}$. Hence $P(APh) = APh$. That is, $PAP = AP$.

    For (b) implies (c), if $P$ is represented as a $2\times 2$ matrix relative to $\mathscr{H} = \mathscr{M}\oplus \mathscr{M}^{\perp}$, then $$P = \begin{bmatrix} I & 0 \\ 0 & 0 \end{bmatrix}$$
    Hence, $$PAP = \begin{bmatrix} W & 0 \\ 0 & 0 \end{bmatrix} = AP = \begin{bmatrix} W & 0 \\ Y & 0 \end{bmatrix}$$
    So $Y = 0$.

    For (c) implies (a), if $Y = 0$ and $h \in \mathscr{M}$, then $$Ah = \begin{bmatrix} W & X \\ 0 & Z\end{bmatrix}\begin{bmatrix} h \\ 0 \end{bmatrix} = \begin{bmatrix} Wh \\ 0 \end{bmatrix} \in \mathscr{M}$$


    Next, for (d) implies (e) since both $\mathscr{M}$ and $\mathscr{M}^{\perp}$ are invariant for $A$, (b) implies that $AP = PAP$ and $A(I-P) = (I-P)A(I-P)$. Multiplying the second equation gives $A-AP = A-AP-PA+PAP$. Thus $PA = PAP = AP$.

    for (e) implies (f) we have again the presentation of $P$ as above, so $$PA = \begin{bmatrix} W & X \\ 0 & 0 \end{bmatrix} = AP = \begin{bmatrix} W & 0 \\ Y & 0 \end{bmatrix}$$
    so $X$ and $Y$ are $0$.

    for (f) implies (g), if $X$ and $Y$ are $0$, then $$A = \begin{bmatrix} W & 0 \\ 0 & Z \end{bmatrix} \;\text{ and }\;A^* = \begin{bmatrix} W^* & 0 \\ 0 & Z^* \end{bmatrix}$$
    By (c), $\mathscr{M}$ is invariant for both $A$ and $A^*$.

    For (g) implies (d), if $h \in \mathscr{M}^{\perp}$ and $g \in \mathscr{M}$, then $\inner{g}{Ah} = \inner{A^*g}{h} = 0$, sicne $A^*g \in \mathscr{M}$. Since $g$ was an arbitrary vector of $\mathscr{M}, Ah \in \mathscr{M}^{\perp}$. That is, $A\mathscr{M}^{\perp} \subseteq\mathscr{M}^{\perp}$.
\end{proof}

If $\mathscr{M}$ reduces $A$, then $X$ and $Y$ are $0$ in our matrix decomposition. This says that a study of $A$ is reduced to the study of the smaller operators $W$ and $Z$. If $A \in \mathscr{B}(\mathscr{H})$ and $\mathscr{M}$ is an invariant subspace, then $\norm{A\vert_{\mathscr{M}}} \leq \norm{A}$, so the restriction is also bounded.


\section{Compact Operators}
\label{sec:comp}

Let $B_{\mathscr{H}}$ denote the closed unit ball in $\mathscr{H}$.

\begin{defn}\index{Compact linear transformation}
    A linera transformation $T:\mathscr{H}\rightarrow \mathscr{K}$ is \textbf{compact} if $T(B_{\mathscr{H}})$ has compact closure in $\mathscr{K}$. The set of compact operators from $\mathscr{H}$ into $\mathscr{K}$ is denoted by $\mathscr{B}_0(\mathscr{H},\mathscr{K})$, and $\mathscr{B}_0(\mathscr{H}) = \mathscr{B}_0(\mathscr{H},\mathscr{H})$.
\end{defn}

\begin{prop}
    \begin{enumerate}
        \item[(a)] $\mathscr{B}_0(\mathscr{H},\mathscr{K}) \subseteq \mathscr{B}(\mathscr{H},\mathscr{K})$
        \item[(b)] $\mathscr{B}_0(\mathscr{H},\mathscr{K})$ is a linear space and if $\{T_n\} \subseteq \mathscr{B}_0(\mathscr{H},\mathscr{K})$ and $T \in \mathscr{B}(\mathscr{H},\mathscr{K})$ such that $\norm{T_n-T} \rightarrow 0$, then $T \in \mathscr{B}_0(\mathscr{H},\mathscr{K})$.
        \item[(c)] If $A\in\mathscr{B}(\mathscr{H}),B \in \mathscr{B}(\mathscr{K})$, and $T \in \mathscr{B}_0(\mathscr{H},\mathscr{K})$, then $TA$ and $BT \in \mathscr{B}_0(\mathscr{H},\mathscr{K})$.
    \end{enumerate}
\end{prop}
\begin{proof}
    (a) If $T \in \mathscr{B}_0(\mathscr{H},\mathscr{K})$, then $\text{cl}[T(B_{\mathscr{H}})]$ is compact in $\mathscr{K}$. Hence, there is a constant $C > 0$ such that $T(B_{\mathscr{H}}) \subseteq \{k \in \mathscr{K}:\norm{k} \leq C\}$. Thus $\norm{T} \leq C$.

    (b) Let $T,S \in \mathscr{B}_0(\mathscr{H},\mathscr{K})$ and $\alpha \in \F$. If $\alpha = 0$ then $\alpha T = 0$, which is trivially compact. Otherwise, if $\alpha \neq 0$, multiplication by $\alpha$ is a homeomorphism on $\mathscr{K}$. Further, if $h \in \text{cl}[\alpha T(B_{\mathscr{H}})]$, then for all $\epsilon > 0$ there exists $k \in B_{\mathscr{H}}$ such that $\norm{h-\alpha T(k)} < \epsilon$. This implies $h/\alpha \in \text{cl}[T(B_{\mathscr{H}})]$, and so $h \in \alpha\text{cl}[T(B_{\mathscr{H}})]$. By symmetry $\text{cl}[\alpha T(B_{\mathscr{H}})] = \alpha \text{cl}[T(B_{\mathscr{H}})]$, we have that the two spaces are homeomorphic, and so as one is compact so must the other be. Hence $\alpha T \in \mathscr{B}_0(\mathscr{H},\mathscr{K})$.

    Next, as $\mathscr{K}$ is complete it is sufficient to show that $(S+T)(B_{\mathscr{H}})$ is totally bounded. For $\epsilon > 0$, there exist $s_1,...,s_n,t_1,...,t_m$ such that $S(B_{\mathscr{H}}) \subseteq \bigcup_{i=1}^nB_{\epsilon/2}(s_i)$ and $T(B_{\mathscr{H}}) \subseteq \bigcup_{j=1}^mB_{\epsilon/2}(t_j)$. I claim $(S+T)(B_{\mathscr{H}}) \subseteq \bigcup_{i,j} B_{\epsilon}(s_i+t_j)$. Let $b \in B_{\mathscr{H}}$. Then there exist $i,j$ such that $S(b) \in B_{\epsilon/2}(s_i)$ and $T(b) \in B_{\epsilon/2}(t_j)$. Then $$\norm{S(b)+T(b)-s_i-t_j} \leq \norm{S(b)-s_i}+\norm{T(b)-t_j} < \epsilon$$
    as desired. Therefore the closure of $(S+T)(B_{\mathscr{H}})$ is compact, so $S+T \in \mathscr{B}_0(\mathscr{H},\mathscr{K})$.

    Next, suppose $T_n$ and $T$ are as in (b). We again show totally boundedness for $T(B_{\mathscr{H}})$. Let $\epsilon > 0$ and choose $n$ such that $\norm{T-T_n} < \epsilon/3$. Since $T_n$ is compact, there are vectors $h_1,...,h_m$ in $B_{\mathscr{H}}$ such that $T_n(B_{\mathscr{H}}) \subseteq \bigcup_{j=1}^m B_{\epsilon/3}(T_nh_j)$. So if $\norm{h} \leq 1$, there is an $h_j$ with $\norm{T_nh_j-T_nh} < \epsilon/3$. Thus \begin{align*}
        \norm{Th_j-Th} &\leq \norm{Th_j-T_nh_j} + \norm{T_nh_j-T_nh} + \norm{T_nh-Th} \\
        &<2\norm{T-T_n}+\epsilon/3 \\
        &< \epsilon
    \end{align*}
    Hence $T(B_{\mathscr{H}}) \subseteq \bigcup_{j=1}^mB_{\epsilon}(Th_j)$, completing the claim.

    For (c), suppose $A,B$, and $T$ are as in the claim. As $A$ is bounded $A(B_{\mathscr{H}}) \subseteq \norm{A}B_{\mathscr{H}}$. But then $T(\norm{A}B_{\mathscr{H}}) = \norm{A}T(B_{\mathscr{H}})$, which has compact closure from part (b). Thus as $\text{cl}[TA(B_{\mathscr{H}})] \subseteq \text{cl}[\norm{A}T(B_{\mathscr{H}})]$, it follows that $\text{cl}[TA(B_{\mathscr{H}})]$ is compact, so $TA$ is compact.

    Finally, it is sufficient to show $BT(B_{\mathscr{H}})$ is totally bounded. If $B = 0$ then this is trivial. Otherwise, let $\epsilon > 0$, so there exist $b_1,...,b_n \in B_{\mathscr{H}}$ such that $T(B_{\mathscr{H}}) \subseteq \bigcup_{j=1}^nB_{\epsilon/\norm{B}}(b_j)$. Now $$BT(B_{\mathscr{H}}) \subseteq \bigcup_{j=1}^nB(B_{\epsilon/\norm{B}}(b_j)) \subseteq \bigcup_{j=1}^n\norm{B}B_{\epsilon/\norm{B}}(b_j) = \bigcup_{j=1}^nB_{\epsilon}(b_j)$$
    This completes the proof.
\end{proof}

\begin{defn}
    An operator $T$ on $\mathscr{H}$ has \textbf{finite rank} if $\ran T$ is finite dimensional. The set of continuous finite rank operators is denoted by $\mathscr{B}_{00}(\mathscr{H},\mathscr{K})$.
\end{defn}

It is clear that $\mathscr{B}_{00}(\mathscr{H},\mathscr{K})$ is a linear space. It is also true that $\mathscr{B}_{00}(\mathscr{H},\mathscr{K}) \subseteq \mathscr{B}_0(\mathscr{H},\mathscr{K})$ since \textbf{TBC}

\begin{thm}
    If $T \in \mathscr{B}(\mathscr{H},\mathscr{K})$, the following statements are equivalent.
    \begin{enumerate}
        \item[(a)] $T$ is compact
        \item[(b)] $T^*$ is compact
        \item[(c)] There is a sequence $\{T_n\}$ of operators of finite rank such that $\norm{T-T_n}\rightarrow 0$.
    \end{enumerate}
\end{thm}
\begin{proof}
    (c) implies (a) follows from the previous result and the fact that $\mathscr{B}_{00}(\mathscr{H},\mathscr{K}) \subseteq \mathscr{B}_0(\mathscr{H},\mathscr{K})$.

    For (a) implies (c), since $\text{cl}[T(B_{\mathscr{H}})]$ is compact, it is separable. Therefore, $\text{cl}(\ran T) = \mathscr{L}$ is a separable subspace of $\mathscr{K}$. Let $\{e_1,e_2,...\}$ be a basis for $\mathscr{L}$ and let $P_n$ be the orthogonal projection of $\mathscr{K}$ onto $\land\{e_j:1\leq j\leq n\}$. Put $T_n = P_nT$; note that each $T_n$ has finite rank. It will be shown that $\norm{T_n-T}\rightarrow 0$.

    \textbf{Claim.} If $h \in \mathscr{H}$, $\norm{T_nh-Th}\rightarrow 0$.

    In fact, $k = Th \in \mathscr{L}$, so $\norm{P_nk-k}\rightarrow 0$ from our previous results, and so $\norm{P_nTh-Th} \rightarrow 0$ and the claim is proved.

    Since $T$ is compact, if $\epsilon > 0$, there are vectors $h_1,...,h_m \in B_{\mathscr{H}}$ such that $T(B_{\mathscr{H}}) \subseteq \bigcup_{j=1}^mB_{\epsilon/3}(Th_j)$. So if $\norm{h} \leq 1$, choose $h_j$ with $\norm{Th-Th_j} < \epsilon/3$. Thus for any integer $n$, \begin{align*}
        \norm{Th-T_nh}&\leq \norm{Th-Th_j}+\norm{Th_j-T_nh_j}+\norm{P_n(Th_j-Th)} \\
        &\leq 2\norm{Th-Th_j}+\norm{Th_j-T_nh_j} \\
        &\leq 2\epsilon/3+\norm{Th_j-T_nh_j}
    \end{align*}
    Using the claim we can find $n_0$ such that $\norm{Th_j-T_nh_j} < \epsilon/3$ for $1 \leq j \leq m$ and $n \geq n_0$. So $\norm{Th-T_nh} < \epsilon$ uniformly for $h \in B_{\mathscr{H}}$. Therefore, $\norm{T-T_n} < \epsilon$ for $n \geq n_0$.

    For (c) implies (b), if $\{T_n\}$ is a sequence in $\mathscr{B}_{00}(\mathscr{H},\mathscr{K})$ such that $\norm{T_n-T}\rightarrow 0$, then $\norm{T_n^*-T^*} = \norm{T_n-T}\rightarrow 0$. But $T_n^* \in \mathscr{B}_{00}(\mathscr{H},\mathscr{K})$ (TBC). Since (c) implies (a), $T^*$ is compact.

    (b) implies (a) (TBC).
\end{proof}

\begin{cor}
    If $T \in \mathscr{B}_0(\mathscr{H},\mathscr{K})$, then $\text{cl}(\ran T)$ is separable and if $\{e_n\}$ is a basis for $\text{cl}(\ran T)$ and $P_n$ is the projection of $\mathscr{K}$ onto $\land\{e_j:1\leq j\leq n\}$, then $\norm{P_nT-T}\rightarrow 0$.
\end{cor}

\begin{prop}
    Let $\mathscr{H}$ be a separable Hilbert space with basis $\{e_n\}$; let $\{\alpha_n\}\subseteq \F$ with $M = \sup\{|\alpha_n|:n\geq 1\} < \infty$. If $Ae_n = \alpha_ne_n$ for all $n$, then $A$ extends by linearity to a bounded operator on $\mathscr{H}$ with $\norm{A} = M$. The operator $A$ is compact if and only if $\alpha_n\rightarrow 0$ as $n\rightarrow \infty$.
\end{prop}
\begin{proof}
    By definition we have that $\norm{A} \geq M$. Further, if $h \in \mathscr{H}$ with $\norm{h} = 1$ we have that $h = \sum_{n}\inner{h}{e_n}e_n$ and $1 = \norm{h}^2 = \sum_n|\inner{h}{e_n}|^2$. Then $Ah_k = \sum_{n=1}^k\alpha_n\inner{h}{e_n}e_n$. As $\sum_n|\alpha_n\inner{h}{e_n}|^2 \leq M^2 \sum_n|\inner{h}{e_n}|^2 < \infty$ converges, we have from previous results that $Ah_k$ converges to $\sum_n\alpha_n\inner{h}{e_n}e_n$, which we define to be $Ah$. Then $\norm{Ah}^2 = \sum_n|\alpha_n\inner{h}{e_n}|^2 \leq M^2\norm{h}^2 = M^2$, so $\norm{Ah} \leq M$ and so $\norm{A} \leq M$ taking supremums. Thus $A$ is bounded and $\norm{A} = M < \infty$.

    Let $P_n$ be the projection of $\mathscr{H}$ onto $\land\{e_1,...,e_n\}$. Then $A_n = A-AP_n$ is seen to satisfy $A_ne_j = \alpha_je_j$ if $j > n$ and $A_ne_j = 0$ if $j \leq n$. So $AP_n \in \mathscr{B}_{00}(\mathscr{H})$ and $\norm{A_n} = \sup\{|\alpha_j|: j > n\}$ from our initial work. If $\alpha_n\rightarrow 0$ then $\norm{A_n}\rightarrow 0$ and so $A$ is compact since it is the limit of a sequence of finite-rank operators. Conversely, if $A$ is compact, then the previous Corollary implies $\norm{A_n}\rightarrow 0$; hence $\alpha_n\rightarrow 0$.
\end{proof}

\begin{prop}
    If $(X,\Omega,\mu)$ is a measure space and $k \in L^2(X\times X,\Omega\times \Omega,\mu\times \mu)$, then $$(Kf)(x) = \int k(x,y)f(y)d\mu(y)$$
    is a compact operator and $\norm{K} \leq \norm{k}_2$.
\end{prop}

We require the following lemma.

\begin{lem}
    If $\{e_i:i \in I\}$ is a basis for $L^2(X,\Omega,\mu)$ and $$\phi_{ij}(x,y) = e_j(x)\overline{e_i(y)}$$
    for $i,j \in I$, $x,y \in X$, then $\{\phi_{ij}:i,j \in I\}$ is an orthonormal set in $L^2(X\times X,\Omega \times \Omega,\mu\times \mu)$. If $k$ and $K$ are as in the preceding proposition, then $\inner{k}{\phi_{ij}} = \inner{Ke_j}{e_i}$.
\end{lem}

We can now prove the proposition.
    
\begin{proof}
    First we show that $K$ defines a bounded operator. In fact, if $f \in L^2(\mu)$, $$\norm{Kf}^2 = \int\left|\int k(x,y)f(y)d\mu(y)\right|^2d\mu(x) \leq \int\left(\int |k(x,y)|^2d\mu(y)\right)\left(\int|f(y)|^2d\mu(y)\right)d\mu(x) = \norm{k}^2\norm{f}^2$$
    Hence $K$ is bounded and $\norm{K} \leq \norm{k}_2$.

    Now let $\{e_i\}$ be a basis for $L^2(\mu)$ and define $\phi_{ij}$ as in the statement of the lemma. Thus $$\norm{k}^2 \geq \sum_{i,j}|\inner{k}{\phi_{ij}}|^2 = \sum_{i,j}|\inner{Ke_j}{e_i}|^2$$
    Since $k \in L^2(\mu\times \mu)$, there are at most a countable number of $i$ and $j$ such that $\inner{k}{\phi_{ij}} \neq 0$; denote these by $\{\psi_{km}:1\leq k,m < \infty\}$. Note that $\inner{Ke_j}{e_i} = 0$ unless $\phi_{ij} \in \{\psi_{km}\}$. Let $P_n$ be the orthogonal projection onto $\land\{e_k:1\leq k \leq n\}$, and put $K_n = KP_n + P_nK - P_nKP_n$; so $K_n$ is a finite rank operator.

    Let $f \in L^2(\mu)$ with $\norm{f}^2 \leq 1$; so $f = \sum_j\alpha_je_j$. Hence \begin{align*}
        \norm{Kf-K_nf}^2 &= \sum_i|\inner{Kf-K_nf}{e_i}|^2 \\
        &= \sum_i\left|\sum_j\alpha_j\inner{(K-K_n)e_j}{e_i}\right|^2 \\
        &= \sum_k\left|\sum_m\alpha_m\inner{(K-K_n)e_m}{e_k}\right|^2 \\
        &\leq \sum_k\left[\sum_m|\alpha_m|^2\right]\left[\sum_m|\inner{(K-K_n)e_m}{e_k}|^2\right] \\
        &\leq \norm{f}^2 \sum_k\sum_m|\inner{Ke_m,e_k}-\inner{KP_ne_m}{P_ne_k} \\
        &\hspace{15pt}-\inner{KP_ne_m}{P_ne_k} + \inner{KP_ne_m}{P_ne_k}|^2 \\
        &\leq \sum_{k=n+1}^{\infty}\sum_{m=n+1}^{\infty}|\inner{Ke_m}{e_k}|^2 \\
        &= \sum_{k=n+1}^{\infty}\sum_{m=n+1}^{\infty}|\inner{k}{\psi_{km}}|^2
    \end{align*}
    SInce $\sum_{k,m}|\inner{k}{\psi_{km}}|^2 < \infty$, $n$ can be chosen sufficiently large such that for any $\epsilon > 0$ this last sum will be smaller than $\epsilon^2$. Thus $\norm{K-K_n}\rightarrow 0$.
\end{proof}


We now begin the spectral theory for our operators.

\begin{defn}
    If $A \in \mathscr{B}(\mathscr{H})$, a scalar $\alpha \in \F$ is an \textbf{eigenvalue} of $A$ if $\ker(A-\alpha) \neq (0)$. If $h \in \ker(A-\alpha)\backslash(0)$, $h$ is called an \textbf{eigenvector} for $\alpha$; thus $Ah = \alpha h$. Let $\sigma_p(A)$ denote the set of eigenvalues of $A$.
\end{defn}

\begin{eg}
    Let $A$ be the diagonalizable operator (i.e. we have a diagonal basis) in the proposition above. Then $\sigma_p(A) = \{\alpha_1,\alpha_2,...\}$. If $\alpha \in \sigma_p(A)$, let $J_{\alpha} = \{j \i n\N:\alpha_j = \alpha\}$. Then $h$ isan eigenvector for $\alpha$ if and only if $h \in \land\{e_j:j \in J_{\alpha}\}$.
\end{eg}

\begin{eg}
    The Volterra operator has no eigenvalues.
\end{eg}

\begin{eg}
    Let $h \in \mathscr{H} = L^2_{\C}(-\pi,\pi)$ and define $K:\mathscr{H}\rightarrow \mathscr{H}$ by $$(Kf)(x) = \frac{1}{\sqrt{2\pi}}\int_{-\pi}^{\pi}h(x-y)f(y)dy$$
    If $\lambda_n = \frac{1}{\sqrt{2\pi}}\int_{-\pi}^{\pi}h(x)\exp(-inx)dx = \hat{h}(n)$, the $n$th Fourier coefficient of $h$, then $Ke_n = \lambda_ne_n$, where $e_n(x) = \frac{1}{\sqrt{2\pi}}\exp(-inx)$.
\end{eg}

Operators on finite dimensional spaces over $\C$ always have eigenvalues. As the Volterra operator illustrates, the analogy between operators on finite dimensional spacs and compact operators breaks down here.

\begin{prop}
    If $T \in \mathscr{B}_0(\mathscr{H})$, $\lambda \in \sigma_p(T)$, and $\lambda \neq 0$, then the eigenspace $\ker(T-\lambda)$ is finite dimensional.
\end{prop}
\begin{proof}
    Suppose there is an infinite orthonormal sequence $\{e_n\}$ in $\ker(T-\lambda)$. Since $T$ is compact there is a subsequence $\{e_{n_k}\}$ such that $\{Te_{n_k}\}$ converges. Thus $\{Te_{n_k}\}$ is a Cauchy sequence. But for $n_k \neq n_j$, $\norm{Te_{n_k}-Te_{n_j}}^2 = \norm{\lambda e_{n_k}-\lambda e_{n_j}}^2 = 2|\lambda|^2 > 0$, sicne $\lambda \neq 0$. This contradiction shows that $\ker(T-\lambda)$ must be finite dimensional.
\end{proof}

\begin{prop}
    If $T$ is a compact operator on $\mathscr{H}$, $\lambda \neq 0$, and $\inf\{\norm{(T-\lambda)h}:\norm{h}=1\} = 0$, then $\lambda \in \sigma_p(T)$.
\end{prop}
\begin{proof}
    By hypothesis, there is a sequence of unit vectors $\{h_n\}$ such that $\norm{(T-\lambda)h_n}\rightarrow 0$. Since $T$ is compact, there is a vector $f$ in $\mathscr{H}$ and a subsequence $\{h_{n_k}\}$ such that $\norm{Th_{n_k}-f}\rightarrow 0$. But $h_{n_k} = \lambda^{-1}[(\lambda-T)h_{n_k}+Th_{n_k}]\rightarrow \lambda^{-1}f$. So $1 = \norm{\lambda^{-1}f} = |\lambda|^{-1}\norm{f}$ and $f \neq 0$. Also, it must be that $Th_{n_k}\rightarrow \lambda^{-1}Tf$. Since $Th_{n_k}\rightarrow f$, $f = \lambda^{-1}Tf$, or $Tf = \lambda f$. That is, $f \in \ker(T-\lambda)$ and $f \neq 0$, so $\lambda \in \sigma_p(T)$.
\end{proof}

\begin{cor}
    If $T$ is a compact operator on $\mathscr{H}$, $\lambda \neq 0$, $\lambda \notin\sigma_p(T)$, and $\overline{\lambda}\notin \sigma_p(T^*)$, then $\ran(T-\lambda) = \mathscr{H}$ and $(T-\lambda)^{-1}$ is a bounded operator on $\mathscr{H}$.
\end{cor}
\begin{proof}
    Since $\lambda \notin \sigma_p(T)$, the preceding proposition implies there exists $c > 0$ such that $\norm{(T-\lambda)h} \geq c\norm{h}$ for all $h \in \mathscr{H}$. If $f \in \text{cl}[\ran(T-\lambda)]$, then there is a sequence $\{h_n\}$ in $\mathscr{H}$ such that $(T-\lambda)h_n\rightarrow f$. Thus $\norm{h_n-h_m}\leq c^{-1}\norm{(T-\lambda)h_n-(T-\lambda)h_m}$, and so $\{h_n\}$ is a Cauchy sequence. Hence $h_n\rightarrow h$ for some $h \in \mathscr{H}$. THus $(T-\lambda)h = f$. So $\ran(T-\lambda)$ is closed, and hence $\ran(T-\lambda) = [\ker(T-\lambda)^*]^{\perp} = \mathscr{H}$, by hypothesis.

    So for $f \in \mathscr{H}$ let $Af = $ the unique $h$ such that $(T-\lambda)h = f$. Thus $(T-\lambda)Af = f$ for all $f \in \mathscr{H}$. From the inequality above, $$c\norm{Af} \leq \norm{(T-\lambda)Af} = \norm{f}$$
    so $\norm{Af} \leq c^{-1}\norm{f}$ and $A$ is bounded. Also, $$(T-\lambda)A(T-\lambda)h = (T-\lambda)h$$
    so $0 = (T-\lambda)[A(T-\lambda)h-h]$. Since $\lambda \notin \sigma_p(T)$, $A(T-\lambda)h = h$. That is, $A = (T-\lambda)^{-1}$.
\end{proof}


\section{Diagonalization of Compact Self-Adjoint Operators}
\label{sec:diag}

The main result we aim to prove in this section is the following.

\begin{thm}\label{thm:DiagCompSelfAdj}
    If $T$ is a compact self-adjoint operator on $\mathscr{H}$, then $T$ has only a countable number of distinct eigenvalues. If $\{\lambda_1,\lambda_2,...\}$ are the distinct nonzero eigenvalues of $T$, and $P_n$ is the projection of $\mathscr{H}$ onto $\ker(T-\lambda_n)$, then $P_nP_m = P_mP_n = 0$ if $n \neq m$, eahc $\lambda_n$ is real, and $$T = \sum_{n=1}^{\infty}\lambda_nP_n$$
    where the series converges to $T$ in the metric defined by the norm of $\mathscr{B}(\mathscr{H})$.
\end{thm}

We first collect some preliminary results after providing some consequences of this theorem.

\begin{cor}
    With the notation as in the theorem, \begin{enumerate}
        \item[(a)] $\ker T = [\land \{P_n\mathscr{H}:n\geq 1\}]^{\perp} = (\ran T)^{\perp};$
        \item[(b)] each $P_n$ has finite rank;
        \item[(c)] $\norm{T} = \sup\{|\lambda_n|:n\geq 1\}$ and $\lambda_n\rightarrow 0$ as $n\rightarrow \infty$.
    \end{enumerate}
\end{cor}
\begin{proof}
    Since $P_n \perp P_m$ for $n \neq m$, if $h \in \mathscr{H}$, then $$\norm{Th}^2 = \sum_{n=1}^{\infty}\norm{\lambda_nP_nh}^2 = \sum_{n=1}^{\infty}|\lambda_n|^2\norm{P_nh}^2$$
    Hence $Th = 0$ if and only if $P_nh = 0$ for all $n$. That is $h \in \ker T$ if and only if $h \perp P_n\mathscr{H}$ for all $n$, whence (a).

    Part (b) follows from a proposition in the previous section on eigenspaces of nonzero eigenvalues.

    For part (c), if $\mathscr{L} = \text{cl}[\ran T]$, $\mathscr{L}$ is invariant for $T$. Since $T = T^*$, $\mathscr{L} = (\ker T)^{\perp}$ and $\mathscr{L}$ reduces $T$. So we can consider the restriction of $T$ to $\mathscr{L}$, $T\vert_{\mathscr{L}}$. Now $\mathscr{L} = \land\{P_n\mathscr{H}:n\geq 1\}$ by (a). Let $\{e_j^{(n)}:1\leq j \leq N_n\}$ be a basis for $P_n\mathscr{H} = \ker(T-\lambda_n)$, so $Te_j^{(n)} = \lambda_ne_j^{(n)}$ for $1 \leq j \leq N_n$. Thus $\{e_j^{(n)}:1\leq j \leq N_n, n\geq 1\}$ is a basis for $\mathscr{L}$ and $T\vert_{\mathscr{L}}$ is diagonalizable with respect to this basis. Part (c) now follows from a result in the previous section.
\end{proof}

\begin{cor}
    If $T$ is a compact self-adjoint operator, then there is a sequence $\{\mu_n\}$ of real numbers and an orthonormal basis $\{e_n\}$ for $(\ker T)^{\perp}$ such that for all $h$, $$Th = \sum_{n=1}^{\infty}\mu_n\inner{h}{e_n}e_n$$
\end{cor}
Note that there may be repitions in the sequence $\{\mu_n\}$.

\begin{cor}
    If $T \in \mathscr{B}_0(\mathscr{H})$, $T = T^*$, and $\ker T = (0)$, then $\mathscr{H}$ is separable.
\end{cor}

Now we begin collecting our results for the proof of our main theroem.

\begin{prop}
    If $A$ is a normal operator and $\lambda \in \F$, then $\ker(A-\lambda) = \ker(A-\lambda)^*$ and $\ker(A-\lambda)$ is a reducing subspace for $A$.
\end{prop}
\begin{proof}
    Since $A$ is normal, so is $A-\lambda$. Hence $\norm{(A-\lambda)h} = \norm{(A-\lambda)^*h}$, so $\ker(A-\lambda) = \ker(A-\lambda)^*$. If $h \in \ker(A-\lambda)$, $Ah = \lambda h \in \ker(A-\lambda)$. Also, $A^*h = \overline{\lambda}h \in \ker(A-\lambda)$. Therefore $\ker(A-\lambda)$ reduces $A$.
\end{proof}

\begin{prop}
    If $A$ is a normal operator and $\lambda,\mu$ are distinct eigenvalues of $A$,then $\ker(A-\lambda)\perp \ker(A-\mu)$.
\end{prop}
\begin{proof}
    If $h \in \ker(A-\lambda)$ and $g \in \ker(A-\mu)$, then the previous proposition implies $A^*g = \overline{\mu}g$ and $\lambda \inner{h}{g} = \mu\inner{h}{g}$. Thus $(\lambda-\mu)\inner{h}{g} = 0$. Since $\lambda - \mu \neq 0$, $h\perp g$.
\end{proof}

\begin{prop}
    If $A = A^* $ and $\lambda \in \sigma_p(A)$, then $\lambda$ is a real number.
\end{prop}
\begin{proof}
    If $Ah = \lambda h$, then $Ah = A^*h = \overline{\lambda}h$, so $(\lambda-\overline{\lambda})h = 0$. Since $h$ can be chosen different from $0$, $\lambda = \overline{\lambda}$.
\end{proof}

\begin{lem}
    If $T$ is a compact self-adjoint operator, then either $\pm\norm{T}$ is an eigenvalue of $T$.
\end{lem}
\begin{proof}
    If $T = 0$, the result is immediate. So suppose $T \neq 0$. As $T$ is self-adjoint there is a sequence $\{h_n\}$ of unit vectors such that $|\inner{Th_n}{h_n}|\rightarrow \norm{T}$. By passing to a subsequence if necessary, we may assume that $\inner{Th_n}{h_n}\rightarrow \lambda$, where $|\lambda| = \norm{T}$. It will be shown that $\lambda \in \sigma_p(T)$. Since $|\lambda| = \norm{T}$, $$0 \leq \norm{(T-\lambda)h_n}^2 = \norm{Th_n}^2-2\lambda \inner{Th_n}{h_n}+\lambda^2 \leq 2\lambda^2 - 2\lambda \inner{Th_n}{h_n}\rightarrow 0$$
    Hence $\norm{(T-\lambda)h_n} \rightarrow 0$. Thus $\lambda \in \sigma_p(T)$.
\end{proof}


We now proceed to the proof of the main theorem.

\begin{proof}
    By the Lemma there is a real number $\lambda_1$ in $\sigma_p(T)$ with $|\lambda_1| = \norm{T}$. Let $\mathscr{E}_1 = \ker(T-\lambda_1)$, $P_1 = $ the projection onto $\mathscr{E}_1$, $\mathscr{H}_2 = \mathscr{E}_1^{\perp}$. Then $\mathscr{E}_1$ reduces $T$, so $\mathscr{H}_2$ reduces $T$. Let $T_2 = T\vert_{\mathscr{H}_2};$ then $T_2$ is a self-adjoint compact operator on $\mathscr{H}_2$.

    By the Lemma there is an eigenvalue $\lambda_2$ for $T_2$ such that $|\lambda_2| = \norm{T_2}$. Let $\mathscr{E}_2 = \ker(T_2-\lambda_2)$. Note that $(0) \neq \mathscr{E}_2 \subseteq \ker(T-\lambda_2)$. Since $\mathscr{E}_1\perp\mathscr{E}_2$, $\lambda_1\neq \lambda_2$. Let $P_2$ be the projection of $\mathscr{H}$ onto $\mathscr{E}_2$ and put $\mathscr{H}_3 = (\mathscr{E}_1\oplus \mathscr{E}_2)^{\perp}$. Note that $\norm{T_2}\leq \norm{T}$, so $|\lambda_2|\leq |\lambda_1|$.

    Using induction we obtain a sequence $\{\lambda_n\}$ of real eigenvalues of $T$ such that \begin{enumerate}
        \item[(i)] $|\lambda_1| \geq |\lambda_2|\geq \cdots$;
        \item[(ii)] If $\mathscr{E}_n = \ker(T-\lambda_n), |\lambda_{n+1}| = \norm{T\vert_{(\mathscr{E}_1\oplus\cdots\oplus\mathscr{E}_n)^{\perp}}}$
    \end{enumerate}
    By (i) there is a nonnegative number $\alpha$ such that $|\lambda_n|\rightarrow \alpha$.

    \textbf{Claim.} $\alpha = 0$.

    In fact, let $e_n \in \mathscr{E}_n$, $\norm{e_n} = 1$. Since $T$ is compact there is an $h \in \mathscr{H}$ and a subsequence $\{e_{n_j}\}$ such that $\norm{Te_{n_j}-h}\rightarrow 0$. But $e_n\perp e_m$ for $n \neq m$ and $Te_{n_j}=\lambda_{n_j}e_{n_j}$. Hence $\norm{Te_{n_j}-Te_{n_i}}^2 = \lambda_{n_j}^2+\lambda_{n_i}^2 \geq 2\alpha^2$. Since $\{Te_{n_j}\}$ is a Cauchy sequence, $\alpha = 0$.

    Now put $P_n = $ the projection of $\mathscr{H}$ onto $\mathscr{E}_n$ and examine $T- \sum_{j=1}^n\lambda_jP_j$. If $h \in \mathscr{E}_k$, $1\leq k \leq n$, then $$\left(T-\sum_{j=1}^n\lambda_jP_j\right)h = T_h-\lambda_kh = 0$$
    Hence $\mathscr{E}_1\oplus\cdots\oplus\mathscr{E}_n\subseteq \ker\left(T-\sum_{j=1}^n\lambda_jP_j\right)$. If $h \in (\mathscr{E}_1\oplus\cdots\oplus \mathscr{E}_n)^{\perp}$, then $P_jh = 0$ for $1\leq j \leq n$; so $\left(T-\sum_{j=1}^n\lambda_jP_j\right)h = Th$. These two statements together with the fact that $(\mathscr{E}_1\oplus\cdots\oplus\mathscr{E}_n)^{\perp}$ reduces $T$, imply that $$\norm{T-\sum_{j=1}^n\lambda_jP_j} = \norm{T\vert_{(\mathscr{E}_1\oplus\cdots\oplus\mathscr{E}_n)^{\perp}}} = |\lambda_{n+1}|\rightarrow 0$$
    Therefore the series $\sum_{n=1}^{\infty}\lambda_nP_n$ converges in the metric of $\mathscr{B}(\mathscr{H})$ to $T$.
\end{proof}

This is called the \textbf{Spectral Theorem} for compact self-adjoint operators.



\section{The Spectral Theorem and Functional Calculus for Compact Normal Operators}
\label{sec:SpecThmFuncCalc}

\begin{prop}
    Let $\{P_i:i \in I\}$ be a family of pairwise orthogonal projections in $\mathscr{B}(\mathscr{H})$. (That is, $P_iP_j = P_jP_i = 0$ for $i \neq j$.) If $ h \in \mathscr{H}$, then $\sum_i\{P_ih:i \in I\}$ converges in $\mathscr{H}$ to $Ph$, where $P$ is the projection of $\mathscr{H}$ onto $\bigvee\{P_i\mathscr{H}:i \in I\}$.
\end{prop}
\begin{proof}
    (TBD)
\end{proof}

If $\{P_i:i \in I\}$ is as in the preceding proposition and $\mathscr{M}_i = P_i\mathscr{H}$, then $P$ is the projection of $\mathscr{H}$ onto $\bigoplus_i\mathscr{M}_i$. Write $P = \sum_iP_i$. Note that $Ph = \sum_iP_ih$ where the convergence is in the norm of $\mathscr{H}$, but $\sum_iP_i$ does not converge to $P$ in the norm of $\mathscr{B}(\mathscr{H})$ unless $I$ is finite.

\begin{defn}\label{Partition of Identity}
    A \textbf{partition of the identity} on $\mathscr{H}$ is a family $\{P_i: i \in I\}$ of pairwise orthogonal projections on $\mathscr{H}$ such that $\bigvee_iP_i\mathscr{H} = \mathscr{H}$. This might be indicated by $1 = \sum_iP_i$.
\end{defn}

\begin{defn}
    An operator $A$ on $\mathscr{H}$ is \textbf{diagonalizable} if there is a partition of the identity on $\mathscr{H}$, $\{P_i:i \in I\}$, and a family of scalars $\{\alpha_i:i \in I\}$ such that $\sup_i|\alpha|_i < \infty$ and $Ah = \alpha_ih$ whenever $h \in \ran P_i$.
\end{defn}

In this case $\norm{A} = \sup_i|\alpha_i|$.

To denote a diagonalizable operator satisfying the conditions of this definition, write $$A = \sum_i\alpha_iP_i$$
Note it was note assumed that the scalars $\alpha_i$ are distinct. However, if $\alpha_i = \alpha_j$ then we can replace $P_i$ and $P_j$ with $P_i+P_j$.

\begin{prop}
    An operator $A$ on $\mathscr{H}$ is diagonalizable if and only if there is an orthonormal basis for $\mathscr{H}$ consisting of eigenvectors for $A$.
\end{prop}
\begin{proof}
    (TBD)
\end{proof}

Also note that if $A = \sum_i\alpha_iP_i$, then $A^* = \sum_i\overline{\alpha}_iP_i$ and $A$ is normal.

\begin{thm}\label{thm:diag}
    If $A = \sum_i\alpha_iP_i$ is diagonalizable and all the $\alpha_i$ are distinct, then an operator $B \in \mathscr{B}(\mathscr{H})$ satisfies $AB = BA$ if and only if for each $i$, $\ran P_i$ reduces $B$.
\end{thm}
\begin{proof}
    If all the $\alpha_i$ are distinct, then $\ran P_i = \ker(A-\alpha_i)$. If $AB = BA$ and $Ah = \alpha_ih$, then $ABh = BAh = B(\alpha_ih) = \alpha_iBh$; hence $Bh \in \ran P_i$ whenever $h \in \ran P_i$. Thus $\ran P_i$ is left invaraint by $B$. Therefore $B$ leaves $\bigvee\{ \ran P_j:j \neq i\} = \mathcal{N}_i$ invariant. But since $\sum_iP_i = 1,\mathcal{N}_i = (\ran P_i)^{\perp}$. Thus $\ran P_i$ reduces $B$.

    Now assume that $B$ is reduced by each $\ran P_i$. Thus $BP_i = P_iB$ for all $i$. If $h \in \mathscr{H}$, then $Ah = \sum_i\alpha_iP_ih$. Hence $BAh = \sum_i\alpha_iBP_ih = \sum_i\alpha_iP_iBh = ABh$.
\end{proof}

Using the notation of the preceding theorem, if $AB = BA$, let $B_i = B\vert_{\ran P_i}$. THen it is appropriate to write $B = \sum_iB_i$ on $\mathscr{H} = \sum_iP_i\mathscr{H}$. One might paraphrase the theorem by saying that $B$ commutes with a diagonalizable operator if and only if $B$ can be ``diagonalized with operator entries."

\begin{nthm}{Spectral Theorem for Compact Normal Operators}\label{thm:specComp}
    If $T$ is a compact normal operator on the complex Hilbert space $\mathscr{H}$, then $T$ has only a countable number of distinct eigenvalues. If $\{\lambda_1,\lambda_2,...\}$ are the distinct nonzero eigenvalues of $T$, and $P_n$ is the projection of $\mathscr{H}$ onto $\ker(T-\lambda_n)$, then $P_nP_m = P_mP_n = 0$ if $n \neq m$ and \begin{equation}
        T = \sum_{n=1}^{\infty}\lambda_nP_n
    \end{equation}
    where this series converges to $T$ in the metric defined by the norm on $\mathscr{B}(\mathscr{H})$.
\end{nthm}
\begin{proof}
    Let $A = (T+T^*)/2, B = (T-T^*)/2i$. So $A,B$ are compact self-adjoint operators, $T = A+iB$, and $AB=BA$ since $T$ is normal. By Theorem \ref{thm:DiagCompSelfAdj}, $A = \sum_n\alpha_nE_n$, where $\alpha_n \in \R$, $\alpha_n \neq \alpha_m$ if $n \neq m$, and $E_n$ is the projection of $\mathscr{H}$ onto $\ker(A-\alpha_n)$. Since $AB = BA$ we can diagonalize $A$ and $B$ simultaneously. 

    For each $n$, $E_n\mathscr{H} = \mathcal{L}_n$ reduces $B$ since $BA = AB$. Let $B_n = B\vert_{\mathcal{L}_n}$. Then $B_n = B_n^*$ and $\dim \mathcal{L}_n < \infty$. Applying the spectral theorem for finite dimensional linear operators to $B_n$, there is a basis $\{e^{(n)}:1\leq j\leq d_n\}$ for $\mathcal{L}_n$ and real numbers $\{\beta_j^{(n)}:1\leq j \leq d_n\}$ such that $B_ne^{(n)}_j = \beta_j^{(n)}e_j^{(n)}$. Thus $Te_j^{(n)} = Ae_j^{(n)}+iBe^{(n)}_j = (\alpha_n+i\beta_j^{(n)})e_j^{(n)}$.

    Therefore, $\{e_j^{(n)}:1\leq j \leq d_n,n\geq 1\}$ is a basis for $\text{cl}(\ran A)$ consisting of eigenvectors for $T$. It may be that $\text{cl}(\ran A)\neq \text{cl}(\ran T)$. Since $B$ is reduced by $\ker A = (\ran A)^{\perp}$ and $B_0 = B\vert_{\ker A}$ is a compact self-adjoint operator there is an orthonormal basis $\{e_j^{(0)}:j\geq 1\}$ for $\text{cl}(\ran B_0)$ and scalars $\{\beta_j^{(0)}:j \geq 1\}$ such that $Be_j^{(0)} = \beta_j^{(0)}e_j^{(0)}$. It follows that $Te_j^{(0)} = i\beta_j^{(0)}e_j^{(0)}$. Moreover, $\ker T^* \supseteq \ker A\cap \ker B_0$, so $\text{cl}(\ran T) \subseteq \text{cl}(\ran A)\oplus\text{cl}(\ran B_0)$.

    Thus, $T$ has a countable number of distinct eigenvalues \textbf{TO BE CONTINUED}
\end{proof}

\begin{cor}
    With notation as in the theorem \begin{enumerate}
        \item[(a)] $\ker T = \left[\bigvee\{P_n\mathscr{H}:n\geq 1\}\right]^{\perp}$;
        \item[(b)] each $P_n$ has finite rank;
        \item[(c)] $\norm{T} = \sup\{|\lambda_n|:n\geq 1\}$ and either $\{\lambda_n\}$ is finite or $\lambda_n\rightarrow 0$ as $n\rightarrow \infty$.
    \end{enumerate}
\end{cor}

\begin{cor}
    If $T$ is a compact operator on a complex Hilbert space, then $T$ is normal if and only if $T$ is diagonalizable.
\end{cor}

We now develop a functional calculus for normal operators $T$. That is an operator $\phi(T)$ will be defined for every bounded Borel function $\phi$ on $\C$ and certain properties of the map $\phi\mapsto \phi(T)$ will be deduced. A projection-valued measure will then be obtained by letting $\mu(\Delta) = \chi_{\Delta}(T)$, where $\chi_{\Delta}$ is the characteristic function of the set $\Delta$. We restrict to the case of compact normal operators for now.

\begin{defn}
    Denote by $l^{\infty}(\C)$ all the bounded functions $\phi:\C\rightarrow \C$. If $T$ is a compact normal operator which has description as in the theorem, define $\phi(T):\mathscr{H}\rightarrow \mathscr{H}$ by $$\phi(T) = \sum_{n=1}^{\infty}\phi(\lambda_n)P_n+\phi(0)P_0$$
    where $P_0 =$ the projection of $\mathscr{H}$ onto $\ker T$.
\end{defn}

Note that $\phi(T)$ is a diagonalizable operator and $\norm{\phi(T)} = \sup_i\{|\phi(0)|,|\phi(\lambda_i)|\}$.

\begin{nthm}{Functional Calculus for Compact Normal Operators}
    If $T$ is a compact normal operator on a $\C$-Hilbert space $\mathscr{H}$, then the map $\phi\mapsto \phi(T)$ of $l^{\infty}(\C)\rightarrow \mathscr{B}(\mathscr{H})$ has the following properties: \begin{enumerate}
        \item[(a)] $\phi\mapsto \phi(T)$ is a multiplicative linear map of $l^{\infty}(\C)$ into $\mathscr{B}(\mathscr{H})$. If $\phi\equiv 1$, $\phi(T) = 1$; if $\phi(z) = z$on $\sigma_p(T)\cup \{0\}$, then $\phi(T) = T$.
        \item[(b)] $\norm{\phi(T)} = \sup\{|\phi(\lambda)|:\lambda \in \sigma_p(T)\}$
        \item[(c)] $\phi(T)^* = \phi^*(T)$ where $\phi^*$ is the function defined by $\phi^*(z) = \overline{\phi(z)}$.
        \item[(d)] If $A \in \mathscr{B}(\mathscr{H})$ and $AT= TA$, then $A\phi(T) = \phi(T)A$ for all $\phi \in l^{\infty}(\C)$.
    \end{enumerate}
\end{nthm}
\begin{proof}
    For (a), if $\phi,\psi \in l^{\infty}(\C)$, then $(\phi\psi)(z) = \phi(z)\psi(z)$ for $z \in \C$. Also, $$\phi(T)\psi(T)h = \left[\phi(0)P_0+\sum_n\phi(\lambda_n)P_n\right]\left[\psi(0)P_0+\sum_m\psi(\lambda_m)P_m\right]h = \left[\phi(0)P_0+\sum_n\phi(\lambda_n)P_n\right]\left[\psi(0)P_0h+\sum_m\psi(\lambda_m)P_mh\right]$$
    Since $P_nP_m = 0$ whenever $n \neq m$, this gives that $$\phi(T)\psi(T)h = \phi(0)\psi(0)P_0h + \sum_n\phi(\lambda_n)\psi(\lambda_n)P_nh = (\phi\psi)(T)h$$
    THus $\phi\mapsto \phi(T)$ is multiplicative. The linearity of the map is immediate. If $\phi(z) = 1$, then $$\phi(T) = 1(T) = P_0 + \sum_nP_n = 1$$
    since $\{P_0,P_1,...\}$ is a partition of the identity. If $\phi(z) = z$, $\phi(\lambda_n) = \lambda_n$ and so $\phi(T) = T$.

    Part (c) is clear and part (b) follows from previous work.

    For part (d) if $AT = TA$ then by Theorem \ref{thm:diag} $P_0\mathscr{H},P_1\mathscr{H},...$ all reduce $A$. Fix $h_n \in P_n\mathscr{H}$, $n \geq 0$. If $\phi \in l^{\infty}(\C)$, then $Ah_n \in P_n\mathscr{H}$ and so $$\phi(T)Ah_N = \phi(\lambda_n)Ah_n = A(\phi(\lambda_n)h_n) = A\phi(T)h_n$$
    If $h \in \mathscr{H}$, then $\sum_{n\geq 0}h_n$, where $h_n \in P_n$. Hence $$\phi(T)Ah = \sum_{n\geq 0}\phi(T)Ah_n = \sum_{n\geq 0}A\phi(T)h_n = A\phi(T)h$$
\end{proof}

\begin{qst}
    Which operators on $\mathscr{H}$ can be expressed as $\phi(T)$ for some $\phi$ in $l^{\infty}(\C)$?
\end{qst}

\begin{thm}
    If $T$ is a compact normal operator on a $\C$-Hilbert space, then $\{\phi(T):\phi \in l^{\infty}(\C)\}$ is equal to $$\{B \in \mathscr{B}(\mathscr{H}):BA = AB\text{ whenever }AT =TA\}$$
\end{thm}
\begin{proof}
    Half of the desired equality is obtained from part (d) of the previous theorem. So let $B \in \mathscr{B}(\mathscr{H})$ and assume that $AB = BA$ whenever $AT = TA$. Thus $B$ must commute with $T$ itself. By Theorem \ref{thm:diag} $B$ is reduced by each $P_n\mathscr{H} =: \mathscr{H}_n, n \geq 0$. But $B_n = B\vert_{\mathscr{H}_n}$. Fix $n \geq 0$ and let $A_n$ be any bounded operator on $\mathscr{B}(\mathscr{H}_n)$. Define $Ah = A_nh$ if $h \in \mathscr{H}_n$ and $A_nh = 0$ if $h \in \mathscr{H}_m, m\neq n$, and extend $A$ to $\mathscr{H}$ by linearity; so $A = \sum_{m\geq 0}A_m$ where $A_m = 0$ if $m \neq n$. By Theorem \ref{thm:diag} $AT =TA$; hence $AB  = BA$. THis implies that $B_nA_n = A_nB_n$. Since $A_n$ was arbitrarily chosen from $\mathscr{B}(\mathscr{H}_n)$, $B_n = \beta_n$ for some $\beta_n$. If $\phi:\C\rightarrow \C$ is defined by $\phi(0) = \beta_0$ and $\phi(\lambda_n) = \beta_n$ for $n \geq 1$, then $B = \phi(T)$.
\end{proof}

\begin{defn}
    If $A \in \mathscr{B}(\mathscr{H})$, then $A$ is \textbf{positive} if $\ip{Ah,h} \geq 0$ for all $h \in \mathscr{H}$.
\end{defn}

Recall that every positive operator on a complex Hilbert space is self-adjoint.

\begin{prop}
    If $T$ is a compact normal operator, then $T$ is positive if and only if all its eigenvalues are non-negative real numbers.
\end{prop}
\begin{proof}
    Let $T = \sum_n\lambda_nP_n$. If $T \geq 0$ and $h \in P_n\mathscr{H}$ with $\norm{h} = 1$, then $Th = \lambda_nh$. Hence $\lambda_n = \ip{Th,h} \geq 0$. Conversely, assume each $\lambda_n \geq 0$. If $h \in \mathscr{H}$, $h = h_0 + \sum_nh_n$, where $h_0 \in \ker T$ and $h_n \in P_n\mathscr{H}$ for $n \geq 1$. Then $Th = \sum_n\lambda_nh_n$. Hence $$\ip{Th,h} = \ip{\sum_n\lambda_nh_n,h_0+\sum_mh_m} = \sum_n\sum_m\lambda_n\ip{h_n,h_m} = \sum_n\lambda_n\norm{h_n}^2\geq 0$$
    since $\ip{h_n,h_m} = 0$ when $n \neq m$.
\end{proof}

\begin{thm}
    If $T$ is a compact self-adjoint operator, then there are unique positive compact operators $A,B$ such that $T = A-B$ and $AB = BA = 0$.
\end{thm}
\begin{proof}
    Let $T = \sum_n\lambda_nP_n$ as in the Spectral theorem. Define $\phi,\psi:\C\rightarrow \C$ by $\phi(\lambda_n) = \lambda_n$ if $\lambda_n > 0$, $\phi(z) = 0$ otherwise; $\psi(\lambda_n) = -\lambda_n$ if $\lambda_n < 0$, $\psi(z) = 0$ otherwise. Put $A = \phi(T)$ and $B = \psi(T)$. Then $A = \sum\{\lambda_nP_n:\lambda_n > 0\}$ and $B = \sum\{-\lambda_nP_n:\lambda < 0\}$. Thus $T = A-B$. Since $\phi\psi = 0$ $AB = BA = 0$. Since $\phi,\psi \geq 0, A,B \geq 0$ by the preceding proposition. It remains to show uniqueness.

    Suppose $T = C-D$ where $C,D$ are compact positive operators and $CD =DC = 0$. Then $C$ and $D$ commute with $T$. Put $\lambda_0 = 0$ and $P_0 = $ the projection of $\mathscr{H}$ onto $\ker T$. Thus $C$ and $D$ are reduced by $P_n\mathscr{H} =: \mathscr{H}_n$ for all $n \geq 0$. Let $C_n = C\vert_{\mathscr{H}_n}$ and $D_n = D\vert_{\mathscr{H}_n}$. So $C_nD_n = D_nC_n = 0$, $\lambda_nP_n = T\vert_{\mathscr{H}_n} = C_n-D_n$, and $C_n,D_n$ are positive. Suppose $\lambda_n > 0$ and let $h \in \mathscr{H}_n$. Since $C_nD_n = 0$, $$\ker C_n \supseteq \text{cl}[\ran D_n] = (\ker D_n)^{\perp}$$
    So if $h \in (\ker D_n)^{\perp}$, then $\lambda_nh = -D_nh$. Hence $\lambda_n\norm{h}^2 = -\ip{D_nh,h} \leq 0$. Thus $h = 0$ since $\lambda_n > 0$. That is $|ker D_n = \mathscr{H}_n$. Thus $D_n = 0 = B\vert_{\mathscr{H}_n}$ and $C_n = \lambda_nP_n = A\vert_{\mathscr{H}_n}$. Similarly, if $\lambda_n < 0$, $C_n = 0 = A\vert_{\mathscr{H}_n}$ and $D_n = -\lambda_nP_n = B\vert_{\mathscr{H}_n}$. On $\mathscr{H}_0$, $T\vert_{\mathscr{H}_0} = 0 = C_0-D_0$. Thus $C_0 = D_0$. But $0 = C_0D_0 = C_0^2$. Thus $0 = \ip{C_0^2h,h} = \norm{C_0h}^2$, so $C_0 = A\vert_{\mathscr{H}_0}$ and $D_0 = 0 = B\vert_{\mathscr{H}_0}$. Therefore $C = A$ and $D = B$.
\end{proof}

\begin{thm}
    If $T$ is a positive compact operator, then there is a unique positive compact operator $A$ such that $A^2 = T$.
\end{thm}
\begin{proof}
    Let $T = \sum_n\lambda_nP_n$ as in the Spectral theorem. Since $T \geq 0$, $\lambda_n > 0$ for all $n$. Let $\phi(\lambda_n) = \sqrt{\lambda_n}$ and $\phi(z) = 0$ otherwise; put $A = \phi(T)$. It is easy to check that $A \geq 0$; $A = \sum_n \sqrt{\lambda_n}P_n$ so that $A$ is compact; and $A^t = T$.

    Uniqueness TBC
\end{proof}


\section{Unitary Equivalence for Compact Normal Operators}
\label{sec:unitEquiv}

The isomorphism between Hilbert spaces induces an equivalence relation between the operators on the spaces.

\begin{defn}
    If $A,B$ are bounded operators on Hilbert spaces $\mathscr{H},\mathscr{K}$, then $A$ and $B$ are \textbf{unitarily equivalent} if there is an isomorphism $U:\mathscr{H}\rightarrow \mathscr{K}$ such that $UAU^{-1} = B$ (note $U^{-1} = U^*$).
\end{defn}

We now give necessary and sufficient conditions that two compact normal operators be unitarily equivalent.

\begin{defn}
    If $T$ is a compact operator, the \textbf{multiplicity function} for $T$ is the cardinal number valued function $m_T$ defined for every complex number $\lambda$ by $m_T(\lambda) = \dim\ker(T-\lambda)$.
\end{defn}

Note $m_T(\lambda) > 0$ if and only if $\lambda$ is an eigenvalue for $T$. Note $m_T(\lambda) < \omega$ if $\lambda \neq 0$.

If $T,S$ are compact operators on Hilbert spaces and $U:\mathscr{H}\rightarrow \mathscr{K}$ is an isomorphism with $UTU^{-1} = S$, then $U\ker(T-\lambda) = \ker(S-\lambda)$ for every $\lambda$ in $\C$. In particular, it must be that $m_T = m_S$.

\begin{thm}
    Two compact normal operators are unitarily equivalent if and only if they have the same multiplicity function.
\end{thm}
\begin{proof}
    Let $T,S$ be compact normal operators on $\mathscr{H},\mathscr{K}$. If $T \equiv S$ then it has already been shown that $m_T = m_S$. Conversely, suppose $m_T = m_S$.

    Let $T = \sum_n\lambda_nP_n$ and let $S = \sum_n\mu_nQ_n$ as in the Spectral Theorem. So if $n \neq m$ then $\lambda_n \neq \lambda_m$ and $\mu_n\neq \mu_m$, and each of the projections $P_n$ and $Q_n$ has finite rank. Let $P_0,Q_0$ be the projections onto the kernels, so $$P_0 = \left(\sum_nP_n\right)^{\perp},\;\text{ and }\;Q_0 = \left(\sum_nQ_n\right)^{\perp}$$
    Put $\lambda_0 = \mu_0 = 0$.

    Since $m_T = m_S$, $0 < m_T(\lambda_n) = m_S(\lambda_n)$. Hence there is a unique $\mu_j$ such that $\mu_j = \lambda_n$. Define $\pi:\N\rightarrow \N$ by letting $\mu_{\pi(n)} = \lambda_n$. Let $\pi(0) = 0$. Note that $\pi$ is one-to-one. Also, since $0 < m_S(\mu_n) = m_T(\mu_n)$, for every $n$ there is a $j$ such that $\pi(j) = n$. Thus $\pi$ is a bijection. Since $\dim P_n = m_T(\lambda_n) = m_S(\mu_{\pi(n)}) = \dim Q_{\pi(n)}$, there is an isomorphism $U_n:P_n\mathscr{H}\rightarrow Q_{\pi(n)}\mathscr{K}$ for $n \neq 0$. Define $U:\mathscr{H}\rightarrow \mathscr{K}$ by letting $U = U_n$ on $P_n\mathscr{H}$ and extending by linearity. Hence $U = \sum_{n\geq 0}U_n$. (TBC) it can be checked that $U$ is an isomorphism. Also if $h \in P_n\mathscr{H}$, $n \geq 0$, then $$UTh = \lambda_nUh = \mu_{\pi(n)}Uh = SUh$$
    Hence $UTU^{-1} = S$.
\end{proof}

If $V$ is the Volterra operator, then $m_V \equiv 0$ and $V$ and the zero operator are definitely not unitarily equivalent, so the preceding theorem only applied to compact normal operators.

%
% \begin{acknowledgement}
% If you want to include acknowledgments of assistance and the like at the end of an individual chapter please use the \verb|acknowledgement| environment -- it will automatically render Springer's preferred layout.
% \end{acknowledgement}
%
% \section*{Appendix}
% \addcontentsline{toc}{section}{Appendix}
%


% Problems or Exercises should be sorted chapterwise
\section*{Problems}
\addcontentsline{toc}{section}{Problems}
%
% Use the following environment.
% Don't forget to label each problem;
% the label is needed for the solutions' environment
\begin{prob}
\label{prob1}
A given problem or Excercise is described here. The
problem is described here. The problem is described here.
\end{prob}


% \begin{prob}
% \label{prob2}
% \textbf{Problem Heading}\\
% (a) The first part of the problem is described here.\\
% (b) The second part of the problem is described here.
% \end{prob}

%%%%%%%%%%%%%%%%%%%%%%%% referenc.tex %%%%%%%%%%%%%%%%%%%%%%%%%%%%%%
% sample references
% %
% Use this file as a template for your own input.
%
%%%%%%%%%%%%%%%%%%%%%%%% Springer-Verlag %%%%%%%%%%%%%%%%%%%%%%%%%%
%
% BibTeX users please use
% \bibliographystyle{}
% \bibliography{}
%


% \begin{thebibliography}{99.}%
% and use \bibitem to create references.
%
% Use the following syntax and markup for your references if 
% the subject of your book is from the field 
% "Mathematics, Physics, Statistics, Computer Science"
%
% Contribution 
% \bibitem{science-contrib} Broy, M.: Software engineering --- from auxiliary to key technologies. In: Broy, M., Dener, E. (eds.) Software Pioneers, pp. 10-13. Springer, Heidelberg (2002)
% %
% Online Document

% \end{thebibliography}

