%%%%%%%%%%%%%%%%%%%%% chapter.tex %%%%%%%%%%%%%%%%%%%%%%%%%%%%%%%%%
%
% sample chapter
%
% Use this file as a template for your own input.
%
%%%%%%%%%%%%%%%%%%%%%%%% Springer-Verlag %%%%%%%%%%%%%%%%%%%%%%%%%%
%\motto{Use the template \emph{chapter.tex} to style the various elements of your chapter content.}
\chapter{Description of Curved Spacetime}
\label{CurvSpac} % Always give a unique label
% use \chaptermark{}
% to alter or adjust the chapter heading in the running head



\abstract{To be completed once done}

\section{Coordinates}
\label{sec:Coord}


A line element specifies a geometry, but many different line elements describe the same spacetime geometry because different coordinate systems can be used. A good coordinate system provides unique labels for each point in spacetime. However, most coordinate systems only do this locally. Even polar coordinates fails to uniquely label points on the $\theta = 0$ axis. The singularities in most coordinate systems mean that different overlapping coordinate patches must be used to cover spacetime so that every point is labeled by a nonsingular set of coordinates.


\section{Metric}
\label{sec:metRic}

To describe a general geometry we use a system of four coordinates, $x^{\alpha}$, to label the points and specify the line element giving the distance, $ds^2$, between nearby points separated by coordinate intervals $dx^{\alpha}$. That line element will have the form \begin{equation*}
    \boxed{ds^2 = g_{\alpha\beta}(x)dx^{\alpha}dx^{\beta}}
\end{equation*}
where $g_{\alpha\beta}(x)$ is a symmetric, position-dependent matrix called the \textbf{metric}.


\section{The Summation Convention}
\label{sec:summConv}

\begin{enumerate}
    \item The location of the indices must be respected: superscripts for coordinates and vector components and subscripts for the metric.
    \item Repeated indices always occur in superscript-subscript pairs and imply summation.
    \item Indices that are not summed are called free indices.
\end{enumerate}



\section{Light Cones and World Lines}
\label{sec:LightWorld}

Points separated from $P$ by infinitesimal coordinate intervals $dx^{\alpha}$ can be timelike separated, spacelike separated, or null separated as the square of their distance away defined by the metric satisfies \begin{align*}
    ds^2 &< 0 \tag{timelike separation} \\
    ds^2 &= 0 \tag{null separation} \\
    ds^2 &> 0 \tag{spacelike separation}
\end{align*}
Light rays move along null curves in spacetime along which $ds^2= 0$.

Particles move on timelike world lines which can be specified parametrically by four functions $x^{\alpha}(\tau)$ of the distance $\tau$ along them.
\begin{defn}
    The distance between a point $A$ and a point $B$ along a timelike worl line is given by \begin{equation*}
        \tau_{AB} = \int_A^B[-g_{\alpha\beta}(x)dx^{\alpha}dx^{\beta}]^{1/2}
    \end{equation*}
    where the integral is along the world line.
\end{defn}

The global arrangement of light cones is called the spacetime's \textbf{causal structure}.


\section{Length, Area, Volume, and Four-Volume for Diagonal Metrics}
\label{sec:measurements}

In this section suppose $ds^2 = g_{\alpha\alpha}dx^{\alpha}dx^{\alpha}$ is a diagonal metric. The proper lengths of a segment will be of the form $d\ell^{1} = \sqrt{g_{11}}dx^{1}$. Since the coordinates are orthogonal, the area element created by the region spanned by two line segments is $$dA = d\ell^2d\ell^3 = \sqrt{g_{11}g_{22}}dx^1dx^2$$
For three-volume $$dV = \sqrt{g_{11}g_{22}g_{33}}dx^1dx^2dx^3$$
For a metric of signature $(1,3)$, the four-volume is $$dv = \sqrt{-\det(g_{\alpha\beta})}d^4x$$







% \begin{acknowledgement}
% If you want to include acknowledgments of assistance and the like at the end of an individual chapter please use the \verb|acknowledgement| environment -- it will automatically render Springer's preferred layout.
% \end{acknowledgement}
%
\section*{Appendix}
\addcontentsline{toc}{section}{Appendix}




% Problems or Exercises should be sorted chapterwise
\section*{Problems}
\addcontentsline{toc}{section}{Problems}
%
% Use the following environment.
% Don't forget to label each problem;
% the label is needed for the solutions' environment
\begin{prob}
\label{prob1}
A given problem or Excercise is described here. The
problem is described here. The problem is described here.
\end{prob}

% \begin{prob}
% \label{prob2}
% \textbf{Problem Heading}\\
% (a) The first part of the problem is described here.\\
% (b) The second part of the problem is described here.
% \end{prob}

%%%%%%%%%%%%%%%%%%%%%%%% referenc.tex %%%%%%%%%%%%%%%%%%%%%%%%%%%%%%
% sample references
% %
% Use this file as a template for your own input.
%
%%%%%%%%%%%%%%%%%%%%%%%% Springer-Verlag %%%%%%%%%%%%%%%%%%%%%%%%%%
%
% BibTeX users please use
% \bibliographystyle{}
% \bibliography{}
%


% \begin{thebibliography}{99.}%
% and use \bibitem to create references.
%
% Use the following syntax and markup for your references if 
% the subject of your book is from the field 
% "Mathematics, Physics, Statistics, Computer Science"
%
% Contribution 
% \bibitem{science-contrib} Broy, M.: Software engineering --- from auxiliary to key technologies. In: Broy, M., Dener, E. (eds.) Software Pioneers, pp. 10-13. Springer, Heidelberg (2002)
% %
% Online Document

% \end{thebibliography}

