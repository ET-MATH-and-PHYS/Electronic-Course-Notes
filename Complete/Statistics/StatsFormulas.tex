\documentclass[12pt]{report}
\usepackage[utf8]{inputenc}
%%%%%%%%%%%%%%%%% Book Formatting Comments:

%%%%%%%%%%%%%%%%%%%%%%%%%%%%%%%%%%%%% for Part

%%%%%%%%%%%%%%%%%%%%%% for chapter

%%%%%%%%%%%%%%%%%%%% for section








%%%%%% PACKAGES %%%%%%%
\usepackage{hyperref}
\hypersetup{
    colorlinks,
    citecolor=black,
    filecolor=black,
    linkcolor=black,
    urlcolor=black
}
\usepackage{amsmath} % Math display options
\usepackage{amssymb} % Math symbols
\usepackage{amsfonts} % Math fonts
\usepackage{amsthm}
\usepackage{mathtools} % General math tools
\usepackage{array} % Allows you to write arrays
\usepackage{empheq} % For boxing equations
\usepackage{mathabx}
\usepackage{mathrsfs}
\usepackage{nameref}

\usepackage{soul}
\usepackage[normalem]{ulem}

\usepackage{txfonts}
\usepackage{cancel}
\usepackage[toc, page]{appendix}
\usepackage{titletoc,tocloft}
\setlength{\cftchapindent}{1em}
\setlength{\cftsecindent}{2em}
\setlength{\cftsubsecindent}{3em}
\setlength{\cftsubsubsecindent}{4em}
\usepackage{titlesec}

\titleformat{\section}
  {\normalfont\fontsize{25}{15}\bfseries}{\thesection}{1em}{}
\titleformat{\section}
  {\normalfont\fontsize{20}{15}\bfseries}{\thesubsection}{1em}{}
\setcounter{secnumdepth}{1}  
  
  

\newcommand\numberthis{\refstepcounter{equation}\tag{\theequation}} % For equation labelling
\usepackage[framemethod=tikz]{mdframed}

\usepackage{tikz} % For drawing commutative diagrams
\usetikzlibrary{cd}
\usetikzlibrary{calc}
\tikzset{every picture/.style={line width=0.75pt}} %set default line width to 0.75p

\usepackage{datetime}
\usepackage[margin=1in]{geometry}
\setlength{\parskip}{1em}
\usepackage{graphicx}
\usepackage{float}
\usepackage{fancyhdr}
\setlength{\headheight}{15pt} 
\pagestyle{fancy}
\lhead[\leftmark]{}
\rhead[]{\leftmark}

\usepackage{enumitem}

\usepackage{url}
\allowdisplaybreaks

%%%%%% ENVIRONMENTS %%%
\definecolor{purp}{rgb}{0.29, 0, 0.51}
\definecolor{bloo}{rgb}{0, 0.13, 0.80}



%%\newtheoremstyle{note}% hnamei
%{3pt}% hSpace above
%{3pt}% hSpace belowi
%{}% hBody fonti
%{}% hIndent amounti
%{\itshape}% hTheorem head fonti
%{:}% hPunctuation after theorem headi
%{.5em}% hSpace after theorem headi
%{}% hTheorem head spec (can be left empty, meaning ‘normal’)i


%%%%%%%%%%%%% THEOREM STYLES

\newtheoremstyle{BigTheorem}
{20pt}
{20pt}
{\slshape}
{}
{\Large\color{purp}\bfseries}
{.}
{\newline}
{\thmname{#1}\thmnumber{ #2}\thmnote{ (#3)}}



\newtheoremstyle{TheoremClassic}
{15pt}
{15pt}
{\slshape}
{}
{\bfseries}
{.}
{.5em}
{}

\newtheoremstyle{Definitions}
{15pt}
{15pt}
{\slshape}
{}
{\bfseries}
{.}
{.5em}
{\thmname{#1}\thmnumber{ #2}\thmnote{ (#3)}}


\newtheoremstyle{Remarks}
{10pt}
{10pt}
{\upshape}
{}
{\bfseries}
{.}
{.5em}
{}

\newtheoremstyle{Examples}
{10pt}
{10pt}
{\upshape}
{}
{\bfseries}
{.}
{.5em}
{}


%%%%%%%%%%%%% THEOREM DEFINITIONS

\theoremstyle{BigTheorem}
\newtheorem{namthm}{Theorem}
\newtheorem{conj}[namthm]{Conjecture}

\theoremstyle{TheoremClassic}
\newtheorem{thm}{Theorem}[section]
\newtheorem*{thm*}{Theorem}
\newtheorem{lem}[thm]{Lemma}
\newtheorem{cor}[thm]{Corollary}
\newtheorem{prop}[thm]{Proposition}
\newtheorem{claim}[thm]{Claim}


\theoremstyle{Definitions}
\newtheorem{defn}{Definition}[section]
\newtheorem{axi}[defn]{Axiom}
\newtheorem{cust}[defn]{}
\newtheorem{cons}[defn]{Construction}
\newtheorem{props}[defn]{Properties}
\newtheorem{proc}[defn]{Process}
\newtheorem*{law}{Law}


\theoremstyle{Examples}
\newtheorem{eg}{Example}[section]
\newtheorem{noneg}[eg]{Non-Example}
\newtheorem{xca}[eg]{Exercise}


\theoremstyle{Remarks}
\newtheorem{rmk}{Remark}[section]
\newtheorem{qst}[rmk]{Question}
\newtheorem*{ans}{Answer}
\newtheorem{obs}[rmk]{Observation}
\newtheorem{rec}[rmk]{Recall}
\newtheorem{summ}[rmk]{Summary}
\newtheorem{nota}[rmk]{Notation}
\newtheorem{note}[rmk]{Note}



\renewcommand{\qedsymbol}{$\blacksquare$}


\numberwithin{equation}{section}

\newenvironment{qest}{
    \begin{center}
        \em
    }
    {
    \end{center}
    }

%%%%%% MACROS %%%%%%%%%
%% New Commands
\newcommand{\ip}[1]{\langle#1\rangle} %%% Inner product
\newcommand{\abs}[1]{\lvert#1\rvert} %%% Modulus
\newcommand\diag{\operatorname{diag}} %%% diag matrix
\newcommand\tr{\mbox{tr}\.} %%% trace
\newcommand\C{\mathbb C} %%% Complex numbers
\newcommand\R{\mathbb R} %%% Real numbers
\newcommand\Z{\mathbb Z} %%% Integers
\newcommand\Q{\mathbb Q} %%% Rationals
\newcommand\N{\mathbb N} %%% Naturals
\newcommand\F{\mathbb F} %%% An arbitrary field
\newcommand\ste{\operatorname{St}} %%% Steinberg Representation
\newcommand\GL{\mathbf{GL}} %%% General Linear group
\newcommand\SL{\mathbf{SL}} %%% Special linear group
\newcommand\gl{\mathfrak{gl}} %%% General linear algebra
\newcommand\G{\mathbf{G}} %%% connected reductive group
\newcommand\g{\mathfrak{g}} %%% Lie algebra of G
\newcommand\Hbf{\mathbf{H}} %%% Theta fixed points of G
\newcommand\X{\mathbf{X}} %%% Symmetric space X
\newcommand{\catname}[1]{\normalfont\textbf{#1}}
\newcommand{\Set}{\catname{Set}} %%% Category set
\newcommand{\Grp}{\catname{Grp}} %%% Category group
\newcommand{\Rmod}{\catname{R-Mod}} %%% Category r-modules
\newcommand{\Mon}{\catname{Mon}} %%% Category monoid
\newcommand{\Ring}{\catname{Ring}} %%% Category ring
\newcommand{\Topp}{\catname{Top}} %%% Category Topological spaces
\newcommand{\Vect}{\catname{Vect}_{k}} %%% category vector spaces'
\newcommand\Hom{\mathbf{Hom}} %%% Arrows

\newcommand{\map}[2]{\begin{array}{c} #1 \\ #2 \end{array}}

\newcommand{\Emph}[1]{\textbf{\ul{\emph{#1}}}}

\newcommand{\mapsfrom}{\mathrel{\reflectbox{\ensuremath{\mapsto}}}}


%% Math operators
\DeclareMathOperator{\ran}{Im} %%% image
\DeclareMathOperator{\aut}{Aut} %%% Automorphisms
\DeclareMathOperator{\spn}{span} %%% span
\DeclareMathOperator{\ann}{Ann} %%% annihilator
\DeclareMathOperator{\rank}{rank} %%% Rank
\DeclareMathOperator{\ch}{char} %%% characteristic
\DeclareMathOperator{\ev}{\bf{ev}} %%% evaluation
\DeclareMathOperator{\sgn}{sign} %%% sign
\DeclareMathOperator{\id}{Id} %%% identity
\DeclareMathOperator{\supp}{Supp} %%% support
\DeclareMathOperator{\inn}{Inn} %%% Inner aut
\DeclareMathOperator{\en}{End} %%% Endomorphisms
\DeclareMathOperator{\sym}{Sym} %%% Group of symmetries


%% Diagram Environments
\iffalse
\begin{center}
    \begin{tikzpicture}[baseline= (a).base]
        \node[scale=1] (a) at (0,0){
          \begin{tikzcd}
           
          \end{tikzcd}
        };
    \end{tikzpicture}
\end{center}
\fi




\newdateformat{monthdayyeardate}{%
    \monthname[\THEMONTH]~\THEDAY, \THEYEAR}
%%%%%%%%%%%%%%%%%%%%%%%
\tikzset{every picture/.style={line width=0.75pt}} %set default line width to 0.75pt       


\begin{document}

%%%%%%%%%%%%%%%%%%%%%%%%%%%%%%%%%%

%%%%%% TITLE PAGE %%%%%

\begin{titlepage}
    \centering
    \scshape
    \vspace*{\baselineskip}
    \rule{\textwidth}{1.6pt}\vspace*{-\baselineskip}\vspace*{2pt}
    \rule{\textwidth}{0.4pt}
    
    \vspace{0.75\baselineskip}
    
    {\LARGE Statistics: Formula Sheet}
    
    \vspace{0.75\baselineskip}
    
    \rule{\textwidth}{0.4pt}\vspace*{-\baselineskip}\vspace{3.2pt}
    \rule{\textwidth}{1.6pt}
    
    \vspace{2\baselineskip}
    Stats \\
    \vspace*{3\baselineskip}
    \monthdayyeardate\today \\
    \vspace*{5.0\baselineskip}
    
    {\scshape\Large Elijah Thompson, \\ Physics and Math Honors\\}
    
    \vspace{1.0\baselineskip}
    \textit{Solo Pursuit of Learning}
\end{titlepage}

%%%%%%%%%%%%%%%%%%%%%%%
\tableofcontents

%%%%%%%%%%%%%%%%%%%%%%%%%%%%%%%%%%


\chapter{Probability Basics}

\section{Terminology}

\begin{defn}{}{}
    A \Emph{random experiment} is a process that leads to a single outcome, which cannot be predicted with certainty.
\end{defn}


\begin{defn}{}{}
    A \Emph{sample space} (denoted $\mathbf{S}$) is the collection of all possible outcomes for a given experiment.
\end{defn}

\begin{defn}{}{}
    An \Emph{event} is a subset of the sample space $\mathbf{S}$ for a given experiment.
\end{defn}


\begin{defn}{}{}
    The \Emph{probability} of an event $\mathbf{X}$, denoted $P(\mathbf{X})$, is the measurement of the likelihood that $\mathbf{X}$ will occur when an experiment is performed.
\end{defn}

\begin{cust*}[separator sign = {}]{Axioms of Probability}{}
    Suppose event $\mathbf{A}$ is a subset of a sample space $\mathbf{S}$. Then:\begin{itemize}[leftmargin = 1in]
        \item[\textbf{Axiom 1:}] $0\leq P(\mathbf{A})\leq 1$
        \item[\textbf{Axiom 2:}] $P(\mathbf{S}) = 1$
        \item[\textbf{Axiom 3:}] If $\mathbf{A}_1,\mathbf{A}_2,\mathbf{A}_3,...$ is a collection of events that do not share the same elements in $\mathbf{S}$ (i.e. they are mutually disjoint as sets), then \begin{equation*}
                P\left(\bigcup\limits_{i=1}^{\infty}A_i\right) = \sum\limits_{i=1}^{\infty}P(\mathbf{A}_i)
        \end{equation*}
    \end{itemize}
\end{cust*}


\section{Set-Theoretic Identities}


\begin{prop}{}{}
    Let $\mathbf{A}$ be a subset of some universal set $\mathbf{S}$ of interest. Then the \Emph{complement rule} states that: \begin{equation*}
        P(\mathbf{A}) + P(\mathbf{A}^C) = 1
    \end{equation*}
\end{prop}


\begin{prop}{}{}
    Let $\mathbf{A}$ and $\mathbf{B}$ be subsets of some universal set $\mathbf{S}$ of interest. Then the \Emph{additive rule} states that: \begin{equation*}
        P(\mathbf{A}\cup\mathbf{B}) = P(\mathbf{A}) + P(\mathbf{A}^C) - P(\mathbf{A}\cap\mathbf{B})
    \end{equation*}
\end{prop}

\begin{prop}{}{}
    Let $\mathbf{A}$ and $\mathbf{B}$ be subsets of some universal set $\mathbf{S}$ of interest. Then the \Emph{law of total probability} states that: \begin{equation*}
        P(\mathbf{A}) = P(\mathbf{A}\cap\mathbf{B}^C) + P(\mathbf{A}\cap\mathbf{B})
    \end{equation*}
\end{prop}



\begin{namthm}{DeMorgan's Laws}{}
    Let $\mathbf{A}$ and $\mathbf{B}$ be subsets of some universal set $\mathbf{S}$ of interest. Then the \Emph{DeMorgan's Laws} state that: \begin{equation*}
        P(\mathbf{A}^C\cap\mathbf{B}^C) = P((\mathbf{A}\cup\mathbf{B})^C)
    \end{equation*}
    and \begin{equation*}
        P(\mathbf{A}^C\cup\mathbf{B}^C) = P((\mathbf{A}\cap\mathbf{B})^C)
    \end{equation*}
\end{namthm}

\begin{prop}{}{}
    Let $\mathbf{A}$, $\mathbf{B}$, and $\mathbf{C}$ be subsets of some universal set $\mathbf{S}$ of interest. Then the \Emph{distributive laws} state that: \begin{equation*}
        P(\mathbf{A}\cap(\mathbf{B}\cup\mathbf{C}) = P((\mathbf{A}\cap\mathbf{B})\cup(\mathbf{A}\cap\mathbf{C}))    
    \end{equation*}
    and \begin{equation*}
        P(\mathbf{A}\cup(\mathbf{B}\cap\mathbf{C}) = P((\mathbf{A}\cup\mathbf{B})\cap(\mathbf{A}\cup\mathbf{C})) 
    \end{equation*}
    The \Emph{associative laws} state that: \begin{equation*}
        P(\mathbf{A}\cap(\mathbf{B}\cap\mathbf{C}) = P((\mathbf{A}\cap\mathbf{B})\cap\mathbf{C})    
    \end{equation*}
    and \begin{equation*}
        P(\mathbf{A}\cup(\mathbf{B}\cup\mathbf{C}) = P((\mathbf{A}\cup\mathbf{B})\cup\mathbf{C}) 
    \end{equation*}
\end{prop}


\section{Conditional Probability}


\begin{defn}{}{}
    A \Emph{conditional probability} is a probability that reflects additional knowledge that may affect the outcome of an experiment.

    Precisely, if $\mathbf{A}$ and $\mathbf{B}$ are events in a sample space $\mathbf{S}$, then the probability of $\mathbf{A}$ given $\mathbf{B}$, written $P(\mathbf{A}\vert\mathbf{B})$, is calculated by:\begin{equation*}
        P(\mathbf{A}\vert\mathbf{B}) = \frac{P(\mathbf{A}\cap\mathbf{B})}{P(\mathbf{B})}
    \end{equation*}
    where $P(\mathbf{B}) > 0$.
\end{defn}

\begin{defn}{}{}
    Let $\mathbf{A}$ and $\mathbf{B}$ be two events in a sample space $\mathbf{S}$. THen we say $\mathbf{A}$ and $\mathbf{B}$ are \Emph{independent} if and only if the three following equivalent conditions hold: \begin{enumerate}
        \item $P(\mathbf{A}\vert\mathbf{B}) = P(\mathbf{A})$
        \item $P(\mathbf{B}\vert\mathbf{A}) = P(\mathbf{B})$
        \item $P(\mathbf{A}\cap\mathbf{B}) = P(\mathbf{A})P(\mathbf{B})$
    \end{enumerate}
\end{defn}

\begin{prop}{}{}
    If $\mathbf{A}$ and $\mathbf{B}$ are independent events, then so are $\mathbf{A}$ and $\mathbf{B}^C$, $\mathbf{A}^C$ and $\mathbf{B}$, and $\mathbf{A}^C$ and $\mathbf{B}^C$.
\end{prop}


\begin{defn}{}{}
    A \Emph{contingency table} for two events $\mathbf{A}$ and $\mathbf{B}$ in a sample space $\mathbf{S}$ is given by:
    \begin{table}[H]
		\centering
		\begin{tabular}{c||c|c||c}
			& $\mathbf{A}$ & $\mathbf{A}^C$ & \\\hline\hline
            $\mathbf{B}$ & $P(\mathbf{A}\cap\mathbf{B})$ & $P(\mathbf{A}^C\cap\mathbf{B})$ & $P(\mathbf{B})$ \\\hline
            $\mathbf{B}^C$ & $P(\mathbf{A}\cap\mathbf{B}^C)$ & $P(\mathbf{A}^C\cap\mathbf{B}^C)$ & $P(\mathbf{B}^C)$ \\\hline\hline
            & $P(\mathbf{A})$ & $P(\mathbf{A}^C)$ & $P(\mathbf{S})$
		\end{tabular}
	\end{table}
\end{defn}

\begin{namthm}{Law of Total Probability}{}
    Let $\mathbf{A}_1,\mathbf{A}_2,...,\mathbf{A}_n$ be events which partition the sample space $\mathbf{S}$. That is, $$\bigcup\limits_{i=1}^n\mathbf{A}_i = \mathbf{S}$$
    and $\mathbf{A}_i\cap\mathbf{A}_j =\emptyset$ for all $i \neq j$, and $P(\mathbf{A}_i) > 0$ for all $i$. Then \begin{equation*}
        P(\mathbf{B}) = \sum\limits_{i=1}^nP(\mathbf{B}\cap\mathbf{A}_i) = \sum\limits_{i=1}^nP(\mathbf{B}\vert\mathbf{A}_i)P(\mathbf{A}_i)
    \end{equation*}
\end{namthm}


\begin{namthm}{Bayes' Theorem}{}
    Given events $\mathbf{A}_1,\mathbf{A}_2,...,\mathbf{A}_n$ which partition the sample space $\mathbf{S}$, for each event $\mathbf{B}$ with $P(\mathbf{B}) > 0$ we have for each $j \in \{1,2,...,n\}$: \begin{equation*}
        P(\mathbf{A}_j,\mathbf{B}) = \frac{P(\mathbf{A}_j\cap\mathbf{B})}{P(\mathbf{B})} = \frac{P(\mathbf{B}\vert\mathbf{A}_j)P(\mathbf{A}_j)}{\sum\limits_{i=1}^nP(\mathbf{B}\vert\mathbf{A}_i)P(\mathbf{A}_i)}
    \end{equation*}
\end{namthm}


\begin{defn}{}{}
    A \Emph{tree diagram} for two events $\mathbf{A}$ and $\mathbf{B}$, where $\mathbf{A}$ is a condition for $\mathbf{B}$ is given by: 
    \begin{center}
\begin{tikzpicture}[x=0.75pt,y=0.75pt,yscale=-1,xscale=1]
%uncomment if require: \path (0,310); %set diagram left start at 0, and has height of 310

%Straight Lines [id:da24684006577735862] 
\draw    (110,180) -- (160,220) ;
%Straight Lines [id:da6495124487889197] 
\draw    (110,180) -- (160,140) ;
%Straight Lines [id:da8016873219117782] 
\draw    (200,140) -- (240,170) ;
%Straight Lines [id:da016234073114670178] 
\draw    (200,140) -- (240,110) ;
%Straight Lines [id:da6130662536830012] 
\draw    (200,220) -- (240,250) ;
%Straight Lines [id:da9838550223138658] 
\draw    (200,220) -- (240,190) ;

% Text Node
\draw (165,132.4) node [anchor=north west][inner sep=0.75pt]  [font=\footnotesize]  {$P( A)$};
% Text Node
\draw (161,212.4) node [anchor=north west][inner sep=0.75pt]  [font=\footnotesize]  {$P\left( A^{C}\right)$};
% Text Node
\draw (241,102.4) node [anchor=north west][inner sep=0.75pt]  [font=\footnotesize]  {$P( B|A)$};
% Text Node
\draw (241,161.4) node [anchor=north west][inner sep=0.75pt]  [font=\footnotesize]  {$P\left( B^{C} |A\right)$};
% Text Node
\draw (241,182.4) node [anchor=north west][inner sep=0.75pt]  [font=\footnotesize]  {$P\left( B|A^{C}\right)$};
% Text Node
\draw (245,241.4) node [anchor=north west][inner sep=0.75pt]  [font=\footnotesize]  {$P\left( B^{C} |A^{C}\right)$};


\end{tikzpicture}
    \end{center}
\end{defn}


\section{Counting}

\begin{prop}{}{}
    Let $\{\mathbf{A}_i\}_{i=1}^n$ be a collection of a sample space $\mathbf{S}$. Then the set of events is \Emph{mutually independent} if and only if for every $k \leq n$, and every $k$-sized subset of events $\{B_i\}_{i=1}^k\subseteq \{A_i\}_{i=1}^k$ we have \begin{equation*}
        P\left(\bigcap_{i=1}^k\mathbf{B}_i\right) = \prod\limits_{i=1}^kP(\mathbf{B}_i)
    \end{equation*}
\end{prop}

\begin{prop}{}{}
    The \Emph{multiplication principle} states that $\mathbf{A}_1,...,\mathbf{A}_n$ are events, then \begin{equation*}
        |\mathbf{A}_1\times...\times\mathbf{A}_n| = |\mathbf{A}_1|\cdot...\cdot|\mathbf{A}_n|
    \end{equation*}
\end{prop}


\begin{defn}{}{}
    A \Emph{permutation} is an arrangement of objects in which \emph{order matters}. If there are a total of $n$ objects, the number of ways we can order $r$ of them is: \begin{equation*}
        {}_{n}P_{r} = \frac{n!}{(n-r)!}
    \end{equation*}
\end{defn}



\begin{defn}{}{}
    A \Emph{combination} is an arrangement of objects in which \emph{order doesn't matters}. If there are a total of $n$ objects, the number of ways we can select $r$ of them is: \begin{equation*}
        {}_{n}C_{r} = \binom{n}{r} =  \frac{n!}{r!(n-r)!}
    \end{equation*}
\end{defn}


\begin{thm}{}{}
    The \Emph{binomial theorem} states that \begin{equation*}
        (x+y)^n = \sum\limits_{i=0}^n\binom{n}{i}x^{n-i}y^i
    \end{equation*}
\end{thm}

\begin{defn}{}{}
    The number of ways of partitioning $N$ objects into $k$ distinct groups containing $n_1,n_2,...,n_k$ objects, respectively, is given by the \Emph{multinomial coefficient} \begin{equation*}
        \binom{N}{n_1n_2...n_k} := \binom{N}{n_1}\binom{N-n_1}{n_2}...\binom{N - \sum_{i=1}^{k-1}n_i}{n_k} = \frac{N!}{n_1!n_2!...n_k!}
    \end{equation*}
\end{defn}






\chapter{Discrete Random Variables}


\section{Basic Terminology}

\begin{defn}{}{}
    A \Emph{random variable} for a sample space $\mathbf{S}$ is a function \begin{equation*}
        \mathbf{X}:\mathbf{S}\rightarrow \R
    \end{equation*}
\end{defn}

\begin{defn}{}{}
    A \Emph{discrete random variable} is a function $\mathbf{X}:\mathbf{S}\rightarrow \R$ with a finite, or countably infinite, image. That is, the possible values $\mathbf{X}$ can take can be arranged in a (possibly infinite) sequence.
\end{defn}


\begin{defn}{}{}
    A \Emph{continuous random variable} are functions $\mathbf{X}:\mathbf{S}\rightarrow \R$ in which the image of $\mathbf{X}$ is uncountable, and $\mathbf{X}$ satisfies certain other conditions to be discussed later.
\end{defn}

\begin{defn}{}{}
    The \Emph{probability distribution} of a discrete random variable is a listing or graph that specifies every possible value that a random variable can assume along with the probabilities associated with each of these values.
\end{defn}

\begin{defn}{}{}
    The \Emph{probability distribution table} for a finite random variable $\mathbf{X}$ is given by: 
    \begin{table}[H]
		\centering
		\begin{tabular}{|c|c|c|c|c|}
            \hline
		    $\mathbf{X} = x$ & $x_1$ & $x_2$ & $\hdots$ & $x_n$ \\\hline
            $P(\mathbf{X} = x)$ & $P(\mathbf{X} = x_1)$ & $P(\mathbf{X} = x_2)$ & $\hdots$ & $P(\mathbf{X} = x_n)$ \\\hline
        \end{tabular}
	\end{table}
    where $P(\mathbf{X} = x) = P(x) = p_{\mathbf{X}}(x)$ is the \Emph{probability mass function (pmf)} for $\mathbf{X}$, which assigns the likelihood that any event or element in the sample space can occur.
\end{defn}

\begin{defn}{}{}
    The probability mass function for a discrete random variable $\mathbf{X}$ must satisfy the following conditions: \begin{enumerate}
        \item $0<P(\mathbf{X}=x_i)<1$ for $i \in \{1,2,...,n\}$
        \item $\sum\limits_{i=1}^nP(\mathbf{X}=x_i) = 1$
    \end{enumerate}
\end{defn}


\begin{defn}{}{}
    A \Emph{probability distribution graph} plots probability on the $y$-axis and the values of the random variable on the $x$-axis.

    A \Emph{symmetric} graph is one that is invariant under reflection about some vertical line, located at it's ``middle."

    A \Emph{right-skewed} graph is one that has a ``tail" pointing off in the right direction (the mass of the graph is on the left-hand side).

    A \Emph{left-skewed} graph is one that has a ``tail" pointing off in the left direction (the mass of the graph is on the right-hand side).
\end{defn}

\section{Expected Values and Variance}

\begin{defn}{}{}
    The \Emph{expected value} of a discrete random variable $\mathbf{X}$ is the long-run average value of $\mathbf{X}$ over an infinite number of repetitions of an experiment: \begin{equation*}
        E[\mathbf{X}] = \mu_{\mathbf{X}} = \mu = \sum\limits_{i=1}^nx_iP(\mathbf{X}=x_i)
    \end{equation*}
    where $x_i$ is the $i$th value that $\mathbf{X}$ can assume.
\end{defn}

\begin{prop}{}{}
    Let $\mathbf{X}$ be a discrete random variable, and let $g(\mathbf{X})$ be a real-valued function of $\mathbf{X}$. Then the expected value of $g(\mathbf{X})$ is given by:
    \begin{equation*}
        E[g(\mathbf{X})] = \sum\limits_{all\;x}g(x)P(\mathbf{X} = x)
    \end{equation*}
\end{prop}

\begin{prop}{}{}
    Let $\mathbf{X}$ be a discrete random variable, $g(\mathbf{X}), g_1(\mathbf{X}),...,g_k(\mathbf{X})$ real-valued functions of $\mathbf{X}$, and $c \in \R$. Then: \begin{enumerate}
        \item $E[c] = c$
        \item $E[cg(\mathbf{X})] = cE[g(\mathbf{X})]$
        \item $E\left[\sum\limits_{i=1}^kg_i(\mathbf{X})\right] = \sum\limits_{i=1}^kE[g_i(\mathbf{X})]$
    \end{enumerate}
\end{prop}

\begin{defn}{}{}
    The \Emph{variance} of a discrete random variable $\mathbf{X}$ is a measure of the variability of $\mathbf{X}$ over an infinite number of repetitions of an experiment. It determines the average squared deviation from the expected value of $\mathbf{X}$: \begin{equation*}
        VAR[\mathbf{X}] = V[\mathbf{X}] = \sigma_{\mathbf{X}}^2 = \sigma^2 = E[(\mathbf{X}-\mu)^2] = E[\mathbf{X}^2] - E[\mathbf{X}]^2
    \end{equation*}
\end{defn}

\begin{defn}{}{}
    The \Emph{standard deviation} of a discrete random variable $\mathbf{X}$ is another measure of variability of $\mathbf{X}$ and is the positive square root of the variance: \begin{equation*}
        SD[\mathbf{X}] = \sigma_{\mathbf{X}} = \sigma = \sqrt{VAR[\mathbf{X}]}
    \end{equation*}
\end{defn}

\begin{prop}{}{}
    Let $\mathbf{X}$ be a discrete random variable and $c \in \R$. Then: \begin{enumerate}
        \item $VAR[c] = 0$
        \item $VAR[c\mathbf{X}] = c^2VAR[\mathbf{X}]$
    \end{enumerate}
\end{prop}


\section{Bernoulli Random Variables}

\begin{defn}{}{}
    A scenario or experiment with the following characteristics produces a \Emph{Bernoulli random variable}: \begin{itemize}
        \item The experiment consists of a single trial
        \item The trial results in one of two outcomes (a ``success" or a ``failure")
        \item The probability of a ``success" - denoted $p$ - is the same for all similar trials 
        \item The \Emph{Bernoulli random variable} itself takes on a value of $1$ if a ``success" is achieved and a value of $0$ if a ``failure" is achieved
    \end{itemize}
    Such an experiment is called a \Emph{Bernoulli trial}.
\end{defn}

\begin{defn}{}{}
    The values that define a random variable's probability distribution are called \Emph{parameters}.
\end{defn}

\begin{defn}{}{}
    A Bernoulli random variable depends on one parameter: \begin{equation*}
        p - \text{the probability of a success}
    \end{equation*}
    If a random variable $\mathbf{X}$ follows a Bernoulli distribution we write $$\mathbf{X}\sim Bernoulli(p)$$
\end{defn}

\begin{defn}{}{}
    The pmf for a Bernoulli random variable $\mathbf{X}$ is \begin{equation*}
        P(\mathbf{X} = x) = p^x(1-p)^{1-x} = \left\{\begin{array}{lc} 1-p, & x=0 \\ p, & x=1\end{array}\right.
    \end{equation*}
\end{defn}

\begin{defn}{}{}
    The MGF for a Bernoulli random variable $\mathbf{X}$ is: \begin{equation*}
        M_{\mathbf{X}}(t) = pe^t+(1-p)
    \end{equation*}
\end{defn}



\begin{defn}{}{}
    Let $\mathbf{X}\sim Bernoulli(p)$. Then \begin{equation*}
        E[\mathbf{X}] = p
    \end{equation*}
    and
    \begin{equation*}
        VAR[\mathbf{X}] = p(1-p) = pq
    \end{equation*}
\end{defn}


\section{Binomial Random Variables}

\begin{defn}{}{}
    A scenario or experiment with the following characteristics produces a \Emph{binomial random variable}: \begin{itemize}
        \item The experiment consists of a fixed number of $n$ independent, identical Bernoulli trials
        \item Each trial results in one of two outcomes (a ``success" or a ``failure")
        \item The probability of a ``success" - denoted $p$ - is the same from trial to trial
        \item The \Emph{binomial random variable} itself is the number of successes out of the $n$ trials
    \end{itemize}
\end{defn}


\begin{defn}{}{}
    A binomial random variable depends on two parameters: \begin{align*}
        p &- \text{the probability of a success} \\
        n &- \text{the number of trials}
    \end{align*}
    If a random variable $\mathbf{X}$ follows a binomial distribution we write $$\mathbf{X}\sim binomial(n,p)$$
\end{defn}

\begin{defn}{}{}
    The pmf for a binomial random variable $\mathbf{X}$ is \begin{equation*}
        P(\mathbf{X} = x) = \binom{n}{x}p^x(1-p)^{n-x} 
    \end{equation*}
\end{defn}

\begin{defn}{}{}
    Let $\mathbf{X}\sim binomial(n,p)$. Then \begin{equation*}
        E[\mathbf{X}] = np
    \end{equation*}
    and
    \begin{equation*}
        VAR[\mathbf{X}] = np(1-p) = npq
    \end{equation*}
\end{defn}


\begin{defn}{}{}
    The MGF for a binomial random variable $\mathbf{X}$ is: \begin{equation*}
        M_{\mathbf{X}}(t) = [pe^t+(1-p)]^n
    \end{equation*}
\end{defn}


\begin{defn}{}{}
    Let $\mathbf{X}\sim binomial(n,p)$. To find $P(\mathbf{X} = a)$ in $R$, write: \begin{equation*}
        \text{dbinom(a, $size = n$, $prob = p$)}
    \end{equation*}
    To find $P(\mathbf{X} \leq a)$ in $R$, write: \begin{equation*}
        \text{pbinom(a, $size = n$, $prob = p$)}\;\text{ or }\;\text{sum(dbinom(start:a, $size = n$, $prob = p$))}
    \end{equation*}
\end{defn}


\section{Negative Binomial Random Variables}

\begin{defn}{}{}
    A scenario or experiment with the following characteristics produces a \Emph{negative binomial random variable}: \begin{itemize}
        \item The experiment consists of independent, identical Bernoulli trials
        \item Each trial results in one of two outcomes (a ``success" or a ``failure")
        \item The probability of a ``success" - denoted $p$ - is the same from trial to trial
        \item The \Emph{negative binomial random variable} itself is the number of trials needed to yield $r$ successes
    \end{itemize}
\end{defn}


\begin{defn}{}{}
    A negative binomial random variable depends on two parameters: \begin{align*}
        p &- \text{the probability of a success} \\
        r &- \text{the number of successes we are interested in observing}
    \end{align*}
    If a random variable $\mathbf{X}$ follows a negative binomial distribution we write $$\mathbf{X}\sim negative\;binomial(r,p)$$
\end{defn}

\begin{defn}{}{}
    The pmf for a binomial random variable $\mathbf{X}$ is \begin{equation*}
        P(\mathbf{X} = x) = \binom{x-1}{r-1}p^r(1-p)^{x-r} 
    \end{equation*}
\end{defn}

\begin{defn}{}{}
    Let $\mathbf{X}\sim negative\;binomial(r,p)$. Then \begin{equation*}
        E[\mathbf{X}] = \frac{r}{p}
    \end{equation*}
    and
    \begin{equation*}
        VAR[\mathbf{X}] = \frac{r(1-p)}{p^2} = \frac{rq}{p^2}
    \end{equation*}
\end{defn}


\begin{defn}{}{}
    The MGF for a negative binomial random variable $\mathbf{X}$ is: \begin{equation*}
        M_{\mathbf{X}}(t) = \left[\frac{pe^t}{1-(1-p)e^t}\right]^r
    \end{equation*}
\end{defn}

\begin{defn}{}{}
    Let $\mathbf{X}\sim negative\;binomial(r,p)$. To find $P(\mathbf{X} = a)$ in $R$, write: \begin{equation*}
        \text{dnbinom(a-r, $size = r$, $prob = p$)}
    \end{equation*}
    To find $P(\mathbf{X} \leq a)$ in $R$, write: \begin{equation*}
        \text{pnbinom(a-r, $size = r$, $prob = p$)}\;\text{ or }\;\text{sum(dnbinom(start:a-r, $size = r$, $prob = p$))}
    \end{equation*}
\end{defn}


\section{Geometric Random Variables}

\begin{defn}{}{}
    A scenario or experiment with the following characteristics produces a \Emph{geometric random variable}: \begin{itemize}
        \item The experiment consists of independent, identical Bernoulli trials
        \item Each trial results in one of two outcomes (a ``success" or a ``failure")
        \item The probability of a ``success" - denoted $p$ - is the same from trial to trial
        \item The \Emph{geometric random variable} itself is the number of the trial on which the first success occurs
    \end{itemize}
\end{defn}


\begin{defn}{}{}
    A geometric random variable depends on one parameters: \begin{align*}
        p &- \text{the probability of a success}
    \end{align*}
    If a random variable $\mathbf{X}$ follows a geometric distribution we write $$\mathbf{X}\sim geometric(p)$$
\end{defn}

\begin{defn}{}{}
    The pmf for a geometric random variable $\mathbf{X}$ is \begin{equation*}
        P(\mathbf{X} = x) = (1-p)^{x-1}p 
    \end{equation*}
\end{defn}

\begin{defn}{}{}
    Let $\mathbf{X}\sim geometric(p)$. Then \begin{equation*}
        E[\mathbf{X}] = \frac{1}{p}
    \end{equation*}
    and
    \begin{equation*}
        VAR[\mathbf{X}] = \frac{(1-p)}{p^2} = \frac{q}{p^2}
    \end{equation*}
\end{defn}


\begin{defn}{}{}
    The MGF for a geometric random variable $\mathbf{X}$ is: \begin{equation*}
        M_{\mathbf{X}}(t) = \frac{pe^t}{1-(1-p)e^t}
    \end{equation*}
\end{defn}


\begin{defn}{}{}
    Let $\mathbf{X}\sim geometric(p)$. To find $P(\mathbf{X} = a)$ in $R$, write: \begin{equation*}
        \text{dgeom(a-1, $prob = p$)}
    \end{equation*}
    To find $P(\mathbf{X} \leq a)$ in $R$, write: \begin{equation*}
        \text{pgeom(a-1, $prob = p$)}\;\text{ or }\;\text{sum(dgeom(start:a-1, $prob = p$))}
    \end{equation*}
\end{defn}


\section{Hypergeometric Random Variables}

\begin{defn}{}{}
    A scenario or experiment with the following characteristics produces a \Emph{hypergeometric random variable}: \begin{itemize}
        \item The experiment involves randomly selecting $n$ elements (without replacement) from a total number of elements $N$
        \item Each trial results in one of two outcomes (a ``success" or a ``failure")
        \item The total number of ``successes" is known to be a certain number $r$
        \item The \Emph{hypergeometric random variable} itself is the number of successes in the set of $n$ selected elements drawn from $N$
    \end{itemize}
\end{defn}


\begin{defn}{}{}
    A hypergeometric random variable depends on three parameters: \begin{align*}
        r &- \text{the the total number of ``successes" in $N$} \\
        N &- \text{the total number of elements we are drawing from (the ``population")} \\
        n &- \text{the number of elements drawn from $N$ (the ``sample")}
    \end{align*}
    If a random variable $\mathbf{X}$ follows a hypergeometric distribution we write $$\mathbf{X}\sim hyper\;geometric(r,N,n)$$
\end{defn}

\begin{defn}{}{}
    The pmf for a binomial random variable $\mathbf{X}$ is \begin{equation*}
        P(\mathbf{X} = x) = \frac{\binom{r}{x}\binom{N-r}{n-x}}{\binom{N}{n}}
    \end{equation*}
\end{defn}

\begin{defn}{}{}
    Let $\mathbf{X}\sim negative\;binomial(r,p)$. Then \begin{equation*}
        E[\mathbf{X}] = \frac{nr}{N}
    \end{equation*}
    and
    \begin{equation*}
        VAR[\mathbf{X}] = n\cdot\frac{r}{N}\cdot\frac{N-r}{N}\cdot\frac{N-n}{N-1}
    \end{equation*}
\end{defn}



\begin{defn}{}{}
    Let $\mathbf{X}\sim hyper\;geometric(r,N,k)$. To find $P(\mathbf{X} = a)$ in $R$, write: \begin{equation*}
        \text{dhyper(a, $m = r$, $n = N-r$, $k = k$)}
    \end{equation*}
    To find $P(\mathbf{X} \leq a)$ in $R$, write: \begin{equation*}
        \text{phyper(a, $m = r$, $n = N-r$, $k=k$)}\;\text{ or }\;\text{sum(dhyper(start:a, $m = r$, $n = N-r$, $k = k$)}
    \end{equation*}
\end{defn}

\section{Poisson Random Variables}

\begin{defn}{}{}
    A scenario or experiment with the following characteristics produces a \Emph{Poisson random variable}: \begin{itemize}
        \item The experiment involves an event occurring during a given interval (of time, area, distance, volume, etc.)
        \item The probability that an event occurs in an interval is the same for all other equal intervals
        \item The number of events that occur in one interval is independent of the number of events that occur in other intervals
        \item There is a known average or expected number of events, $\lambda$, that occur during/in the interval
        \item The \Emph{Poisson random variable} itself is the number of times an event has occurred in a given interval
    \end{itemize}
\end{defn}


\begin{defn}{}{}
    A Poisson random variable depends on one parameter: \begin{align*}
        \lambda &- \text{the average number of events during a specified interval}
    \end{align*}
    If a random variable $\mathbf{X}$ follows a Poisson distribution we write $$\mathbf{X}\sim Poisson(\lambda)$$
\end{defn}

\begin{defn}{}{}
    The pmf for a Poisson random variable $\mathbf{X}$ is \begin{equation*}
        P(\mathbf{X} = x) = \frac{\lambda^xe^{-\lambda}}{x!}
    \end{equation*}
\end{defn}

\begin{defn}{}{}
    Let $\mathbf{X}\sim Poisson(\lambda)$. Then \begin{equation*}
        E[\mathbf{X}] = \lambda
    \end{equation*}
    and
    \begin{equation*}
        VAR[\mathbf{X}] = \lambda
    \end{equation*}
\end{defn}


\begin{defn}{}{}
    The MGF for a Poisson random variable $\mathbf{X}$ is: \begin{equation*}
        M_{\mathbf{X}}(t) = \operatorname{exp}[\lambda(e^t-1)]
    \end{equation*}
\end{defn}


\begin{defn}{}{}
    Let $\mathbf{X}\sim Poisson(\lambda)$. To find $P(\mathbf{X} = a)$ in $R$, write: \begin{equation*}
        \text{dpois(a, $lambda = lambda$)}
    \end{equation*}
    To find $P(\mathbf{X} \leq a)$ in $R$, write: \begin{equation*}
        \text{ppois(a, $lambda = lambda$)}\;\text{ or }\;\text{sum(dpois(start:a, $lambda = lambda$))}
    \end{equation*}
\end{defn}



\section{Moment-Generating Functions}

\begin{defn}{}{}
    The $k$th moment of a random variable $X$ about the origin (the \Emph{standardized moment}) is defined as $E[X^k]$ where \begin{equation*}
        E[X^k] = \sum\limits_{\text{all } x} x^kP(X=x)
    \end{equation*}
    The $k$th moment of a random variable $X$ about its mean (the \Emph{central moment}) is defined as $E[(X-\mu)^k]$ where \begin{equation*}
        E[(X-\mu)^k] = \sum\limits_{\text{all } x} (x-\mu)^kP(X=x)
    \end{equation*}
\end{defn}

\begin{defn}{}{}
    The first standardized moment is the \Emph{mean}. It is a measure of central tendency and gives an idea of where our distribution is centered.
\end{defn}


\begin{defn}{}{}
    The second central moment is the \Emph{variance}. It is a measure of spread, and gives the squared deviation of the random variable from its mean.
\end{defn}

\begin{defn}{}{}
    The third central moment is the \Emph{skewness}. It gives an idea of the symmetry of the probability distribution about the mean. A perfectly symmetric distribution will have a skewness of $0$, a left-skewed distribution will have a negative skewness, and a right-skewed distribution will have a positive skewness.
\end{defn}

\begin{defn}{}{}
    The fourth central moment is \Emph{kurtosis}. It gives an idea of how ``thick" the tails of a distribution are in comparison to the normal distribution of the same variance.
\end{defn}


\begin{defn}{}{}
    A \Emph{moment generating function} (MGF) is a function that allows us to find any moment of a distribution. In particular, for a random variable $X$ its MGF exists if there is a constant $b > 0$ such that for $|t| \leq b$ the expectation $E[e^{tX}]$ exists, and it is defined as: \begin{equation*}
        M_X(t) = E[e^{tX}] = \left\{\begin{array}{lc} \sum\limits_{all\;x}e^{tx}P(X=x) & when\;discrete \\ \int e^{tx}P(X=x)dx & when\;continuous\end{array}\right.
    \end{equation*}
    We can find the $k$th moment of $X$ as follows: \begin{equation*}
        E[X^k] = \frac{d^k}{dt^k}[M_X(t)]\Big\rvert_{t=0} = M_X^{(k)}(t)\Big\rvert_{t=0}
    \end{equation*}
\end{defn}

\begin{rmk}{}{}
    THe moment-generating function uniquely determines the distribution. In particular if $X$ and $Y$ are random variables with cdf $F_X$ and $F_Y$, then if $M_X(t) = M_Y(t)$ for all values of $t$, then $F_X(x) = F_Y(x)$ for all $x$.
\end{rmk}
     



\chapter{Continuous Random Variables}


\section{Basic Definitions: Continuous Random Variables}

\begin{props}{}{}
    Let $X$ be a continuous random variable. Then: \begin{itemize}
        \item For all $x \in \R$, $P(X= x) = 0$.
        \item For all $x \in \R$, $P(X\leq x) = P(X < x)$ and $P(X\geq x) = P(X > x)$.
    \end{itemize}
\end{props}


\begin{defn}{}{}
    The \Emph{cumulative distribution function} (cdf) $F(x)$ of a random variable $X$ gives the probability that $X$ will take a value less than or equal to that value $x$: \begin{equation*}
        F(X = x) = F(x) = P(X\leq x)
    \end{equation*}
    If $X$ is continuous, then \begin{equation*}
        F(x) = P(X\leq x) = P(X < x) = \int_{-\infty}^xf(t)dt
    \end{equation*}
    where $f(x) = \frac{d}{dx}F(x) = F'(x)$ is the \Emph{probability density function} (pdf) of $X$.
\end{defn}

\begin{rmk}{}{}
    For a continuous random variable $X$ we have that \begin{equation*}
        P(a\leq X\leq b) = P(X\leq b) - P(X<a) = F(b)-F(a) = \int_a^bf(x)dx
    \end{equation*}
\end{rmk}


\begin{props}{}{}
    Let $F(x)$ be a cdf. Then \begin{itemize}
        \item $F(-\infty):= \lim\limits_{x\rightarrow -\infty}F(x) = 0$
        \item $F(\infty) := \lim\limits_{x\rightarrow \infty}F(x) = 1$
        \item $F(x)$ is a nondecreasing, nonnegative function of $x$, meaning that $x_1 < x_2 \implies F(x_1) < F(x_2)$
    \end{itemize}
    If $X$ is a continuous random variable, then: \begin{itemize}
        \item $f(x) \geq 0$ for all $x \in \R$.
        \item $\int_{-\infty}^{\infty}f(x)dx = 1$
    \end{itemize}
\end{props}


\begin{defn}{}{}
    The expected value for a continuous random variable $X$ is given by: \begin{equation*}
        E[X] = \mu_X = \mu = \int_{-\infty}^{\infty}xf(x)dx
    \end{equation*}
    The limits of integration are called the \Emph{support} of $X$.
\end{defn}

\begin{thm}{}{}
    Let $X$ be a continuous random variable with pdf $f(x)$ and let $g(X)$ be a real-valued function of $X$. Then the expected value of $g(X)$ is given by: \begin{equation*}
        E[g(X)] = \int_{-\infty}^{\infty}g(x)f(x)dx
    \end{equation*}
\end{thm}

\begin{defn}{}{}
    The variance for a continuous random variable $X$ is defined as follows: \begin{equation*}
        VAR[X] = V[X] = \sigma^2_X = \sigma^2 = E[(X-\mu)^2] = E[X^2]-(E[X])^2
    \end{equation*}
    The standard deviation of $X$ is defined as: \begin{equation*}
        SD[X] = \sigma_X = \sigma = \sqrt{VAR[X]}
    \end{equation*}
\end{defn}

\section{Uniform Random Variables}

\begin{defn}{}{}
    The continuous \Emph{uniform distribution} is a family of symmetric probability distributions such that all intervals of the same length on the distribution's support are equally likely/probable.
\end{defn}

\begin{defn}{}{}
    A uniform random variable depends on two parameters: \begin{align*}
        a &- \text{the minimum value of the support} \\
        b &- \text{the maximum value of the support}
    \end{align*}
    If a random variable $\mathbf{X}$ follows a uniform distribution we write $$\mathbf{X}\sim uniform(a,b)$$
\end{defn}

\begin{defn}{}{}
    The pdf for a uniform random variable $\mathbf{X}$ is \begin{equation*}
        f(x) = \left\{\begin{array}{cc} \frac{1}{b-a} & a \leq x \leq b \\ 0 & elsewhere \end{array}\right.
    \end{equation*}
    The cdf for a uniform  random variable $\mathbf{X}$ is \begin{equation*}
        F(X = x) = P(X\leq x) =  \left\{\begin{array}{cc} 0 & x < a \\ \frac{x-a}{b-a} & a \leq x < b \\ 1 & x \geq b \end{array}\right.
    \end{equation*}
\end{defn}

\begin{defn}{}{}
    Let $\mathbf{X}\sim uniform(a,b)$. Then \begin{equation*}
        E[\mathbf{X}] = \frac{a+b}{2}
    \end{equation*}
    and
    \begin{equation*}
        VAR[\mathbf{X}] = \frac{(b-a)^2}{12}
    \end{equation*}
\end{defn}


\begin{defn}{}{}
    The MGF for a uniform random variable $\mathbf{X}$ is: \begin{equation*}
        M_{\mathbf{X}}(t) = \frac{e^{tb}-e^{ta}}{t(b-a)}
    \end{equation*}
\end{defn}


\begin{defn}{}{}
    Let $\mathbf{X}\sim uniform(a,b)$. To find $P(\mathbf{X} \leq a)$ in $R$, write: \begin{equation*}
        \text{punif(a, $min = min$, $max = max$)}
    \end{equation*}
    To find $x_0$ such that $P(\mathbf{X} \leq x_0) = p$ in $R$, write: \begin{equation*}
        \text{qunif(p, $min = min$, $max = max$)}
    \end{equation*}
\end{defn}



\section{Normal Random Variables}

\begin{defn}{}{}
    The \Emph{normal or Gaussian distribution} is a family of symmetric probability distributions that are bell-shaped.
\end{defn}

Note not all bell-shaped distributions are normal.


\begin{defn}{}{}
    A normal random variable depends on two parameters: \begin{align*}
        \mu &- \text{the mean} \\
        \sigma &- \text{the standard deviation}
    \end{align*}
    If a random variable $\mathbf{X}$ follows a normal distribution we write $$\mathbf{X}\sim normal(\mu,\sigma)$$
\end{defn}

\begin{defn}{}{}
    The pdf for a normal random variable $\mathbf{X}$ is \begin{equation*}
        f(x\vert\mu,\sigma) = \frac{1}{\sigma\sqrt{2\pi}}e^{\frac{-(x-\mu)^2}{2\sigma^2}}
    \end{equation*}
    The cdf for a normal random variable $\mathbf{X}$ is \begin{equation*}
        F(X = x) = P(X\leq x) = \int_{-\infty}^x\frac{1}{\sigma\sqrt{2\pi}}e^{\frac{-(t-\mu)^2}{2\sigma^2}}dt
    \end{equation*}
\end{defn}

\begin{defn}{}{}
    Let $\mathbf{X}\sim normal(\mu,\sigma)$. Then \begin{equation*}
        E[\mathbf{X}] = \mu \approx \frac{\sum_{i=1}^nx_i}{n}\;\;\;(\text{sample mean})
    \end{equation*}
    and
    \begin{equation*}
        VAR[\mathbf{X}] = \sigma \approx \frac{\sum_{i=1}^n(x_i-\mu)^2}{n}\;\;\;(\text{sample variance})
    \end{equation*}
\end{defn}


\begin{defn}{}{}
    The MGF for a normal random variable $\mathbf{X}$ is: \begin{equation*}
        M_{\mathbf{X}}(t) = \operatorname{exp}\left\{\mu t + \frac{\sigma^2t^2}{2}\right\}
    \end{equation*}
\end{defn}


\begin{defn}{}{}
    Let $\mathbf{X}\sim normal(\mu,\sigma)$. To find $P(\mathbf{X} \leq a)$ in $R$, write: \begin{equation*}
        \text{pnorm(a, $mean = \mu$, $sd = \sigma$)}
    \end{equation*}
    To find $x_0$ such that $P(\mathbf{X} \leq x_0) = p$ in $R$, write: \begin{equation*}
        \text{qnorm(p, $mean = \mu$, $sd = \sigma$)}
    \end{equation*}
\end{defn}

\begin{defn}{}{}
    The \Emph{standard normal distribution} is a special case of the normal distribution in which $\mu = 0$ and $\sigma = 1$. 

    If we let $X$ be a normal random variable with mean $\mu$ and standard deviation $\sigma$, we define the \Emph{standardized score} or \Emph{z-score} of $X$ to be: \begin{equation*}
        Z = \frac{X-\mu}{\sigma}
    \end{equation*}
\end{defn}

\section{Gamma Random Variables}

\begin{defn}{}{}
    The \Emph{gamma distribution} is a two-parameter family of continuous probability distributions which are always non-negative and right-skewed.
\end{defn}

\begin{defn}{}{}
    A gamma random variable depends on two parameters: \begin{align*}
        \alpha &- \text{the shape parameter (a format of skewness)} \\
        \beta &- \text{the scale parameter (breadth of viable scope)}
    \end{align*}
    If a random variable $\mathbf{X}$ follows a gamma distribution we write $$\mathbf{X}\sim gamma(\alpha,\beta)$$
\end{defn}

\begin{defn}{}{}
    The pdf for a gamma random variable $\mathbf{X}$ is \begin{equation*}
        f(x\vert\alpha,\beta) = \frac{x^{\alpha-1}e^{-x/\beta}}{\beta^{\alpha}\Gamma(\alpha)}
    \end{equation*}
    The cdf for a gamma random variable $\mathbf{X}$ is \begin{equation*}
        F(X = x) = P(X\leq x) = \int_{0}^x\frac{t^{\alpha-1}e^{-t/\beta}}{\beta^{\alpha}\Gamma(\alpha)}dt
    \end{equation*}
    where $\Gamma(\alpha)$ is the gamma function.
\end{defn}

\begin{defn}{}{}
    The \Emph{gamma function} is defined for all complex numbers $\alpha \in \C$ such that $\mathfrak{R}(\alpha) > 0$, and is defined by: \begin{equation*}
        \Gamma(\alpha) = \int_0^{\infty}x^{\alpha - 1}e^{-x}dx,\;\;\;\;\mathfrak{R}(\alpha) > 0
    \end{equation*}
    The following are some properties of the gamma function: \begin{itemize}
        \item $\Gamma(n) = (n-1)!$ for all $n \in \Z^+$
        \item $\Gamma(\alpha+1) = \alpha\Gamma(\alpha)$ for all $\mathfrak{R}(\alpha) > 0$
        \item $\Gamma\left(\frac{1}{2}\right) = \sqrt{\pi}$
        \item By definition we have the identity: \begin{equation*}
                \int_0^{\infty}x^{\alpha-1}e^{-x/\beta}dx = \beta^{\alpha}\Gamma(\alpha)
        \end{equation*}
            or equivalently \begin{equation*}
                \int_0^{\infty}x^{\alpha}e^{-x/\beta}dx = \beta^{\alpha+1}\Gamma(\alpha+1)
            \end{equation*}
    \end{itemize}
\end{defn}





\begin{defn}{}{}
    Let $\mathbf{X}\sim gamma(\alpha,\beta)$. Then \begin{equation*}
        E[\mathbf{X}] = \alpha\beta
    \end{equation*}
    and
    \begin{equation*}
        VAR[\mathbf{X}] = \alpha\beta^2
    \end{equation*}
\end{defn}


\begin{defn}{}{}
    The MGF for a gamma random variable $\mathbf{X}$ is: \begin{equation*}
        M_{\mathbf{X}}(t) = (1-\beta t)^{-\alpha}
    \end{equation*}
\end{defn}


\begin{defn}{}{}
    Let $\mathbf{X}\sim gamma(\alpha,\beta)$. To find $P(\mathbf{X} \leq a)$ in $R$, write: \begin{equation*}
        \text{pgamma(a, $shape = \alpha$, $scale = \beta$)} (or\;\; rate = 1/\beta)
    \end{equation*}
    To find $x_0$ such that $P(\mathbf{X} \leq x_0) = p$ in $R$, write: \begin{equation*}
        \text{qgamma(p, $shape = \alpha$, $scale = \beta$)}
    \end{equation*}
\end{defn}


\section{Exponential Random Variables}

\begin{defn}{}{}
    The \Emph{exponential distribution} is a special case of the gamma distribution when $\alpha = 1$
\end{defn}

\begin{defn}{}{}
    An exponential random variable depends on one parameter: \begin{align*}
        \beta &- \text{the scale parameter (breadth of viable scope)}
    \end{align*}
    In certain applications we write $\beta = 1/\lambda$ to emphasize the relationship between the Exponential and Poisson distributions. If a random variable $\mathbf{X}$ follows an exponential distribution we write $$\mathbf{X}\sim exponential(\beta)$$
\end{defn}

\begin{defn}{}{}
    The pdf for an exponential random variable $\mathbf{X}$ is \begin{equation*}
        f(x\vert\beta) = \frac{e^{-x/\beta}}{\beta}
    \end{equation*}
    The cdf for an exponential random variable $\mathbf{X}$ is \begin{equation*}
        F(X = x) = P(X\leq x) = \int_{0}^x\frac{e^{-t/\beta}}{\beta}dt
    \end{equation*}
    A useful result is that for $x > 0$, \begin{equation*}
        F(X = x) = 1 - e^{-x/\beta}
    \end{equation*}
\end{defn}



\begin{defn}{}{}
    Let $\mathbf{X}\sim exponential(\beta)$. Then \begin{equation*}
        E[\mathbf{X}] = \beta = \frac{1}{\lambda}
    \end{equation*}
    and
    \begin{equation*}
        VAR[\mathbf{X}] = \beta^2 = \frac{1}{\lambda^2}
    \end{equation*}
\end{defn}


\begin{defn}{}{}
    The MGF for an exponential random variable $\mathbf{X}$ is: \begin{equation*}
        M_{\mathbf{X}}(t) = (1-\beta t)^{-1}
    \end{equation*}
\end{defn}

\begin{props}{}{}
    If $X \sim exponential(\beta)$, then for any $a,b \in \R$, \begin{equation*}
        P(X > t+s\vert X >t) = P(X > s)
    \end{equation*}
    and \begin{equation*}
        P(X < t+s\vert X > t) = P(X < s)
    \end{equation*}
\end{props}


\begin{defn}{}{}
    Let $\mathbf{X}\sim exponential(\beta)$. To find $P(\mathbf{X} \leq a)$ in $R$, write: \begin{equation*}
        \text{pexp(a, $rate = 1/\beta$)} 
    \end{equation*}
    To find $x_0$ such that $P(\mathbf{X} \leq x_0) = p$ in $R$, write: \begin{equation*}
        \text{qexp(p, $rate = 1/\beta$)}
    \end{equation*}
\end{defn}




\section{Beta Random Variables}

\begin{defn}{}{}
    The \Emph{beta distribution} is a two-parameter family of continuous probability distributions defined on the interval $[0,1]$. 
\end{defn}

\begin{defn}{}{}
    A beta random variable depends on two parameters: \begin{align*}
        \alpha &- \text{a shape parameter} \\
        \beta &- \text{a shape parameter}
    \end{align*}
    If a random variable $\mathbf{X}$ follows a beta distribution we write $$\mathbf{X}\sim beta(\alpha,\beta)$$
\end{defn}

\begin{defn}{}{}
    The pdf for a beta random variable $\mathbf{X}$ is \begin{equation*}
        f(x\vert\alpha,\beta) = \left\{\begin{array}{cc} \frac{x^{\alpha-1}(1-x)^{\beta-1}}{B(\alpha,\beta)} & 0 \leq x \leq 1 \\ 0 & elsewhere, \end{array}\right.
    \end{equation*}
    The cdf for a beta random variable $\mathbf{X}$ is \begin{equation*}
        F(X = x) = P(X\leq x) = \int_{0}^x\frac{t^{\alpha-1}(t-1)^{\beta-1}}{B(\alpha,\beta)}dt
    \end{equation*}
    where $B(\alpha,\beta)$ is the beta function.
\end{defn}

\begin{defn}{}{}
    The \Emph{beta function} is defined for all complex numbers $\alpha,\beta \in \C$ such that $\mathfrak{R}(\alpha),\mathfrak{R}(\beta) > 0$, and is defined by: \begin{equation*}
        B(\alpha,\beta) = \int_0^1x^{\alpha-1}(1-x)^{\beta-1}dx = \frac{\Gamma(\alpha)\Gamma(\beta)}{\Gamma(\alpha+\beta)}
    \end{equation*}
\end{defn}





\begin{defn}{}{}
    Let $\mathbf{X}\sim beta(\alpha,\beta)$. Then \begin{equation*}
        E[\mathbf{X}] = \frac{\alpha}{\alpha+\beta}
    \end{equation*}
    and
    \begin{equation*}
        VAR[\mathbf{X}] = \frac{\alpha\beta}{(\alpha+\beta)^2(\alpha+\beta+1)}
    \end{equation*}
\end{defn}



\begin{defn}{}{}
    Let $\mathbf{X}\sim beta(\alpha,\beta)$. To find $P(\mathbf{X} \leq a)$ in $R$, write: \begin{equation*}
        \text{pbeta(a, $shape1 = \alpha$, $shape2 = \beta$)}
    \end{equation*}
    To find $x_0$ such that $P(\mathbf{X} \leq x_0) = p$ in $R$, write: \begin{equation*}
        \text{qbeta(p, $shape1 = \alpha$, $shape2 = \beta$)}
    \end{equation*}
\end{defn}

\chapter{Bivariate Probabilities}

\section{Bivariate Data Structures}

\begin{defn}{}{}
    The \Emph{joint probability function} for discrete random variables $X$ and $Y$ is \begin{equation*}
        P(X = x\cap Y = y) = P(X = x, Y = y)
    \end{equation*}
    and it has the following properties: \begin{enumerate}
        \item $0 \leq p(x,y) \leq 1$; $\forall x \in Dom(X), \forall y \in Dom(Y)$.
        \item $\sum\limits_{all\;x}\sum\limits_{all\;y}p(x,y) = 1$
    \end{enumerate}
\end{defn}

\begin{defn}{}{}
    The \Emph{joint distribution function} for discrete random variables $X$ and $Y$ is \begin{equation*}
        F(X = x\cap Y = y) = P(X \leq x, Y \leq y) = \sum\limits_{t_1\leq x}\sum\limits_{t_2\leq y}p(t_1,t_2)
    \end{equation*}
\end{defn}


\begin{defn}{}{}
    Two random variables are \Emph{jointly continuous} if there exist a \Emph{joint density function} $f(x,y)$ which satisfies the density function axioms: \begin{enumerate}
        \item $f(x,y) \geq 0$, $\forall x \in Dom(X),\forall y \in Dom(Y)$
        \item $\int_{-\infty}^{\infty}\int_{-\infty}^{\infty}f(x,y)dxdy =1$
    \end{enumerate}
\end{defn}


\begin{defn}{}{}
    The \Emph{joint distribution function} for joint continuous random variables $X$ and $Y$ is \begin{equation*}
        F(X = x\cap Y = y) = P(X \leq x, Y \leq y) = \int_{-\infty}^x\int_{-\infty}^yf(t_1,t_2)dt_2dt_1
    \end{equation*}
\end{defn}


%%%%%%%%%%%%%%%%%%%%%%%%%%%%%%%%%%



%-------------- End ---------------

\end{document}
