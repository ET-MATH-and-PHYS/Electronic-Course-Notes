%%%%%%%%%%%%%%%%%%%%% chapter.tex %%%%%%%%%%%%%%%%%%%%%%%%%%%%%%%%%
%
% sample chapter
%
% Use this file as a template for your own input.
%
%%%%%%%%%%%%%%%%%%%%%%$% Springer-Verlag %%%%%%%%%%%%%%%%%%%%%%%%%%
%\motto{Use the template \emph{chapter.tex} to style the various elements of your chapter content.}
\chapter{Using Prefixes, Bases, and Suffixes to Describe the Body}
\label{Body} % Always give a unique label
% use \chaptermark{}
% to alter or adjust the chapter heading in the running head


%%% Questions to think about
%The \textbf{Thesis} or general sense of the article is ...

%The \textbf{method} the author uses to argue their point is ...

%In their \textbf{analysis} the author uses tools such as ... 

%Additionally they conclude ...

%What connections does the author portray with regard to \textbf{space}, \textbf{relationships}, \textbf{occupation}, and \textbf{religion}.


\abstract{}


\section{Notes}
\label{sec:NOTE2}

\subsection{Prefixes}

Remember, the prefix is added to the front of the base. It modifies or adds extra information about the base, telling us how, where, or to what degree something happens.

\begin{longtable}{c | c | c}
    \caption{Prefixes for the body.}
    \hline 
    Prefix & Meaning & Example(s) \\ \hline
    ambi-, ambo- & `both' & ambidextrous \\ 
    apo- & `away from' & apogy \\
    contra- & `opposite,' `outer' & contrapositive, contradiction \\
    ecto- & `outside,' `outer' & ectoplasm \\
    endo- & `inside,' `inner' & endomorphism \\
    inter- & `between' & intermolecular \\
    intra- & `within' & intramolecula \\
    meso- & `middle' & mesosphere \\
    sub- & `below,' `underneath' & subzero \\
    super- & `upper,' `above,' `beyond' & superhuman \\
    trans- & `across,' `through' & transgender \\
    \label{tab:Ch2Prefix}
\end{longtable}


\subsection{Suffixes}

Remember, the suffix is added to the end of the base to make meaningful sense. THe base and the suffix together form a complete noun, adjective, or verb.

\begin{longtable}{c | c | c}
    \caption{Suffixes for the body.}
    \hline 
    Suffix & Meaning & Example(s) \\ \hline
    -ad & `toward' & ahead, nomad \\
    -an & `pertaining to' & Russian \\
    -al & `pertaining to' & natural \\
    -ary & `pertaining to' & cautionary \\
    -ial & `pertaining to' & substantial \\
    -ic & `pertaining to' & acidic \\
    -ion & `action,' `condition,' `act of ' & fusion, creation \\
    -ous & `pertaining to,' `like,' `full of,' `having' & glorious, humorous \\
    -tic & `pertaining to' & analytic \\
    -verse & `to turn,' `to travel,' `turned' & transverse, neurotic \\
    \label{tab:Ch2Suffix}
\end{longtable}


\subsection{Bases}

Some bases have two meanings that are entirely different.

\begin{longtable}{c | p{0.4\textwidth} | p{0.4\textwidth}}
    \caption{Bases for the body.}
    \hline 
    Base & Meaning(s) & Example(s) \\ \hline
    VENTR- & `front,' `abdomen,' `belly' & ventral (VENTR-al) - pertaining to the front, ventrad (VENTR-ad) - toward the front \\
    DORS- & `back' & mesodorsal (meso-DORS-al) - pertaining to the middle of the back, dorsoventral (DORS-O-VENTR-al) - pertaining to the front and the back, dorsad (DORS-ad) - toward the back \\
    ANTER- & `front,' `before' & anterior (ANTER-ior) - pertaining to the front \\
    POSTER- & `back,' `behind' & posterior (POSTER-ior) - pertaining to the back \\
    FRONT- & `front,' `forehead' & frontal (FRONT-al) - pertaining to the front \\
    LATER- & `side' & contralateral (contra-LATER-al) - pertaining to the opposite side, ambilateral (ambi-LATER-al) - pertaining to both sides \\
    MEDI- & `middle,' `midline' & median (MEDI-an) - pertaining to the middle, mediad (MEDI-ad) - toward the middle \\
    MES- & `middle,' `midline' & mesal (MES-al) - pertaining to the middle, mesial (MES-I-al) - pertaining to the middle \\
    CENTR- & `center,' `mid-point' & apocentric (apo-CENTR-ic) - pertaining to away from the center \\
    DEXTR- & `right,' `right-handed' & dextrad (DEXTR-ad) - toward the right \\
    SINISTR- & `left,' `left-handed' & sinistrad (SINISTR-ad) - toward the left \\
    LAEV- or LEV- & `left,' `left-handed' & levoversion (LEV-O-VERS-ion) - action of turning to the left \\
    SUPER- & `above,' `in the top part of' & superior (SUPER-ior) - pertaining to above \\
    INFER- & `below,' `in the bottom part of' & inferior (INFER-ior) - pertaining to below \\
    CEPHAL- & `head' & cephalic (CEPHAL-ic) - pertaining to the head, intracephalic (intra-CEPHAL-ic) - pertaining to the inside of the head, cephalad (CEPHAL-ad) - toward the head, caudocephalad (CAUD-O-CEPHAL-ad) - toward the head from the tail \\
    CAUD- & `tail' & caudal (CAUD-al) - pertaining to the tail, caudad (CAUD-ad) - toward the tail, cephalocaudad (CEPHAL-O-CAUD-ad) - toward the tail from the head \\
    PROXIM- & `near to,' `near a point of attachment,' `near the beginning of a structure' & proximal (PROXIM-al) - pertaining to near the point of attachment \\
    DIST- & `away from,' `away from a point of attachment,' `away from the beginning of a structure' & distal (DIST-al) - pertaining to away from the point of attachment \\
    INTERN- & `inside' & internal (INTERN-al) - pertaining to the inside \\
    EXTERN- & `outside' & external (EXTERN-al) - pertaining to the outside \\
    FACI- or -FICI- & `face,' `surface' & facial (FACI-al) - pertaining to the face, superficial (SUPER-FICI-al) - pertaining to the upper surface \\
    PARIET- & `wall' (usually, of a body cavity) & parietal (PARIET-al) - pertaining to a cavity wall, interparietal (inter-PARIET-al) - pertaining to between cavity walls, intraparietal (intra-PARIET-al) - pertaining to inside cavity walls, transparietal (trans-PARIET-al) - pertaining to across cavity walls \\
    AX- & `axis,' `central line' & axial (AX-ial) - pertaining to a central line, subaxial (sub-AX-ial) - pertaining to below the central line \\
    VERS- or VERT- & `travel,' `turn' & transverse (trans-VERS-e/trans-verse) - travel across, version (VERS-ion) - the action of turning \\
    CORON- & `crown,' `like a crown' & coronal (CORON-al) - pertaining to a crown, coronary (CORON-ary) - pertaining to like a crown \\
    SAGITT- & `arrow' & sagittal (SAGGIT-al) - pertaining to an arrow \\
    MORPH- & `form,' `shape' & morphic (MORPH-ic) - pertaining to shape, morphous (MORPH-ous) - pertaining to shape, morphotic (MORPH-O-tic) - pertaining to shape \\
    DERM- or DERMAT- & `skin,' `layer' & endodermic (endo-DERM-ic) & pertaining to the inside layer, mesodermic (meso-DERM-ic) - pertaining to a middle layer, ectodermic (ecto-DERM-ic) - pertaining to an outside layer \\
    SOM- or SOMAT- & `body' & somal (SOM-al) - pertaining to the body, somatic (SOMAT-ic) - pertaining to the body \\
    CORP-, CORPOR-, or CORPUS- & `body' & corporal (CORPOR-al) - pertaining to the body \\
    \label{tab:Ch2Bases}
\end{longtable}





\section{Questions and Remarks}
\label{sec:QR2}






%
% \begin{acknowledgement}
% If you want to include acknowledgments of assistance and the like at the end of an individual chapter please use the \verb|acknowledgement| environment -- it will automatically render Springer's preferred layout.
% \end{acknowledgement}
%
% \section*{Appendix}
% \addcontentsline{toc}{section}{Appendix}
%


% Problems or Exercises should be sorted chapterwise
\section*{Problems}
\addcontentsline{toc}{section}{Problems}
%
% Use the following environment.
% Don't forget to label each problem;
% the label is needed for the solutions' environment
\begin{prob}
\label{prob1}
A given problem or Excercise is described here. The
problem is described here. The problem is described here.
\end{prob}

% \begin{prob}
% \label{prob2}
% \textbf{Problem Heading}\\
% (a) The first part of the problem is described here.\\
% (b) The second part of the problem is described here.
% \end{prob}

%%%%%%%%%%%%%%%%%%%%%%%% referenc.tex %%%%%%%%%%%%%%%%%%%%%%%%%%%%%%
% sample references
% %
% Use this file as a template for your own input.
%
%%%%%%%%%%%%%%%%%%%%%%%% Springer-Verlag %%%%%%%%%%%%%%%%%%%%%%%%%%
%
% BibTeX users please use
% \bibliographystyle{}
% \bibliography{}
%


% \begin{thebibliography}{99.}%
% and use \bibitem to create references.
%
% Use the following syntax and markup for your references if 
% the subject of your book is from the field 
% "Mathematics, Physics, Statistics, Computer Science"
%
% Contribution 
% \bibitem{science-contrib} Broy, M.: Software engineering --- from auxiliary to key technologies. In: Broy, M., Dener, E. (eds.) Software Pioneers, pp. 10-13. Springer, Heidelberg (2002)
% %
% Online Document

% \end{thebibliography}

