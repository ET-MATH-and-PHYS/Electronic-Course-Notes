%%%%%%%%%%%%%%%%%%%%% chapter.tex %%%%%%%%%%%%%%%%%%%%%%%%%%%%%%%%%
%
% sample chapter
%
% Use this file as a template for your own input.
%
%%%%%%%%%%%%%%%%%%%%%%$% Springer-Verlag %%%%%%%%%%%%%%%%%%%%%%%%%%
%\motto{Use the template \emph{chapter.tex} to style the various elements of your chapter content.}
\chapter{Ears}
\label{Ears} % Always give a unique label
% use \chaptermark{}
% to alter or adjust the chapter heading in the running head


%%% Questions to think about
%The \textbf{Thesis} or general sense of the article is ...

%The \textbf{method} the author uses to argue their point is ...

%In their \textbf{analysis} the author uses tools such as ... 

%Additionally they conclude ...

%What connections does the author portray with regard to \textbf{space}, \textbf{relationships}, \textbf{occupation}, and \textbf{religion}.


\abstract{}


\section{Notes}
\label{sec:NOTE6}


\subsection{Prefixes}

\begin{longtable}{c | p{0.4\textwidth} | p{0.4\textwidth}}
    \caption{Prefixes for the ears.}
    \hline
    Prefix & Meaning(s) & Example(s) \\ \hline
        ana-, an- & `up,' `upward,' `again' & anabolic, \\
        dis-, dif-, di- & `apart,' `away from,' `separation,' `lack of' & dissimilar, dissapointment, distrust \\
        enanti- & `opposite' & enantimer \\
        ex-, ef-, e- & `out,' `outside,' `from' & exit, escape \\
        exo- & ex- & `outside,' `outer,' `external' & external, exoplanet \\
        in-, il-, im-, ir- & `not' & indecisive, incredulent \\
        juxta- & `close to' & juxtapose \\
        pre-, prae- & `in front of,' `before' & prenatal,preface \\
        pro- & `in front of,' `forward' & pronounce \\
        se- & `away,' `aside' & separate, \\
        ultra- & `beyond' & ultraman \\
    \label{tab:Ch6Prefix}
\end{longtable}


\subsection{Suffixes}

\begin{longtable}{c | p{0.4\textwidth} | p{0.4\textwidth}}
    \caption{Suffixes for the ears.}
    \hline
    Suffix & Meaning(s) & Example(s) \\ \hline
        -able, -abil- & `able to be' & capable, ability \\
        -ac & `pertaining to' & sac \\
        -ance, -ancy & `state of' & stance, happenstance \\
        -ature & `system composed of' & \\
        -culus & `small' & \\
        -ent & `pertaining to' & entertainment \\
        -er & `person who (does ...),' `thing that (does...)' & barber, climber \\
        -fic & `causing,' `making' & \\
        -ician & `specialist' & physician, musician \\
        -ics, -tics & `art of,' `science of,' `study of' & physics \\
        -id & `pertaining to,' `having' & horrid \\
        -ine & `pertaining to' & clementine \\
        -ism & `condition of' & marxism \\
        -ist & `person who (does...),' `specialist' & artist \\
        -ization & `process of making' & realization, \\
        -ory & `having the function of' & story, laboratory \\
    \label{tab:Ch6Suffix}
\end{longtable}




\subsection{Bases} 

\begin{longtable}{c | p{0.4\textwidth} | p{0.4\textwidth}}
    \caption{Bases for the ear.}
    \hline
    Base & Meaning(s) & Example(s) \\ \hline
        AUD-, AUDI-, AUDIT- & `to hear,' `hearing' & audile (AUD-ile) - pertaining to hearing (also, able to hear), audiometry (AUDI-O-metry) - process of measuring hearing, auditory (AUDIT-ory) - having the function of hearing \\
        ACOU-, ACOUS- & `hearing,' `sound' & acoumeter (ACOU-meter) - instrument used to measure hearing, anacousia (an-ACOUS-ia) - pertaining to not hearing (i.e. deafness), acousticophobia (ACOUS-tic-O-phobia) - abnormal fear pertaining to sound \\
        SON- & `sound' & asonia (a-SON-ia) - pertaining to no sound (i.e. deafness), ultrasonography (ultra-SON-O-graphy) - process of recording with beyond (normal) sound, sonographer (SON-O-GRAPH-er) - person who records sound, dissonance (dis-SON-ance) - state of separation of sound \\
        AUR- & `ear' & dextraural (DEXTR-AUR-al) - pertaining to the right ear, auriscopy (AUR-I-scopy) - process of examining the ear with an instrument, auricle (AUR-I-cle) - small ear \\
        AURICUL- & `auricle,' (part of the outer ear that projects outward from the head, also the small ear-shaped projections found in the heart) `outer ear' & auricular (AURICUL-ar) - pertaining to the outer ear, pertaining to an auricle, pre-auricular (pre-AURICUL-ar) - pertaining to in front of the outer ear, auriculocranial (AURICUL-O-CRANI-al) - pertaining to the cranium and the outer ear, auriculoid (AURICUL-oid) - shaped like the outer ear \\
        OT- & `ear' & endotitis (end-OT-itis) - inflammation of the inner ear, prootic (pro-OT-ic) - pertaining to in front of the ear, parotid (par-OT-id) - pertaining to beside the ear \\
        PENN-, PINN- & `feather,' `wing' & penniform (PENN-I-form) - like a feather or wing, pinnate (PINN-ate) - pertaining to a feather or wing, having feathers or wings \\
        PINN- & `pinna,' `auricle' (the outer ear) & pinnal (PINN-al) - pertaining to the pinna \\
        CHONDR- & `granule,' `cartilage' & synchondrosis (syn-CHONDR-osis) - condition of (a joint with) connecting cartilage, hypochondriac (hypo-CHONDR-I-ac) - pertaining to below the cartilage region, hypochondriasis (hypo-CHONDR-iasis) - abnormal condition below the cartilage region \\
        HELIC- & `spiral,' `helix' & helicine (HELIC-ine) - pertaining to a spiral or helix, helicoid (HELIC-oid) - resembling a spiral or helix \\
        FOSS- & `ditch,' `trench,' `to dig' & efossion (e-FOSS-ion) - act of removal by digging \\
        FOSS- & `fossa' (a pit, a trench-like depression, or a hollow in the surface of a structure in the body) & fossula (FOSS-ula) - a little fossa \\
        SCAPH- & `boat shaped' & scaphocephalism (SCAPH-O-CEPHAL-ism) - condition of the head being boat shaped, scaphoid (SCAPH-oid) - resembling a boat \\
        TUBER- & `swelling,' `nodule' & tubercle (TUBER-cle) - small swelling \\
        TUBERCUL- & `tubercle,' `small swelling or nodule' & tuberculitis (TUBERCUL-itis) - swelling of any tubercle, tuberculosis (TUBERCUL-osis) - abnormal condition of the tubercles \\
        CONCH- & `sea-shell,' `spiral shell' & conchate (CONCH-ate) - pertaining to a sea-shell, conchiform (CONCH-I-form) - having the form of a spiral shell \\
        CONCH- & `concha' (a structure similar in shape to a sea-shell) & conchitis (CONCH-itis) - inflammation of a concha, conchoscope (CONCH-O-scope) - an instrument used to examine the conchae \\
        TURBIN- & `cone shaped,' `spiraled' & turbinate (TURBIN-ate) - having cone shaped (features), turbinated (TURBIN-ated) - composed of spiraled (parts), turbinectomy (TURBIN-ectomy) - surgical removal of a turbinated (bone) \\
        MAST- & `breast' & amastia (a-MAST-ia) - condition of without breasts, mastoid (MAST-oid) - resembling a breast \\
        MASTOID- & `mastoid process' & mastoiditis (MASTOID-itis) - inflammation of the mastoid process, premastoidal (pre-MASTOID-al) - pertaining to in front of the mastoid process \\
        CED-, CESS- & `to go,' `to be in motion' & `antecedent (ante-CED-ent) - pertaining to before going, processive (pro-CESS-ive) - tending to forward moving, secession (se-CESS-ion) - act of away moving \\
        STYL- & `column,' `pillar' & styloid (STYL-oid) - resembling a colum, styliform (STYL-I-form) - having the form of a column \\
        STYL-, STYLOID- & `styloid process' & stylomastoidal (STYL-O-MASTOID-al) - pertaining to the mastoid and styloid processes, styloiditis (STYLOID-itis) - inflammation of a styloid process \\
        ME- & `to go,' `to pass,' `to travel' & permeation (per-ME-ation) - process of through passage, permeable (per-ME-able) - tending to (allow) thorugh passage, impermeable (im-per-ME-able) - tending to not (allow) through passage \\
        MEAT- & `opening,' `passageway,' `meatus' (a bodily passage or channel, especially the external opening of a canal) & meatal (MEAT-al) - pertaining to an opening, meatoscopy (MEAT-O-scopy) - process of examining a meatus with an instrument, meatometry (MEAT-O-metry) - process of measuring a meatus \\
        TYMPAN- & `drum' `stretched membrane' & tympanal (TYMPAN-al) - pertaining to a drum, tympanism (TYMPAN-sim) - condition of (being like) a drum \\
        TYMPAN- & `tympanic membrane,' `tympanic cavity' & tympanotemporal (TYMPAN-O-TEMPOR-al) - pertaining to the tympanic cavity and the temporal region or bone, tympanities (TYMPAN-itis) - inflammation of the tympanic membrane, tympanometry (TYMPAN-O-metry) - process of measuring the tympanic membrane, tympanogenous (TYMPAN-O-GEN-ous) - pertaining to production by the tympanic cavity \\
        MYRING- & `tympanic membrane' & myringitis (MYRING-itis) - inflammation of the tympanic membrane, myringoscope (MYRING-O-scope) - instrument used to examine the tympanic membrane \\
        OSS-, OSSE- & `bone' & ossature (OSS-ature) - system composed of bones, ossicle (OSS-I-cle) - small bone, osseous (OSSE-ous) - pertaining to bone \\
        LABYRINTH- & `labyrinth,' `maze' & labyrinthine (LABYRINTH-ine) - pertaining to a labyrinth \\
        LABYRINTH- & `inner ear,' & labyrinthitis (LABYRINTH-itis) - inflammation of the inner ear \\
        VESTIBUL- & `vestibule,' (a vestibule or vestibulum is a small entrance to a canal) `entrance' & vestibulate (VESTIBUL-ate) - having a vestibule, vestibular (VESTIBUL-ar) - pertaining to a vestibule  \\
        COCHLE- & `snail-shell,' `spiral shell' & cochleous (COCHLE-ous) - like a snail-shell, cochleate (COCHLE-ate) - pertaining to a snail shell \\
        COCHLE- & `cochlea' (spiral-shaped canal surrounding a core of spongy bone in the inner ear) & cochlear (COCHLE-ar) - pertaining to the cochlea, cochleitis (COCHLE-itis) - inflammation of the cochlea, vestibulocochlear (VESTIBUL-O-COCHLE-ar) - pertaining to the cochlea and vestibule \\
        CANAL- & `channel' & canaliform (CANAL-I-form) - having the form of a channel, canaliculus  (CANAL-I-culus) - a small channel, a canaliculus \\
        CANALICUL- & `canaliculus,' `small channel' & canalicular (CANALICUL-ar) - pertaining to a canaliculus, canaliculization (CANALICUL-ization) - process of making caniculi or small channels in a tissue \\
        AMPULL- & `flask,' `bottle' & ampulliform (AMPULL-I-form) - having the form of a flask or bottle \\
        AMPULL- & `ampulla' (a flask-shaped enlargement of a canal or duct) & ampullitis (AMPULL-itis) - inflammation of an ampulla, juxtaampullary (juxta-AMPULL-ary) - pertaining to close to an ampulla, ampullula (AMPULL-ula) - small ampulla \\
    \label{tab:Ch6Base}
\end{longtable}

In anatomy processes are bony projections that provide attachment points for muscles and ligaments.


\begin{longtable}{c | p{0.4\textwidth} | p{0.4\textwidth}}
    \caption{Bases for the pain, disease, and treatment.}
    \hline
    Base & Meaning(s) & Example(s) \\ \hline
        ALG- & `pain' & algogenesis (ALG-O-GEN-esis) - process of producing pain, otalgia (OT-ALG-ia) - condition of pain in the ear (i.e. earache), algophobia (ALG-O-phobia) - abnormal fear of pain \\
        ALGES- & `pain sensation' & analgesic (an-ALGES-ic) - pertaining to without pain sensation (i.e. something that alleviates pain), algesimetry (ALGES-I-metry) - process of measuring pain sensation, hyperalgesia (hyper-ALGES-ia) - condition of more than normal pain sensation \\
        AGR- `pain,' `painful seizure' & cephalagra (CEPHAL-AGRa) - pain in the head, ophthalmagra (OPHTHALM-AGRa) - pain in the eyes \\
        ODYN- & `pain' & otodynia (OT-ODYN-ia) - condition of pain in the ear, anodynia (an-ODYN-ia) - condition of without pain, cephalodynic (CEPHAL-ODYN-ic) - pertaining to pain in the head \\
        DOL-, DOLOR- & `pain' & indolent (in-DOL-ent) - pertaining to not painful, dolorific (DOLOR-I-fic) - causing pain \\
        NOS- & `disease,' `sickness' & nosogenic (NOS-O-GEN-ic) - pertaining to producing disease, gynenosia (GYN-E-NOS-ia) - condition of disease that (affects) women (mostly), nosometry (NOS-O-metry) - process of measuring disease \\
        MORB- & `disease' & morbid (MORB-id) - pertaining to disease, morbigenous (MORB-I-GEN-ous) - pertaining to the production of disease \\
        PATH- & `disease,' `suffering,' `feeling' & andropathy (ANDR-O-PATH-y) - state of a disease that (affects) men (mostly), enantiopathic (enanti-O-PATH-ic) - pertaining to opposite feelings, exopathic (exo-PATH-ic) - pertaining to an external disease \\
        AESTHE-, ESTHE- & `to feel,' `sensation,' `feeling' & anaesthetic (an-AESTHE-tic) - pertaining to without sensation, acouesthesia (ACOU-ESTHE-sia) - condition of sensation of sound, hyperphotesthesia (hyper-PHOT-ESTHE-sia) - condition of more than normal sensation to light, anacatesthesia (ana-cat-ESTHE-sia) - condition of up down sensation \\
        THERAP-, THERAPEUT- & `to provide treatment,' `treatment' & therapist (THERAP-ist) - specialist who provides treatment, therapeutic (THERAPEUT-ic) - pertaining to treatment, therapeutician (THERAPEUT-ician) - specialist who provides treatment \\
        IATR- & `physician,' `medical treatment' & psychiatry (PSYCH-IATR-y) - act of medical treatment of the mind, psychiatrics (PSYCH-IATR-ics) - art of medical treatment of the mind, dermiatric (DERM-IATR-ic) - pertaining to medical treatment of the skin, iatrophobia (IATR-O-phobia) - abnormal fear of physicians
    \label{tab:Ch6Base2}
\end{longtable}


\subsection{Compound Suffixes}

\begin{longtable}{c | p{0.4\textwidth} | p{0.4\textwidth}}
    \caption{Compound suffixes for the ears.}
    \hline
    Suffix & Meaning(s) & Example(s) \\ \hline
        -agra & `pain,' `painful seizure' & cephalagra \\
        -algesia & `sensation of pain' & \\
        -algia & `painful condition' & mialgia \\
        -acousia & `condition of hearing' & \\
        -esthesia & `condition of sensation' & \\
        -genesis & `production' & \\
        -genic & `producing,' `produced' & \\
        -iatrics, -iatry & `medical treatment' & psychiatrics \\
        -nosia & `disease' & \\
        -odynia & `painful condition' & \\
        -pathia, -pathy & `disease,' `treatment of disease' & \\
        -therapia, -therapy & `treatment' & \\
    \label{tab:Ch6Suffix2}
\end{longtable}


\section{Questions and Remarks}
\label{sec:QR6}






%
% \begin{acknowledgement}
% If you want to include acknowledgments of assistance and the like at the end of an individual chapter please use the \verb|acknowledgement| environment -- it will automatically render Springer's preferred layout.
% \end{acknowledgement}
%
% \section*{Appendix}
% \addcontentsline{toc}{section}{Appendix}
%


% Problems or Exercises should be sorted chapterwise
\section*{Problems}
\addcontentsline{toc}{section}{Problems}
%
% Use the following environment.
% Don't forget to label each problem;
% the label is needed for the solutions' environment
\begin{prob}
\label{prob1}
A given problem or Excercise is described here. The
problem is described here. The problem is described here.
\end{prob}

% \begin{prob}
% \label{prob2}
% \textbf{Problem Heading}\\
% (a) The first part of the problem is described here.\\
% (b) The second part of the problem is described here.
% \end{prob}

%%%%%%%%%%%%%%%%%%%%%%%% referenc.tex %%%%%%%%%%%%%%%%%%%%%%%%%%%%%%
% sample references
% %
% Use this file as a template for your own input.
%
%%%%%%%%%%%%%%%%%%%%%%%% Springer-Verlag %%%%%%%%%%%%%%%%%%%%%%%%%%
%
% BibTeX users please use
% \bibliographystyle{}
% \bibliography{}
%


% \begin{thebibliography}{99.}%
% and use \bibitem to create references.
%
% Use the following syntax and markup for your references if 
% the subject of your book is from the field 
% "Mathematics, Physics, Statistics, Computer Science"
%
% Contribution 
% \bibitem{science-contrib} Broy, M.: Software engineering --- from auxiliary to key technologies. In: Broy, M., Dener, E. (eds.) Software Pioneers, pp. 10-13. Springer, Heidelberg (2002)
% %
% Online Document

% \end{thebibliography}

