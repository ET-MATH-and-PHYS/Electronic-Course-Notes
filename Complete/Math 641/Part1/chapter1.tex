%%%%%%%%%%%%%%%%%%%%% chapter.tex %%%%%%%%%%%%%%%%%%%%%%%%%%%%%%%%%
%
% sample chapter
%
% Use this file as a template for your own input.
%
%%%%%%%%%%%%%%%%%%%%%%%% Springer-Verlag %%%%%%%%%%%%%%%%%%%%%%%%%%
%\motto{Use the template \emph{chapter.tex} to style the various elements of your chapter content.}
\chapter{Number Fields}
\label{Numb} % Always give a unique label
% use \chaptermark{}
% to alter or adjust the chapter heading in the running head



\abstract{Summary of material in chapter (to be completed after chapter)}

\section{Basic Concepts}
\label{sec:BC1}

A \textbf{number field} is a subfield of $\C$ having finite degree over $\Q$. Every such field has the form $\Q(\alpha)$ for some algebraic number $\alpha \in \C$. Note as the extension is finite $\Q(\alpha) = \Q[\alpha]$. If $\alpha$ is a root of an irreducible polynomial over $\Q$ having degree $n$, then $$\Q[\alpha] = \{a_0+a_1\alpha+\cdots+a_{n-1}\alpha^{n-1}:a_i \in \Q\}$$
and $\{1,\alpha,...,\alpha^{n-1}\}$ is a basis for $\Q[\alpha]$ as a vector space over $\Q$.

Consider $\omega = e^{2\pi i/m}$. The field $\Q[\omega]$ is called the $m^{th}$ \textbf{cyclotomic field}. In general, for odd $m$, the $m^{th}$ cyclotoic field is equal to the $2m^{th}$. On the other hand, cyclotomic fields for $m$ even, $m > 0$, are all distinct.

Another infinite class of number fields consists of the \textbf{quadratic fields} $\Q[\sqrt{m}]$, $m \in \Z$, $m$ not a perfect square. These fields have degree $2$ over $\Q$, having basis $\{1,\sqrt{m}\}$. We need only consider squarefree $m$ since, for example, $\Q[\sqrt{12}] = \Q[\sqrt{3}]$. The $\Q[\sqrt{m}]$, for $m$ squarefree, are all distinct. The $\Q[\sqrt{m}]$, $m > 0$, are called the \textbf{real quadratic fields}; the $\Q[\sqrt{m}]$, $m < 0$, the \textbf{imaginary quadratic fields}.

\begin{defn}
    A complex number is an \textbf{algebraic integer} if and only if it is a root of some monic (leading coefficient $1$) polynomial with coefficients in $\Z$.
\end{defn}

Note we do note require the polynomial to be irreducible over $\Q$.

\begin{thm}
    Let $\alpha$ be an algebraic integer, and let $f$ be a monic polynomial over $\Z$ of least degree having $\alpha$ as a root. Then $f$ is irreducible over $\Q$.
\end{thm}

\begin{lem}
    Let $f$ be a monic polynomial with coefficients in $\Z$, and suppose $f = gh$ where $g$ and $g$ are monic polynomials with coefficients in $\Q$. Then $g$ and $h$ actually have coefficients in $\Z$.
\end{lem}
\begin{proof}
    Let $m$ (resp. $n$) be the smallest positive integer such that $mg$ (resp. $nh$) has coefficients in $\Z$. Then the coefficients of $mg$ have no common factor. The same is true of the coefficients of $nh$. Using this, we can show that $m=n=1$: If $mn > 1$, take any prime $p$ dividing $mn$ and consider the equation $mnf = (mg)(nh)$. Reducing coefficients $\mod p$, we obtain $0 \equiv (mg)(nh)\mod p$. But $\Z_p[x]$ is an integral domain, so either $mg \equiv 0$ or $nh \equiv 0 \mod p$. But then $p$ divides all coefficients of either $mg$ or $nh$; as we showed above, this is impossible. Thus $m = n = 1$, and hence $g,h \in \Z[x]$.
\end{proof}

We now can prove the preceding theorem.
\begin{proof}
    If $f$ is not irreducible, then $f = hg$ where $g,h \in \Q[x]$ are nonconstant polynomials. Without loss of generality we can assume that $g,h$ are monic. Then $g,h \in \Z[x]$ by the lemma. But $\alpha$ is a root of either $g$ or $h$, and both have degree less than that of $f$. This is a contradiction.
\end{proof}

\begin{cor}
    The only algebraic integers in $\Q$ are the ordinary integers.
\end{cor}

\begin{cor}
    Let $m$ be a squarefree integer. The set of algebraic integers in the quadratic field $\Q[\sqrt{m}]$ is $$\{a+b\sqrt{m}:a,b \in \Z\},\;\text{ if }\;m\equiv 2\text{ or }3\mod 4$$
    $$\left\{\frac{a+b\sqrt{m}}{2}:a,b \in \Z,a\equiv b \mod 2\right\},\;\text{ if }m\equiv 1\mod 4$$
\end{cor}
\begin{proof}
    Let $\alpha = r+s\sqrt{m}, r,s \in \Q$. If $s \neq 0$, then the monic irreducible polynomial over $\Q$ having $\alpha$ as a root is $$x^2-2rx+r^2-ms^2$$
    Thus $\alpha$ is an algebraic integer if and only if $2r$ and $r^2-ms^2$ are both integer. This can be used to obtain the result.
\end{proof}

\begin{thm}
    The following are equivalent for $\alpha \in \C$: \begin{enumerate}
        \item[(1)] $\alpha$ is an algebraic integer;
        \item[(2)] $\Z[\alpha]$ is a finitely generated $\Z$-module;
        \item[(3)] There exists a subring $B$ of $\C$ containing $\alpha$ which is finitely generated as a $\Z$-module;
        \item[(4)] $\alpha A \subseteq A$ for some finitely generated $\Z$-submodule of $\C$.
    \end{enumerate}
\end{thm}
\begin{proof}
    (1) implies (2) follows from the fact that $\alpha$ is a root of a monic polynomial over $\Z$ of some degree $n$, so $\Z[\alpha]$ is generated by $1,\alpha,...,\alpha^{n-1}$. (2) implies (3) implies (4) is immediate.

    Suppose (4). Let $a_1,...,a_n$ generate $A$ over $\Z$. Expressing each $\alpha a_i$ as a linear combination of $a_1,...,a_n$ with coefficients in $\Z$ we obtain $\alpha a = Ma$, for $a = (a_1,...,a_n)$, where $M$ is an $n\times n $ matrix over $\Z$. Equivalently, $(\alpha I-M)a = 0$. Since the $a_i$ are not all zero, it follows that $\alpha I-M$ has determinant zero when we multiply on the left of by the adjugate. Expressing this determinant in terms of the $n^2$ coordinates of $\alpha I-M$ we obtain a monic polynomial in $\alpha$ with coefficients in $\Z$. Thus $\alpha$ is a algebraic integer.
\end{proof}

\begin{cor}
    If $\alpha,\beta$ are algebraic integers, then so are $\alpha+\beta,\alpha\beta$.
\end{cor}
\begin{proof}
    We know that $\Z[\alpha]$ and $\Z[\beta]$ are finitely generated $\Z$-modules. Then so is the ring $\Z[\alpha,\beta]$. Finally, $\Z[\alpha,\beta]$ contains $\alpha+\beta$ and $\alpha\beta$. By the previous theorem this implies that they are algebraic integers.
\end{proof}

Hence the set of algebraic integers in $\C$ is a ring, which we denote by $\mathbb{A}$. In particular $\mathbb{A}\cap K$ is the subring of algebraic integers in $K$ for any number field $K$.


\section{The Cyclotomic Fields}
\label{sec:cycloField}

Let $\omega = e^{2\pi i/m}$. 

\begin{thm}
    All $\omega^k, 1\leq k\leq m, \gcd(k,m) = 1$, are conjugates of $\omega$.
\end{thm}
\begin{proof}
    It will be enough to show that for each $\theta = \omega^k$, and for each prime $p$ not dividing $m$, $\theta^p$ is a conjugate of $\theta$. Let $f$ be a monic irreducible polynomial for $\theta$ over $\Q$. Then $x^m-1 = f(x)g(x)$ for some monic $g \in \Q[x]$, and from before we know $f,g \in \Z[x]$. Note $\theta^p$ is a root of $x^m-1$, so $\theta^p$ is a root of $f$ or $g$. Suppose $\theta^p$ is a root of $g$. Then $\theta$ is a root of the polynomial $g(x^p)$. It follows that $g(x^p)$ is divisible by $f(x)$ in $\Q[x]$. Applying the lemma again we obtain that $g(x^p)$ is divisible by $f(x)$ in $\Z[x]$. Reducing coefficients $\mod p$, we obtain $g(x^p)+(p)$ is divisible by $f(x)+(p)$. But $g(x^p)+(p) = (g(x))^p+(p)$, and $\Z_p[x]$ is a UFD; it follows that $f+(p)$ and $g+(p)$ have a common factor $h$ in $\Z_p[x]$. Then $h^2\vert fg+(p) = x^m-1+(p)$. This implies that $h$ divides the derivative of $x^m-1$, which is $mx^{m-1}+(p)$. Since $p$ does not divide $m$, $m+(p)\neq 0+(p)$; then in fact $h(x)$ is just a monomial. But this is impossible since $h\vert x^m-1+(p)$.
\end{proof}

\begin{cor}
    $\Q[\omega]$ has degree $\varphi(m)$ over $\Q$.
\end{cor}
\begin{proof}
    $\omega$ has $\varphi(m)$ conjugates, hence the irreducible polynomial for $\omega$ over $\Q$ has degree $\varphi(m)$.
\end{proof}

\begin{cor}
    The Galois group of $\Q[\omega]$ over $\Q$ is isomorphic to the multiplicative group of integers $\mod m$ $$\Z^*_m = \{k:1\leq k \leq m,\gcd(k,m) = 1\}$$
    For each $k \in \Z^*_m$, the corresponding automorphism in the Galois group sends $\omega$ to $\omega^k$.
\end{cor}
\begin{proof}
    An automorphism of $\Q[\omega]$ is uniquely determined by the image of $\omega$, and by our previous results $\omega$ can be sent to any of the $\omega^k,\gcd(k,m) = 1$. This establishes the one-to-one correspondence between the Galois group and the multiplicative group $\mod m$.
\end{proof}

\begin{cor}
    Let $\omega = e^{2\pi i/m}$. If $m$ is even, the only roots of $1$ in $\Q[\omega]$ are the $m^{th}$ roots of $1$. If $m$ is odd, the only ones are the $2m^{th}$ roots of $1$.
\end{cor}
\begin{proof}
    It is enough to prove the statement for even $m$. Suppose $\theta$ is a primitive $k$th root of unity in $\Q[\omega]$. Then $\Q[\omega]$ contains a primitive $r$th root of unity, where $r$ is the least common multiple of $k$ and $m$. But then $\Q[\omega]$ contains the $r$th cyclotomic field, implying $\varphi(r) \leq \varphi(m)$. This is a contradiction unless $r = m$. Hence $k\vert m$ and $\theta$ is an $m$th root of unity.
\end{proof}

\begin{cor}
    The $m$th cyclotomic fields, for $m$ even, are all distinct and in fact pairwise non-isomorphic.
\end{cor}



\section{Embeddings in $\C$}

Let $K$ be a number field of degree $n$ over $\Q$. Since this is a separable extension there are exactly $n$ embeddings of $K$ into $\C$.

\begin{eg}
    The quadratic field $\Q[\sqrt{m}]$, $m$ squarefree, has two embeddings in $\C$: the identity mapping, and also the one which sends $a+b\sqrt{m}\mapsto a-b\sqrt{m}$. 
\end{eg}

\begin{eg}
    The $m$th cyclotomic field has $\varphi(m)$ embeddings in $\C$, the $\varphi(m)$ automorphisms.
\end{eg}

If $K, L$ are two number fields with $K \subset L$, then we know that every embedding of $K$ in $\C$ extends to exactly $[L:K]$ embeddings of $L$ in $\C$. In particular, $L$ has $[L:K]$ embeddings in $\C$ which leave each point of $K$ fixed. To replace embeddings of a number field $K$ with automorphisms is to extend $K$ to a normal extension $L$ of $\Q$; each embedding of $K$ extends to $[L:K]$ embeddings of $L$, all of which are automorphisms of $L$ since $L$ is normal.

\section{The Trace and the Norm}
\label{sec:TrNorm}

Let $K$ be a number field throughout. We define two functions $T := T_{K/\Q}$ and $N := N_{K/\Q}$ (the \textbf{trace} and the \textbf{norm}) on $K$, as follows: Let $\sigma_1,...,\sigma_n$ denote the embeddings of $K$ in $\C$, where $n = [K:\Q]$. For each $\alpha \in K$, set $$T(\alpha) = \sum_i\sigma_i(\alpha),\; N(\alpha) = \prod_i\sigma_i(\alpha)$$
From the definition we obtain $T(\alpha+\beta) = T(\alpha)+T(\beta)$ and $N(\alpha\beta) = N(\alpha)N(\beta)$ for all $\alpha,\beta \in K$. Moreover, for $r \in \Q$ we have $T(r) = nr, N(r) = r^n$. Also for $r \in \Q$ and $\alpha \in K$, $T(r\alpha) = rT(\alpha)$ and $N(r\alpha) = r^nN(\alpha)$.

Let $\alpha$ have degree $d$ over $\Q$. Let $t(\alpha)$ and $n(\alpha)$ denote the sum and product, respectively, of the $d$ conjugates of $\alpha$ over $\Q$. Then we have 

\begin{thm}
    $T(\alpha) = \frac{n}{d}t(\alpha)$ and $N(\alpha) = (n(\alpha))^{n/d}$ where $n = [K:\Q]$. Note $n/d = [K:\Q(\alpha)]$.
\end{thm}
\begin{proof}
    $t(\alpha)$ and $n(\alpha)$ are the trace and norm $T_{\Q[\alpha]/\Q}$ and $N_{\Q[\alpha]/\Q}$ of $\alpha$. Each embedding of $\Q[\alpha]$ in $\C$ extends to exactly $n/d$ embeddings of $K$ in $\C$. This establishes the formulas.
\end{proof}


\begin{cor}
    $T(\alpha)$ and $N(\alpha)$ are rational.
\end{cor}

If $\alpha$ is an algebraic integer, then its monic irreducible polynomial over $\Q$ has coefficients in $\Z$; hence we obtain
\begin{cor}
    If $\alpha$ is an algebraic integer, then $T(\alpha)$ and $N(\alpha)$ are integers.
\end{cor}

\begin{eg}
    For the quadratic field $K = \Q[\sqrt{m}]$, we have $$T(a+b\sqrt{m}) = 2a$$
    and $$N(a+b\sqrt{m}) = a^2-mb^2$$
    for $a,b \in \Q$.
\end{eg}





%
% \begin{acknowledgement}
% If you want to include acknowledgments of assistance and the like at the end of an individual chapter please use the \verb|acknowledgement| environment -- it will automatically render Springer's preferred layout.
% \end{acknowledgement}
%
% \section*{Appendix}
% \addcontentsline{toc}{section}{Appendix}
%


% Problems or Exercises should be sorted chapterwise
\section*{Problems}
\addcontentsline{toc}{section}{Problems}
%
% Use the following environment.
% Don't forget to label each problem;
% the label is needed for the solutions' environment
\begin{prob}
\label{prob1}
A given problem or Excercise is described here. The
problem is described here. The problem is described here.
\end{prob}

% \begin{prob}
% \label{prob2}
% \textbf{Problem Heading}\\
% (a) The first part of the problem is described here.\\
% (b) The second part of the problem is described here.
% \end{prob}

%%%%%%%%%%%%%%%%%%%%%%%% referenc.tex %%%%%%%%%%%%%%%%%%%%%%%%%%%%%%
% sample references
% %
% Use this file as a template for your own input.
%
%%%%%%%%%%%%%%%%%%%%%%%% Springer-Verlag %%%%%%%%%%%%%%%%%%%%%%%%%%
%
% BibTeX users please use
% \bibliographystyle{}
% \bibliography{}
%


% \begin{thebibliography}{99.}%
% and use \bibitem to create references.
%
% Use the following syntax and markup for your references if 
% the subject of your book is from the field 
% "Mathematics, Physics, Statistics, Computer Science"
%
% Contribution 
% \bibitem{science-contrib} Broy, M.: Software engineering --- from auxiliary to key technologies. In: Broy, M., Dener, E. (eds.) Software Pioneers, pp. 10-13. Springer, Heidelberg (2002)
% %
% Online Document

% \end{thebibliography}

