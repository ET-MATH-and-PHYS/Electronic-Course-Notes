%%%%%%%%%%%%%%%%%%%%% chapter.tex %%%%%%%%%%%%%%%%%%%%%%%%%%%%%%%%%
%
% sample chapter
%
% Use this file as a template for your own input.
%
%%%%%%%%%%%%%%%%%%%%%%$% Springer-Verlag %%%%%%%%%%%%%%%%%%%%%%%%%%
%\motto{Use the template \emph{chapter.tex} to style the various elements of your chapter content.}
\chapter{Characterizing Roman Artifacts to Investigate Gendered Practices in Contexts Without Sexed Bodies}
\label{GendArt} % Always give a unique label
% use \chaptermark{}
% to alter or adjust the chapter heading in the running head


%%% Questions to think about
%The \textbf{Thesis} or general sense of the article is ...

%The \textbf{method} the author uses to argue their point is ...

%In their \textbf{analysis} the author uses tools such as ...
% How do they look at the evidence? Do they place it in some theoretical framework? (i.e. gender studies, music studies, etc.)

%Additionally they conclude ...
% How does this compare to others throughout time? What is the societal context?

%What connections does the author portray with regard to \textbf{space}, \textbf{relationships}, \textbf{occupation}, and \textbf{religion}.


\abstract{}

\section{Questions and Remarks}
\label{sec:QR14}





\section{First Reading}
\label{sec:FirRead14}



\subsection{Introduction}

Gender as a sociocultural construct with ``constantly negiotiated relationships" constituted in historically specific ways, is not inherent in archaeological data. In the geographically and chronologically diverse Roman world, where social status and ethnicity often played more significant roles in social hierarchies and socioeconomic practices than did biological sex, gender as a defining characteristic of identity and practice is problematic. (p.103)


\begin{rmk}
    While the existence of changing and differing gender identities across the Roman world is for certain, any apparent consistencies of gendered practices in artifact use across that world have important ramifications for understanding how sociocultural practices spread.
\end{rmk}


\subsection{Approaches to Gender in Roman Archaeology}

Feminist and gender studies in Roman classical archaeology in the 1990s focused on elite women. While approaches to gender across Roman archaeology are converging, they are still reliant on the sexed bodies as represented in the sources.


\subsection{Gendered Approaches to Roman Artifacts and Lived Spaces}

A major concern for gender archaeology has been the assumed maleness of many Roman archaeological remains. 

\begin{nte}
    The main material sources used by feminist archaeologists to develop insights into the hidden voices across the Roman world are representational, epigraphical, and funerary.
\end{nte}

Dress-related artifacts have been used to identify gendered practices in some lived contexts that lack sexed bodies. Van Driel-Murray used the size ranges of leather shoes found in Early Impertial military bases to argue for the presence of women and children inside soldiers' barracks.

\begin{rmk}
    The author argues that interpretative links can be found between artifacts and gender in contexts with sexed bodies and that such artifact types can be systematically analyzed, characterized, and used critically as tools for investigating gendered identities and practices within archaeological contexts that lack such bodies. (p.107)
\end{rmk}


\subsection{Gendered Characterizations of Artifact Types and Gendered Space}

The presence of specific brooch types in military contexts has traditionally been used to argue that these were types worn by Roman soldiers. However, such an argument gives precedence to preconceived assumptions about who occupied these military bases over specific evidence for how different types of brooches would have been worn by different status and gender groups.

\begin{rmk}
    Brooches were part of both male and female dress in much of pre-Roman Europe and were adopted and adapted during the Roman period.
\end{rmk}


By the Augustan period, a distinction had developed such that some types of brooches and ways of wearing them were indicative of status and sex.

The \textbf{Distelfibel} or thistle-shaped brooch is a massive, heavy brooch with a ribbed semicircular bow that had a large shield decorated with curved and incised pressed sheet metal. These represented less than $5\%$ of brooches found inside military fortifications, while in oppida (i.e., local settlements) double that percentage was found. Gechter argues that this distribution implies that it was a civilian, and quite possibly a distinctively female, fastener.

While there are exceptions, there is therefore strong evidence for Distelfibeln as female attributes. This brooch type may also have been an age and status attribute.


Artifacts associated with personal hygiene, health, and beauty often served as female attributes, especially of elite women. Care of the body and bodily adornment were seen to ``soften Roman citizens," and toilet activities served to ``display the adorned female body."

\begin{nte}
    The symbolic association of toilet activities with female beauty does not necessarily represent actual practice. Many toilet items found in excavations, such as spatulas, probes, and tweezers, could equally have been medical implements and so cannot be easily gendered.
\end{nte}

The small ceramic and glass bottle, which is widely considered to have been used to container for cosmetics and perfumed oils, is one type of artifact that seems more specifically associated with women's toilet activities. These are often represented in Roman art as parts of cosmetic sets, but are also found in associations with medical equipment, concurring with the lack of distinction between cosmetics and medical remedies.


\begin{rmk}
    Despite attention to male grooming being considered a vice in Roman society, in ROme elite men used perfumed oils after the bath, and perfumed oils could be used for anointing military regalia and statues of deities.
\end{rmk}

There is considerable written, representational, and burial evidence that cloth working was predominantly a female task in the Roman world, specifically with regard to spinning cloth. However, there is not as much evidence for the gendering of needlework, and indeed in imperial household during the Early Empire, male \textbf{vestifici} and \textbf{sarcinatores} (cloth menders) were recorded.


In burial contexts we see hints that needles were female attributes but were not strongly gendered, at least symbolically. 

\subsection{Concluding Comments}

This article demonstrates that the investigation of artifact assemblages is important for better understandings of gendered sociospatial practices. The author argues that the consideration of different levels of gendered characterization for particular artifact types constitutes a useful interpretative tool for investigating how gender was played out in lived spaces in the Roman world.

The evident patterns and habitual practice in this material and its contexts are important here, rather than how individual items might be sexed anecdotally. 

\begin{rmk}
    This article aims to present approaches, analytical tools, and some case studies that can help increase ``conversations between social and material traces of the past."
\end{rmk}





\section{Notes on Analysis and Societal Context}
\label{sec:SocCont14}


Author attempts to characterize Roman artifacts so that remains from lived spaces can be used to greater effect for insights into the presence, roles, and identities of women within these spaces.

\section{Terms}
\label{sec:terms14}

\begin{enumerate}
	\item
\end{enumerate}


%
% \begin{acknowledgement}
% If you want to include acknowledgments of assistance and the like at the end of an individual chapter please use the \verb|acknowledgement| environment -- it will automatically render Springer's preferred layout.
% \end{acknowledgement}
%
% \section*{Appendix}
% \addcontentsline{toc}{section}{Appendix}
%


% Problems or Exercises should be sorted chapterwise
\section*{Problems}
\addcontentsline{toc}{section}{Problems}
%
% Use the following environment.
% Don't forget to label each problem;
% the label is needed for the solutions' environment
\begin{prob}
\label{prob1}
A given problem or Excercise is described here. The
problem is described here. The problem is described here.
\end{prob}

% \begin{prob}
% \label{prob2}
% \textbf{Problem Heading}\\
% (a) The first part of the problem is described here.\\
% (b) The second part of the problem is described here.
% \end{prob}

%%%%%%%%%%%%%%%%%%%%%%%% referenc.tex %%%%%%%%%%%%%%%%%%%%%%%%%%%%%%
% sample references
% %
% Use this file as a template for your own input.
%
%%%%%%%%%%%%%%%%%%%%%%%% Springer-Verlag %%%%%%%%%%%%%%%%%%%%%%%%%%
%
% BibTeX users please use
% \bibliographystyle{}
% \bibliography{}
%


% \begin{thebibliography}{99.}%
% and use \bibitem to create references.
%
% Use the following syntax and markup for your references if 
% the subject of your book is from the field 
% "Mathematics, Physics, Statistics, Computer Science"
%
% Contribution 
% \bibitem{science-contrib} Broy, M.: Software engineering --- from auxiliary to key technologies. In: Broy, M., Dener, E. (eds.) Software Pioneers, pp. 10-13. Springer, Heidelberg (2002)
% %
% Online Document

% \end{thebibliography}

