%%%%%%%%%%%%%%%%%%%%% chapter.tex %%%%%%%%%%%%%%%%%%%%%%%%%%%%%%%%%
%
% sample chapter
%
% Use this file as a template for your own input.
%
%%%%%%%%%%%%%%%%%%%%%%$% Springer-Verlag %%%%%%%%%%%%%%%%%%%%%%%%%%
%\motto{Use the template \emph{chapter.tex} to style the various elements of your chapter content.}
\chapter{Masculinity, Appearance, and Sexuality: Dandies in Roman Antiquity}
\label{mascApp} % Always give a unique label
% use \chaptermark{}
% to alter or adjust the chapter heading in the running head


%%% Questions to think about
%The \textbf{Thesis} or general sense of the article is ...

%The \textbf{method} the author uses to argue their point is ...

%In their \textbf{analysis} the author uses tools such as ...
% How do they look at the evidence? Do they place it in some theoretical framework? (i.e. gender studies, music studies, etc.)

%Additionally they conclude ...
% How does this compare to others throughout time? What is the societal context?

%What connections does the author portray with regard to \textbf{space}, \textbf{relationships}, \textbf{occupation}, and \textbf{religion}.


\abstract{}

\section{Questions and Remarks}
\label{sec:QR12}

\begin{qst}
    Who was \textbf{Hermaphroditus}? How did they cross between the realms of woman and man?
\end{qst}


\begin{qst}
    How does dress place Hermaphroditus in a \textbf{liminal/transitional} status?
\end{qst}

\begin{qst}
    How do dress and the body figure into representations of Hermaphroditus?
\end{qst}


\begin{qst}
    How could Olson's work be used to enhance or counter our understanding of Hermaphroditus?
\end{qst}


\begin{qst}
    How can Olson be used to unpack Hermaphroditus?
\end{qst}


\begin{qst}
    What can Roman male clothing tell us about cross-dressing and transgender identity?
\end{qst}

\begin{qst}
    What are \textbf{Ancient Dandies}?
\end{qst}






\section{First Reading}
\label{sec:FirRead12}


In the early second century CE the biographer \textbf{Suetonius} wrote of Julius Caesar that 
\begin{quotation}
    he was somewhat overnice in the care of his person, being not only carefully trimmed and shaved, but even supposedly having superfluous hair plucked out. ... They say too that he was remarkable in his dress, that he wore the broad-striped tunic, with fringed sleeves reaching to the wrist, and always had a belt overtop [\textbf{super eum cingeretur}], though rather a loose one [\textbf{quidem fluxiore cinctura}], and this, they say, was the occasion of Sulla's \textbf{mot}, when he often warned optimates to beware the ill-girt boy [\textbf{ut male praecinctum puerum cauerent}]
\end{quotation}

\begin{nte}
    \textbf{Plutarch} was a Greek biographer who lived and wrote in the late first century CE.
\end{nte}

Plutarch talked of Caesar's soft white skin and related how he used to scratch his head with one finger and daintily arrange his hair. 

\begin{rmk}
    The elder Curio in one of his speeches famously referred to Caesar as ``every woman's man and every man's woman"; other authors note his reported sexual liaison with Nicomedes, king of Bithynia. (p.182)
\end{rmk}

Caesar has a reputation as an \textbf{androgyne}, \textbf{catamite}, and wearer of effeminate clothing. \textbf{Anthony Corbeill} claims Caesar's choice of dress was ``political self-advertisement," because ``the popular politicians [\textbf{populares}] became aligned with feminine traits" since the tradionalist politicians (optimates) adopted masculine-coded walk and dress. (p.183) Thus Caesar's effeminacy was part of a political identity, and by transgressing normal male aristocratic behaviour, he ``fashioned himself as a proponent of political change." (p.183) - \textbf{emphasis on political change and liminal position between genders}


\begin{qst}
    Was Caesar's effeminacy part of a political identity, as Corbeill claims, and was he transgressing normal male aristocratic behaviour?
\end{qst}

The author's contention is no, in both instances.


\begin{rmk}
    The Romans operated on a system of \textbf{gender identity} rather than one of \textbf{sexual orientation}. Instead of categorizing their sexual world into identities based on the preferred gender of someone's partner, as we do, Roman sexual ideology seems to have divided the world up into ``penetrators" and ``those penetrated" (\textbf{Can connect to Kamen and Richardson}) (p.184)
\end{rmk}

The penetrator was an adult male of citizen status who by his active sexual role also configured himself as dominant and masculine. It mattered little whom he was penetrating, or which orifice, as long as he took the active role. The penetrated partner was characterized as womanish, servile, and emasculated---a role well suited to slaves, prostitutes, and women but problematic if filled by another adult citizen male.

\begin{rmk}
    The Romans liked to see social and sexual roles collapsed. Adult men who enjoyed being penetrated or giving fellatio or cunnilingus were mercilessly lampooned and censured in the dominant discourse. What bothered Roman writers in male homoerotic relations was an assimilation to the female role. (p.184)
\end{rmk}

\begin{defn}
    \textbf{Effeminatus} (effeminate) and \textbf{mollis} (soft) refer to a man who did not embody traditional masculine looks.
\end{defn}

\begin{defn}
    \textbf{Pathicus} was a ``blunt term" referring pejoratively to a man who had been or who continued to be anally penetrated.
\end{defn}

\begin{defn}
    \textbf{Delicatus} and \textbf{deliciae} often allude to slave boys kept for visual and sexual pleasure.
\end{defn}

\begin{defn}
    The \textbf{cinaedus} was a man who wore loos colorful clothing, perfume, and curled hair, who walked along with a mincing gait, and who was apt to be anally penetrated and enjoy it.
\end{defn}

Scholars have suggested that what made a cinaedus was his general lack of self-control and the abrogation of sartorial masculinity, both forms of gender deviance, rather than any specific sexual practice or preference.

\begin{nte}
    We have no first-person statements from a cinaedus in Roman antiquity; these are always words the Roman authors use to hurl at another person. The voices of the passive, as the Romans would have called them, are absent from our sources (p.186)
\end{nte}


\subsection{Roman Male Clothing}

The normal appearance for an elite Roman man was \textbf{staid, even plain}. Roman citizen men fulfilled normal social and sartorial expectations if they wore the tunic (\textbf{tunica}), the simple short-sleeved or sleeveless garment worn by men of all ages and ranks, the basic male garment for both public and private wear. It was normally girded with a cord at the waist, and the tunic material bloused over the cord so that the tunic fell to knee-length. Over it the Roman citizen would wear a \textbf{toga}; but these garments were ideally made of unbleached, undyed woolen cloth.


\begin{nte}
    Roman antiquity was more or less a sartorially static society, and there were few real clothing changes over the centuries that comprised the bulk of its history.
\end{nte}

\begin{quotation}
    aristocratic maleness was to be expressed by independence from the servitude of fashion (p.187)
\end{quotation}

Roman ethicists saw aesthetics and morality as being inextricably linked, which meant that deviation from the male vestimentary code at Rome could bring social censure. (p.187) An ideal Roman masculinity did not equal deliberate untidiness---a certain amount, a degree, of refinement was in order.


\subsection{The Signs of Effeminacy}

The Roman statesman \textbf{Cicero} in the first century BCE depicted his effeminate political enemy \textbf{Aulus Gabinius} as having cheeks ``bright with rouge"; the first-century CE novelist Petronius (also a politician at the court of the emperor Nero) described a male slave-prostitute who wore makeup and had nicely combed hair (\textbf{quo enim spectant flexae pectine comae, quo facies medicamine attrita}). \textbf{Quintilian}, a ROman advocate and famous authority on rhetoric who lived in the late first century CE, condemned the use of womanish cosmetics on men indirectly in his censure of overly embellished oratory.

\begin{rmk}
    Cosmetics were bound up with constructions of sexuality and power in Roman antiquity: persons who used cosmetics---boy slaves used for sexual pleasure, women, male whores---were located outside traditional legal power structures.
\end{rmk}

Long hair, or long curly hair was a sign of desirability and sexual availability and is mentioned most often in reference to \textbf{delicati}. Such slaves were termed \textbf{capillati}. The effeminate statesman \textbf{Maecenas}'s hair was described by \textbf{Suetonius} as ``ringlets ... dripping with perfume".

\textbf{Depilation} described a man who removed the hair from his legs, chest, buttocks, even genitals by means of plucking, pitch, or other depilatory and was said to be hairless or ``smooth" (\textbf{levis, glaber, expolitus}). \textbf{Maud Gleason} observed that a man's natural hair was thought to be the product of the same abundance of inner heat that concocted his sperm (\textbf{connection to Jones and Flemming}) (p.189)

\begin{quotation}
    those who depilated themselves were rightly suspected of undermining the symbolic language in which male privilege was written.
\end{quotation}

Long hair and depilation were practices held to be womanish, but also with status dissonance, connected with the confusion of gender boundaries.

\begin{rmk}
    Much of the ancient vitriol against male use of perfume comes from \textbf{Cicero}: effeminate men reeked of unguents or had cheeks ``moist with unguent"; the Roman rebel Catiline's insurgents glistened in perfumed oils. To use a certain type of perfume called \textbf{opobalsamum} could also be a sign of effeminacy, as well as the expensive concoctions normally used by women.
\end{rmk}

Wearing more than one ring was a conventional sign of \textbf{mollitia}. Quintilian recommended that the hand of an orator should not be loaded with rings and that the orator should especially eschew any that did not fit over the middle finger joint; presumably, wearing rings in this fashion meant one was able to display more of them.

The ancients thought that it was by his clothing that a man most clearly indicated his sexual proclivities (p.190). The sartorial blurring of gender boundaries was ridiculed and censured by many. Quintilian observed that 
\begin{quotation}
    a tasteful and magnificent dress [\textbf{cultus}] ... lends added dignity to its wearer; but effeminate and luxurious apparel fails to adorn the body and merely reveals the foulness of the mind.
\end{quotation}

The emperor \textbf{Tiberius} (13-37 CE) legislated against the wearing of silk by men, citing the fabric's inherent effeminacy as his reason. Quintilian believed that ``womanish attire" (\textbf{vestem muliebrem}) was ``an indication of an effeminate and unmanly character" (\textbf{mollis et parum viri signa}).

Clothing color could also be an indication of a lack of masculinity: only somber hues were worn by ``real" men.

Tunics with long sleeves were also considered effeminate. A tunic that was girded too short was also cause for censure. A short tunic could also indicate that one was a manual laborer: artisans and workers wore tunics that fell above the knees. 

For a nobleman to wear a tunic loosely girt, as Caesar did, or one without a belt altogether also indicated an effeminate nature.\textbf{Margaret Graver} notes that the state of ``unbeltedness" was near to being the exact opposite of ideal masculinity in antiquity. ``As a point of dress, the absence of cincture indicates defiance of convention and also unreadiness for action, in the particular, the inability to wear a weapon."

\begin{quotation}
    For if someone, drenched daily in perfumes, adorns himself before a mirror, shaves his eyebrows, walks about with his beard plucked and thigh hairs pulled out, who, as a young boy with his lover, wearing a long-sleeved tunic [\textbf{chiridota tunica}], was accustomed to lie in the low spot at banquets, who is not only fond of wine, but fond of men also, then would anyone doubt that he has done the same thing that \textbf{cinaedi} usually do [\textbf{quod cinaedi facere solent}]? - Second-century BCE general and politician Publius Scipio Africanus
\end{quotation}


\subsection{Youth, Urbanity, Heterosexual Activity}

Effeminacy was often closely linked to passive homoerotic activity in Roman antiquity. \textbf{Catherine Edwards} notes how scholars ``used references to men behaving in an effeminate manner to determine, firstly, how widespread homosexual practices were in ancient Rome." Although some descriptions of effeminacy did connect this sort of self-presentation with cinaedi or pathici, as we have seen, there is also an intriguing cluster of references that associated ``effeminate" visual images with youth, urbanity, and heterosexual activity.

\textbf{Gellius} suggested in the second century CE that Publius Africanus, who was ``habituated to this ancestral fashion" (\textbf{hac antiquitate indutus}) of wearing a short-sleeved tunic, reproved Sulpicius Gallus for wearing a long-sleeved one, implying perhaps that it was trendily youthful. Diodorus Siculus, a Greek from Sicily who wrote a universal history in teh first century BCE, reported that after 146 BCE, the ``younger generation" (\textbf{hoi neoi}) wore garments in the Roman marketplace that were soft and delicate and resembled women's garments.

Quintilian was of the opinion that ``purple and deep red garments do not suit old men; in the young, however, we can endure a rich and even perhaps a risky style."  Despite teh range of authors and genres, we may generally note some tension between the generations as to what was acceptable in the way of clothing.


\begin{rmk}
    Attention paid to personal appearance was often related to male urbanity and sophistication.
\end{rmk}

Plutarch reported that fashionable shades in men's clothing changed; thus, ``when [the stern politician Cato the Younger] saw a purple which was excessively red and much in vogue, he himself would wear a dark shade," implying that the refined man kept up with changes in clothing colors.

\begin{rmk}
    Ken Gelder suggests that one useful way of understanding dandyism in history is as an ``anachronism that refuses to go away, as a mode of fashionability that survives against the odds and makes a point of being out of step with its context" (note 92)
\end{rmk}

The younger Seneca wrote that when luxury spreads, men first begin by paying more attention to their personal appearance. The elder Pliny held that ``unguents are among the most elegant and also most honorable pleasures in life" but railed against men wearing gold bracelets called \textbf{dardania}, apparently the fashion. The sophisticated urbanite thus paid some attention to his appearance. Williams says that ``perfume and depilation in themselves were not necessarily markers of excessive effeminacy."

\begin{nte}
    Many of the conventional characteristics of effeminacy were attached to men who were said to be trying to attract women.
\end{nte}

The elegiac poet TIbullus who wrote in the alte first century BCE, stated that richly dressed men were attractive to women: ``whoever dresses his hair with art and whose voluminous toga falls with a rommy fold." \textbf{Ovid}, a love poet who wrote in the late first century BCE, also suggested the connection between fashionable male appearance and adultery with the term \textbf{cultus adulter}, ``well-groomed adulterer."

\begin{defn}
    Martial's epigrams contain the fullest description of a \textbf{bellus homo}: ``A beautiful man curls his hair and arranges it carefully, always smells of balsam or cinnamon, hums tunes from the Nile and from Cadiz, moves his plucked arms in time with changing measures, lounges all day among ladies' chairs and is forever murmuring into some ear; reads billets sent from this quarter and that, and writes them, and shrinks from the cloak on a neighbor's elbow."
\end{defn}

\textbf{Arrian}, a Greek writer and lecturer in the second century CE, claimed that a smooth man depilated himself to be attractive to women.

The sophist and physiognomist Polemon, writing in the second century CE, claimed that men sometimes assume items of personal adornment and clothing in order ``to please other men and women." The author offers the alternative label of \textbf{dandy}.

\textbf{Edwards} states that in the ancient sources the same men were often accused of effeminacy and adultery (men like Marc Antony, friend to Julius Caesar) or effeminacy and uxoriousness (men like Maecenas, friend to Caesar Agustus). An effeminate appearance could be a mode of self-presentation associated with youth, urban  sophistication, and heterosexual activity. 

\begin{rmk}
    In Lucian's dialogues from the second century CE, \textbf{Eros} bluntly tells Zeus that if he wants reciprocal enamorment, he must adopt an effeminate manner; his tough-guy looks are unattractive to women. Women's erotic desires were apparently directed at the softer sort of male.
\end{rmk}

\textbf{Graver} has noted that ``effeminate behaviour is not so much behaviour of one kind or orientation as it is behaviour which calls attention to itself. Masculinity appears in purely negative terms." (p.199) Seneca the younger complained of such men who would ``even put up with censure, provided that they can advertise themselves."

\begin{qst}
    Were such men necessarily \textbf{pathici}, that is, passive in their sexual relations with men or with women? Why would a Roman man sport the kinds of effeminate characeristics that might lead to social censure and the label of \textbf{pathicus}?
\end{qst}

Such an effete self-presentation indicated pretensions to upper-class status, at least in some circles. Perfume, jewelry, and expensive, excess, or colourful fabrics indicated status and rank rather than merely effeminacy, as the items were expensive. In another epigram, an effeminate man is perhaps an \textbf{equestrian}: status symbols mentioned here include a garment (\textbf{trabea}) fastened with a brooch (\textbf{fibula}), and he points with a ``smoothed...hand" (\textbf{pumicata...manu}). 

\begin{nte}
    There clearly existed both a correlation and a confusion between the signs of wealth and status and signs of ``effeminacy" (p.200)
\end{nte}

\subsection{Effeminacy, Dandies, and Class in Premorder Societies}

\textbf{Gleason} has noted that the ancients thought excessive sexual indulgence with either sex was thought to cool the body down too much; since proper masculine warmth could not be maintained, effeminacy was the result. (\textbf{link to Jones and Flemming}). \textbf{Alan Sinfield} writes that mostly effeminacy ``meant being emotional and spending too much time with women. Often it involved excessive cross-sexual [that is, heterosexual] attachment."

At even the time of the late nineteenth century and the Wilde trials in England, effeminacy was still flexible, ``with the potential to refute homosexuality as well as to imply it." Oscar Wilde's ``effeminate" manner and interests had excited comment and hostility, but they ``had not led either his friends or strangers to regard him as obviously, even probably queer" but rather as an aristocrat and aesthete. The Wilde trials are said to have produced a major shift in perception of the signs of same-sex passion.

\begin{rmk}
    While effeminacy figured upper-class uselessness and debauchery, and ``dandy" was also often used to describe aristocratic foppishness, it simultaneously embodied aspirations toward refinement, sensitivity, and taste.
\end{rmk}

Effeminacy was thus coded as a signifier of class or of excessive heterosexual activity, rather than exclusively of homoerotic dissidence.


\subsection{Ancient Dandies}

\begin{defn}
    A \textbf{dandy} or what we might call today a \textbf{metrosexual} is an urban young man of fashion whose sexuality may be ambiguous, whose defining features include a love of self-display, and who seeks to create social spectacle through his appearance.
\end{defn}

\begin{qst}
    Can we locate such men in Roman antiquity?
\end{qst}

Derogatory terms for such a man did exist in Latin: \textbf{trossulus} and \textbf{comptulus}. Varro, an antiquarian, satirist, and essayist who lived and wrote in the first century BCE, associated trossuli with personal adornment and paying outrageous sums for hroses. The younger Seneca described the trossulus as ``rather refined and often surrounded by friends".

\begin{rmk}
    Use of either the culter or the gladius would have bee nshameful pastimes for a noble youth, the culter, used for trimming the hair and nails, because of its associations with female adornment and attention to appearance, and the gladius because it was the weapon of the gladiator. 
\end{rmk}

That the dandy or fashionable man was connected in some way with the aristocratic lifestyle in antiquity is perhaps evident from the association of the trossulus with intellectualism: poetry recitations and philosophy lectures thrilled them, and the younger Seneca described them as being complete from the book-storage box. 


\section{Notes on Analysis and Societal Context}
\label{sec:SocCont12}

The author examines the nexus of effeminacy and masculinity in Roman antiquity by first setting out the conventional signs of effeminacy and its implied connection to sexual passivity in men, and they then go on to detail the instances in which an appearance conventionally held to be effeminate was also linked with youth, urbanity, and even heterosexual activity: indeed, many of the conventional characteristics of effeminacy were attached to men who are said to be trying to attract women.

The author also argues for the existence of a male figure on Rome's urban scene seldom acknowledged by scholars: the \textbf{dandy}, or \textbf{urban young man of fashion}. (p.183)



Effeminate male appearance could indicate dandyism rather than pathic sexuality, although certain Roman authors equated or confused urbanity and dandyism with sexual passivity. Dandyism and effeminacy may also have indicated membership in or aspirations toward the upper class.

To the Roman moralists there was no difference between dandies and cinaedi because in Roman society, affect served as a visualization of morality; indeed, even from a modern standpoint it is hard to distinguish them. (p.204)

We can note that there were derogatory overtones even in the words for ``dandy"; thus, may equites, according to the elder Pliny, were ashamed of being called trossuli. There was more than a whiff of sexual ambiguity about the dandy even in Roman antiquity.

The author concludes that when Caesar wore an effeminate-style unbelted tunic in the first century BCE, daintily arranged his hair, and depilated himself, comparative evidence suggests that it might have been not a political or even a sexual announcement but mere dandyism, which itself was in part a visual declaration of his patrician status. In Roman antiquity, class, dandyism, and effeminacy were all linked in a nexus of ideas about masculinity, status, and sexuality




\section{Terms}
\label{sec:terms12}

\begin{enumerate}
	\item
\end{enumerate}

%
% \begin{acknowledgement}
% If you want to include acknowledgments of assistance and the like at the end of an individual chapter please use the \verb|acknowledgement| environment -- it will automatically render Springer's preferred layout.
% \end{acknowledgement}
%
% \section*{Appendix}
% \addcontentsline{toc}{section}{Appendix}
%


% Problems or Exercises should be sorted chapterwise
\section*{Problems}
\addcontentsline{toc}{section}{Problems}
%
% Use the following environment.
% Don't forget to label each problem;
% the label is needed for the solutions' environment
\begin{prob}
\label{prob1}
A given problem or Excercise is described here. The
problem is described here. The problem is described here.
\end{prob}

% \begin{prob}
% \label{prob2}
% \textbf{Problem Heading}\\
% (a) The first part of the problem is described here.\\
% (b) The second part of the problem is described here.
% \end{prob}

%%%%%%%%%%%%%%%%%%%%%%%% referenc.tex %%%%%%%%%%%%%%%%%%%%%%%%%%%%%%
% sample references
% %
% Use this file as a template for your own input.
%
%%%%%%%%%%%%%%%%%%%%%%%% Springer-Verlag %%%%%%%%%%%%%%%%%%%%%%%%%%
%
% BibTeX users please use
% \bibliographystyle{}
% \bibliography{}
%


% \begin{thebibliography}{99.}%
% and use \bibitem to create references.
%
% Use the following syntax and markup for your references if 
% the subject of your book is from the field 
% "Mathematics, Physics, Statistics, Computer Science"
%
% Contribution 
% \bibitem{science-contrib} Broy, M.: Software engineering --- from auxiliary to key technologies. In: Broy, M., Dener, E. (eds.) Software Pioneers, pp. 10-13. Springer, Heidelberg (2002)
% %
% Online Document

% \end{thebibliography}

