%%%%%%%%%%%%%%%%%%%%% chapter.tex %%%%%%%%%%%%%%%%%%%%%%%%%%%%%%%%%
%
% sample chapter
%
% Use this file as a template for your own input.
%
%%%%%%%%%%%%%%%%%%%%%%$% Springer-Verlag %%%%%%%%%%%%%%%%%%%%%%%%%%
%\motto{Use the template \emph{chapter.tex} to style the various elements of your chapter content.}
\chapter{Performing Blurred Gender Lines: Revisiting Omphale and Hercules in Pompeian Dionysian Theatre Gardens}
\label{BlurGender} % Always give a unique label
% use \chaptermark{}
% to alter or adjust the chapter heading in the running head


%%% Questions to think about
%The \textbf{Thesis} or general sense of the article is ...

%The \textbf{method} the author uses to argue their point is ...

%In their \textbf{analysis} the author uses tools such as ...
% How do they look at the evidence? Do they place it in some theoretical framework? (i.e. gender studies, music studies, etc.)

%Additionally they conclude ...
% How does this compare to others throughout time? What is the societal context?

%What connections does the author portray with regard to \textbf{space}, \textbf{relationships}, \textbf{occupation}, and \textbf{religion}.


\abstract{}

\section{Questions and Remarks}
\label{sec:QR15}

\begin{qst}
    What are \textbf{Dionysian Theatre Gardens}?
\end{qst}


\begin{qst}
    Who was \textbf{Omphale} and what was their relation to Hercules?
\end{qst}


\begin{qst}
    What was Ovid's \textbf{Fasti}?
\end{qst}

\begin{qst}
    What was the \textbf{Julio-Claudian Period}?
\end{qst}



\section{First Reading}
\label{sec:FirRead15}


\subsection{Constructing the Dionysian Theatre Garden}

\begin{defn}
    The Pompeian \textbf{domus} was a type of upper class house.
\end{defn}

\begin{rmk}
    Dionysus was worshipped as the embodiment of transformation through wine, revelry and performance (p. 143). These personae appear in visual representations of many Pompeian homes, as wall paintings and sculptures in or near garden space where dining took place.
\end{rmk}

\begin{nte}
    Representations of Dionysus included himself, satyrs, maenads, and theatrical masks, as well as subsidiary characters/
\end{nte}

One explanation of these representations in garden settings maintains that the garden evokes or represents a \textbf{bucolic} setting for cultic or ritual performances that were synonymous with the god Dionysus. The second explanation connects garden space to theatre space in order to convey a means of self-representation for the patron of the home.

\begin{defn}
    The author has found four distinctive features that constitute the \textbf{Dionysian Theatre Garden}: \begin{itemize}
        \item[(1)] dining areas (\textbf{triclinia}, \textbf{oeci}, \textbf{cenationes}, \textbf{cenacula})
        \item[(2)] stages/raised platforms
        \item[(3)] sculptures, frescoes, mosaics and other artifacts related to Dionysian myth
        \item[(4)] plantings or representations of plantings related to Dionysus.
    \end{itemize}
\end{defn}

It has been established that dining rooms and gardens proper were used for performance within the \textbf{domus}. The presence of Dionysus (as the multifaceted god of theatre and wine) in visual representations in and around the dining areas of select Roman homes could also do more than act as a subtle reference to the theatre.


\subsection{Situating Omphale and Hercules within the Pompeian Dionysian Theatre Garden}

We focus on a sample of Pompeian artistic representations of Omphale and Hercules: three fresco painting and one full-length sculpture in the round, all dating roughly to the Julio-Claudian period (the first half of the first centery CE).

\begin{rmk}
    The \textbf{House of M. Lucretius} displayed a fresco of the cross-dressed pair. The pair stands in the presence of Dionysian characters such as erotes, maenads and Dionysus himself.
\end{rmk}


\begin{rmk}
    The \textbf{House of the Golden Cupids} displays a marble statue depicting a standing figure of Omphale, represented in a similar fashion to the representation found on the previous fresco.
\end{rmk}


\begin{rmk}
    From the \textbf{House of the Prince of Montenegro}, Omphale and Hercules appear convivially in a grove setting.
\end{rmk}

\begin{rmk}
    From the \textbf{House of Siricus} we see a variant of the representation found in the previous house. Set within a grove, female attendants accompany Omphale, seated on the viewer's upper left.
\end{rmk}

These four examples appear in houses that feature gardens, dining areas and Dionysian-themed artifacts. Situating the representations of Omphale and Hercules near or within these spaces is highly suggestive of the pair's performative roles as seen in the \textbf{Fasti}.

\begin{nte}
    In these depictions Omphale's sartorial features inlcude: Hercules' Nemean lion headdress covering her head, or lying near her side; Hercules' club appearing in her hand or near her side; an ankle-length tunic and a mantle draped around her body. Hercules, on the other hand, can appear standing or reclining, with sartorial features including a mantle draping his otherwise unclothed body, or Omphale's tunic.
\end{nte}

Ovid references the pair dining in a garden/grove in the \textbf{Fasti}.

\begin{rmk}
    \textbf{Priapus} is the god of fruit plants and guardian of gardens, who is identified sartorially by his tunic laden with fruit. His visual reference provides an allusion to physical gardens/groves, as well as conflates references to other comic rape narratives involving this deity in Ovid's Fasti.
\end{rmk}

The statue of Omphale was originally found in the west end of the peristyle garden, near the raised stage structure that housed a \textbf{triclinium}, and therefore in close proximity to a dining space. In the last two figures mentioned, both individuals are reclining and dining in a grove. Notably, in both images Omphale and Hercules are reclining in different ares. The alter piece in the imagery replaces the conventional \textbf{mensa} (table) as the centrepiece.

In a triclinium we would often see permanent or movable couches (\textbf{lecti}) arranged around a central table (\textbf{mensa}). Going from left to right the couches were called \textbf{imus} (lowest), \textbf{medius} (middle), and \textbf{summus} (highest). The host and his family would be delegated to the \textbf{lectus imus}, high-status guests the \textbf{lectus medius}, and low-status guests the \textbf{lectus summus}.

In our imagery Omphale takes the primary role at the \textbf{lectus imus} and assumes the role of \textbf{domina}, blurring conventional gender lines for traditional dining practices. The inclusion of not only Omphale and Hercules, but also other mythic figures within a Dionysian setting would allow diners to consider the experiences of other social classes, foreigners and women, thereby promoting a sense of identity and community.

\begin{rmk}
    Clothing was a clear marker of social status in the Roman world. According to Kelly Olson `the sartorial blurring of gender boundaries was ridiculed and censured by many' (p.149).
\end{rmk}

The roles of Dionysian allusions and contexts should not be underestimated for their ability to subvert conventional social and gender roles. (p. 150)


\subsection{Situating Omphale and Hercules in Ovid's Fasti}


The \textbf{Lupercalia}'s patron diety, \textbf{Faunus}, enters the sleeping chamber and unsuccessfully attempts to assault the cross-dressed Hercules in Ovid's Fasti, mistaken for Omphale. Faunus' distaste for sartorial deception subsequently serves as the basis for the rites' unclothed attendees.

\begin{rmk}
    The scene takes place in a Lydian vineyard grove dedicated to Bacchus and forms a basis for Ovid's acknowledgement of Dionysian themes.
\end{rmk}

Plutarch's \textbf{Antony and Demetrius} serves to provide an analogy between Omphale and Hercules and Cleopatra and Antony.


\subsection{Weaving Omphale's and Hercules' Theatricality in the Julio-Claudian Period}

Representations of Omphale and Hercules reinforce gender and social inclusivity not only through a ritual, but also within a theatrical setting. (p. 153)

\begin{rmk}
    \textbf{Maecenas} was a patron of the Augustance poets, and a purveyor of Greek culture as well as Roman theatre. Maecenas staged performances of Omphale and Hercules on his estate. 
\end{rmk}


It is suggestive that under Augustus, Dionysus belonged to the elite sphere of Roman society. During the Augustan period, we catch glimpses of the interconnected relationships between Omphale, hercules, Dionysus, gardens, theatrical perfomances and cross-dressing.

During Nero's reign, the emperor \textbf{quo} actor came to embody the philhellenic model, which encompassed a much broader sphere of community and inclusivity. Dionysus became a diety open to all spheres of Roman society.

\begin{rmk}
    Nero revived Antony's Hellenic cultural aspirations, and continued to build associations with Dionysus. Nero was able to reach out to and create community amongst those typically excluded from traditional elite practices (p.154)
\end{rmk}

The \textbf{Dionysia Megala festival} included ritual processions involving Dionysian cross-dressing, as well as performances of tragedy and comedy to ultimately include citizens, foreigners, slaves, women and children alike. This coincides with Dionysus' liminal function `as a progenitor of \textbf{communitas}', wherevy `the community is stripped of all social barriers and social distinctions so that members of the community can experience one another ``concretely" as equal'. (p. 154)


The \textbf{Domus Aurea} provided the optimal setting to stage gender reversal roles and thereby promote community and inclusivity.

\begin{rmk}
    According to Stephanie Wyler, `the fantasy of a primitive lack of differentiation between sexes and species seems to have been at the heart of the artistic experimentations of Nero himself, both in his theatrical behaviour and in the conception of his palace.' (p. 155)
\end{rmk}


\subsection{Conclusions}









\section{Notes on Analysis and Societal Context}
\label{sec:SocCont15}

The author focuses on the foreign Lydian Queen Omphale and the hero Hercules, who offer a glimpse into how gender reversal can potentially reinforce or break down perceived social and cultural barriers for ancient Roman diners.

The author suggests that an adaptation of Ovid's \textbf{Fasti} is pertinent to our understanding of the visual representations of the pair in or near garden settings, which I call `Dionysian Theatre Gardens'. The pair serve as props in the backdrops of small-scale Julio-Claudian performances. (e.g. \textbf{pantomime}, \textbf{poetic recitations})

The author argues that the inclusion of the cross-dressed pair in a theatrical context made gender reversal an instrinsic part of both domestic and public life (p. 143)




\section{Terms}
\label{sec:terms15}

\begin{enumerate}
	\item
\end{enumerate}



%
% \begin{acknowledgement}
% If you want to include acknowledgments of assistance and the like at the end of an individual chapter please use the \verb|acknowledgement| environment -- it will automatically render Springer's preferred layout.
% \end{acknowledgement}
%
% \section*{Appendix}
% \addcontentsline{toc}{section}{Appendix}
%


% Problems or Exercises should be sorted chapterwise
\section*{Problems}
\addcontentsline{toc}{section}{Problems}
%
% Use the following environment.
% Don't forget to label each problem;
% the label is needed for the solutions' environment
\begin{prob}
\label{prob1}
A given problem or Excercise is described here. The
problem is described here. The problem is described here.
\end{prob}

% \begin{prob}
% \label{prob2}
% \textbf{Problem Heading}\\
% (a) The first part of the problem is described here.\\
% (b) The second part of the problem is described here.
% \end{prob}

%%%%%%%%%%%%%%%%%%%%%%%% referenc.tex %%%%%%%%%%%%%%%%%%%%%%%%%%%%%%
% sample references
% %
% Use this file as a template for your own input.
%
%%%%%%%%%%%%%%%%%%%%%%%% Springer-Verlag %%%%%%%%%%%%%%%%%%%%%%%%%%
%
% BibTeX users please use
% \bibliographystyle{}
% \bibliography{}
%


% \begin{thebibliography}{99.}%
% and use \bibitem to create references.
%
% Use the following syntax and markup for your references if 
% the subject of your book is from the field 
% "Mathematics, Physics, Statistics, Computer Science"
%
% Contribution 
% \bibitem{science-contrib} Broy, M.: Software engineering --- from auxiliary to key technologies. In: Broy, M., Dener, E. (eds.) Software Pioneers, pp. 10-13. Springer, Heidelberg (2002)
% %
% Online Document

% \end{thebibliography}

