%%%%%%%%%%%%%%%%%%%%% chapter.tex %%%%%%%%%%%%%%%%%%%%%%%%%%%%%%%%%
%
% sample chapter
%
% Use this file as a template for your own input.
%
%%%%%%%%%%%%%%%%%%%%%%$% Springer-Verlag %%%%%%%%%%%%%%%%%%%%%%%%%%
%\motto{Use the template \emph{chapter.tex} to style the various elements of your chapter content.}
\chapter{Fertility control in ancient Rome}
\label{FertContInRome} % Always give a unique label
% use \chaptermark{}
% to alter or adjust the chapter heading in the running head


%%% Questions to think about
%The \textbf{Thesis} or general sense of the article is ...

%The \textbf{method} the author uses to argue their point is ...

%In their \textbf{analysis} the author uses tools such as ...
% How do they look at the evidence? Do they place it in some theoretical framework? (i.e. gender studies, music studies, etc.)

%Additionally they conclude ...
% How does this compare to others throughout time? What is the societal context?

%What connections does the author portray with regard to \textbf{space}, \textbf{relationships}, \textbf{occupation}, and \textbf{religion}.


\abstract{}

\section{Questions and Remarks}
\label{sec:QR6}


\begin{qst}
    How was Fertility percieved and understood in ancient Rome?
\end{qst}


\begin{qst}
    What methods did ancient Romans use to control women through fertility?
\end{qst}


\begin{qst}
    Were the methods deployed sufficiently effective to qualify as `control', and was it `fertility' that was being acted on through adoption and exposure?
\end{qst}




\section{First Reading}
\label{sec:FirRead6}


\begin{nte}
    One key shift in the history of human procreation is from societies in which the dominant fertility project was the production of helathy children to those in which the limitation of that production dominates.
\end{nte}

The reproduction of some groups is always enabled and encouraged more than others.

Agency in the fertility domain, particularly female agency, should not be restricted to action around contraception and abortion, but understood more holistically, as recent scholarship on medicine and childbearing in medieval Europe has emphasized.

\begin{rmk}
    Behaviour was not bound simply to the number of children already born, but also to their sex and survivorship, among other considerations.
\end{rmk}


The author notes the following quote with regard to the term `fertility control':

\begin{quotation}
    While helpful in linking the prevention and promotion of procreation, the term may be too modern for centuries before the twentieth. People have always aimed to achieve certain objectives for family continuity and population size, individual health and happiness, but their conceptual and practical tools have changed.
\end{quotation}


`Control' is a modern reproductive term. There is an issue about whether `control' sets the efficacy bar too high for the pre-modern world: whether, or to what extent, success in respect to or at least real purchase on the challenges and aims involved is required to use this language. There is also the sense in which `control' has now become an aim in itself rather than a means to an end, and so perhaps lacks the categorical stability necessary to do the requisite heuristic work.


\subsection{Fertility \textit{Control}}

Circa 100 CE in Rome, the noted physician \textbf{Soranus of Ephesus} composed his \textbf{Gynecology}, the only such dedicated treatise to survive from the early Roman Empire. Soranus traveled from his birthplace to the imperial capital via the medical schools of \textbf{Alexandria}, and continued to write in his native Greek. Greek was still the dominant language of learned medicine.

\begin{rmk}
    Soranus offered instructions about how to have healthy children to Roman elite, through opposition to and criticism of past medical authorities. He positioned himself against the traditional Hippocratic view that female health depended on generation, arguing instead that women's physical well-being was undermined by her `child-production' (\textbf{teknopoia}).
\end{rmk}

Soranus' pro-procreative program started with female anatomy and moved onto a systematic study of all the processes involved in generation, from menstruation to birth and the care of the newborn. Soranus insisted that girls pass the first occurrence of menstruation and become physically mature before marrying. This was somewhat at odds with elite practice in the Roman empire. Soranus argued that questions about the fertility of any prospective pride should accompany the customary inquiries.

\begin{nte}
    All evidence indicates that the Roman elite stuck to their traditional interests in birth, money, and looks, instead of heavily considering the fertility of their bride. The women's childbearing prowess was something to be proved. THe only women who posses the virtue of ``\textbf{fecunditas}", in the Annals of the Roman historian Tacitus, for instance, have already born children.
\end{nte}

Soranus' answer to conception followed the Hippocratic view that women are most likely to concieve as their periods are dwindling and stopping. For the rest, body and soul must be in the right condition, feeling good and appropriately inclined.

\begin{rmk}
    In modern medicine, the `fertile window' refers to the six days during which heterosexual intercourse can result in pregnancy, those being the five days before and the day of ovulation itself. So this does not align with Soranus' best time. However, both the menstrual and ovulatory cycles are somewhat variable.
\end{rmk}

Guidance of care for the pregnant woman had three stages: 
\begin{enumerate}
    \item guarding the deposited seed
    \item alleviating the ensuing symptoms, such as those associated with \textbf{kissa} (characterized by cravings, nausea, and general digestive disarray)
    \item Aim at perfecting the embryo and preparing for the demans of birth.
\end{enumerate}
Every aspect of a woman's life was to be regulated.



\begin{nte}
    The first book of the \textbf{Gynecology} ends with a chapter on contraception and abortion. Soranus believed that childbearing uses up resources, saps vigor, and causes premature aging. Thus Soranus opened up conceptual space in which talk of family limitation could occur, within the pro-procreative program.
\end{nte}

The items and actions which prevent conception were called \textbf{sullepsis} or `non-birthers' \textbf{atokia}, and those which `destroy what has been conceived' were called \textbf{phthoria}.

\begin{rmk}
    `Destruction of what is carried' was controversial at the time. The opposition called Hippocrates as a witness, who said `I will give no woman an abortive', and asserted that the medical art must guard and preserve what has been generated by nature.
\end{rmk}

The proponents of judgement were mainly motivated by preventing dangers in birth, and they said the same about contraceptives. Soranus concurred with this. 

Soranus' contraceptive presecriptions can be roughly divided into three:
\begin{enumerate}
    \item The first was that the `best time' for procreative sex should be avoided.
    \item The second involved applications to the mouth of the womb prior to intercourse, preventing the entry or retention of the seed.
    \item The last were oral contraceptives.
\end{enumerate}

For the thirty days after conception do the opposite of what Soranus advised to guard the deposited seed.

\begin{nte}
    Several of the ingredients listed by Soranus have been identified as having fertility suppressing effects in a range of ethnobotanical and laboratory studies.
\end{nte}

The work of John Riddle, who was the first to survey this evidennce in relation to ancient medical writings, has been subject to sustained criticism ever since: its orientation, presuppositions, methodology, and conclusions have all been called into question. For instance, discovering what modern species might be designated by ancient plant names is far from straightforward.

Soranus explicitly located his discussion of contraceptives and abortives within marriage. In pharmacological contexts or works on medical materials, actual engagement with the business of prevention or destruction occurred in association with prostitution. 

\begin{rmk}
    The philosophical poet Lucretius, writing his Latin epic \textbf{On the Nature of THings} in the last decades of the Roman Republic, had asserted that women themselves can `prevent or resist' conception, by pulling away and becoming limp as a man climaxes. This technique however belongs to `\textbf{scorta}' (`whores'), who wish to minimize their chances of becoming pregnant and maximize their client's pleasure.
\end{rmk}

The second book of the \textbf{Gynecology} covers the business of normal birth and the ensuing care of both mother and baby. There are two important points in the detailed descriptions and instructions:
\begin{enumerate}
    \item First is the section on how the midwife (\textbf{maia}) was to recognize whether the infant she had just delivered was fit for rearing or not. The main positive indicators were that the mother had enjoyed good health during pregnancy, birth had occurred at the proper time, the newborn had cried vigorously when placed on the ground, and was well-formed in all its parts. At the end of the day it was the father's decision to rear or expose (i.e. to put the new-born out to die or for someone else to raise).
    \item The other issue of interest is the nutrition of the newborn. Soranus favored wet-nursing, aligning himself with the dominant elite practice of the early Empire, and against arguments by some philosophers and traditional moralists that women should nurse their own infants.
\end{enumerate}

The later half of the \textbf{Gynecology} deals with the diseases of women, in which dangers and damaging impact of pregnancy and parturition loom large. Difficult birth is referred to as \textbf{dustokia}. The sections on several of these uterine ailments are not preserved in their original Greek, but, apart from their headings, survive only in the later `Latinizations' of the \textbf{Gynecology} of the fifth-centery CE North African physician Caelius Aureliunus and his less firmly located successor \textbf{Muscio}. Similarly, the contents of the final chapter in book three of Soranus' composition, listed as `On non-generation (\textbf{agonia}) and non-conception (\textbf{asullepsia})' are transmitted only in Latin. 

All of the failures associated with being `sterile', in latin `\textbf{Sterilitas}', occur in the female body according to Soranus, the cause may lie with either party. All of the reasons can be treated, mostly dietetically if addressing the overall somatic condition, and through pharmacological applications or surgery if the problem is more localized and specific.

There were non-medical courses of action available to those struggling to have children in the Roman Empire. Generative failure could be caused by some sort of incongruity or incompatibility between the couple having intercourse. The suggested remedy was changing partners for better results. The formulations were mostly vague, but Lucretius clearly recommended divorce and remarriage in contexts where no progeny had been forthcoming.


\begin{nte}
    By the time Lucretius wrote his didactic epic in the first centure BCE, divors and remarriage were legally (if not practically) straightforward for both parties at Rome, especially if there no no surviving offspring. This was a variation on a key theme in Roman matrimony---the main reason for divorce in the late Republic and early empire was to remarry, for political, economic, or generative purposes.
\end{nte}

Soranus had an apparent omission of the reltational aspects of infertility. A range of texts from the imperial period demonstrate that dream interpreters, astrologers and fortune-tellers were often consulted about the production of children, pregnancy, birth, and the prospects of the new-born. The point here is simply to return to the pro-procreative shape of Roman society, with which this section opened.

Note that although not all the resources for the generative project were accessible to those below the elite, many were, at lesat in some form. Maximum effectiveness still resides in infant exposure and adult adoption, however, so it is to these phenomena we now turn.

\subsection{\textit{Fertility} Control}

As \textbf{Soranus} assumed, in the Roman world birth was followed by a decision about whether to rear the newborn. A positive judgement meant being welcomed into the family and community, while a negative one entailed the separation of the child from their natal famly through exposure, their being put out (\textbf{ekthesis}) either to die or be picked up and raised by someone else. The main reason for third party rescue was to bring up the infant as a slave. \textbf{Exposure} was about separation or rejection, not about the fate of the child. It was a means of regulating family size and family composition.

\begin{rmk}
    Soranus described a physical assessment of suitability to rear, one that was entirely gender neutral, but other ancient sources and modern scholarship raise the possibility of selectivity by sex in these post-parturition judgements, a selectivity that favored boys over girlds.
\end{rmk}

Issues of sex and disability surely played a role in Roman decision making about raising children, but in complex and relative rather than absolute ways.

Control can be exercised over quantity and quality, and the efficacy of \textbf{expositio} is obvious in respect to both. Until the development of reliable fetal sex discernment tests in the twentieth century, exposure and infanticide were the only means of sex selection in relation to offspring. 

\begin{qst}
    Did the Roman sources themselves include \textbf{expositio} with other forms of family limitation or considered it as a distinct practice? Where did it fit in the overall demographic system of the Roman world?
\end{qst}

Soranus' approach was essentially inclusive, covering contraception, abortion, and exposure, as well as infertility treatments, in a single treatise. Soranus' role was limited by the role of the \textbf{maia} as reporter's of the newborn's physical condition to those in the family who would make the actual decision: most critically, the father, in whose power (\textbf{patria potestas}) any child raised would most likely be.

Roman law made all legitimate offspring, female and male, automatic heirs (\textbf{sui heredes}) who had to be left a fair share of the estate unless explicitly disinherited.

\begin{nte}
    A couple decades before Soranus was writing, the Stoic moralist \textbf{Musonius Rufus} argued strongly in support of the thesis that all children born should be raised, which was more or less the positon of the \textbf{Stoa}.
\end{nte}

Musonius mainly had an issue with the wealthy who chose not to rear later-born offspring so that those earlier born may inherit greater wealth. This was essentially a civic argument. Having lots of children was an obligation citizens owed to the state and the gods, though the benefits accrued to both the community and the family concerned, far outweighing the pragmatic excuses for limiting offspring that he dealth with. Musonius also praised a variety of measures against abortion and contraception, public rewards for parents of multiple progeny and penalties for the childless.

\begin{rmk}
    The end of marriage, through death or divorce, could have resulted in the exposure of any progeny born in the aftermath. Both pragmatic and emotional reasons seem to have been in play, including matters of inheritance.
\end{rmk}

For example, there is the question of `fatherless' children, those born to a woman not in a ROman marriage (\textbf{iustum matrimonium}), so who were not born in \textbf{patria potestas} with all that entailed. These were not babies born to a `single' woman, one who society deemed should not be having children or was having them by the wrong man, for example in adultery. So, though those latter women would likely have exposed their offspring, the numbers involved were probably small.

The raising of foundlings, a kind of `fostering', became a regular and to some extent regulated occurrence in the Roman world. It seems that certain local places became informally known as spots where newborns would be put out and could be taken up, by anyone who wanted to. Another possibility beyond slavery is that \textbf{expositi} might be smuggled into reasonably wealthy, even positively elite households lacking offspring and presented as the product of their marriages by wives unable or unwilling to bear children for themselves. 

\begin{nte}
    Legislation and juristic discussions condemned the practice---there was no time-limit on fraud accusations concerning the introduction of such children, for instance---but they also recognized that husbands might collude in such undertakings as well as their primary victims.
\end{nte}

Under classical Roman law, exposure did not affect the birth status of the infant. It remained free if born to a freewoman, and remained in \textbf{patria potestas} if that woman was in a Roman marriage. It allowed these redemptions as long as the person who had raised the foundling was compensated for what they had spent on maintenance by the natal family. While some imperial rulers permitted this kind of purchase of freedom to be enforced in parts of Greece, the emperor \textbf{Trajan} preferred the principle. He stresses the inviolability of free birth; if the status were proven, they should not have to `buy back their freedom'.

Some \textbf{expositi} did return to their original homes. In fact, that may have been the plan all along. This chimes with the idea that among the married poor exposure mostly a response to a specific crisis, rather than to poverty as such. If desperate circumstances compelled them to put out a newborn it may well have been in the hope of future recovery, when things had improved, thus locating \textbf{expositio} among the adaptive strategies developed to spread the burden of childbearing and improve procreative outcomes as well as among the methods of family limitation.

In Roman adoption a man who lacked a direct heir could acquire one, more or less fully formed, from another lineage to inherit his family name and cult as well as property. Adoptive households should roughly replicate natural ones. The model adopter was over sixty or otherwise known to be unable to procreate, had tried to have and maintain his own children, without lasting success. He should adopt an adult male at least eighteen years his junior, of similar social status if not actually part of the same kin group. The adoptee should also come from a family which could bear his transfer elsewhere, indeed his move would ideally benefit both parties.

\begin{rmk}
    Adopted children were legally in the same relationship to their \textbf{paterfamilias} as children who had been born to him in a legitimate marriage, but they had been raised by someone else. That raising, the emotional and material resources invested in it, its formative effects, the physical and moral resemblance between parents and offspring it forged, left its mark and was neither wiped out nor replaced by the formal transfer to a new family.
\end{rmk}

Less formal practices of fostering, of raising the offspring of others, might produce closer emotional ties, but without the same legal results: foster-children could not be heirs in the same way that adopted sons were. The consent of the adoptee was only relevant if the father was dead. This also meant that any offspring born to a master by his slave women, since they followed the status of the mother, could not be adopted and while it would have been theoretically possible to adopt children produced outside marriage, if the mother were a citizen, there is no evidence that this happened.

\begin{nte}
    The position of the adoptee, at least in elite circles, would have been socially untenable, and his inheritance would undoubtedly have been challenged in the courts with some chance of success.
\end{nte}


\subsection{Conclusions}

The author has aimed to enable a fuller assessment of questions of control over those matters of the procreative project in the Roman world, as part of a longer history of fertility control. To summarize, everybody was in the business of family continuity, of having children to pass their name, status, cult, and whatever property they might have owned on to, of forging links to posterity. The slaves, wanting to have free children, to establish and then enact the possibility of family continuity after a period of generalized, definitional lack of control, including over their fertility.

It is important to distinguish between the elite and the rest. For the vast majority while there would have been definite advantages to birth spacing, achieved through abstinence and breatsfeeding, absolute limits were not an issue. Parents seem generally to have wanted both sons and daughters, a son first and foremost to ensure the continuity fo the paternal line but also daughters, who made a range of important contributions to the family enterprise. Sex-selective exposure might have been deployed in such circumstances of single sex offsprings, but decisions to raise children were largely in response to crisis, albeit in a precarious world, where food shortages and famine were not infrequent, and with some wishful hopes of retrieving those given up when fortunes improved.

Birth spacing for the elite was neither suffcient nor so easily organized, given the reliance on wet-nurses; a pattern of rapid generation of some sons and daughters and then stopping, with the possibility of re-starting after either child mortality or a new marriage was more suited to family needs. Though Soranus attempted to facilitate this through making contraception and abortion available to respectable married women and not just prostitutes, to protect those women from the most damaging effects of repeated childbearing, his recommendations would have been of limited efficacy, in respect to either pregnancy or well-being. Control would have come from abstinence or exposure, ultimately relying on the latter, without any benefits to female health. 

\begin{rmk}
    This was, as Musonius indicated, the dominant means of family limitation but one that operated within a wider suite of actions with the same aims, all of which he opposed while promoting moves encouraging childbearing.
\end{rmk}





\section{Notes on Analysis and Societal Context}
\label{sec:SocCont6}


The author provides a survey of methods used to promote but also prevent pregnancy in ancient Rome. The author also discusses the pracices of adult adoption and infant exposure in more detail in order to interrogate the notion of `fertility control'.


The author argues that the Roman case has plenty to offer wider debates about the history of reproduction as it includes the desires to have and not to have children, to limit and increase offspring, to shape families in different ways.

The author argues that the fact that in all cases families and individuals and communities had procreative aims toward which they consciously worked suggests a long-term narrative in which the reproductive project itself, whether more expansive or restrictive, provides the unifying thread to be tracked and analyzed.

\begin{nte}
    The author pays careful attention to definitional issues.
\end{nte}



\section{Terms}
\label{sec:terms6}

\begin{enumerate}
	\item
\end{enumerate}

%
% \begin{acknowledgement}
% If you want to include acknowledgments of assistance and the like at the end of an individual chapter please use the \verb|acknowledgement| environment -- it will automatically render Springer's preferred layout.
% \end{acknowledgement}
%
% \section*{Appendix}
% \addcontentsline{toc}{section}{Appendix}
%


% Problems or Exercises should be sorted chapterwise
\section*{Problems}
\addcontentsline{toc}{section}{Problems}
%
% Use the following environment.
% Don't forget to label each problem;
% the label is needed for the solutions' environment
\begin{prob}
\label{prob1}
A given problem or Excercise is described here. The
problem is described here. The problem is described here.
\end{prob}

% \begin{prob}
% \label{prob2}
% \textbf{Problem Heading}\\
% (a) The first part of the problem is described here.\\
% (b) The second part of the problem is described here.
% \end{prob}

%%%%%%%%%%%%%%%%%%%%%%%% referenc.tex %%%%%%%%%%%%%%%%%%%%%%%%%%%%%%
% sample references
% %
% Use this file as a template for your own input.
%
%%%%%%%%%%%%%%%%%%%%%%%% Springer-Verlag %%%%%%%%%%%%%%%%%%%%%%%%%%
%
% BibTeX users please use
% \bibliographystyle{}
% \bibliography{}
%


% \begin{thebibliography}{99.}%
% and use \bibitem to create references.
%
% Use the following syntax and markup for your references if 
% the subject of your book is from the field 
% "Mathematics, Physics, Statistics, Computer Science"
%
% Contribution 
% \bibitem{science-contrib} Broy, M.: Software engineering --- from auxiliary to key technologies. In: Broy, M., Dener, E. (eds.) Software Pioneers, pp. 10-13. Springer, Heidelberg (2002)
% %
% Online Document

% \end{thebibliography}

