%%%%%%%%%%%%%%%%%%%%% chapter.tex %%%%%%%%%%%%%%%%%%%%%%%%%%%%%%%%%
%
% sample chapter
%
% Use this file as a template for your own input.
%
%%%%%%%%%%%%%%%%%%%%%%$% Springer-Verlag %%%%%%%%%%%%%%%%%%%%%%%%%%
%\motto{Use the template \emph{chapter.tex} to style the various elements of your chapter content.}
\chapter{Beyond Ritual: Cross-dressing between Greece and the Orient}
\label{BeyondRitual} % Always give a unique label
% use \chaptermark{}
% to alter or adjust the chapter heading in the running head


%%% Questions to think about
%The \textbf{Thesis} or general sense of the article is ...

%The \textbf{method} the author uses to argue their point is ...

%In their \textbf{analysis} the author uses tools such as ...
% How do they look at the evidence? Do they place it in some theoretical framework? (i.e. gender studies, music studies, etc.)

%Additionally they conclude ...
% How does this compare to others throughout time? What is the societal context?

%What connections does the author portray with regard to \textbf{space}, \textbf{relationships}, \textbf{occupation}, and \textbf{religion}.


\abstract{}

\section{Questions and Remarks}
\label{sec:QR11}

\begin{qst}
    What did Romans consider to be effeminate behaviour in view of manly ideals?
\end{qst}


\begin{qst}
    Who was \textbf{Hermaphroditus}? How did they cross between the realms of woman and man?
\end{qst}


\begin{qst}
    How does dress place Hermaphroditus in a \textbf{liminal/transitional} status?
\end{qst}

\begin{qst}
    How do dress and the body figure into representations of Hermaphroditus?
\end{qst}


\begin{qst}
    How could Facella's work be used to enhance or counter our understanding of Hermaphroditus?
\end{qst}


\begin{qst}
    How can Facella be used to unpack Hermaphroditus?
\end{qst}


\begin{qst}
    What is \textbf{Hybristika}?
\end{qst}


\begin{qst}
    What is \textbf{Le crime des Lemniennes}?
\end{qst}


\section{First Reading}
\label{sec:FirRead11}


Cross-dressing episodes in the ancient Greek and Latin texts has been explained as \textbf{strategemata}, \textbf{exempla}, \textbf{virtutum}, \textbf{thaumasta}, or \textbf{aitia}.

\begin{rmk}
    There is a risk in analyzing this evidence of exponentially departing from the specific situation pertaining to each fact and of closely linking phenomena which can be connected only up to a certain point.
\end{rmk}


\subsection{Hybristika: ritual cross-dressing, aetiology, and the comparative approach}

\textbf{Plutarch} in \textbf{De Mulierum Virtutibus} 245 C-F mentions the existence at \textbf{Argos} of a festival called \textbf{Hybristika} (``Outrageous Acts"), during which men and women exchanged dress. Plutarch associates these celebrations with the struggle between the Spartans and the Argives (c. 494 BCE), and relates that the Argive women led by Telesilla successfully defended their city against the enemy. 

\begin{nte}
    \textbf{Hybristika} also translates to the ``Festival of Impudence."
\end{nte}

\begin{rmk}
    To correct the loss of population, the women were united with the best of the people living around (\textbf{perioeci}), whom they made citizens. The women, however, seemed to despise their husband, hence the law which orders married women to wear beards when they go to bed with their husbands.
\end{rmk}

\textbf{Herodotus} reports in his version that the oracle consulted before the battle by the Argives in conjunction with the Milesians declares the victory of the \textbf{teleia} (feminine element) over the \textbf{arsen} (masculine element); but Cleomenes does not attack Argos. The attention is focused instead on the \textbf{oliganthropia} of Argos and the consequent conquest of the city by the slaves.

\begin{rmk}
    The introduction of the women warriors and the poetess Telesilla, the Argive perspective permeating the entire passage, and the differences compared with the Herodotean version (in particular, the substitution of slaves with perioikoi) suggest that it was a later re-elaboration by Socrates and other local historians aimed at improving the reputation of their city.
\end{rmk}

The account has no historical reliability, and the role reversal in the story is meant to explain the role reversal of the ritual. The story of Telesilla and the women warriors was suitable for explaining both the origin of a celebration where men and women experienced a temporary reversal of social roles, and of a bizarre custom in which Argive brides wore a false beard on their wedding night.

\textbf{Martin Nilsson} compares this tradition with others involving the wearing of clothes of the opposite sex. Nilsson recalls the cult of the bearded Aphrodite at Cyprus, whose worshippers wore clothes of the opposite sex; the tradition of a Spartan bride wearing a man's cloak and sandals; and the sacrifices to Mutunus Tutunus by Roman women dressed in male clothes.

\begin{nte}
    Nilsson explains this rituals as marriage customs in which the enactment of cross-dressing was directed at confusing the powers of evil. (p.110)
\end{nte}

\textbf{Ernest Crawley} had observed the wedding practices seen in Argos, Cos, and Sparta in various `primitive' societies and which he described as `inoculation': a means of overcoming sexual taboos. \textbf{James Frazer} Noted that ritual transvestism went beyond the marriage sphere. Frazer explored the custom of transvestite priests in the Pelew Islands (Western Pacific) and collected comparanda in the ancient and contemporary world, to realise that a single solution applicable to all cases was unlikely.

\begin{quotation}
    the custom of men dressing as women and of women dressing as men has been practised from a variety of superstitious motives, among which the principal would seem to be the wish to please certain powerful spirits or to deceive others. (p.110)
\end{quotation}

\textbf{Robert Halliday} viewed the donning of the clothes of the opposite sex as a typical `rite de passage', aimed at creating a sense of unity in those who performed it. We see the exchange of clothes between boys and girls at the circumcision ceremonies of the Egyptians and of the Nandi in East Africa, the masquerade of men in female attire in some Northern African carnivals, and the `Geese dancing' of Cornwall and the Isles of Scilly in the UK were aligned with several traditions recorded by Graeco-Roman sources.

\begin{rmk}
    Halliday observed that the donning of the clothes of the other sex takes place at \textbf{transitional moments} which can concern the individual (circumcision, marriage, mourning, initiation of seers), as well as the entire community (seasonal and renewal feasts, festivals with social reversal). (p.110)
\end{rmk}

The construction of a theoretical paradigm based on ethnological comparisons caused an oversimplification of a multifaceted phenomenon and its confinement to the ritual sphere.


\subsection{Le crime des Lemniennes and functional disguise}

\textbf{Georges Dum\'{e}zil} investigated the proverbial myth of the \textbf{dysodia} of the Lemnian women, the ``foul smell" which had caused a sexual refusal by their husbands and the consequent massacre of the male population of the island by the outraged females. Dum\'{e}zil showed that we have a pseudo-historical projection of a ritual which, as \textbf{Philostratus} records, took place on the island every year: the legend of the \textbf{malodorous Lemniads}, who killed their husbands, ruled their country alone, and then were reconciled with the male sex when Jason and his companions arrived.

The children born from the union of the Lemniads with the Argonauts are referred to as the Mynians.

\begin{rmk}
    The Mynians married Spartan women, and were cast into prison due to their arrogance and impious. Their wives helped them escape by dressing them in women's clothing.

    Dum\'{e}zil follows Fredrich, and maintains that in the Lemnian ritual we see an aetiological explanation of an ancient Spartan ritual; an exchange of clothes between the sexes would be in its place.
\end{rmk}

Neither Herodotus nor Plutarch make a correlation between this legend and a Spartan festival. In other words, the existence of Spartan ceremonies associated with the Mynian deeds is a matter of guesswork and so is the cross-dressing procession which is supposed to have taken place in these ceremonies.

The episode of the Mynians does show some socio-political dynamics which typically develop between incomers and the indigenous population: new marriages, a new ethnic mix, and new conflicts. Herodotus narrates how some Macedonian men, disguised in women's clothes, killed their Persian hosts who had tried to take advantage of them. This strategy during war was seen quite often. These episodes cannot be reasonably classified as examples of ``ritual cross-dressing."

\begin{rmk}
    In the episode of the Mynians as well as in the others, what is described is a functional disguise, which must be distinguished from ritual cross-dressing or transvestism. (p.112-113)
\end{rmk}


\subsection{Ritual cross-dressing or socially subversive dressing?}

The story of \textbf{Aristodemus Malakos turant of Cumae} is often cited as an example of ritual cross-dressing. Our main source is \textbf{Dionysus of Halicarnassus}.

\begin{rmk}
    Dionysus says that Malakos ordered the boys to wear their hair long like girls, to keep it curled and to bind up tresses with hair nets, to wear embroidered robes, and over these thin and soft mantles, and to pass their lives in the shade.
\end{rmk}

Plutarch recounts that Aristodemus ``accustomed the boys to wear long hair and golden ornaments, and he compelled the girls to cut their hair short around the neck, and to wear youths' cloaks over their short chitons". Jacques Boulogne remarks that we may be in the presence of old rites, and compares this case with the story of Hybristika and that of the Mynians. 

\begin{nte}
    Plutarch describes the tyranny at Cumae as an overturning of normal social relations. Dionysus on the other hand argues that Aristodemus's impositions suited a precise political design, and were intended to weaken the young Cumaeans so as to make them unsuitable for the government of the city.
\end{nte}

A passage by \textbf{Athenaeus} on the people of Tarentum shows how the gender of determination of a garment could vary (or become restricted) over time: ``He [= Clearchus of Soli] says that all men wore transparent garments with a purple border, which are today a refinement of women's life."


\begin{rmk}
    Aristodemus is likely to have extended to the young citizens a refined education, which until that time had been restricted to the aristocracy, and this change was seen negatively by the later tradition. It is likely therefore that fashion, rather than a mysterious ritual, was behind Aristodemus's agency. (p.114)
\end{rmk}


We can compare to a passage on the tyrant \textbf{Ortyges} and his followers preserved by \textbf{Athenaeus}.

\begin{rmk}
    Ortyges and his companions, having possessed themselves of the supreme power in \textbf{Chios}, destroyed all who opposed their proceedings, and they subverted the laws, and themselves managed the whole of the affairs of the state. They tried all actions, sitting as judges, clothed in purple cloaks, and in tunics with purple borders, and they wore sandals with many slits, but in winter they always walked about in women's shoes; and they let their hair grow, and took great care of it so as to have ringlets, dividing it on the top of their head with fillets of yellow and purple. 
\end{rmk}

Once again we find the historiographical pattern that associates tyranny with a reversal of social conventions (p.114). The fashion of long hair recalls Aristodemus and the \textbf{Koronistai}, the long-haired young men with whom he fought against the barbarians.

\begin{nte}
    The description of these tyrants has been deeply moulded by the later historiographical tradition, so it is the negative attitude towards them which attributes a female connotation to what was actually a fashion, or an outward way of distinguishing themselves by certain elite groups. (p.114)
\end{nte}

\subsection{Transvestites for love}

\textbf{Effeminacy} is one of the main ingredients constituting the Graeco-Roman stereotypical image of a decadent `Oriental' court society. It is in this setting that examples of women who cross-dress and assume a male role may be found.

\begin{eg}
    \textbf{Aelianus} records in \textbf{Varia Historia} of the case of \textbf{Aspasia of Phocea}, a girl of humble origin, who first became the concubine of the younger Cyrus and then of his brother Artaxerxes. When Tiridates Artaxerxes' favourite eunuch died, Aspasia was the only one who managed to console the grieving king. Artaxerxes put the eunuch's cloak over Aspasia's black dress.
\end{eg}

Aspasia is also mentioned by Xenophon and Plutarch. In the case of Aspasia, the cross-dressing is enacted for erotic reasons and is limited in time and space (p.115).

\begin{eg}
    \textbf{Hypsicrateia} was the concubine of \textbf{Mithradates Eupator}, and she cut her hair short and was accustomed to ride a horse and to use weapons, so that it was easier to take part in his fatigues and dangers. Plutarch specifies that the woman wore the garment of a Persian male, and ``the king called her \textbf{Hypsicrates}". In 2005, in the Taman peninsula, at Phanagoreia, a marble base, part of a funerary monument, was uncovered. The \textbf{gyne} is here recorded with the name in the masculine form.

    Amazonian traditions can provide at most reasons for the adoption of a `Persian' male costume by the concubine; what they fail to do is to explain her adoption or acceptance of a male name.
\end{eg}



\subsection{Conclusions}

The collection of cases assembled over time to support an exclusively ritualistic basis for cross-dressing includes incongruous exampls and inevitably inducs us to oversimplify a more complex reality. The predominance of this type of cross-dressing in our evidence is a reflection of the Greek approach to this practice (p.116).

Beyond the ritual or strategic spheres, cross-dressing behaviours were considered socially disturbing, as the association between tyranny and transvestism in our sources reveals. Positive comments are expressed only for two female cross-dressers, concubines of Persian kings, who were hence considered as belonging to an eccentric `Oriental' world and presented as having renounced their femininity for love.

\begin{rmk}
    Artemidorus says ``A woman's attire is auspicious only for bachelors and those who act on the stage. . . . But dreaming that one is wearing a colourful, or a woman's, garment at feasts and festivals does not hurt anyone."
\end{rmk}




\section{Notes on Analysis and Societal Context}
\label{sec:SocCont11}

The author focuses on a few items of evidence which should alert us to the temptation of associating episodes of cross-dressing of a different nature and always explaining them as an expression of religious ritual. 


\section{Terms}
\label{sec:terms11}

\begin{enumerate}
	\item
\end{enumerate}


%
% \begin{acknowledgement}
% If you want to include acknowledgments of assistance and the like at the end of an individual chapter please use the \verb|acknowledgement| environment -- it will automatically render Springer's preferred layout.
% \end{acknowledgement}
%
% \section*{Appendix}
% \addcontentsline{toc}{section}{Appendix}
%


% Problems or Exercises should be sorted chapterwise
\section*{Problems}
\addcontentsline{toc}{section}{Problems}
%
% Use the following environment.
% Don't forget to label each problem;
% the label is needed for the solutions' environment
\begin{prob}
\label{prob1}
A given problem or Excercise is described here. The
problem is described here. The problem is described here.
\end{prob}

% \begin{prob}
% \label{prob2}
% \textbf{Problem Heading}\\
% (a) The first part of the problem is described here.\\
% (b) The second part of the problem is described here.
% \end{prob}

%%%%%%%%%%%%%%%%%%%%%%%% referenc.tex %%%%%%%%%%%%%%%%%%%%%%%%%%%%%%
% sample references
% %
% Use this file as a template for your own input.
%
%%%%%%%%%%%%%%%%%%%%%%%% Springer-Verlag %%%%%%%%%%%%%%%%%%%%%%%%%%
%
% BibTeX users please use
% \bibliographystyle{}
% \bibliography{}
%


% \begin{thebibliography}{99.}%
% and use \bibitem to create references.
%
% Use the following syntax and markup for your references if 
% the subject of your book is from the field 
% "Mathematics, Physics, Statistics, Computer Science"
%
% Contribution 
% \bibitem{science-contrib} Broy, M.: Software engineering --- from auxiliary to key technologies. In: Broy, M., Dener, E. (eds.) Software Pioneers, pp. 10-13. Springer, Heidelberg (2002)
% %
% Online Document

% \end{thebibliography}

